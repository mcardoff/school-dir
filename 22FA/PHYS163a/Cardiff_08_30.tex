\documentclass{beamer}

\usepackage{hyperref}
\usepackage{url}

\newenvironment{itemframe}[1]{\begin{frame}{#1}\begin{itemize}}   {\end{itemize}\end{frame}}

\title{Aug 30 Prep Work}
\author{Michael Cardiff}
% \logo{\large \LaTeX{}}
\subtitle{PHYS 163a}

% Changes style of actual slides
\usetheme{Dresden}
% Changes color of slides
\usecolortheme{spruce}
% removes controls at bottom right side
\usenavigationsymbolstemplate{}
\begin{document}

\begin{frame}
  \titlepage
\end{frame}

\section{What is Thermodynamics?}
\begin{frame}{Given Definition}
  The definition given for thermodynamics is:
  \begin{center}
    \textit{\textbf{Thermodynamics} is a \underline{phenomenological} description of properties of \underline{macroscopic systems} in \underline{thermal equilibrium}}
  \end{center}
  We are asked the definition of the underlined terms
\end{frame}

\begin{frame}{What do these words mean?}
  \begin{itemize}
  \item \underline{Phenomenological} comes from phenomenology, which I think of as the theory behind a particular experimental field. 
  \item \underline{Macroscopic system} This warrants breaking down the phrase:
    \begin{itemize}
    \item A \underline{system} is a particular portion of the universe being studied\footnote{https://www.britannica.com/science/system-physics}
    \item Something which is \underline{macroscopic} is on the same scale as humans, it is not the scale of something like quantum mechanics
    \end{itemize}
  \item \underline{Thermal equilibrium} refers to when heat is not flowing in a system. This is shown by the system being at the same temperature throughout its constituents. 
  \end{itemize}
\end{frame}

\section{Thermodynamic Observables}
\begin{frame}{Thermodynamic Observables}
  In order to qualify as an observable, we should be able measure the value of the associated variable. Some examples include:
  \begin{itemize}
  \item Temperature
  \item Pressure
  \item Volume
  \item Internal Energy
  \end{itemize}
\end{frame}

\section{Intensive vs Extensive}
\begin{frame}{Intensive vs Extensive}
  \begin{itemize}
  \item An \underline{extensive} variable is one that depends on the amount of substance there is, such as the mass of a sample, or its volume.
  \item In contrast, an \underline{intensive} variable is one intrinsic to the substance, such as boiling point or density. 
  \end{itemize}
\end{frame}


\end{document}