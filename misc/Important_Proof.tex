\documentclass[12pt]{article}

\title{Proof}
\author{Michael Cardiff}
\date{\today}

%% science symbols 
\usepackage{amsmath}
\usepackage{amssymb}
\usepackage{physics}

%% general pretty stuff
\usepackage{bm}
\usepackage{enumitem}
\usepackage{float}
\usepackage[margin=1in]{geometry}
\usepackage{graphicx}

% figures
\graphicspath{ {./figs/} }

\newcommand{\fig}[3]
{
  \begin{figure}[H]
    \centering
    \includegraphics[width=#1cm]{#2}
    \caption{#3}
  \end{figure}
}

\newcommand{\figref}[4]
{
  \begin{figure}[H]
    \centering
    \includegraphics[width=#1cm]{#2}
    \caption{#3}
    \label{#4}
  \end{figure}
}

\renewcommand{\L}{\mathcal{L}}

\begin{document}
\maketitle
We start with the following identity:
\begin{equation}
  \sin(\frac{\pi}{6})=\frac{1}{2}
\end{equation}
As well as the Taylor series for $\sin$:
\begin{equation}
  \sin(x)=\sum_{n=0}^\infty(-1)^{n}\frac{x^{2n+1}}{(2n+1)!}
\end{equation}
Truncating to first order allows us to say:
\begin{equation}
  \sin(x)=x
\end{equation}
Plugging in allows us to find the following relation:
\begin{equation}
  \sin(\frac{\pi}{6})=\frac{\pi}{6}
\end{equation}
Truncating to first order makes sense, as $\frac{\pi}{6}$ is small, so this approximation is sesnsible. We can then using the first relation:
\begin{equation}
  \sin(\frac{\pi}{6})=\frac{1}{2}=\frac{\pi}{6}
\end{equation}
Multiplying each side by 6 gives our final result:
\begin{equation}
  \pi=3
\end{equation}
\end{document}