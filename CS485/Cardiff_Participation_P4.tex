\documentclass[12pt]{article}

\title{\vspace{-3em}CS 485 Participation 4}
\author{Michael Cardiff}
\date{\today}

%% science symbols 
\usepackage{amsmath}
\usepackage{amssymb}
\usepackage{physics}

%% general pretty stuff
\usepackage{bm}
\usepackage{enumitem}
\usepackage{float}
\usepackage{graphicx}

% figures
\graphicspath{ {./figs/} }

\newcommand{\fig}[3]
{
  \begin{figure}[H]
    \centering
    \includegraphics[width=#1cm]{#2}
    \caption{#3}
  \end{figure}
}

\newcommand{\figref}[4]
{
  \begin{figure}[H]
    \centering
    \includegraphics[width=#1cm]{#2}
    \caption{#3}
    \label{#4}
  \end{figure}
}


\begin{document}
\maketitle
\section*{Question 1}
Visibility is easy to define in a sense of physics. If you need to be visible, then you need to be able to interact with photons. In order to be seen, you need to have light be reflected off of you, and then for someone else (or rather their eyes) to be there and capture those photons.

In a societal sense, we can discuss the concept of visibility as what I would consider to be a 'presense' as in online presense. In this sense, visibility means a bit more. It means that you exist in the technological world, on the internet for example. This would be something like having a facebook profile, a twitter account, or any social media. Even if you are dead, people owuld be able to see your accounts on the internet and see that you, at least to some extent, existed. In that sense, that is what visibility is, existing. To that end, we could also talk about what it means to exist, which could lead us to the sense of what it means to be perceived, and even back to where we started, with what it means to be visible or to be seen. 
\section*{Question 2}
With the dawn of the internet age, one thing has been made clear to me: It is impossible to truly delete anything. To delete something means we need to get rid of it completely, and there is no way it could possibly exist further. However, even if you are able to erase your physical media such as hard drives, etc., your online presence still exists, which you need to get rid of as well. The problem with this is that the social media companies will often back up their data for safety, (likely on physical disks so back to square 1) so your data likely still exists even if you scrub your account from the internet. You can see how quickly this spirals, but very clearly it is hard to be invisible in the internet age. 
\end{document}