\documentclass[letterpaper, 12pt]{article}

\title{CS 485 Research Paper Draft}
\author{Michael Cardiff}
\date{\today}

%% science symbols 
\usepackage{amsmath}
\usepackage{amssymb}
\usepackage{physics}

%% general pretty stuff
\usepackage{bm}
\usepackage{enumitem}
\usepackage{float}
\usepackage[margin=1in]{geometry}
\usepackage{graphicx}
\usepackage{url}
\usepackage{hyperref}
\usepackage[labelfont=bf]{caption}

% figures
\graphicspath{ {./figs/} }
\numberwithin{figure}{section}

\newcommand{\fig}[3]
{
  \begin{figure}[H]\centering
    \includegraphics[width=#1cm]{#2}
    \caption{#3}
  \end{figure}
}

\newcommand{\figref}[4]
{
  \begin{figure}[H]
    \centering
    \includegraphics[width=#1cm]{#2}
    \caption{#3}
    #4
  \end{figure}
}


\begin{document}
\maketitle\nocite{*}
\section{Introduction}
This section most likely will not be in the final paper, I intend it as an explanation/outline of the general structure of my essay. I decided to discuss a rather extreme situation, how Formula 1 (F1) decided to become more safe by adding what is called a 'halo' to the top of the driver's cockpit, made of carbon fiber and titanium, which is meant to be unobstructive to the driver while keeping them safe. I want to study a few cases here that resulted in drivers either getting seriously injured, or in the worst case, dying. I am going to look at the crashes of 3 drivers over 3 different F1 seasons. First the crash that resulted in the death of Jules Bianchi at the 2014 Japanese Grand Prix (GP). I use the BBC news article \cite{bianchi} from after the crash which detailed important information from the official incident report which, I do not have access to. This immediate produces a problem (my first section), F1 drivers are dying from horrible crashes. After Jules Bianchi's death, the halo was introduced (the start of the section 'How did we learn?'), I want to then dive deep into two cases since the halo was introduced, and how drivers have been saved by the halo, this will be another section asking whether or not the introduction of the Halo succeeded. After this I want to compare and contrast this to how machines 'learn' which I believe it is eerily similar to. In order to facilitate discussion on improving the efficiency of this type of learning, it would be nice to discuss a few of the other improvements to safety other than the halo, since there have been quite a few, and the halo is one of many innovations. With that, we can determine a method for which computers could possibly find a new safety innovation which went unseen by the engineers or race specialist from before. I want to conclude with a small discussion on whether or not that solution is viable, and how safety improvements can be made without having to risk the lives of actual humans.
\section{The Problem}
Formula 1 (F1) is bar none the fastest, most intense and prestigious motor-sport in the world. Twenty drivers from ten teams each race to be crowned a champion of the world. The drivers participate in a number of races each year, with their final position in the race determining a number of points they get in the championship, and whoever gets the most points in a given year is the (drivers) champion. In order to be competitive, you have to drive fast, and these drivers definitely drive fast, at over 200 miles per hour, no vehicles that race on complex circuits are faster. This includes the inherent risk of those drivers doing something wrong, humans are imperfect after all. One case where this could go terribly wrong is the case of Jules Bianchi during the 2014 Japanese Grand Prix. 

Bianchi crashed into a medical car which was attending to another crashed vehicle at the time. The driver of the other vehicle was relatively okay, considering that a number of months later, Bianchi died of the injuries sustained from this crash. During this time F1 cars had an open cockpit, with the driver's head (seen in their helmet) sticking out. This is a massive risk to the driver's safety, as anything that is flung at the driver's head, whether it would be debris from crashes, would impact the head directly. In the case of Jules Bianchi, he sustained a "catastrophic head injury" \cite{halogood} from his crash, which, in the end, would have been prevented if there was sufficient protection of his head. This presents the governing body of F1, known as the FIA with a problem, as deaths like that of Jules Bianchi are completely preventable, and should not happen again if at all possible.

The circumstances of Bianchi's accident are relevant here. The article in \cite{bianchi} lists the important details of his crash. The most important detail is the weather, on the lap of Bianchi's crash, the track was completely wet, and it was still raining. This makes the F1 driver's job 1000x harder than it already is, so conditions already are not in his favor. The next important detail was the fact that Bianchi was going a touch too fast, and had not slowed down when faced with the yellow flags that tell a driver to slow down, as there was an accident up ahead. This, especially in the rain, could lead to a driver easily losing control of their car. This is important to note, as it is something that Bianchi did which contributed to the crash. However, he is an F1 driver, so of course he is going to go as fast as possible despite the conditions. The report mentions multiple details which lead to an unfortunate set of circumstances for Jules Bianchi, including issues with the software of the car, the position of vehicles attempting to make the track safer. Ultimately, the FIA determined that Bianchi's injuries were more than anything, a result of the unfortunate circumstances which he was put in. This then begs the question, what could have been done to prevent such circumstances.

\figref{5.0}{halo}{The Halo Shown on a 2018 F1 Car \cite{halopic}}{\label{fig:1}}
Many solutions were proposed that could have prevented this crash. The first was a software override of the car, which allowed for the race stewards to set a virtual speed limit on the car so speeding in this case would not be a problem. However, software fails, just as it did with this specific incident. Another was an age limit for drivers, which would inevitably require drivers to have more experience in order to drive in F1. This is irrelevant to the Bianchi incident, since Bianchi had years of experience prior to his crash, so the crash is unrelated at all to years of experience. The final did not come until years later, in 2018, when F1 introduced a titanium "halo" (called a secondary roll structure by \cite{fia2018}) meant to protect the drivers in the cockpit while being minimally obtrusive. Pictured below in Figure \ref{fig:1}, the Halo is a y-shaped structure that connects the nose of the car to the rear section which supports the engine. 
Since its introduction, the Halo has saved numerous lives. To name a few, I can immediately think of crashes like those of Charles Leclerc and Marcus Ericsson in 2018, where if the halo was not there, the drivers may not have survived their respective crashes. However two major crashes come to mind when I think of the halo. The first is the crash of Romain Grosjean in the 2020 Bahrain Grand Prix, the second is a crash between two drivers, Max Verstappen and Lewis Hamilton, in the 2021 Italian grand Prix. These two crashes are exemplary of how the FIA has learned from the crash of Jules Bianchi by preventing the certain death of two drivers. 
\section{How Did We Learn?}
It is important to see the result of crashes in a world that has experience the preventable death of Jules Bianchi. The first crash is in the Bahrain grand prix, where conditions were not nearly as bad as the 2014 Japanese GP, but the actual crash is much worse. Next we talk about a crash that is not as bad, but once again the result could have been so much worse if the halo was not involved.

\figref{5}{grosjeancrash}{A Figure from \cite{grosjeancrashsim} Depicting how the Halo Saved Romain Grosjean's Life}{\label{fig:2}}
The 2020 Bahrain GP had nothing wrong, there was no rain (how could there be if its in the middle of the desert?). An on track battle occurred in the first lap, when there is usually chaos. Grosjean was attempting to overtake his teammate, when the car behind him was trying to overtake as well, in the process, Grosjean hit the front tires of the car behind him, sending him spinning off towards the barriers \cite{grosjeanvideo}. The result of this was Grosjean being sent into the barriers. However, unfortunate circumstances struck once again, as the lower of 2 barriers collapsed, which caught the top of the engine, causing the engine to separate at the point where fuel is injected, causing the fuel to catch fire. We will discuss the fire safety later, but the main importance of this crash can be seen in the video \cite{grosjeancrashsim}. Grosjean went straight into the barriers, however, his halo being so rigid and strong, it was able to push the barrier out of the way of his head, allowing Grosjean to pass unharmed from a physical crash, see Figure \ref{fig:2}. At best, without the halo, grosjean would have a very bad migraine, and at worst, he could have been decapitated. This shows that the FIA have learned from the mistakes that lead to the Bianchi crash, and have taken steps which, if not taken, would have clearly resulted in the death of Grosjean.

We will now move on to talk about the crash in the 2021 Italian GP
\section{Did it Succeed?}
This is a clear, although extreme example of reinforcement learning. The clear comparison to machine learning here is that stimulus that a computer would be programmed to see as bad occurred in the death of Jules Bianchi. As a response the FIA (or our hypothetical computer) took direct steps to prevent this, even taking a couple tries before it was perfected. The contrast is in the way that the solution was arrived at seems to not be reachable by a computer, as it seems to require extensive knowledge of humans, cars, and crashes.

Reinforcement learning is a section of machine learning which is Pavlovian. This is in the sense that good 'behavior' is rewarded and 'bad' behavior is punished. The punishment to the FIA is less a physical one, not like a slap on the wrist or not getting a treat, but rather it is a moral one. A moral punishment is something that weighs on your conscious more than anything, in our case having a life on your hands. This is an interesting case where it would be useful for a machine to have fear. We would want the fear to be intense enough so that the machine does not want someone to die, but not too intense as to stop the cars entirely as it thinks driving is too dangerous. Even in humans this balance could be difficult to achieve, so it is hard to realize in terms of machines as well.

The solution reached by both a computer and the FIA may have drastically differed. This is even true between different racing series. In the NTT IndyCar series, they use a windscreen, which has its own advantages, but nonetheless a different solution. This does not mean that a computer could not have produced an equally effective solution as the FIA did, but instead it could produce a less conventional one, which the FIA could not think of. Another clear cut difference is the time frame. It took the FIA years to develop this solution, and even in a time where machine learning was not as advanced, so developing a machine-generated solution could have taken just as long, but more likely much longer, since we are not even sure if that solution would be effective, so the FIA would need to conduct its own tests etc. 
\subsection{Could Machines Have Prevented This?}
In essence, no. Given the details in the last section, this sort of ML takes time, and even then, producing this solution and rigorously testing it would take even longer. So, could machine learning have prevented the death of Jules Bianchi, or the great harm that was caused to Romain Grosjean? Probably not. This leads to the question, what is the value of producing machine learning solutions to F1 crashes? The answer is quite simple, for the future.

In order to be proactive in the prevention of future crashes, F1 needs to gather more information on crashes, these all could be used to train some sort of AI to learn what the chief causes of crashes are, and thus the most optimal places to insert driver protection. (EXPAND THIS SECTION)
\section{The Future}
The future should include some sort of AI that is trained wtih the fatal and nonfatal crashes from F1's history, and should be able to figure out what can be done on the end of the FIA's end to ensure the safety of the drivers. This notion is clear from various sections of \cite{fia2021} which lists the various regulations related to the safety of drivers in F1. With the innovations of the future, we may not need machine learning to prevent crashes, if simulations become realistic and more advanced, drivers could become good enough to prevent any crash entirely! (EXPAND THIS SECTION)
% \newpage
\bibliographystyle{plain}
\bibliography{sources}
\end{document}