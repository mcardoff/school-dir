\documentclass[12pt, letterpaper]{article}

\title{\vspace{-3em}CS 485 Participation 6}
\author{Michael Cardiff}
\date{\today}

%% science symbols 
\usepackage{amsmath}
\usepackage{amssymb}
\usepackage{physics}

%% general pretty stuff
\usepackage{bm}
\usepackage{enumitem}
\usepackage{float}
\usepackage[margin=1in]{geometry}
\usepackage{graphicx}

% figures
\graphicspath{ {./figs/} }

\newcommand{\fig}[3]
{
  \begin{figure}[H]
    \centering
    \includegraphics[width=#1cm]{#2}
    \caption{#3}
  \end{figure}
}

\newcommand{\figref}[4]
{
  \begin{figure}[H]
    \centering
    \includegraphics[width=#1cm]{#2}
    \caption{#3}
    \label{#4}
  \end{figure}
}

\renewcommand{\L}{\mathcal{L}}

\begin{document}
\maketitle
\section{Learning Something New}
I am not a CS major, but rather a physics major, so I decided to learn a bit of new physics, since you can never learn enough math. In the curriculum at IIT, there is not too much Solid State physics covered, so I decided to delve into a small intro for this. I believe then my teaching method would be a bit skewed, since it is something I am more familiar with, however I feel that my understanding of physics is a little less than my knowledge of the English language, so I think it is a fair transfer. 

There are many resources for learning solid state physics online, so I decided to just use google and find a shorter document that didn't cover too much. I found a short book from the Institute for Theoretical Physics, and decided to take a look at the first chapter. The first chapter covered the basic structure of solids in terms of electrons and protons/neutrons. These constituents are something I am already familiar with from studies in my other classes. This means my brain, instead of having to form new knowledge, can build upon old knowledge. The key point here being that I am familiar with the concepts that solid state physics are built upon, so I can build up the knowledge in the same way that the people that actually discovered the concept. This does not mean I understand it well, or as deep as someone who does this on the daily, but I can get a level of understanding nonetheless.
\section{Teaching Someone}
Even though I have friends who are also in Physics, I decided to teach someone else who is less familiar, so I could really test my method of teaching. I decided to approach my student by just giving a more surface explanation of the topic, and then try to explain a bit deeper, a slight bit of the math.

It was a bit of a challenge, as I was a bit unsure of what my student already knew, so I could not make many assumptions. I told them about electrons and nucleons (neutrons and protons) and how they make up like everything we care about in physics. I went on to explain what a solid is, and why it is important to talk about in physics. I eventually went to talk about the hamiltonian, which is an expression of the energy in the system. I wanted to explain the specific form of the solid state Hamiltonian, each individual part, and why it had to be that way specifically.
\section{Teaching a Computer?}
This would be very hard to teach to a computer, since it is more conceptual rather than systematic, like in a recipe or a word. However I could teach the USE of an equation, rather than what the equation means. In some way I could show the computer what the equation means by attaching a graphical end to it, but that would not really be teaching but instead just piping it instructions on how to show something. 
\section{Biases}
Obviously my bias immediately comes from my choice of topic, since I understand it more than other things. My other bias shows up in my choice of student, even though I chose someone who is not too familiar with physics, they still have taken classes like high school chemistry, so they know about the basic structure of matter in terms of neutrons, protons, and electrons. So my method would not be comprehensive if I were to teach someone who has no education at that level. 
\end{document}