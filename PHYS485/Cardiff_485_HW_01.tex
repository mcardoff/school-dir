\documentclass[12pt]{article}

\usepackage{physics}

\title{Colloquium Summary: Dr. Zasadzinski}
\author{Michael Cardiff}
\date{\today}

\begin{document}
\maketitle
Dr Zasadzinksi gave a very interesting talk regarding quantum computing and especially with relating cocepts from when he was a graduate student to what he is currently doing as a full research professor. I really enjoyed the connection back to his PhD work, which was surprisingly related to his current research! Dr Z's main research focuses on superconductivity, in which Niobium (Nb) is very important as it has a relatively high $T_C$ that allows for physicists to actually make use of the superconducting properties in applications such  as electronics. However, when Nb oxidizes, it produces Nb Oxide (NbO) which dampens this effect, and since air tends to oxidize metals, and electronics (or SRF cavities) tend to be in air, this becomes a problem.

This problem was solved in the 1980s by Dr Z by coating the Nb in Aluminum so that instead of NbO, AlO would be produced, something that has much better properties while still keeping the Nb intact. Dr Z mentioned in more recent research, he had encountered the same problem, where the quality factor of superconducting RF cavities would be diminished by the production of NbO. This caused Dr Z to think about his research timeline more like a circle than a straight line, which is a fascinating way to think about it at least philosphically.

Dr Z then went on to explain how the problem was tackled in the superconducting cavities. The result of simply removing the NbO was an improvement in the quality factor of a Qubit by a multiplier of $10^2$. Near the end of the talk he supposed that possibly the complete solution to the problem is to coat it in Aluminum just like in his PhD research, but the results are yet to be revealed of this.

The main application that Dr Z had talked about in the modern sense was qubits in quantum information science. The transmon qubits use a large area of Nb, so the oxidation effect is magnified here. Since the medium used is a small film of Niobium, the small bit of oxidation has an even larger effect. Even further on we can detect these using a scanning PCT and we can even find magnetic effects.

To conclude, Dr Z gave a very interesting talk on not only his current research, but the effect of his previous research on that research he is doing today. This talk was very engaging the way he used jokes throughout as a means of keeping the audience, and aided in the more jargon-heavy sections of the talk. Overall this was a very good talk and I look forward to seeing more throughout the semester. 
\begin{equation}
  
\end{equation}
\end{document}
