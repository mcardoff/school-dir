\documentclass[12pt]{article}

\title{\vspace{-3em}Colloquium Review: Muirhead}
\author{Michael Cardiff}
\date{\today}

%% science symbols 
\usepackage{amsmath}
\usepackage{amssymb}
\usepackage{physics}

%% general pretty stuff
\usepackage{bm}
\usepackage{enumitem}
\usepackage{float}
\usepackage[margin=1in]{geometry}
\usepackage{graphicx}

% figures
\graphicspath{ {./figs/} }

\newcommand{\fig}[3]
{
  \begin{figure}[H]
    \centering
    \includegraphics[width=#1cm]{#2}
    \caption{#3}
  \end{figure}
}

\newcommand{\figref}[4]
{
  \begin{figure}[H]
    \centering
    \includegraphics[width=#1cm]{#2}
    \caption{#3}
    \label{#4}
  \end{figure}
}

\renewcommand{\L}{\mathcal{L}}

\begin{document}
\maketitle
\section{Outline}
\begin{itemize}
\item General arguments
  \begin{itemize}
  \item Exoplants, smaller than earth, orbiting star that is smaller than the sun
  \item Why are these systems interesting?
  \end{itemize}
\item Appeals to knowledge of general audience, Kepler's 3 laws
\item Discussed Kepler mission, need to see as many stars as possible outside of galactic plane
\item Kepler was an important mission, discovered $2/3$ of known exoplanets
\item Primary mission of Kepler: transiting planets around sun-like stars. 
\item Method used is known as transiting planets. 
\item Mostly interested in Transiting planets around low-mass stars
\item Smaller planets have transits more often
\item Lower mass starts are not black bodies
\item Lower mass requires infrared observations to observe transits
\item Most importantly, these short period terrestrial planets are common in the universe
\item Search for biosignatures continue here 
\item Such habitable planets should orbit even lower-mass stars (brown dwarfs), motivation of PINES project. 
\end{itemize}
\pagebreak
\section{Review}
This was a very interesting colloquium, especially coming from the viewpoint of a non-astrophysics major. The speaker, Dr Muirhead talked primarily about transiting planets across low mass stars, starting by pondering why/if these planets were important. Before he went into depth however, he talked about different methods of detection, and explained why transiting is the more fruitful of these methods. Dr. Muirhead then went on to explain a couple of missions that resulted in the discovery of these exoplanets. The conclusion of the talk was an interesting discussion on the implications of these planets, why it is worth talking about them, and why we should be exploring smaller ones.

The speaker started off by showing an interesting graph discussing a cumulative number of exoplanets discovered in a year, distinguished by method of discovery. The majority of these planets were discovered by what is called transiting planets. This detection method, as the speaker describes, requires a bit of luck. First you need to know where the star is, and point your telescope there. The luck comes in when you have to take your 'picture,' you need to take it during a period in which the planet is traveling through the line of sight of the sun. Even then, how are we sure that this is in fact a transiting planet and not a fluke in the image? We need to take more pictures. The problem with taking multiple pictures during one orbit period, is that in the time it takes for the information to travel from the telescope to us, the period could be over, and if the planet has a large transit period, it could be years before we see it again. This leads straight into the speaker's argument regarding smaller planets! These planets tend to have smaller transit periods, hence more pictures. I thought it especially interesting how many of these planets were discovered from one mission, specifically $2/3$ of all of the exoplanets we know of were discovered by NASA's Kepler mission, which was also described by the speaker.

The Kepler mission is responsible for a number of exoplanet discoveries. The primary objective of Kepler was to find transiting planets which were orbiting sun-like stars. Despite this, there were plenty of bigger planets which were orbiting stars much greater than the sun, and plenty which were orbiting stars smaller than the sun. Planets were even discovered using the Hubble telescope. I was interested in how this process was different than the detection method of Kepler, it was not described in depth, but I assume it is fairly trivial. The speaker even mentioned how the follow-up to Kepler would be used to discover even more exoplanets. The speaker argued that the smallest of these planets are the most important. These smaller planets, as mentioned in the previous section, already have a smaller transit period, but there is greater importance to this as the speaker described near the end of the talk.

The most interesting part of the talk was definitely the talks of life beyond earth. The planets which were on the smaller end and orbitting sub-solar mass stars. We can search for planets in the habitable zone with these smaller mass planets/stars. Further, this motivates the need for the new project that the speaker is finding, called PINES. The project targets these smaller mass stars so that we can find not only more exoplanets, but get reliable analysis data from them. Exploration is even done on earth, which is highly convenient. The search also goes down in another direction. Smaller stars in general imply lower energy stars, so lower energy detection methods are required. In this case, we need detections in the infrared range. I was interested in why this was done on earth as opposed to a telescope in space, I figured that there would be a lot of background infrared radiation from the sun, but there must be some way for the radiation to be filtered out at the right time.

To conclude, this was a very interesting talk about astrophysics that had be asking a few questions to myself. The main takeaways that I had were that low mass stars and their planets were very important, as they are abundant and long lasting. Another is that the planets orbiting these stars very well can lie in the habitable planet zone, as well as being easily detected by us. By trends we can determine that planets can orbit stars as small as brown dwarfs and objects with masses comparable to planets. The speaker connected this to previous projects as well as present and upcoming ones, which made it relevant to the contemporary physicist as well as one which is just learning the material. This was an excellent talk. 
\end{document}
