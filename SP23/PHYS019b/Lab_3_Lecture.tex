\documentclass{beamer}

\usepackage[italicdiff]{physics}
\usepackage{hyperref}
\usepackage{url}
\usepackage{float,multicol}

\newcommand{\dbar}{d\hspace*{-0.08em}\bar{}\hspace*{0.1em}}

\newenvironment{itemframe}[1]{\begin{frame}{#1}\begin{itemize}}   {\end{itemize}\end{frame}}

% Changes style of actual slides
\usetheme{Dresden}
% Changes color of slides
\usecolortheme{spruce}
% removes controls at bottom right side
\usenavigationsymbolstemplate{}

% for figures
\graphicspath{ {./figs/} }

\title{Lab 3}
\author{Michael Cardiff}
\subtitle{Equipotential Surfaces}

\begin{document}
\begin{frame}
  \titlepage
\end{frame}

\section{What is Potential?}
\begin{frame}{Electric Potential}
  \begin{columns}
    \begin{column}{0.5\textwidth}
      \begin{itemize}
      \item The Electric Potential is to the Electric Potential Energy what the Electric Field is to the Electric Force:
        \begin{align*}
          \vb{E}\equiv\frac{\vb{F}}{q}\iff V=\frac{U}{q}
        \end{align*}
      \item Where $V$ is the electric potential, and $U$ the energy
      \item When you use a \textbf{voltmeter}, you are measuring electric potential differences
      \end{itemize}
    \end{column}
    \begin{column}{0.5\textwidth}
      \begin{figure}[H]
        \centering
        \includegraphics[width=4.0cm]{dmm}
        \caption{Digital Multimeter}
      \end{figure}
    \end{column}
  \end{columns}
\end{frame}

\section{What is an Equipotential?}
\begin{frame}{Equipotential Lines}
  \begin{columns}
    \begin{column}{0.5\textwidth}
      \begin{itemize}
      \item This lab has you finding \textbf{Equipotential Lines} for different arrangement of charges. 
      \item An Equipotential line is are like the lines on a contour map that shows altitude.
        \begin{itemize}
        \item Map: Same Altitude along line
        \item Equipotential Lines: Same Potential along line
        \end{itemize}
      \item The next slide has some examples for equipotentials
      \end{itemize}
    \end{column}
    \begin{column}{0.5\textwidth}
      \begin{figure}[H]
        \centering
        \includegraphics[width=5.0cm]{topomap}
        \caption{Example of Contour Map}
      \end{figure}
    \end{column}
  \end{columns}
\end{frame}

\begin{frame}{Examples of Equipotential Lines}
  \begin{columns}
    \begin{column}{0.5\textwidth}
      \begin{figure}[H]
        \centering
        \includegraphics[width=5.0cm]{eq1}
        \caption{Equiptential for a Capacitor}
      \end{figure}
    \end{column}
    \begin{column}{0.5\textwidth}
      \begin{figure}[H]
        \centering
        \includegraphics[width=5.0cm]{eq2}
        \caption{Equiptentials for a Charge}
      \end{figure}
    \end{column}
  \end{columns}
  \begin{itemize}
  \item Note that the equipotentials are perpendicular to the field lines
  \end{itemize}
\end{frame}
\begin{frame}{How Do We Find Them?}
  \begin{itemize}
  \item Simply determined by the formula for the potential of the geometry you are working with. 
  \item For example, for the Parallel Plate Capacitor and Point Charge
    \begin{align*}
      V_{cap}(x)=\frac{\sigma}{2\varepsilon_0}x\qquad
      V_{chg}(r)=\frac{1}{4\pi\varepsilon_0}\frac{q}{r}
    \end{align*}
  \item Since $V_{cap}$ only depends on $x$, the equipotential lines should be perpendicular to $x$
  \item Similarly $V_{chg}$ depends on $r$, so equipotential lines should be circles with radius $r$
  \item This is simple when we know the form of the field
  \end{itemize}
\end{frame}

\section{Experiment}
\begin{frame}{Details of Experiment}
  \begin{columns}
    \begin{column}{0.5\textwidth}
      \begin{itemize}
      \item Areas with no sound correspond to equipotential lines
      \item Connecting a banana jack to the top part of the board picks out an equipotential to measure
      \item Lab Manual has more details/Questions you need to answer
      \end{itemize}
    \end{column}
    \begin{column}{0.5\textwidth}
      \begin{figure}[H]
        \centering
        \includegraphics[width=5.0cm]{mapping_device}
        \caption{Mapping Device}
      \end{figure}
    \end{column}
  \end{columns}
\end{frame}

\begin{frame}{Setup}
  \begin{columns}
    \begin{column}{0.5\textwidth}
      \begin{figure}[H]
        \centering
        \includegraphics[width=5.0cm]{upsidedown}
        \caption{Upside Down Setup}
      \end{figure}
    \end{column}
    \begin{column}{0.5\textwidth}
      \begin{figure}[H]
        \centering
        \includegraphics[width=5.0cm]{rightsideup}
        \caption{Rightside Up Setup}
      \end{figure}
    \end{column}
  \end{columns}
  \begin{itemize}
  \item In the righthand side figure, we will have a signal generator instead of a voltage supply, and the headphones instead of a voltmeter
  \end{itemize}
\end{frame}

\begin{frame}{Key Notes of Experiment}
  \begin{itemize}
  \item Technically, the waveform you use (on signal generator) does not matter, but the other two included produce odd sounds, so play at your own risk
  \item There are 7 places to put your banana jacks, but the 2 on the edge may be hard to find, so try to find the inner 5
  \end{itemize}
\end{frame}

\section{Reminders}
\begin{frame}{Reminders}
  \begin{itemize}
  \item This lab is due Friday, March 10
  \item This is a short report, there are questions littered throughout the lab manual
  \item Labs 1 and 2 Will be graded soon.
  \end{itemize}
\end{frame}

\end{document}