\documentclass{beamer}

\title{Lab 6: Faraday's Law}
\author{Michael Cardiff}
\subtitle{PHYS 19b}

\usepackage[italicdiff]{physics}
\usepackage{hyperref}
\usepackage{url}
\usepackage{multicol}

% Changes style of actual slides
\usetheme{Dresden}
% Changes color of slides
\usecolortheme{spruce}
% removes controls at bottom right side
\usenavigationsymbolstemplate{}

% for figures
\graphicspath{ {./figs/} }

\begin{document}

\begin{frame}
  \titlepage
\end{frame}

\begin{frame}{Faraday's Law}
  \begin{itemize}
  \item Faraday's Law of Induction is given by:
    \begin{align*}
      \mathcal{E}=-\frac{\Delta \Phi_B}{\Delta t}=-\dv{\Phi_B}{t}
    \end{align*}
  \item In words: a change in magnetic flux induces an EMF\@.
  \item Magnetic Flux is given by:
    \begin{align*}
      \Phi_B=\int\vb{B}\vdot\dd{\vb{a}}=\int\vb{B}\vdot\dv{\vb{a}}{t}\dd{t}
    \end{align*}
  \end{itemize}
\end{frame}

\begin{frame}{What Does this Mean?}
  \begin{itemize}
  \item Many ways to change the magnetic flux:
    \begin{itemize}
    \item Change the magnet you're using (Changing $\vb{B}$)
    \item Change the area the magnet is going through (Change $\vb{a}$)
    \item Change the way the magnet is moving through $\vb{a}$ (Change $\dd{\vb{a}}$)
    \end{itemize}
  \item If I generate magnetic flux through some area which is conducting, current will be generated!
  \item The minus sign in Faraday's law tells you the direction the current will be in! 
  \end{itemize}
\end{frame}

\begin{frame}{Basics of Magnets}
  \begin{columns}
    \begin{column}{0.5\textwidth}
      \begin{itemize}
      \item Two types of magnets we will encounter: electromagnets and  permanent magnets
      \item Permanent magnets are always magnets
      \item Electromagnets are only magnets when a current is supplied
      \end{itemize}
    \end{column}
    \begin{column}{0.5\textwidth}
      \begin{figure}[H]
        \centering
        \includegraphics[width=4.0cm]{magnet}
        \includegraphics[width=4.0cm]{solenoid}
        \caption{Permanent vs Electromagnet}
      \end{figure}
    \end{column}
  \end{columns}
\end{frame}

\begin{frame}{Continued}
  \begin{itemize}
  \item Electromagnets Work the other way too!
  \item Put a magnetic field in a wire loop
  \item There will be an induced current. 
  \end{itemize}
\end{frame}

\begin{frame}{What the Manual Left out}
  \begin{itemize}
  \item The type of electromagnet we are using is many loops of wire (a solenoid)
  \item Think of one loop of wire as the smallest unit of magnetic field
  \item More loops = stronger magnetic field
  \item This way, we can say Faraday's Law is:
    \begin{align*}
      \mathcal{E}=-N\dv{\Phi^1_{B}}{t}
    \end{align*}
  \item Where $\Phi_B^1$ is the magnetic flux through a single loop:
    \begin{align*}
      \Phi_B^1=B*A_{circle}
    \end{align*}
  \end{itemize}
\end{frame}

\begin{frame}{Lab Reminders}
  \begin{itemize}
  \item Part 1 is Qualitative
  \item Part 2 is kinda messed up
  \item Only 5 working setups for part 2
  \item You may have to get into bigger groups
  \end{itemize}
\end{frame}

\begin{frame}{Misc. Reminders}
  \begin{itemize}
  \item This is the last lab
  \item Long Report
  \item Hints throughout manual for how to write report
  \item Due Date of Lab?
    \begin{itemize}
    \item Friday April 28 (normal date)
    \item Friday May 5 (Friday before finals week)
    \end{itemize}
  \end{itemize}
\end{frame}

\end{document}