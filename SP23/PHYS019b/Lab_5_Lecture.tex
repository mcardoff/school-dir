\documentclass{beamer}

\usepackage[italicdiff]{physics}
\usepackage{hyperref}
\usepackage{url}

\newcommand{\dbar}{d\hspace*{-0.08em}\bar{}\hspace*{0.1em}}

\newenvironment{itemframe}[1]{\begin{frame}{#1}\begin{itemize}}   {\end{itemize}\end{frame}}

% Changes style of actual slides
\usetheme{Dresden}
% Changes color of slides
\usecolortheme{spruce}
% removes controls at bottom right side
\usenavigationsymbolstemplate{}

% for figures
\graphicspath{ {./figs/} }

\title{Lab 5, Capacitors and RC Circuits}
\author{Michael Cardiff}
\subtitle{PHYS 19b}

\begin{document}

\begin{frame}
  \titlepage{}
\end{frame}

\section{Reminders}
\begin{frame}{Important Notes!}
  \begin{itemize}
  \item For Experiment 4 we will be using different capacitors, get one from me at the front
  \item In case you didn't get questions 2 and 3 on the Quiz:
    \begin{align*}
     C_{parallel}=C_1+C_2\\
     \frac1{C_{series}}=\frac1{C_1}+\frac1{C_2}
    \end{align*}
  \item The switches for the circuit are a bit confusing:
    \begin{itemize}
    \item The switch is open when it is touching neither side
    \item When the switch is closed on one side, the same color leads on either side are connected to one another (figure 2)
    \end{itemize}
  \end{itemize}
\end{frame}

\begin{frame}{Other Reminders}
  \begin{itemize}
  \item Short Lab Due Next Friday, April 7th (If you have problems with this email me)
  \item Notes on Capacitors? Specifically Deriving (1) in manual?
  \end{itemize}
\end{frame}

\section{Derivation}
\begin{frame}{Equation in the Manual}
  \begin{itemize}
  \item An important equation in the lab manual is the decay of the RC Circuit in Fig. 1:
    \begin{align*}
      V(t)=V_0 e^{-t/\tau}
    \end{align*}
  \item It is important to look at what happens first when we have S1 closed and S2 Open, lets examine that first
  \end{itemize}
\end{frame}
\begin{frame}{What does S1 do?}
  \begin{itemize}
  \item In this case, a battery is simply acting as a current source to the capacitor, which stores voltage equal to:
    \begin{align*}
      V=\frac{q}{C}
    \end{align*}
  \item A charge can be thought of in terms of a current times a time
  \item So if we supply a constant current to the capacitor, it will charge up to the voltage of the battery
  \item So once everything settles, $V_{cap}$ when S1 is closed is the same as $V_{battery}$
  \item We can now open S1 and examine the circuit with S2 Closed
  \end{itemize}
\end{frame}

\begin{frame}{Closing S2}
  \begin{itemize}
  \item Initially, the capacitor simply acts as a voltage source, but it does NOT keep its voltage as time goes on, it discharges.
  \item However, the loop rule tells us that the voltage in the closed loop should be 0, so:
    \begin{align*}
      V_{cap}+V_{res}=0
    \end{align*}
  \item We know a relationship between charge and capacitance:
    \begin{align*}
      V_{cap}=\frac{q}{C}
    \end{align*}
  \item And a relationship between current and resistance:
    \begin{align*}
      V_{res}=iR=R\dv{q}{t}
    \end{align*}
  \end{itemize}
\end{frame}
\begin{frame}{Continued}
  \begin{itemize}
  \item Put these in the original equation to get an equation for $q(t)$:
    \begin{align*}
      \frac{1}{C}q(t)+R\dv{q(t)}{t}=0
    \end{align*}
  \item ODEs are hard, so I'm just going to guess the solution:
    \begin{align*}
      q(t)=Ae^{Bt}
    \end{align*}
  \item Since $e^t$ is a nice function in physics, we get:
    \begin{align*}
      \frac{A}{C}e^{Bt}+RABe^{Bt}=0
    \end{align*}
  \item Some Algebra can give you one of the constants:
    \begin{align*}
      B=-\frac{1}{RC}
    \end{align*}
  \item Hey, thats the time constant!
  \end{itemize}
\end{frame}
\begin{frame}{The Charge}
  \begin{itemize}
  \item The Next step is applying the initial condition, which gives us $A$ from the capacitor relationship:
    \begin{align*}
      q(0)=CV_0\implies q(t)=CV_0e^{-t/RC}
    \end{align*}
  \item However notice that the same capacitor relationship gives us the voltage law we want:
    \begin{align*}
      V(t)=\frac{q(t)}{C}=V_0e^{-t/RC}
    \end{align*}
  \item Which is exactly eq. (2) in the manual:
    \begin{align*}
      \boxed{V(t)=V_0e^{-t/RC}}
    \end{align*}
  \end{itemize}
\end{frame}

\end{document}