\documentclass{beamer}

\usepackage[italicdiff]{physics}
\usepackage{amsmath,amsthm}
\usepackage{hyperref,url}
\usepackage{multicol}

\newcommand{\dbar}{d\hspace*{-0.08em}\bar{}\hspace*{0.1em}}

\theoremstyle{remark}
\newtheorem{law}{Law}

\newenvironment{itemframe}[1]{\begin{frame}{#1}\begin{itemize}}   {\end{itemize}\end{frame}}

% Changes style of actual slides
\usetheme{Dresden}
% Changes color of slides
\usecolortheme{spruce}
% removes controls at bottom right side
\usenavigationsymbolstemplate{}

% for figures
\graphicspath{ {./figs/} }

\title{Lab 4}
\author{Michael Cardiff}
\subtitle{DC Circuits}

\begin{document}
\begin{frame}
  \titlepage
\end{frame}

\section{Ohm's Law}
\begin{frame}{Ohm's Law}
  \begin{itemize}
  \item Central to any circuit with resistors, it relates the current and resistance to the voltage drop:
    \begin{align*}
      V=IR
    \end{align*}
  \item Using Ohm's Law, and Kirchhoff's Laws, we can solve any circuit, no matter how complex
  \end{itemize}
\end{frame}

\section{Kirchhoff's Laws}
\begin{frame}{Kirchhoff's Loop Law}
  \begin{columns}
    \begin{column}{0.5\textwidth}
      \begin{law}[Kirchhoff's Loop Law]
        The sum of voltages around a loop is 0:
        \begin{align*}
          \sum V=0
        \end{align*}
        Due to this, the current around a loop is equal
      \end{law}
      \begin{itemize}
      \item Used at every single loop in conjunction with the next law
      \end{itemize}
    \end{column}
    \begin{column}{0.5\textwidth}
      \begin{figure}[H]
        \centering
        \includegraphics[width=5.0cm]{loop}
        \caption{Circuit with a loop}
      \end{figure}
    \end{column}
  \end{columns}
\end{frame}

\begin{frame}{Kirchhoff's Junction Law}
  \begin{columns}
    \begin{column}{0.6\textwidth}
      \begin{law}[Kirchhoff's Junction Law]
        The sum of currents at a junction is 0:
        \begin{align*}
          \sum I=0
        \end{align*}
      \end{law}
      \begin{itemize}
      \item Used to solve parallel circuits with multiple loops
      \item In the figure, we know: $I_1+I_2+I_3=0$
      \end{itemize}
    \end{column}
    \begin{column}{0.4\textwidth}
      \begin{figure}[H]
        \centering
        \includegraphics[width=4.0cm]{junction}
        \caption{Circuit with a junction}
      \end{figure}
    \end{column}
  \end{columns}
\end{frame}

\begin{frame}{What do they mean?}
  \begin{itemize}
  \item Loop Law: The circuit is a closed system, so summing voltages is like summing energies
  \item Just energy conservation!
  \item Junction Law: If they did not sum to $0$, then one end has more charge than the other
  \item So the junction law is just conservation of charge!
  \item All Kirchhoff did was apply these general principles to circuit systems
  \end{itemize}
\end{frame}

\section{Measuring Current/Voltage}
\begin{frame}{How does an Ammeter work?}
  \begin{columns}
    \begin{column}{0.5\textwidth}
      \begin{itemize}
      \item Because of the loop law, components connected in series see the same current
      \item Essentially, an ammeter must be placed \textbf{within the circuit} to see the current
      \item Ideally, the ammeter does not affect the rest of the circuit, so it has $0$ resistance
      \end{itemize}
    \end{column}
    \begin{column}{0.5\textwidth}
      \begin{figure}[H]
        \centering
        \includegraphics[width=5.0cm]{ammeter}
        \caption{Ammeter in a circuit}
      \end{figure}
    \end{column}
  \end{columns}
\end{frame}

\begin{frame}{How does a Voltmeter work?}
  \begin{columns}
    \begin{column}{0.6\textwidth}
      \begin{itemize}
      \item Voltmeters work because they have very high resistances
      \item Due to this, if we placed it in series, not voltage would be measured
      \item Instead, the loop and junction law allows us to put it in parallel
      \item Essentially, the voltmeter must \textbf{externally probe the circuit} to measure a voltage
      \end{itemize}
    \end{column}
    \begin{column}{0.4\textwidth}
      \begin{figure}[H]
        \centering
        \includegraphics[width=4.0cm]{voltmeter}
        \caption{Voltmeter in a circuit}
      \end{figure}
    \end{column}
  \end{columns}
\end{frame}

\section{Wheatstone Bridge}
\begin{frame}{Example Circuit}
  \begin{columns}
    \begin{column}{0.6\textwidth}
      \begin{itemize}
      \item The Lab Manual tells you to solve the Wheatstone Bridge circuit to find the following:
        \begin{align*}
          \frac{R_2}{R_1}=\frac{R_x}{R_3}
        \end{align*}
      \item The circuit can be solved using everything we have gone over so far!
      \end{itemize}
    \end{column}
    \begin{column}{0.4\textwidth}
      \begin{figure}[H]
        \centering
        \includegraphics[width=4.0cm]{wheat}
        \caption{Wheatstone Bridge Circuit}
      \end{figure}
    \end{column}
  \end{columns}
\end{frame}

\begin{frame}{Junctions}
  \begin{columns}
    \begin{column}{0.6\textwidth}
      \begin{itemize}
      \item Consider Junctions $B$ and $D$, use the junction law:
        \begin{align*}
          \text{B:}\quad I_3+I_g+I_x=0\\
          \text{D:}\quad I_1+I_g+I_2=0
        \end{align*}
      \item Lab manual: Bridge is 'balanced' when $I_g=0$, so we can use these equations:
        \begin{align*}
          I_3=-I_x\\
          I_1=-I_2
        \end{align*}
      \end{itemize}
    \end{column}
    \begin{column}{0.4\textwidth}
      \begin{figure}[H]
        \centering
        \includegraphics[width=4.0cm]{wheat}
        \caption{Wheatstone Bridge Circuit}
      \end{figure}
    \end{column}
  \end{columns}
\end{frame}

\begin{frame}{Loops}
  \begin{columns}
    \begin{column}{0.6\textwidth}
      \begin{itemize}
      \item Now consider the voltages in loops ABD and DCB:
        \begin{align*}
          \text{ABD:}\quad 0=I_1R_1+I_3R_3+I_gR_g\\
          \text{DCB:}\quad 0=I_2R_2+I_xR_x+I_gR_g
        \end{align*}
      \item Take $I_g=0$, and do some algebra:
        \begin{align*}
          \frac{I_3R_3}{I_1R_1}=-1=\frac{I_xR_x}{I_2R_2}
        \end{align*}
      \end{itemize}
    \end{column}
    \begin{column}{0.4\textwidth}
      \begin{figure}[H]
        \centering
        \includegraphics[width=4.0cm]{wheat}
        \caption{Wheatstone Bridge Circuit}
      \end{figure}
    \end{column}
  \end{columns}
\end{frame}

\begin{frame}{Finishing up}
  \begin{itemize}
  \item Plug the equations from the junction laws:
    \begin{align*}
      \frac{I_xR_3}{I_2R_1}&=\frac{I_xR_x}{I_2R_2}\\
      \frac{I_x}{I_2}\frac{R_3}{R_1}&=\frac{I_x}{I_2}\frac{R_x}{R_2}
    \end{align*}
  \item Thus for a balanced bridge, we have what is in the manual:
    \begin{align*}
      \boxed{\frac{R_3}{R_1}=\frac{R_x}{R_2}\implies
      \frac{R_2}{R_1}=\frac{R_x}{R_3}}
    \end{align*}
  \end{itemize}
\end{frame}

\section{The Lab}
\begin{frame}
  \frametitle{The Lab}
  \begin{itemize}
  \item 3 Experiments Total
    \begin{itemize}
    \item Measuring internal resistance of a battery
    \item Measuring the EMF of a Lemon (How strong is a lemon battery?)
    \item Simulating a Wheatstone Bridge circuit
    \end{itemize}
  \item All are fairly simple
  \item Short Lab Report!
  \item Due Next Week, March 24
  \end{itemize}
\end{frame}

\end{document}