\documentclass{beamer}

\usepackage[italicdiff]{physics}
\usepackage{hyperref}
\usepackage{url}

\newcommand{\dbar}{d\hspace*{-0.08em}\bar{}\hspace*{0.1em}}

\newenvironment{itemframe}[1]{\begin{frame}{#1}\begin{itemize}}   {\end{itemize}\end{frame}}

% Changes style of actual slides
\usetheme{Dresden}
% Changes color of slides
\usecolortheme{spruce}
% removes controls at bottom right side
\usenavigationsymbolstemplate{}

% for figures
\graphicspath{ {./figs/} }

\title{Error Analysis Review}
\author{Michael Cardiff}
\subtitle{PHYS 19b}

\begin{document}

\begin{frame}
  \titlepage
\end{frame}

\begin{frame}{What is This?}
  \begin{itemize}
  \item I am going to hopefully be periodically updating this set of slides with review material for the error analysis needed in your labs.
  \item This is meant to be a quick reference so you can refresh yourself on how to calculate certain error statistics
  \end{itemize}
\end{frame}

\section{Error Propagation}
\begin{frame}{Error Propagation}
  \begin{itemize}
  \item We have certain measurements $x,y,z$ with associated errors $\sigma_x,\sigma_y,\sigma_z$
  \item We want to calcuate $a$, which is related to $x,y,z$ by some function $f$, we could say that:
    \begin{align*}
      a = f(x,y,z)
    \end{align*}
  \item Since $a$ is found through measurements with error, it should have its own error.
  \end{itemize}
\end{frame}

\begin{frame}{Continued}
  \begin{itemize}
  \item We calculate the error on a by \textbf{propagating} the error from $x,y,z$
  \item In order to calculate it, we use the following formula:
    \begin{align*}
      \sigma_a^2=\qty(\pdv{f}{x})^2\sigma_x^2
      +\qty(\pdv{f}{y})^2\sigma_y^2
      +\qty(\pdv{f}{z})^2\sigma_z^2
    \end{align*}
  \item So if a calculated variable depends on some number of measured variables, you will have a partial derivative for each measurement the calculation depends on, multiplied by the square of the error for that measurement.
  \end{itemize}
\end{frame}

\begin{frame}{Interpreting the Formula}
  \begin{itemize}
  \item The formula we use has the basic unit of:
    \begin{align*}
      \qty(\pdv{f}{x})^2\sigma_x^2
    \end{align*}
  \item We can intepret this using the basic definition of derivatives:
  \item $\pdv{f}{x}$ is the change in $f$  when we make a small change to $x$
  \item $\sigma_x$ is how uncertain we are in that change of $x$
  \item This means the basic unit above determines how uncertain we are in $f$ when we specifically change $x$.
  \item So the reason we need to square everything is because we are finding a distance in errors, we want the end result to be an error after all
  \end{itemize}
\end{frame}

\section{How good is your answer?}
\begin{frame}{Why Should we care?}
  \begin{itemize}
  \item The whole purpose of our experiments is to see if the mathematical framework we use to describe real events is accurate
  \item Essentially, when we describe how 'good' a fit is, we want to compare the error between the mathematical answer in the fit, to our measured data
  \item There are a number of ways to determine goodness of a fit
  \end{itemize}
\end{frame}

\begin{frame}{Goodness of Fit}
  \begin{itemize}
  \item The $\chi^2$ statistic is one way to determine if your fit is 'good'
  \item Another way we can say if a fit is good is the $R$ or $R^2$ value
    \begin{itemize}
    \item Excel can instantly give you the $R^2$ value
    \end{itemize}
  \item These both are ways of determining how good your fit is
  \item Let's Discuss each of these
  \end{itemize}
\end{frame}

\section{$R^2$ Statistic}
\begin{frame}{The $R^2$ Statistic}
  \begin{itemize}
  \item This statistic refers only to a fit
  \item It tells you how well, on a scale of 0 to 1, your data fits the curve
  \item Classical example: figure below:
    \begin{figure}[H]
      \centering
      \includegraphics[width=7.0cm]{rval}
      \caption{Different Values of $R^2$}
    \end{figure}
  \item We can put any number of lines through the middle one, but only one line captures all the points in the other two
  \end{itemize}
\end{frame}

\begin{frame}{How to Calculate}
  \begin{itemize}
  \item Given by the Formula:
    \begin{align*}
      R^2=1-\frac{\sum_i\qty(y^{data}_i-y^{fit}_i)^2}
      {\sum_i\qty(y^{data}_i-\bar{y})^2}
    \end{align*}
  \item Where $y^{data}$ is your data, $y^{fit}$ is your fitted data, and $\bar{y}$ is the mean of your data.
  \item Numerator is called the Sum Squared of the residuals
  \item Denominator is the total of the sum of squares, and is related to the variance of the data.
  \item Since the variance is related to the error on the mean, we are trying to see how well the fit residuals mimic the error on the mean.
  \end{itemize}
\end{frame}

\begin{frame}{Pros and Cons}
  \begin{columns}
    \begin{column}{0.5\textwidth}
      \textbf{Pros}
      \begin{itemize}
      \item Gives an idea if your guess for the fit was right
      \item Quantifies how big the fit residuals are
      \end{itemize}
    \end{column}
    \begin{column}{0.5\textwidth}
      \textbf{Cons}
      \begin{itemize}
      \item Does not take into account error of measurements
      \item Only a relationship between chosen fit to data
      \item Multiple different fits can give a good value of $R^2$
      \end{itemize}
    \end{column}
  \end{columns}
\end{frame}

\section{$\chi^2$ Statistic}
\begin{frame}{The $\chi^2$ statistic}
  \begin{itemize}
  \item This refers to 3 pieces of data:
    \begin{itemize}
    \item Your Data
    \item Your Fit
    \item Your Errors
    \end{itemize}
  \item The Formula to find it is:
    \begin{align*}
      \chi^2=\sum_{i=1}^N\frac{(y_i^{data}-y_i^{fit})^2}{\sigma_i^2}
    \end{align*}
  \item $\sigma_i$ is the error on the $i^{th}$ measurement, $y^{data}$ are your measurement values, and $y^{curve}$ are your fit evaluated at the same data points that your $y^{data}$ were measured at. 
  \end{itemize}
\end{frame}

\begin{frame}{Continued}
  \begin{itemize}
  \item The numerator and denominator are both errors
    \begin{align*}
      \chi^2=\sum_{i=1}^N\frac{(y_i^{data}-y_i^{fit})^2}{\sigma_i^2}
    \end{align*}
  \item The numerator is how far off your data is from the supposed mathematical answer (your fit)
  \item The denominator describes the error you said it should be based on either error propagation or error on your measurement devices.
  \item How to Interpret the value of $\chi^2$:
    \begin{itemize}
    \item $\chi^2<1$ You overestimated the error
    \item $\chi^2\approx1$ You estimated the error correctly
    \item $\chi^2>1$ You underestimated the error 
    \end{itemize}
  \end{itemize}
\end{frame}
\begin{frame}{Pros and Cons}
  \begin{columns}
    \begin{column}{0.5\textwidth}
      \textbf{Pros}
      \begin{itemize}
      \item Includes all relevant measurements in a science experiment
      \item Tells you if your choice of error is correct
      \item Fairly easy to calculate
      \end{itemize}
    \end{column}
    \begin{column}{0.5\textwidth}
      \textbf{Cons}
      \begin{itemize}
      \item Overall value of statistic changes based on number of points (this is why we have $\chi^2/N_{dof}$)
      \item Not well defined when there are small amounts of degrees of freedom
      \end{itemize}
    \end{column}
  \end{columns}
\end{frame}

\begin{frame}{Which to Choose}
  \begin{itemize}
  \item If you just want to comment on how good your fit is, use $R^2$
  \item If you want to discuss experimental results, i.e. errors, use $\chi^2$
  \item Most of the time in lab we have errors, so most of the time we are looking to calculate $\chi^2$
  \end{itemize}
\end{frame}

\section{Dangers}
\begin{frame}{Why Does it Matter?}
  \begin{columns}
    \begin{column}{0.5\textwidth}
      \begin{itemize}
      \item To the right are 4 data sets with different Data, but
        \begin{itemize}
        \item The same fit params
        \item The same $R^2$ value
        \end{itemize}
      \item So Completely different data gives the same overall results
      \end{itemize}
    \end{column}
    \begin{column}{0.5\textwidth}
      \begin{figure}[H]
        \centering
        \includegraphics[width=5cm]{anscombe}
        \caption{Anscombe Data Sets}
      \end{figure}
    \end{column}
  \end{columns}
\end{frame}

\end{document}