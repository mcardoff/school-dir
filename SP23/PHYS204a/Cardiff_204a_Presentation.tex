\documentclass{beamer}

\title{The Martin-Siggia-Rose Formalism}
\author{Michael Cardiff}
\subtitle{PHYS 204a}

\usepackage[italicdiff]{physics}
\usepackage{hyperref}
\usepackage{url}

\def\D{\partial}

% Changes style of actual slides
\usetheme{Dresden}
% Changes color of slides
\usecolortheme{dolphin}
% removes controls at bottom right side
\usenavigationsymbolstemplate{}
\setbeamertemplate{footline}[frame number]

% for figures
\graphicspath{ {./figs/} }

\begin{document}

\begin{frame}
  \titlepage
\end{frame}

\section{Problem Overview}
\begin{frame}{A Problem}
  \begin{columns}
    \begin{column}{0.5\textwidth}
      \begin{itemize}
      \item Many non-equilibrium processes involve random variables.
      \item Examples include Brownian Motion, Turbulent flow, and even Financial Markets. 
      \item These all lead to stochastic differential equations (SDEs)
      \item Surprise surprise, SDEs are hard to solve.
      \end{itemize}
    \end{column}
    \begin{column}{0.5\textwidth}
      \begin{figure}[H]
        \centering
        \includegraphics[width=\textwidth]{brownian}
        \includegraphics[width=0.75\textwidth]{stock}
        \caption{Brownian Motion and the Stock Market}
      \end{figure}
    \end{column}
  \end{columns}
\end{frame}

\begin{frame}{How to Tackle This?}
  \begin{itemize}
  \item There are many forms of SDE, but can be boiled down to the form:
    \begin{align*}
      \D_t x(t)=F(x(t))+g(x(t))\xi(t)
    \end{align*}
  \item Where $F$ governs the deterministic time evolution, $g$ the coupling to noise, and $\xi$ is (usually) Gaussian noise.
  \item From this SDE, we can write an equivalent Fokker-Planck Equation
  \item The FP equation gives a probability distribution $p(x,t)$ for the random variable $x(t)$
  \end{itemize}
\end{frame}

\begin{frame}{Is there another way?}
  \begin{itemize}
  \item While the Fokker-Planck equation is fun, it cannot be applied to every situation.
  \item One option is to move to a path integral formalism, which is essentially an equivalent problem, but is computationally different.
  \item Martin, Siggia, and Rose (MSR) did this, and found that observables $\mathcal{O}$ averaged over solutions of any SDE of the form:
    \begin{align*}
      \D_t x(t)=F(x(t),t)+\xi(x(t),t)
    \end{align*}
  \item Can be written as a path integral with some action $S[x,\tilde{x}]$:
    \begin{align*}
      \ev{\mathcal{O}[x(t)]}=\int\mathcal{D}[x,\tilde{x}]\mathcal{O}[x(t)]
      e^{-S[x,\tilde{x}]}
    \end{align*}
  \end{itemize}
\end{frame}

\section{The Framework}
\begin{frame}{``Deriving'' The Path Integral}
  \begin{itemize}
  \item The Path integral from before can be ``derived'' by simply enforcing that all paths must obey the SDE
  \item With no constraints of an SDE, the average of an observable is simply:
    \begin{align*}
      \ev{\mathcal{O}[x(t)]}=\ev{\int\mathcal{D}[x]\mathcal{O}[x(t)]}
    \end{align*}
  \item However, we must enforce our SDE at all points in space, so we use a $\delta$ functional:
    \begin{align*}
      \ev{\mathcal{O}[x(t)]}=\ev{\int\mathcal{D}[x]\mathcal{O}[x(t)]
        \delta[\text{SDE}]}
    \end{align*}
  \end{itemize}
\end{frame}

\begin{frame}{Continued}
  \begin{itemize}
  \item The next step is introducing a form of the $\delta$ functional.
  \item The $\delta$ functional is just a product of $\delta$ functions, use the integral definition of $\delta(x)$:
    \begin{align*}
      \delta(x)\propto\int\dd{k}e^{ikx}
    \end{align*}
  \item The functional version is then found by turning $k$ into another field, usually called $\tilde{x}$, and it is simply integrated out:
    \begin{align*}
      \ev{\mathcal{O}[x(t)]}&=\ev{\int\mathcal{D}[x,\tilde{x}]\mathcal{O}[x(t)]
        \exp{-\int\dd{t}i\tilde{x}(t)\qty[\text{SDE}]}}\\
      [\text{SDE}]&=\D_t x(t)-F(x(t),t)-\xi(x(t),t)
    \end{align*}
  \end{itemize}
\end{frame}

\begin{frame}{The MSR Action}
  \begin{itemize}
  \item From here, properties of the distribution $\xi$ can be invoked, to arrive at the MSR action for this general equation:
    \begin{align*}
      S[x,\tilde{x}]=&\int\dd{t}i\tilde{x}(t)\qty[\D_t x(t)-F(x(t),t)]\\
      &+\frac12\int\dd{t}\dd{t'}\tilde{x}(t)\ev{\xi(x,t)\xi(x',t')}x(t)
    \end{align*}
  \item Finding this action and the corresponding Path Integral allows for Field Theory methods to be employed.
  \item This form of the action can be solved exactly in many cases
  \end{itemize}
\end{frame}

\section{Examples}
\begin{frame}{Multiplicative Noise}
  \begin{itemize}
  \item We encountered this when discussing the Langevin Equation, we had noise of the form:
    \begin{align*}
      \xi(x(t),t)=H(x(t))\xi(t)
    \end{align*}
  \item This is a common form, covering:
    \begin{itemize}
    \item Chemical Reactions
    \item The Stock Market
    \item Turbulence
    \item Neuron Dynamics
    \end{itemize}
  \item We even had an example of this in class
  \end{itemize}
\end{frame}

\begin{frame}{Continued}
  \begin{itemize}
  \item Where we had that $\xi(t)$ is zero mean and $\ev{\xi(t)\xi(t')}=(2\sigma)\delta(t-t')$. 
  \item This reduces the action to a path integral which is local in time:
    \begin{align*}
      S[x,\tilde{x}]=\int\dd{t}i\tilde{x}(t)\qty[\D_t x(t)-F(x(t))+\frac\sigma2H(x(t))^2\tilde{x}]
    \end{align*}
  \item We can then introduce a form for $H(x(t))$ to analyze some properties.
  \end{itemize}
\end{frame}

\begin{frame}{Ornstein–Uhlenbeck Processes}
  \begin{itemize}
  \item These processes have $H(x)=1$, and a fixed form of $F$:
    \begin{align*}
      \D_t x(t)=-\alpha x(t)+\xi(t)
    \end{align*}
  \item Applications
    \begin{itemize}
    \item Brownian Motion Subject to Friction
    \item Interest Models in finance.
    \item Exchange Rates
    \item Evolutionary Biology
    \end{itemize}
  \item This form of $H(x)$ can be analytically solved
  \end{itemize}
\end{frame}
\begin{frame}{Solved Path Integral}
  \begin{itemize}
  \item The generating functional can be found for noise with mean $0$ and variance $\sigma$:
    \begin{align*}
      \ev{e^{\int\dd{t}\lambda(t)x(t)}}=
      \exp[\frac\sigma{2\alpha}\int\dd{s_1}\dd{s_2}e^{-\alpha\abs{s_1-s_2}}
      \lambda(s_1)\lambda(s_2)]
    \end{align*}
  \item Where $\lambda(t)$ is acting like a ``source'' as in Field Theory
  \item Just like in Field Theory, derivatives can be taken to find moments or cumulants
  \item Specifically, if we find the connected Correlator by taking:
    \begin{align*}
      \ev{x(t)x(t')}_c=
      \eval{\frac{\delta^2}{\delta\lambda(t)\delta\lambda(t')}
      \log\ev{e^{\int\dd{t}\lambda(t)x(t)}}}_{\lambda=0}
    \end{align*}
  \item Evaluated, we immediately find the result from operator dynamics:
    \begin{align*}
      \ev{x(t)x(t')}_c=\frac{\sigma}{2\alpha}e^{-\alpha\abs{s_1-s_2}}
    \end{align*}
  \end{itemize}
\end{frame}

\section{Conclusions}
\begin{frame}{Concluding Remarks}
  \begin{itemize}
  \item The MSR Formalism is a powerful tool in studying stochastic dynamics
  \item Despite working with these stochastic variables, we can find exact expressions for moments and cumulants
  \item MSR is applicable to a wide variety of systems, relevant to Physics, Economics, and Biology
  \item Despite this, it is a bit complex to compute the path integral in general
  \item However, if you can compute it, the wide world of QFT methods can be applied
  \end{itemize}
\end{frame}

\begin{frame}
  \begin{center}
    \Huge{Thank You!}
  \end{center}
\end{frame}

\end{document}