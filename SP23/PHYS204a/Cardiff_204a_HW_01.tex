\documentclass[12pt]{article}

\title{\vspace{-3em}PHYS 204 HW 1}
\author{Michael Cardiff}
\date{\today}

%% science symbols 
\usepackage{amssymb,amsthm,bm,physics,slashed}

%% general pretty stuff
\usepackage{caption,enumitem,float,geometry,graphicx,tikz}

% setups
\graphicspath{ {./figs/} }
\captionsetup{labelfont=bf}
\geometry{margin=1in}

% macros
\renewcommand{\L}{\mathcal{L}}
\newcommand{\D}{\partial}
\newcommand{\circled}[1]{\tikz[baseline=(char.base)]{
    \node[shape=circle,draw,inner sep=2pt](char){#1};}}

\begin{document}
\maketitle

\section{Spin Echo}
The Hamiltonian we are using is given by:
\begin{align*}
  H=-\sum_i\qty(B+\Delta B_i)\sigma_i^z
\end{align*}
And the operator $X$, the site averaged $x$ component spin is:
\begin{align*}
  X=\frac1n\sum_i\sigma_i^x
\end{align*}
\subsection{Quantum average of $X$}
The initial stae of the system is that all spins are aligned in the $\ket{+x}$ direction:
\begin{align*}
  \ket{\psi}=\underbrace{\ket{+x}\otimes\cdots\otimes\ket{+x}}_{n\text{ times}}
\end{align*}
We then want to measure $x$ in this state, but we note that the operator evolves according to the time evolution operator $U(t)$:
\begin{align*}
  U(t)&=e^{-itH}\\
  X(t)&=U^\dag(t)XU(t)
\end{align*}
\subsection{Ensemble average of $X$}

\subsection{Ensemble variance}

\subsection{Decay Time}

\subsection{Quantum Average after $T$}

\subsection{Transformation as Time Reversal}

\subsection{Ising Hamiltonian}


\section{Environmental Effects}

\subsection{Completely Depolarizing Channel}

\subsection{Properties of a Depolarizing Channel}

\subsection{Kraus Operators which Depolarize}

\subsection{Depolarized Quantum Average}

\subsection{Quantum Average after One Depolarization}

\subsection{Expectation value over time}

\subsection{Rate of Change of von Neumann Entropy}


\end{document}