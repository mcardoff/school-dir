\documentclass[12pt]{article}

%++++++++++++++++++++++++++++++++++++++++
% Don't modify this section unless you know what you're doing!
\documentclass[letterpaper,12pt]{article}
\usepackage{tabularx} % extra features for tabular environment
\usepackage{multirow} % multiple rows in the table
\usepackage{physics}  % improve math presentation
\usepackage{graphicx} % takes care of graphic including machinery
\usepackage[margin=1in,letterpaper]{geometry} % decreases margins
\usepackage{cite} % takes care of citations
\usepackage{float}
\usepackage[final]{hyperref} % adds hyper links inside the generated pdf file
\usepackage{url}
\usepackage{natbib}
\usepackage[labelfont=bf]{caption}
\hypersetup{
	colorlinks=true,       % false: boxed links; true: colored links
	linkcolor=blue,        % color of internal links
	citecolor=blue,        % color of links to bibliography
	filecolor=magenta,     % color of file links
	urlcolor=blue         
}
% ++++++++++++++++++++++++++++++++++++++++

\newcommand{\labfig}[4]{
  \begin{figure}[H]
    \centering
    \includegraphics[width=#1cm]{#2}
    \caption{#3}
    \label{#4}
  \end{figure}}


\begin{document}

\title{Title of the Report}
\author{A. Partner, B. Partner, and C. Partner}
\date{\today}
\maketitle

\begin{abstract}
In this experiment we studied a very important physical effect by measuring the
dependence of a quantity $V$ of the quantity $X$ for two different sample
temperatures.  Our experimental measurements confirmed the quadratic dependence
$V = kX^2$ predicted by Someone's first law. The value of the mystery parameter
$k = 15.4\pm 0.5$~s was extracted from the fit. This value is
not consistent with the theoretically predicted $k_{theory}=17.34$~s. We attribute this
discrepancy to low efficiency of our $V$-detector.
\end{abstract}


\section{Introduction/Objective}

The very important physical effect has applications to astronomy, nuclear physics, condensed matter, and more. 


\section{Theory/Background}

Here give a brief summary of the physical effect of interest and provide
necessary equations. Here is how you insert an equation. According to
references~\cite{melissinos, Cyr, Wiki} the dependence of interest is given

\section{Procedures}

Give a schematic of the experimental setup(s) used in the experiment (see
figure~\ref{fig:Fig1}). Give the description of  abbreviations
either in the figure caption or in the text. Write a description of what is
going on. 

Don't forget to list all important steps in your experimental procedure!

Use active voice either in past or present through all the report and be
consistent with it:
The laser light comes  from to ... and eventually arrived to the
balanced photodiode as seen in the figure~\ref{fig:Fig1}.

Sentences in the past voice while correct are generally considered hard to read
in large numbers. The laser light was directed to ..., wave plates were set
to ... etc.


\section{Data and Analysis}

In this section you will need to show your experimental results. Use tables and
graphs when it is possible. Table~\ref{tbl:bins} is an example.

\begin{table}[ht]
\begin{center}
\caption{Every table needs a caption.}
\label{tbl:bins} % spaces are big no-no withing labels
\begin{tabular}{|cc|} 
\hline
\multicolumn{1}{|c}{$x$ (m)} &
\multicolumn{1}{c|}{$V$ (V)} \\
\multicolumn{1}{c|}{$V$ (V)} \\
\multirow{2}{*}{Multirow}&X\\
\multirow{2}{*}{Multirow}&Y\\
\multirow{2}{*}{Multirow}&Z\\
\hline
0.0044151 &   0.0030871 \\
0.0021633 &   0.0021343 \\
0.0003600 &   0.0018642 \\
0.0023831 &   0.0013287 \\
\hline
\end{tabular}
\end{center}
\end{table}

Analysis of equation~\ref{eq:aperp} shows ...

Note: this section can be integrated with the previous one as long as you
address the issue. Here explain how you determine uncertainties for different
measured values. Suppose that in the experiment you make a series of
measurements of a resistance of the wire $R$ for different applied voltages
$V$, then you calculate the temperature from the resistance using a known
equation and make a plot  temperature vs. voltage squared. Again suppose that
this dependence is expected to be linear~\cite{Cyr}, and the proportionality coefficient is extracted from the graph. Then what you need to explain is that for the
resistance and the voltage the uncertainties are instrumental (since each
measurements in done only once), and they are $\dots$. Then give an equation
for calculating the uncertainty of the temperature from the resistance
uncertainty. Finally explain how the uncertainty of the slop of the graph was
found (computer fitting, graphical method, \emph{etc}.)

If in the process of data analysis you found any noticeable systematic
error(s), you have to explain them in this section of the report.

It is also recommended to plot the data graphically to efficiently illustrate
any points of discussion. For example, it is easy to conclude that the
experiment and theory match each other rather well if you look at
Fig.~\ref{fig:Fig1} and Fig.~\ref{fig:Fig2}.

\labfig{8}{second_plot.png}{Every plot must have axes labeled}{fig:Fig2}

\section{Conclusion}
Here you briefly summarize your findings.

%++++++++++++++++++++++++++++++++++++++++
% References section will be created automatically 
% with inclusion of "thebibliography" environment
% as it shown below. See text starting with line
% \begin{thebibliography}{99}
% Note: with this approach it is YOUR responsibility to put them in order
% of appearance.

% \begin{thebibliography}{99}

% \bibitem{melissinos}
% A.~C. Melissinos and J. Napolitano, \textit{Experiments in Modern Physics},
% (Academic Press, New York, 2003).

% \bibitem{Cyr}
% N.\ Cyr, M.\ T$\hat{e}$tu, and M.\ Breton,
% "All-optical microwave frequency standard: a proposal,"
% IEEE Trans.\ Instrum.\ Meas.\ \textbf{42}, 640 (1993).

% \bibitem{Wiki} \emph{Expected value},  available at
% \texttt{http://en.wikipedia.org/wiki/Expected\_value}.

% \end{thebibliography}

\bibliographystyle{abbrv}
\bibliography{template}

\end{document}



\title{\vspace{-3em}}
\date{\today}

% setups
\graphicspath{ {./figs/} }

\begin{document}
\maketitle
Sure, here are the MSR equations in \LaTeX:

The starting point is the path integral representation of the time evolution operator:

\begin{equation}
    \mathcal{U}(t,t_0) = \int \mathcal{D}[\phi] \exp \left\{ -S[\phi] + \int_{t_0}^t dt' \, j(t') \cdot \phi(t') \right\},
\end{equation}

where $\phi(t)$ represents the dynamical variables of the system, $j(t)$ is an external force or fluctuation, and $S[\phi]$ is the action functional.

The MSR formalism introduces a set of auxiliary fields $h(t)$ that are coupled to the original dynamical variables via a set of linear equations:

\begin{equation}
    \frac{\delta S}{\delta \phi(t)} h(t) = j(t) - \int_{t_0}^t dt' \, M(t-t') \cdot h(t'),
\end{equation}

where $M(t)$ is a memory kernel that describes the influence of the past values of the auxiliary field on its current value.

Integrating out the auxiliary fields, one obtains a set of nonlinear partial differential equations for the probability distribution $P[\phi]$ of the dynamical variables:

\begin{equation}
    \frac{\partial P[\phi,t]}{\partial t} = -\int_{t_0}^t dt' \int \mathcal{D}[\xi] \, \frac{\delta}{\delta \phi(t')} \cdot \left[ \frac{\delta S}{\delta \phi(t)} \cdot \frac{\delta S}{\delta \phi(t')} \right]^{-1} \cdot \frac{\delta^2 S}{\delta \phi(t') \delta \phi(t)} \cdot P[\phi,t'] \cdot P[\xi,t'] \cdot e^{\xi(t') \cdot h(t')},
\end{equation}

where $\xi(t)$ is a set of Gaussian random variables that represents the fluctuations of the auxiliary fields.

These equations provide a powerful tool for studying the time evolution and statistical properties of nonequilibrium systems. However, they can be difficult to solve exactly, and various approximation schemes and numerical methods have been developed to tackle specific problems.
\end{document}