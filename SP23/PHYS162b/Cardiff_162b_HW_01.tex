\documentclass[12pt]{article}

\title{\vspace{-3em}PHYS 162b HW 1}
\author{Michael Cardiff}
\date{\today}

%% science symbols 
\usepackage{amssymb,amsthm,bm,physics,slashed}

%% general pretty stuff
\usepackage{caption,enumitem,float,geometry,graphicx,tikz}

% setups
\graphicspath{ {./figs/} }
\captionsetup{labelfont=bf}
\geometry{margin=1in}

% macros
\renewcommand{\L}{\mathcal{L}}
\newcommand{\D}{\partial}
\newcommand{\veps}{\varepsilon}
\newcommand{\circled}[1]{\tikz[baseline=(char.base)]{
    \node[shape=circle,draw,inner sep=2pt](char){#1};}}
\renewcommand{\l}{\ell}

\begin{document}
\maketitle
\section{Rigid Rotator}
The Hamiltonian we are working with is:
\begin{align*}
  H=A\vb{L}^2+BL_z+CL_y
\end{align*}
So we need to solve the Eigenvalue equation:
\begin{align*}
  H\ket{\psi}=A\vb{L}^2\ket{\psi}+BL_z\ket{\psi}+CL_y\ket{\psi}=E\ket{\psi}
\end{align*}
We know Eigenvalues of Angular Momentum operators are indexed by $\ell,m$, with the vector $\ket{\ell,m}$ having the following eigenvalues:
\begin{align*}
  \vb{L}^2\ket{\l,m}&=\hbar^2\l(\l+1)\ket{\l,m}\\
  L_z\ket{\l,m}&=\hbar m\ket{\l,m}
\end{align*}
So we are trying to find eigenvalues and eigenfunctions of:
\begin{align*}
  A\vb{L}^2\ket{\l,m}+BL_z\ket{\l,m}+CL_y\ket{\l,m}=E\ket{\l,m}
\end{align*}
However we cannot directly find eigenvalues of $L_y$, we would need to rotate about the $x$ axis again so that the operator $BL_z+CL_y$ is the new operator $L_{z'}$. Since under rotations, the total angular momentum $\vb{L}^2$ is conserved, the $\vb{L}^2$ operator remains the same.

This rotation about the $x$ axis is parameterized by some angle $\theta$, if we define:
\begin{align*}
  \tan\theta=\frac{C}{B}
\end{align*}
We have the following:
\begin{align*}
  \cos\theta&=\frac{B}{\sqrt{B^2+C^2}}\\
  \sin\theta&=\frac{C}{\sqrt{B^2+C^2}}
\end{align*}

Then our Hamiltonian looks like:
\begin{align*}
  A\vb{L}^2\ket{\l,m}+\sqrt{B^2+C^2}(\cos\theta L_z+\sin\theta L_y)\ket{\l,m}
\end{align*}
The rotation operator about the $x$ axis is given by the $L_x$ operator:
\begin{align*}
  \mathfrak{D}_x(\theta)=\exp{-\frac{iL_x\theta}{\hbar}}\\
  \mathfrak{D}_x^{-1}(\theta)=\exp{\frac{iL_x\theta}{\hbar}}
\end{align*}
If we want to transform the $L_z$ operator with respect to the $x$-axis rotation, we need to compute:
\begin{align*}
  L_{z'}\equiv\mathfrak{D}_x^{-1}(\theta)L_z\mathfrak{D}_x(\theta)
\end{align*}
We can use eq. 2.168 to evaluate these, with terms up to third order:
\begin{align*}
  \exp{iG\lambda} A \exp{-iG\lambda}=
  A+i\lambda\comm{G}{A}-\frac{\lambda^2}{2}\comm{G}{\comm{G}{A}}
  +\qty(\frac{i^3\lambda^3}{3!})\comm{G}{\comm{G}{\comm{G}{A}}}
\end{align*}
We note that the values are:
\begin{gather*}
  G=\frac{L_x}{\hbar}\\
  \lambda=\theta\\
  A=L_{z}
\end{gather*}
Insert these values
\begin{align*}
  \exp{iL_x\theta/\hbar} L_z \exp{-iL_x\theta/\hbar}=
  L_z+i\frac{\theta}{\hbar}\comm{L_x}{L_z}
  -\frac{\theta^2}{2\hbar^2}\comm{L_x}{\comm{L_x}{L_z}}
  -\frac{i\theta^3}{3!\hbar^3}\comm{L_x}{\comm{L_x}{\comm{L_x}{L_z}}}
\end{align*}
Note the following commutators:
\begin{gather*}
  \comm{L_x}{L_z} = -i\hbar L_y\\
  \comm{L_x}{L_y} = i\hbar L_z\\
  \comm{L_x}{\comm{L_x}{L_z}} = -i\hbar\comm{L_x}{L_y}=-\hbar^2L_z\\
  \comm{L_x}{\comm{L_x}{\comm{L_x}{L_z}}} = -\hbar^2\comm{L_x}{L_z}=i\hbar^3L_y
\end{gather*}
So the transformation is:
\begin{align*}
  \exp{iL_x\theta/\hbar} L_z \exp{-iL_x\theta/\hbar}&=
  L_z+\frac{i\theta}{\hbar}\qty(-i\hbar L_y)
  +\qty(\frac{i^2\theta^2}{2!\hbar^2})\qty(\hbar^2L_z)
  +\qty(\frac{i^3\theta^3}{3!\hbar^3})(-i\hbar^3L_y)\\
  &=\qty(1-\frac{\theta^2}{2!})L_z+\qty(\theta-\frac{\theta^3}{3!})L_y
\end{align*}
If we had included higher order terms, we would see the Taylor series for $\sin\theta$ and $\cos\theta$ show up:
\begin{align*}
  \exp{iL_x\theta/\hbar} L_z \exp{-iL_x\theta/\hbar}=
  L_z\cos\theta+L_y\sin\theta\equiv L_{z'}
\end{align*}
Hence the Hamiltonian is of the form:
\begin{align*}
  A\vb{L}^2\ket{\l,m}+\sqrt{B^2+C^2}L_{z'}\ket{\l,m}
\end{align*}
And we have identified $L_{z'}$ as:
\begin{align*}
  L_{z'}\equiv\mathfrak{D}_x^{-1}L_z\mathfrak{D}_x
\end{align*}
We must consider that $L_{z'}$ has the following eigenvalues:
\begin{align*}
  L_{z'}\ket{\l,m'}=\hbar m'\ket{\l,m'}
\end{align*}
However, we can explicitly identify what $L_{z'}$ is with respect to our operation $\mathfrak{D}_x$, since the vector as well as the operator were rotated:
\begin{align*}
  L_{z'}\ket{\l,m'}&=\mathfrak{D}_x^{-1}L_{z}\mathfrak{D}_x
  \mathfrak{D}_x^{-1}\ket{\l,m}=\mathfrak{D}_x^{-1}L_{z}\ket{\l,m}\\
  &=\mathfrak{D}_x^{-1}\hbar m\ket{\l,m}=\hbar m\mathfrak{D}_x^{-1}\ket{\l m}\\
  &=\hbar m\ket{\l m'}
\end{align*}
Hence, $m'=m$
\begin{align}
  \boxed{E_{\l,m}=A\hbar^2\l(\l+1)+\sqrt{B^2+C^2}\hbar m}
\end{align}
With the eigenfunctions $\ket{\l,m'}$ being the normal $\ket{\l,m}$ functions rotated by $\theta$ along the $x$-axis:
\begin{align}
  \boxed{\ket{\l,m'}=\qty(\exp{\frac{i\theta L_x}{\hbar}})\ket{\l,m}}
\end{align}

\section{Nucleon Tensor Force}
We are not going to be changing the definition of $V(r)$ at all, since we are only redefining the spin variables in this problem, so we only need to worry about the spin terms.

We note that we can rewrite $\vb{S}$ as:
\begin{align*}
  \vb{S}=\vb{s}_1+\vb{s}_2=\frac{\hbar}{2}\qty(\bm\sigma_1+\bm\sigma_2)
\end{align*}
So that its square is:
\begin{align*}
  \vb{S}^2&=\frac{\hbar^2}{4}\qty(\bm\sigma_1+\bm\sigma_2)^2\\
  &=\frac{\hbar^2}{4}
  \qty(\bm\sigma_1^2+\bm\sigma_2^2+2\bm\sigma_1\vdot\bm\sigma_2)
\end{align*}

The dot with $\vb{r}$:
\begin{align*}
  \vb{S}\vdot\vb{r}&=\frac{\hbar}{2}\qty(\bm\sigma_1+\bm\sigma_2)\vdot\vb{r}\\
  &=\frac{\hbar}{2}\qty(\bm\sigma_1\vdot\vb{r}+\bm\sigma_2\vdot\vb{r})\\
  \qty(\vb{S}\vdot\vb{r})^2&=
  \frac{\hbar^2}{4}\qty(\bm\sigma_1\vdot\vb{r}+\bm\sigma_2\vdot\vb{r})^2\\
  &=\frac{\hbar^2}{4}\qty(
  (\bm\sigma_1\vdot\vb{r})^2+(\bm\sigma_2\vdot\vb{r})^2
  +2(\bm\sigma_1\vdot\vb{r})(\bm\sigma_2\vdot\vb{r}))\\
  \frac{3}{r^2}\qty(\vb{S}\vdot\vb{r})^2&=\frac{3\hbar^2}{4}\qty(
  (\bm\sigma_1\vdot\vu{r})^2+(\bm\sigma_2\vdot\vu{r})^2
  +2(\bm\sigma_1\vdot\vu{r})(\bm\sigma_2\vdot\vu{r}))
\end{align*}
So the difference of the two terms we have found:
\begin{align*}
  \frac{3}{r^2}\qty(\vb{S}\vdot\vb{r})^2-\vb{S}^2=
  \frac{3}{r^2}\qty(\vb{S}\vdot\vb{r})^2&=
  \frac{3\hbar^2}{4}\qty(
  (\bm\sigma_1\vdot\vu{r})^2+(\bm\sigma_2\vdot\vu{r})^2
  +2(\bm\sigma_1\vdot\vu{r})(\bm\sigma_2\vdot\vu{r}))
  -\frac{\hbar^2}{4}
  \qty(\bm\sigma_1^2+\bm\sigma_2^2+2\bm\sigma_1\vdot\bm\sigma_2)\\
  &=\frac{\hbar^2}{4}\qty(
  3(\bm\sigma_1\vdot\vu{r})^2+3(\bm\sigma_2\vdot\vu{r})^2
  +6(\bm\sigma_1\vdot\vu{r})(\bm\sigma_2\vdot\vu{r})-
  \bm\sigma_1^2-\bm\sigma_2^2-2\bm\sigma_1\vdot\bm\sigma_2)
\end{align*}
Since the Pauli matrices are their own inverses, that is their square is the identity, we have:
\begin{align*}
  \bm\sigma_{i}^2=(\sigma_1^1)^2+(\sigma_2^2)^2+(\sigma_3^3)^2&=3I\\
  (\bm\sigma_1\vdot\vu{r})^2=\sigma_{r}^2&=I
\end{align*}
Hence we can rewrite the expression as:
\begin{align*}
  \frac{3}{r^2}\qty(\vb{S}\vdot\vb{r})^2-\vb{S}^2&=
  \frac{\hbar^2}{4}\qty(3I+3I-3I-3I
  +6(\bm\sigma_1\vdot\vu{r})(\bm\sigma_2\vdot\vu{r})
  -2\bm\sigma_1\vdot\bm\sigma_2)\\
  &=\frac{\hbar^2}{2}\qty(
  3(\bm\sigma_1\vdot\vu{r})(\bm\sigma_2\vdot\vu{r})
  -\bm\sigma_1\vdot\bm\sigma_2)\\
  &=\frac{\hbar^2}2\qty(
  \frac{3(\bm\sigma_1\vdot\vu{r})(\bm\sigma_2\vdot\vu{r})}{r^2}
  -\bm\sigma_1\vdot\bm\sigma_2)
\end{align*}
If we multiply through by $2/\hbar^2$ to cancel the factor we have:
\begin{align*}
  \frac2{\hbar^2}\qty(\frac{3}{r^2}\qty(\vb{S}\vdot\vb{r})^2-\vb{S}^2)&=
  \frac{3(\bm\sigma_1\vdot\vu{r})(\bm\sigma_2\vdot\vu{r})}{r^2}
  -\bm\sigma_1\vdot\bm\sigma_2
\end{align*}
All that is left is to add in the $V(r)$ term:
\begin{align}
  \boxed{
    \frac2{\hbar^2}V(r)\qty(\frac{3}{r^2}\qty(\vb{S}\vdot\vb{r})^2-\vb{S}^2)=
    V(r)\qty(\frac{3(\bm\sigma_1\vdot\vu{r})(\bm\sigma_2\vdot\vu{r})}{r^2}
    -\bm\sigma_1\vdot\bm\sigma_2)}
\end{align}

\end{document}