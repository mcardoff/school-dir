\documentclass[12pt]{article}

\title{\vspace{-3em}PHYS 162b HW 5}
\author{Michael Cardiff}
\date{\today}

%% science symbols 
\usepackage{amssymb,amsthm,bm,physics,slashed}

%% general pretty stuff
\usepackage{caption,enumitem,float,geometry,graphicx,tikz}

% setups
\graphicspath{ {./figs/} }
\captionsetup{labelfont=bf}
\geometry{margin=1in}

% macros
\renewcommand{\L}{\mathcal{L}}
\renewcommand{\H}{\mathcal{H}}
\newcommand{\D}{\partial}
\newcommand{\circled}[1]{\tikz[baseline=(char.base)]{
    \node[shape=circle,draw,inner sep=2pt](char){#1};}}

\begin{document}
\maketitle

\section{Sakurai 5.13}
\subsection{Energies for 3 Lowest States}
The energy spectrum for this system is simply the sum of two independent infinite square wells:
\begin{align*}
  E_{nm}=\frac{\hbar^2}{2ma^2}\qty(n^2+m^2)
\end{align*}
With $n$ corresponding to $x$ and $m$ to $y$, the wavefunctions are:
\begin{align*}
  \psi_{nm}=\frac2a\sin(\frac{n\pi}ax)\sin(\frac{m\pi}ay)
\end{align*}
Thus the first three states by their $n$ and $m$ labels are:
\begin{align}
\boxed{\{n,m\}=
  \begin{cases}
    \{1,1\} &\text{non-degenerate} \\
    \{1,2\} &\text{twice degenerate} \\
    \{2,2\} &\text{non-degenerate}
  \end{cases}}
\end{align}

\subsection{Adding Perturbation}
\subsubsection{Energy Shift order in $\lambda$}
For each of the states, we have $\lambda$ times the correction, so it should only be linear in $\lambda$

\subsubsection{Order $\lambda$ Corrections}
For the non-degenerate corrections we have:
\begin{align*}
  \Delta_1^{(1)}&=\lambda\qty[\frac2a\int_0^ax\sin^2\frac{\pi x}a\dd{x}]
  \qty[\frac2a\int_0^ay\sin^2\frac{\pi y}a\dd{y}]=
  \lambda\frac{a^2}4\\
  \Delta_3^{(1)}&=\lambda\qty[\frac2a\int_0^ax\sin^2\frac{2\pi x}a\dd{x}]
  \qty[\frac2a\int_0^ay\sin^2\frac{2\pi y}a\dd{y}]=
  \lambda\frac{a^2}4
\end{align*}
For the Degenerate case, we have the 2 states we need to construct the degeneacy matrix from:
\begin{align*}
  \psi_a&=\frac2a\sin\frac{\pi x}a\sin\frac{2\pi y}a\\
  \psi_b&=\frac2a\sin\frac{2\pi x}a\sin\frac{\pi y}a
\end{align*}
The diagonal elements of the degeneracy matrix are:
\begin{align*}
  W_{aa}=\lambda\frac4{a^2}\int_0^a\int_0^a\dd{x}\dd{y}
  xy\sin^2\frac{\pi x}a\sin^2\frac{2\pi y}a
\end{align*}
Which is easily done in mathematica:
\begin{align*}
  W_{bb}=W_{aa}=\lambda\frac{a^2}{4}
\end{align*}
Due to renaming integration variables we can say $W_{bb}=W_{aa}$, the off diagonal elements are found in mathematica
\begin{align*}
  W_{ab}=W_{ba}=\frac{256a^2\lambda}{81\pi^4}
\end{align*}
\begin{figure}[H]
  \centering
  \includegraphics[width=10.0cm]{integral1}
  \caption{Mathematica for Integral}
\end{figure}
We then construct $[W]$:
\begin{align*}
  [W]=\frac{\lambda a^2}{4}\pmqty{1 & 1024/81\pi^2 \\ 1024/81\pi^2 & 1}
\end{align*}
And can diagonalize it in Mathematica:
\begin{align*}
  \ket{w_\pm}=\frac1{\sqrt{2}}\qty(\psi_a\pm\psi_b)\\
  w_\pm=\frac{\lambda a^2}{4}\qty(1\pm\frac{1024}{81\pi^4})
\end{align*}
\begin{figure}[H]
  \centering
  \includegraphics[width=10.0cm]{diagw}
  \caption{Mathematica for Diagonalization}
\end{figure}
To summarize:
\begin{equation}
  \boxed{\begin{aligned}
      \Delta_1^{(1)}&=\lambda\frac{a^2}{4}\\
      \Delta_{2a}^{(1)}&=\lambda\frac{a^2}{4}\qty(1+\frac{1024}{81\pi^4})\\
      \Delta_{2b}^{(1)}&=\lambda\frac{a^2}{4}\qty(1-\frac{1024}{81\pi^4})\\
      \Delta_1^{(1)}&=\lambda\frac{a^2}{4}
  \end{aligned}}
\end{equation}

\subsubsection{Energy Diagram}
This is fairly simple, since $\Delta_1^{(1)}$ and $\Delta_3^{(1)}$ increase the eenergy by a fixed amount, and $\Delta_2^{(1)}$ increases by that same amount before adding or subtracting the same amount, we find that:
\begin{figure}[H]
  \centering
  \includegraphics[width=10.0cm]{ediagram}
  \caption{Energy Diagram}
\end{figure}

\section{Sakurai 5.14}
As a reminder, we are working with the following Hamiltonian:
\begin{align*}
  \H=\pmqty{E_1^0&\lambda\Delta\\\lambda\Delta&E_2^0}
\end{align*}

\subsection{Exact Solution}
For the exact solution we need to solve:
\begin{align*}
  \H\ket{\psi}=E\ket{\psi}
\end{align*}
So we need to solve:
\begin{align*}
  \det{H-I\mu}&=0\\
  E_1^0E_2^0-\Delta^2\lambda^2-(E_1^0+E_2^0)\mu+\mu^2&=0
\end{align*}
We then find the eigenvalues are:
\begin{align}
  \boxed{E_{1,2}=\frac{E_1^0+E_{2}^0}2\pm
  \sqrt{\frac{(E_1^0-E_2^0)^2}{4}+\lambda^2\Delta^2}}
\end{align}
And the Eigenvectors can we written as:
\begin{align}
  \boxed{\begin{aligned}
    \psi_1&=\pmqty{\cos(\beta/2)\\\sin(\beta/2)}\\
    \psi_2&=\pmqty{-\sin(\beta/2)\\\cos(\beta/2)}
  \end{aligned}}
\end{align}
Where
\begin{align*}
  \tan\beta=\frac{2\lambda\Delta}{E_1^0-E_2^0}
\end{align*}
Equating this essentially to a spin system.

\subsection{Perturbative Solution}
We now have the following for $H_0$ and $V$:
\begin{align*}
  H_0&=\pmqty{\dmat[0]{E_1^0,E_2^0}}\\
  V&=\pmqty{0&\lambda\Delta\\\lambda\Delta&0}
\end{align*}
With the unperturbed basis states as:
\begin{align*}
  \phi_1^{0}=\pmqty{1\\0}\quad\phi_2^{0}=\pmqty{0\\1}
\end{align*}
We can immediately see that the first order correction to the energies are $0$, since $V_{11}=V_{22}=0$. The second order energy shifts are given by eq. 5.42:
\begin{align*}
  \Delta_n^{(1)}&=\sum_{k\neq n}\frac{(V_{nk})^2}{E_{n}^0-E_k^0}\\
  \Delta_1^{(1)}&=\sum_{k\neq 1}\frac{(V_{1k})^2}{E_{1}^0-E_k^0}
  =\frac{(V_{12})^2}{E_{1}^0-E_2^0}\\
  \Delta_2^{(1)}&=\sum_{k\neq 2}\frac{(V_{2k})^2}{E_{2}^0-E_k^0}
  =\frac{(V_{21})^2}{E_{2}^0-E_1^0}
\end{align*}
Note that $V_{12}=V_{21}=\lambda\Delta$, so the energy corrections are:
\begin{align}
  \boxed{\begin{aligned}
    \Delta_1^{(1)}&=\frac{(\lambda\Delta)^2}{E_1^0-E_2^0}\\
    \Delta_2^{(1)}&=-\frac{(\lambda\Delta)^2}{E_1^0-E_2^0}
  \end{aligned}}
\end{align}
Then the corrections to the wavefunctions:
\begin{align}
  \ket{n^{(1)}}&=\sum_{k\neq n}\ket{k^{0}}\frac{V_{kn}}{E_{n}^0-E_k^0}\notag\\
  \ket{\phi_1^{(1)}}&=\phi_1^0\notag
  \sum_{k\neq 1}\ket{k^{0}}\frac{V_{k1}}{E_1^0-E_k^0}\\
  &=\phi_1^0+\phi_2^0\frac{V_{21}}{E_1^0-E_2^0}=
  \boxed{\pmqty{1\\\frac{\lambda\Delta}{E_1^0-E_2^0}}}\\
  \ket{\phi_2^{(1)}}&=\sum_{k\neq 2}\ket{k^{0}}\frac{V_{k2}}{E_2^0-E_k^0}\notag\\
  &=\phi_2^0-\phi_1^0\frac{V_{12}}{E_1^0-E_2^0}=
  \boxed{\pmqty{-\frac{\lambda\Delta}{E_1^0-E_2^0}\\1}}
\end{align}
Which if we take $\beta\to0$ in our exactly solved eigenstates, we will get $\cos\beta/2\to1$ and $\sin\beta/2\to\beta/2$ The energy Eigevalues in this limit goes to:
\begin{align*}
  E_{1/2}=\frac{E_1^0+E_2^0}{2}\pm\frac{E_1^0-E_2^0}{2}
  \sqrt{1+\frac{4\lambda^2\Delta^2}{(E_1^0-E_2^0)}}
\end{align*}
The square root reduces to:
\begin{align*}
  E_{1/2}=\frac{E_1^0+E_2^0}{2}\pm\frac{E_1^0-E_2^0}{2}
  \qty[1+\frac{2\lambda^2\Delta^2}{(E_1^0-E_2^0)}]
\end{align*}
Which isz the same as we found in second order PT

\subsection{Almost Degenerate Limit}
If we instead take an opposite limit, we have:
\begin{align*}
  \tan\beta\to\infty\implies\beta=\pi/2
\end{align*}
Giving the following eigenvectors:
\begin{align}
  \boxed{\psi_{1,2}=\frac1{\sqrt{2}}\pmqty{\pm1\\1}}
\end{align}
And the Eigenvalues now become:
\begin{align*}
  E_{1,2}&=\frac{E_1^0+E_2^0}{2}\pm\lambda\Delta
  \sqrt{1+\frac{(E_1^0-E_2^0)^2}{4\lambda^2\Delta^2}}\\
  &\to \frac{E_1^0+E_2^0}{2}\pm\lambda\Delta
  \qty[1+\frac{(E_1^0-E_2^0)^2}{8\lambda^2\Delta^2}]
\end{align*}
So to first order in the degenerate case where $E_1^0=E_2^0$, we have splitting $\pm\lambda\Delta$, but in this limit we have:
\begin{align}
  \boxed{E_{1,2}=\frac{E_1^0+E_2^0}{2}\pm\lambda\Delta
  \qty[1+\frac{(E_1^0-E_2^0)^2}{8\lambda^2\Delta^2}]}
\end{align}

\section{Sakurai 3.5}
\subsection{Turning off Interaction}
Denote the state $\chi^{e^-}_+\chi^{e^+}_-$ as $\ket{+-}$, if $A=0$ we have:
\begin{align*}
  H\ket{+-}=\frac{eB}{mc}\qty[S_z^{e^-}-S_z^{e^+}]\ket{+-}
\end{align*}
The $\ket{+-}$ is simply $\ket{+}^{e^{-}}\ket{-}^{e^{+}}$, which are eigenfunctions of both spin operators with eigenvalues $\pm\frac\hbar2$:
\begin{align*}
  \frac{eB}{mc}\qty[S_z^{e^-}-S_z^{e^+}]\ket{+-}=
  \frac{eB}{mc}\qty[\frac\hbar2-\qty(-\frac\hbar2)]\ket{+-}
\end{align*}
Hence we do have an eigenstate of $H$, since:
\begin{align}
  \boxed{H\ket{+-}=\hbar\frac{eB}{mc}\ket{+-}}
\end{align}

\subsection{Making Interaction Stronger}
If we instead of turning the interaction off, make it strong enough to overpower the previous term, we get:
\begin{align*}
  H\ket{+-}=A
  \qty[S_z^{e^-}S_z^{e^+}+\frac12S_+^{e^-}S_-^{e^+}+\frac12S^{e^-}_-S^{e^+}_+]
  \ket{+-}=
  -A\frac{\hbar^2}4\ket{+-}+0+\frac{A\hbar^2}2\ket{-+}
\end{align*}
The term from raising the $e^{-}$ spin and lowering the $e^{-}$ spin comes from the fact that you cannot lower further than $-$ and cannot raise $+$. We then do not have an eigenstate, instead we have:
\begin{align}
  \boxed{H\ket{+-}=A\frac{\hbar^2}4\qty(2\ket{-+}-\ket{+-})}
\end{align}
We can then find the expected value of the Hamiltonian:
\begin{align}
  \boxed{\ev{+-}{H}{+-}=-A\frac{\hbar^2}{4}}
\end{align}

\end{document}