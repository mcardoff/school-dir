% -*- TeX-master: "master.tex" -*-
\def\bd{\bm{\delta}}
\section{Electromagnetism}
\begin{problem}
  A plane electromagnetic wave of angular frequency $\omega$ propagates in a uniform plasma with electron density $N_e$. The plasma is locally neutral, $\rho=0$. The electromagnetic wave generates periodic currents within the plasma that, as the problem will show, modify the index of refraction of the medium compared to that of the vacuum.
Assume the plasma has no resistivity and neglect radiation pressure effects as well as currents due to the ions.
\begin{enumerate}[label=(\alph*)]
\item Relate the current $\vb{J}(\vb{r}, t)$ in the plasma to the wave’s electric field $\vb{E}(\vb{r}, t)$ or derivatives thereof. Assume magnetic forces can be neglected.
\item Write down the appropriate Maxwell equations and derive the wave equation. Find the dispersion relation $\omega(k)$ and the lowest frequency electromagnetic wave that can propagate the plasma.
\item Find the phase and group velocity for electromagnetic waves in the plasma. Compare those velocities with $c$, the speed of light in vacuum.
\item Find the index of refraction $n$ of the plasma as a function of frequency.
\item If a plane electromagnetic wave is incident on a plane interface between a vacuum and the plasma, what is the critical angle for total reflection, measured from the normal to the interface?
\end{enumerate}
\end{problem}

\subsection{Current Density}
Note that the current density $\vb{J}$ is related to the velocity of small drifts of the electrons, call these small drifts $\bd(\vb{r},t)$. The wave causes periodic currents, so we know that these drifts look like:
\begin{align*}
  \ddot{\bd}=-\omega^2\bd
\end{align*}
However, this motion according to Newton's law for electric forces is given by:
\begin{align*}
  m\ddot{\bd}=-e\vb{E}
\end{align*}
Subbing in what we know for $\ddot{\bd}$:
\begin{align*}
  m\omega^2\bd=e\vb{E}
\end{align*}
The current density is a macroscopic property, opposite in direction to electron flow and proportional to the number charge density of electrons and their velocity, from this we can derive:
\begin{align*}
  \vb{J}=-eN_e\dot{\bd}
\end{align*}
Rearranging what we know for $\bd$ allows us to get a sense of the velocity in terms of $\vb{E}$:
\begin{align*}
  \bd=\frac{e}{m\omega^2}\vb{E}\implies
  \dot{\bd}=\frac{e}{m\omega^2}\pdv{\vb{E}}{t}
\end{align*}
And we get the current density as:
\begin{align}
  \boxed{\vb{J}=-\frac{N_ee^2}{m\omega^2}\pdv{\vb{E}}{t}}
\end{align}

\subsection{Maxwell Equations}
The full Maxwell equations in SI units are:
\begin{align*}
  \div{\vb{E}}=\frac{\rho}{\veps_0} \quad&\quad
  \curl{\vb{E}}=-\pdv{\vb{B}}{t}\\
  \div{\vb{B}}=0 \quad&\quad
  \curl{\vb{B}}=\mu_0\vb{J}+\frac1{c^2}\pdv{\vb{E}}{t}
\end{align*}
Substituting properties of the plasma we have:

\begin{align*}
  \div{\vb{E}}=0 \quad&\quad
  \curl{\vb{E}}=-\pdv{\vb{B}}{t}\\
  \div{\vb{B}}=0 \quad&\quad
  \curl{\vb{B}}=\qty(\frac1{c^2}-\frac{\mu_0N_ee^2}{m\omega^2})\pdv{\vb{E}}{t}
\end{align*}
Factoring out the $c^{-2}$ in the Ampere-Maxwell law gives:
\begin{align*}
  \curl{\vb{B}}&=\qty(1-\frac{N_ee^2}{m\veps_0\omega^2})
  \frac1{c^2}\pdv{\vb{E}}{t}\\
  &=\qty(1-\frac{\omega_p^2}{\omega^2})\frac1{c^2}\pdv{\vb{E}}{t}
\end{align*}
Where $\omega_p$ is the plasma frequency:
\begin{align*}
  \omega_p^2\equiv\frac{N_ee^2}{m\veps_0}
\end{align*}
Thus the Maxwell Equations are given by:
\begin{equation}
\boxed{\begin{aligned}
    \div{\vb{E}}=0 \quad&\quad
    \curl{\vb{E}}=-\pdv{\vb{B}}{t}\\
    \div{\vb{B}}=0 \quad&\quad
    \curl{\vb{B}}=\qty(1-\frac{\omega_p^2}{\omega^2})\frac1{c^2}\pdv{\vb{E}}{t}
  \end{aligned}}
\end{equation}
The wave equation for $\vb{E}$ can be found by taking the curl of Faraday's law, and subbing in the Ampere-Maxwell law for the curl of $\vb{B}$ granted that mixed derivatives commute:
\begin{align*}
  \curl{\curl{\vb{E}}}=-\pdv{t}\curl{\vb{B}}
\end{align*}
The curl of the curl can be expanded:
\begin{align*}
  \curl{\curl{\vb{E}}}=\grad{(\div{\vb{E}})}-\laplacian{\vb{E}}
  =-\laplacian{\vb{E}}
\end{align*}
So that the equation is now:
\begin{align*}
  \laplacian{\vb{E}}&=\pdv{t}\curl{\vb{B}}\\
  &=\pdv{t}\qty[
  \qty(1-\frac{\omega_p^2}{\omega^2})\frac1{c^2}\pdv{\vb{E}}{t}]\\
  &=\qty(1-\frac{\omega_p^2}{\omega^2})\frac1{c^2}\pdv[2]{\vb{E}}{t}
\end{align*}
So we have a modified wave equation:
\begin{equation}
  \boxed{\laplacian{\vb{E}}=
  \qty(1-\frac{\omega_p^2}{\omega^2})\frac1{c^2}\pdv[2]{\vb{E}}{t}}
\end{equation}
These waves will have a similarly modified dispersion relation:
\begin{align*}
  k^2=\frac{\omega^2}{c^2}\qty(1-\frac{\omega_p^2}{\omega^2})
  =\frac{1}{c^2}\qty(\omega^2-\omega_p^2)
\end{align*}
Note we can still use $\omega$ as the frequency parameter here as the frequencies/wavenumbers at which the plasma will oscillate are determined by the input wave frequency. In terms of $k$, we can solve for $\omega$:
\begin{align*}
  \omega^2=c^2k^2+\omega_p^2
\end{align*}

\subsection{Phase and Group Velocities}
The phase velocity is the movement of high frequency oscillations within the ``envelope'' of a wave, so is given by:
\begin{align*}
  v_p=\frac{\omega}{k}
\end{align*}
This can be identified by the $v$ in the general wave equation:
\begin{align*}
  \laplacian{f}=\frac1{v^2_p}\pdv[2]{f}{t}
\end{align*}
Hence we can identify:
\begin{align*}
  v_p=c\qty(1-\frac{\omega_p^2}{\omega^2})^{-1/2}
\end{align*}
Note that the wavenumber $k$ is imaginary if $\omega<\omega_p$, so the parenthetical quantity is less than one, and so the inverse of its square root is greater than one, so the \underline{\emph{phase velocity is greater than $c$}}. This sounds problematic, however

The group velocity on the other hand is characterized by the movement of the envelope itself, characterized by:
\begin{align*}
  v_g=\dv{\omega}{k}
\end{align*}
Using the dispersion relation, we get that:
\begin{align*}
  2\omega\dv{\omega}{k}&=2c^2k\\
  \dv{\omega}{k}&=c^2\frac{k}{\omega}=\frac{c^2}{v_p}=
  c\sqrt{1-\frac{\omega_p^2}{\omega^2}}
\end{align*}
The quantity under the square root is once again less than one, hence the \underline{\emph{group velocity is less than $c$}}, and we can summarize our results:
\begin{equation}
  \boxed{\begin{aligned}
      v_p&=c\qty(1-\frac{\omega_p^2}{\omega^2})^{-1/2}>c\\
      v_g&=c\qty(1-\frac{\omega_p^2}{\omega^2})^{+1/2}<c
  \end{aligned}}
\end{equation}
The difference between this is seen by the following plot:
\begin{figure}[H]
  \centering
  \includegraphics[width=10.0cm]{phasegroup}
  \caption{Phase vs Group Velocity}
\end{figure}
The velocity of the ``Envelope'' is the group velocity, where information is transmitted, and the phase velocity is the velocity of the high frequency modulation

\subsection{Index of Refraction}
The index of refraction is defined as the ratio of the speed of light and the phase velocity $v_p$:
\begin{align}
  \boxed{n(\omega)=\sqrt{1-\frac{\omega_p^2}{\omega^2}}}
\end{align}
Note that since $v_p>c$, the \underline{\emph{index of refraction is less than 1}}.

\subsection{Critical Angle}
We can use Snell's law to determine a refraction angle from some incident angle:
\begin{align*}
  n_{\text{in}}\sin\theta_{\text{in}}=n_{\text{refr}}\sin\theta_{\text{refr}}
\end{align*}
The wave is incident on the vacuum with $n=1$, and since $n_{\text{refr}}=n_{\text{plasma}}<1$, the refracted angle will be $\ang{90}$ before the incident angle does, hence for total internal reflection, we require that the incident angle be greater than $\theta_c$, where:
\begin{align}
  \boxed{\sin\theta_c=n_{\text{plasma}}=\sqrt{1-\frac{\omega_p^2}{\omega^2}}}
\end{align}

