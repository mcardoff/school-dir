\documentclass[12pt]{article}

\title{\vspace{-3em}PHYS 202a HW 4}
\author{Michael Cardiff}
\date{\today}

%% science symbols 
\usepackage{amssymb,amsthm,bm,physics,slashed}

%% general pretty stuff
\usepackage{caption,enumitem,float,geometry,graphicx,tikz}

% setups
\graphicspath{ {./figs/} }
\captionsetup{labelfont=bf}
\geometry{margin=1in}

% macros
\renewcommand{\L}{\mathcal{L}}
\newcommand{\D}{\partial}
\newcommand{\circled}[1]{\tikz[baseline=(char.base)]{
    \node[shape=circle,draw,inner sep=2pt](char){#1};}}

\begin{document}
\maketitle

\section{Practice with Matrix Elements}

\section{Shift-Symmetric Interaction}

\section{Crossing}

\section{Propagator in a Heat Bath}
\section{Propagator in a Heat Bath}
\subsection{Deriving the Propagator}
We can start by the given definition of the propagator:
\begin{align*}
    D_\beta(x)=\ev{T(\phi(x)\phi(0)}
\end{align*}
And the definition of time ordering is:
\begin{align*}
    T(\phi(x)\phi(0))=\theta(-t)\phi(0)\phi(x)+\theta(t)\phi(x)\phi(0)
\end{align*}
Using the mode expansion, we can see that:
\begin{align*}
    \phi(x)&=\int\frac{\dd[3]{p}}{(2\pi)^3}\frac1{\sqrt{2\omega_p}}
    \qty(\hat{a}_{\vb{p}}e^{-ip\vdot x}+\hat{a}^\dag_{\vb{p}}e^{ip\vdot x})\\
    \phi(0)&=\int\frac{\dd[3]{p}}{(2\pi)^3}\frac1{\sqrt{2\omega_p}}
    \qty(\hat{a}_{\vb{p}}+\hat{a}^\dag_{\vb{p}})
\end{align*}
Such that their products are:
\begin{align*}
    \phi(x)\phi(0)&=\int\frac{\dd[3]{p}\dd[3]{p'}}{(2\pi)^6}\frac{1}{2\sqrt{\omega_p\omega_{p'}}}
    \qty(\hat{a}_{\vb{p}}\hat{a}_{\vb{p}'}e^{-ip\vdot x}+\hat{a}_{\vb{p}}\hat{a}^\dag_{\vb{p}'}e^{-ip\vdot x}
    +\hat{a}^\dag_{\vb{p}}\hat{a}_{\vb{p}'}e^{ip\vdot x}+\hat{a}^\dag_{\vb{p}}\hat{a}^\dag_{\vb{p}'}e^{ip\vdot x})\\
    \phi(0)\phi(x)&=\int\frac{\dd[3]{p}\dd[3]{p'}}{(2\pi)^6}\frac{1}{2\sqrt{\omega_p\omega_{p'}}}
    \qty(\hat{a}_{\vb{p}}\hat{a}_{\vb{p}'}e^{-ip'\vdot x}+\hat{a}_{\vb{p}}\hat{a}^\dag_{\vb{p}'}e^{ip'\vdot x}
    +\hat{a}^\dag_{\vb{p}}\hat{a}_{\vb{p}'}e^{-ip'\vdot x}+\hat{a}^\dag_{\vb{p}}\hat{a}^\dag_{\vb{p}'}e^{ip'\vdot x})
\end{align*}
Note that upon taking the inner product with the $\ket{\beta}$ state, the first and last terms will disappear since they will end up with a different number of particles than $\ket{\beta}$, and we can rewrite the middle terms as:
\begin{align*}
    \hat{a}_{\vb{p}}\hat{a}^\dag_{\vb{p}'}e^{-ip\vdot x}+\hat{a}^\dag_{\vb{p}}\hat{a}_{\vb{p}'}e^{ip\vdot x}
    &=\hat{a}^\dag_{\vb{p}}\hat{a}_{\vb{p}'}(e^{ip\vdot x}+e^{-ip\vdot x})+(2\pi)^3\delta^{(3)}(\vb{p-p}')e^{-ip\vdot x}\\
    \hat{a}_{\vb{p}}\hat{a}^\dag_{\vb{p}'}e^{ip'\vdot x}+\hat{a}^\dag_{\vb{p}}\hat{a}_{\vb{p}'}e^{-ip'\vdot x}
    &=\hat{a}^\dag_{\vb{p}}\hat{a}_{\vb{p}'}(e^{ip'\vdot x}+e^{-ip'\vdot x})+(2\pi)^3\delta^{(3)}(\vb{p-p}')e^{ip'\vdot x}
\end{align*}
From the commutator of the creation and annihilation operators. Hence when we evaluate $\ev{T(\phi(x)\phi(0))}$ we will have the following two terms:
\begin{align*}
    \ev{\phi(x)\phi(0)}_\beta &= \int\frac{\dd[3]{p}\dd[3]{p'}}{(2\pi)^6}\frac{1}{2\sqrt{\omega_p\omega_{p'}}}
    \qty(\ev{\hat{a}^\dag_{\vb{p}}\hat{a}_{\vb{p}'}}_\beta(e^{ip\vdot x}+e^{-ip\vdot x})+
    (2\pi)^3\delta^{(3)}(\vb{p-p}')e^{-ip\vdot x}\ip{\beta})\\
    &= \int\frac{\dd[3]{p}\dd[3]{p'}}{(2\pi)^6}\frac{(2\pi)^3\delta^{(3)}(\vb{p-p}')}{2\sqrt{\omega_p\omega_{p'}}}
    \qty(n_B(E_p)(e^{ip\vdot x}+e^{-ip\vdot x})+e^{-ip\vdot x})\\
    \ev{\phi(0)\phi(x)}_\beta &= \int\frac{\dd[3]{p}\dd[3]{p'}}{(2\pi)^6}\frac{1}{2\sqrt{\omega_p\omega_{p'}}}
    \qty(\ev{\hat{a}^\dag_{\vb{p}}\hat{a}_{\vb{p}'}}_\beta(e^{ip'\vdot x}+e^{-ip'\vdot x})+
    (2\pi)^3\delta^{(3)}(\vb{p-p}')e^{ip'\vdot x}\ip{\beta})\\
    &= \int\frac{\dd[3]{p}\dd[3]{p'}}{(2\pi)^6}\frac{(2\pi)^3\delta^{(3)}(\vb{p-p}')}{2\sqrt{\omega_p\omega_{p'}}}
    \qty(n_B(E_p)(e^{ip'\vdot x}+e^{-ip'\vdot x})+e^{ip'\vdot x})
\end{align*}
We can then reduce these a bit using the delta function:
\begin{align*}
    \ev{\phi(x)\phi(0)}_\beta &= \int\frac{\dd[3]{p}}{(2\pi)^3}\frac{1}{2\omega_p}
    \qty(n_B(E_p)(e^{ip\vdot x}+e^{-ip\vdot x})+e^{-ip\vdot x})\\
    \ev{\phi(0)\phi(x)}_\beta &= \int\frac{\dd[3]{p}}{(2\pi)^3}\frac{1}{2\omega_p}
    \qty(n_B(E_p)(e^{ip\vdot x}+e^{-ip\vdot x})+e^{ip\vdot x})
\end{align*}
So the form of the original definition is:
\begin{align*}
    \ev{T(\phi(x)\phi(0))}_\beta=\theta(-t)\ev{\phi(0)\phi(x)}_\beta+\theta(t)\ev{\phi(x)\phi(0)}_\beta
\end{align*}
Note that we have 2 distinct terms, the first being that which includes the Boson distribution $n_B$, and the one with only a complex exponential. We can identify the bare complex exponential with the solution we found using the vacuum, giving:
\begin{align*}
    \int\frac{\dd[3]{p}}{(2\pi)^3}\frac1{2\omega_p}\qty(\theta(-t)e^{ip\vdot x}+\theta(t)e^{-ip\vdot x})=
    \int\frac{\dd[4]{p}}{(2\pi)^4}\frac{e^{-ip\vdot x}}{p^2-m^2+i\epsilon}
\end{align*}
All that is left is the other term:
\begin{align*}
    \int\frac{\dd[3]{p}}{(2\pi)^3}\frac1{2\omega_p}n_B(E_p)(e^{ip\vdot x}+e^{-ip\vdot x})(\theta(t)+\theta(-t))
\end{align*}
Note that the second parenthetical term, the sum of heaviside functions, is just 1:
\begin{align*}
    \int\frac{\dd[3]{p}}{(2\pi)^3}\frac1{2\omega_p}n_B(E_p)(e^{ip\vdot x}+e^{-ip\vdot x})
\end{align*}
We note that the dot product in the exponentials can be written as:
\begin{align*}
    \int\frac{\dd[3]{p}}{(2\pi)^3}\frac{e^{i\vb{p}\vdot\vb{x}}}{2\omega_p}n_B(E_p)(e^{i\omega_pt}+e^{-i\omega_p t})
\end{align*}
We can then define a dummy integration variable, $p_0$, which will allow us to integrate over off-shell particles as well as on-shell ones:
\begin{align*}
    2\pi\int\frac{\dd[3]{p}}{(2\pi)^3}\frac{e^{i\vb{p}\vdot\vb{x}}}{2\omega_p}n_B(E_p)
    \int\frac{\dd{p_0}}{2\pi}e^{-ip_0t}(\delta(p_0-\omega_p)+\delta(p_0+\omega_p))
\end{align*}
Then we can use the delta function identity:
\begin{align*}
    \frac1{2\omega_p}\qty(\delta(p_0-\omega_p)+\delta(p_0+\omega_p))=\delta(p_0^2-\omega_p^2)=\delta(p_0-\vb{p}^2-m^2)=\delta(p^2-m^2)
\end{align*}
We then get:
\begin{align*}
    2\pi\int\frac{\dd[3]{p}}{(2\pi)^3}\frac{\dd{p_0}}{2\pi}e^{-ip_0t}e^{i\vb{p}\vdot\vb{x}}n_B(E_p)
    \delta(p^2-m^2)
    =2\pi \int\frac{\dd[4]{p}}{(2\pi)^4}e^{-ip_0t}e^{i\vb{p}\vdot\vb{x}}n_B(E_p)\delta(p^2-m^2)
\end{align*}
Which is the remaining term we need. The total term we then get, taking into account that this specific term is $iD_\beta$:
\begin{align}
    \boxed{D_\beta(x)=\int\frac{\dd[4]{p}}{(2\pi)^4}\qty[\frac1{p^2-m^2+i\epsilon}-2\pi i\delta(p^2-m^2)n_B(E_p)]e^{-ip\vdot x}}
\end{align}
\subsection{Verifying it is a Green Function}
We already know that the first term is a Green function, so we need only check that:
\begin{align*}
    (\Box+m^2)\qty(-2\pi i\int\frac{\dd[4]{p}}{(2\pi)^4}\delta(p^2-m^2)n_B(E_p)e^{-ip\vdot x})=0
\end{align*}
This is because the other term already includes the delta, so this should contribute nothing. 

The action of the Box operator will only bring a $-p^2$. The delta function will then do all of the work, replacing the $p^2$ with an $m^2$, contributing the opposite sign term, giving us $0$.

Therefore $D_\beta$ is in fact a Green function
\end{document}