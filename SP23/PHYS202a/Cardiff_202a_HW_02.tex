\documentclass[12pt]{article}

\title{\vspace{-3em}PHYS 202a HW 2}
\author{ATLAS All-Stars}
\date{\today}

%% science symbols 
\usepackage{amssymb,amsthm,bm,physics,slashed}

%% general pretty stuff
\usepackage{caption,enumitem,float,geometry,graphicx,tikz}

% setups
\graphicspath{ {./figs/} }
\captionsetup{labelfont=bf}
\geometry{margin=1in}

% macros
\renewcommand{\L}{\mathcal{L}}
\newcommand{\D}{\partial}
\newcommand{\circled}[1]{\tikz[baseline=(char.base)]{
    \node[shape=circle,draw,inner sep=2pt](char){#1};}}

\begin{document}
\maketitle
\section{Higher Order Euler-Lagrange}
We are given two different forms of the Euler-Lagrange Equations:
\begin{gather*}
  \D_\mu\qty(\pdv{\L}{(\D_\mu\phi)})-\pdv{\L}{\phi}=0\\
  \Box\qty(\pdv{\L}{(\Box\phi)})
  -\D_\mu\qty(\pdv{\L}{(\D_\mu\phi)})+\pdv{\L}{\phi}=0
\end{gather*}
With the first coming from eq. 3.30 and the second from eq. 3.29.

We then have two Lagrangian densities to evaluate with one of these, each with either first or second order terms, designated by the superscript:
\begin{align*}
  \L^{(1)}&=\frac12\qty(\D_\mu\phi\D^\mu\phi-m^2\phi^2)\\
  \L^{(2)}&=-\frac12\phi\qty(\Box+m^2)\phi=-\frac12\qty(\phi\Box\phi+m^2\phi^2)
\end{align*}
We will use the ELE with only the $\D_\mu$ term for the first order Lagrangian $\L^{(1)}$. The derivative with respect to $\D_\mu\phi$ must be handled carefully since the $\mu$ indices in the Lagrangian are contracted:
\begin{align*}
  \pdv{\L^{(1)}}{(\D_\mu\phi)}&=
  \frac12\pdv{(\D_\mu\phi)}\qty(\D_\nu\phi\D^\nu\phi)
  =\frac12\pdv{(\D_\mu\phi)}\qty(\eta_{\nu\lambda}\D_\nu\phi\D_\lambda\phi)\\
  &=\frac12\qty(\eta_{\nu\lambda}\D_\nu\phi\delta_{\mu\lambda}
  +\eta_{\nu\lambda}\delta_{\mu\nu}\D_\lambda\phi)
  =\frac12\qty(\eta_{\nu\mu}\D_\nu\phi
  +\eta_{\mu\lambda}\D_\lambda\phi)\\
  &=\frac12\qty(2\D^\mu\phi)=\D^\mu\phi
\end{align*}
The rest is more trivial derivatives
\begin{gather*}
  \pdv{\L^{(1)}}{(\D_\mu\phi)}=\D^\mu\phi\implies
  \D_\mu\qty(\pdv{\L^{(1)}}{(\D_\mu\phi)})=\D_\mu\D^\mu\phi=\Box\phi\\
  \pdv{\L^{(1)}}{\phi}=-m^2\phi
\end{gather*}
We then find the equations of motion are;
\begin{align}
  \D_\mu\qty(\pdv{\L^{(1)}}{(\D_\mu\phi)})-\pdv{\L^{(1)}}{\phi}=0
  \implies\boxed{\Box\phi+m^2\phi=0}
\end{align}
Now with $\L^{(2)}$ and the ELE with the $\Box$ term:
\begin{gather*}
  \pdv{\L^{(2)}}{(\Box\phi)}=-\frac12\phi\implies
  \Box\qty(\pdv{\L^{(2)}}{(\Box\phi)})=-\frac12\Box\phi\\
  \pdv{\L^{(2)}}{(\D_\mu\phi)}=0\implies
  \D_\mu\qty(\pdv{\L^{(2)}}{(\D_\mu\phi)})=0\\
  \pdv{\L^{(2)}}{\phi}=-m^2\phi-\frac12\Box\phi
\end{gather*}
Which has equations of motion:
\begin{align}
  \Box\qty(\pdv{\L^{(2)}}{(\Box\phi)})-\D_\mu\qty(\pdv{\L^{(2)}}{(\D_\mu\phi)})
  +\pdv{\L^{(2)}}{\phi}=0&\implies-\Box\phi-m^2\phi=0\nonumber\\
  &\implies\boxed{\Box\phi+m^2\phi=0}
\end{align}
Which means for all intents and purposes these Lagrangians are the same, they give identical equations of motion.
\section{Landau Theory}
The free energy for the Landau theory is given by:
\begin{align*}
  \mathcal{E}=\frac12\qty(\grad{\phi})^2+\frac12m^2\phi^2
  +\frac14\lambda_4\phi^4+\frac16\lambda_6\phi^6
\end{align*}
With the following conditions:
\begin{itemize}
\item The constant $\lambda_6>0$
\item The parameter $m^2$ has the form of $m^2=a(T-T_0)$ with $a$, $T$, and $T_0$ all positive. 
\item Assume $\lambda_4$ and $\lambda_6$ do not depend on $T$
\end{itemize}

\subsection{Configuration of Minimum Energy}
The form of the Free energy is such that the contribution by the gradient is always positive, so in a minimum configuration, the gradient term can only at minimum be $0$, which motivates us to say that $\grad{\phi}=0$, or:
\begin{align*}
  \phi_{min}(t,\vb{x})=\bar{\phi}
\end{align*}
Where $\bar{\phi}$ is a constant.

We can then minimize the following with respect to constant $\bar\phi$:
\begin{align*}
  m^2\phi+\lambda_4\phi^3+\lambda_6\phi^6&=0\\
  \bar\phi\qty(m^2+\lambda_4\bar\phi^2+\lambda_6\bar\phi^4)&=0\\
  \implies\left\{
  \begin{array}[H]{c}
    \bar\phi\\
    m^2+\lambda_4\bar\phi^2+\lambda_6\bar\phi^4  
  \end{array}\right\}
  &=0
\end{align*}
The first solution is trivial, however the other has the following solution in terms of $\bar\phi^2$:
\begin{align*}
  \bar\phi^2=\frac{-\lambda_4\pm\sqrt{\lambda_4^2-4m^2\lambda_6}}{2\lambda_6}
\end{align*}
The number of additional solutions is dependent on the discriminant:
\begin{align*}
  \Delta=\lambda_4^2-4m^2\lambda_6
\end{align*}
The value of the discriminant has meaning:
\begin{itemize}
\item If $\Delta>0$, there are two distinct real solutions
\item If $\Delta=0$, the two solutions are real, but degenerate
\item If $\Delta<0$, there are two complex solutions
\end{itemize}
Note that since $\phi$ is a real valued field, the complex solutions are not minima, so the only minimum would be $\bar\phi=0$. We can then use the temperature dependence of $m^2$ to solve for some $T_c>T_0$ below which there exists a minimum value of $\bar\phi$ that is not at $\bar\phi=0$, which is suggested by the fact that the discriminant is positive above this $T_c$:
\begin{align*}
  \Delta=0\implies4a(T_c-T_0)\lambda_6&=\lambda_4^2\\
  T_c-T_0&=\frac{\lambda_4^2}{4a\lambda_6}
\end{align*}
Therefore, in terms of $T_0$, $T_c$ is given by:
\begin{align}
  \boxed{T_c=T_0+\frac{\lambda_4^2}{4a\lambda_6}}
\end{align}
\section{A Theory with Higher Derivatives}


\end{document}