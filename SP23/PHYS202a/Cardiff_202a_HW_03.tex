\documentclass[12pt]{article}

\title{\vspace{-3em}PHYS 202a HW 3}
\author{Michael Cardiff}
\date{\today}

%% science symbols 
\usepackage{amssymb,amsthm,bm,physics,slashed}

%% general pretty stuff
\usepackage{caption,enumitem,float,geometry,graphicx,tikz,varwidth}

% setups
\graphicspath{ {./figs/} }
\captionsetup{labelfont=bf}
\geometry{margin=1in}

% macros
\renewcommand{\L}{\mathcal{L}}
\newcommand{\D}{\partial}
\newcommand{\circled}[1]{\tikz[baseline=(char.base)]{
    \node[shape=circle,draw,inner sep=2pt](char){#1};}}

\begin{document}
\maketitle

\section{Maxwell Energy Momentum Tensor}
The Lagrangian Density from the book with no source is given by:
\begin{align*}
  \L=-\frac14F_{\mu\nu}F^{\mu\nu}
\end{align*}
With $F_{\mu\nu}$ given by:
\begin{align*}
  F_{\mu\nu}=\D_\mu A_\nu-\D_\nu A_\mu
\end{align*}
So the contraction in the Lagrangian is given by:
\begin{align*}
  F_{\mu\nu}F^{\mu\nu}&=(\D_\mu A_\nu-\D_\nu A_\mu)(\D^\mu A^\nu-\D^\nu A^\mu)\\
  &=2(\D_\mu A_\nu\D^\mu A^\nu-\D_\mu A_\nu\D^\nu A^\mu)
\end{align*}
So the more useful Lagrangian for the purpose of doing derivatives is:
\begin{align*}
  \L=\frac12(\D_\mu A_\nu\D^\nu A^\mu-\D_\mu A_\nu\D^\mu A^\nu)
\end{align*}

\subsection{Form of $T_{\mu\nu}$}
The energy momentum tensor is given as:
\begin{align*}
  T^{\mu\nu}=\pdv{\L}{(\D_\mu A_\alpha)}\D^\nu A_\alpha-\eta^{\mu\nu}\L
\end{align*}
The derivative is given by:
\begin{align*}
  \pdv{\L}{(\D_\beta A_\alpha)}&=\frac12
  \pdv{(\D_\beta A_\alpha)}(\D_\mu A_\nu\D^\nu A^\mu-\D_\mu A_\nu\D^\mu A^\nu)\\
  &=\frac12\qty(2\D^\alpha A^\beta-2\D^\beta A^\alpha)\\
  &=\D^\alpha A^\beta-\D^\beta A^\alpha=F^{\alpha\beta}
\end{align*}
Hence what we have in the energy momentum tensor is:
\begin{align*}
  \pdv{\L}{(\D_\mu A_\alpha)}\D^\nu A_\alpha=F^{\alpha\mu}\D^\nu A_\alpha
\end{align*}
So the Energy Momentum tensor is:
\begin{align}
  \boxed{T^{\mu\nu}=F^{\alpha\mu}\D^\nu A_\alpha
    +\frac14\eta^{\mu\nu}F^{\alpha\beta}F_{\alpha\beta}}
\end{align}

\subsection{Conservation}
We can check it is conserved by finding the equations of motion for our Lagrangian $\L$:
\begin{align*}
  \D_\mu\qty(\pdv{\L}{(\D_\mu A_\nu)})-\pdv{\L}{A_\nu}=0
\end{align*}
Note we have already calculated the first derivative, and the second one is $0$:
\begin{align*}
  \D_\mu\qty(\pdv{\L}{(\D_\mu A_\nu)})=\D_\mu F^{\nu\mu}
\end{align*}
And then we have the equations of motion:
\begin{align*}
  \D_\mu F^{\mu\nu}=0
\end{align*}
Since the field strength tensor is antisymmetric in $\mu,\nu$. Apply the derivative to the energy momentum tensor. The derivative of the first term is zero due to the equations of motion above, the second term is $0$ since $\eta_{\mu\nu}$ is a constant with respect to position variables, since we have a flat spacetime. 
\begin{align}
  \boxed{\D_\mu T^{\mu\nu}=0}
\end{align}

\subsection{Bad Properties of $T_{\mu\nu}$}
\subsubsection{Non-Symmetric in Indices}
We can check that it is not symmetric by seeing:
\begin{align*}
  T^{\mu\nu}\neq T^{\nu\mu}
\end{align*}
From above we have:
\begin{align*}
  T^{\mu\nu}&=F^{\alpha\mu}\D^\nu A_\alpha
  +\frac14\eta^{\mu\nu}F^{\alpha\beta}F_{\alpha\beta}\\
  T^{\nu\mu}&=F^{\alpha\nu}\D^\mu A_\alpha
  +\frac14\eta^{\nu\mu}F^{\alpha\beta}F_{\alpha\beta}
\end{align*}
We should take the difference of these two to see if it is $0$, noting that the terms with the metric will cancel out since it is symmetric
\begin{align*}
  \Delta T\equiv T^{\mu\nu}-T^{\nu\mu}=
  (\D^\nu A^\alpha\D^\mu A_\alpha-\D^\alpha A^\nu\D^\mu A_\alpha)-
  (\D^\mu A^\alpha\D^\nu A_\alpha-\D^\alpha A^\mu\D^\nu A_\alpha)
\end{align*}
Note that the first two terms are equal by inserting a metric:
\begin{align*}
  \D^\mu A^\alpha\D^\nu A_\alpha=\D^\mu\eta^{\alpha\gamma}A_\gamma\D^\nu A_\alpha=
  \D^\mu A_\gamma\D^\nu\eta^{\alpha\gamma}A_\alpha=
  \D^\mu A_\gamma\D^\nu A^\gamma
\end{align*}
This is equal to the second term of the second parenthetical, since multiplication commutes. Therefore the difference is:
\begin{align*}
  \Delta T=\D^\alpha(A^\mu \D^\nu-A^\nu\D^\mu)A_\alpha
\end{align*}
Which is clearly not symmetrix in its indices, and therefore not 0, so $T^{\mu\nu}$ is NOT symmetric

\subsubsection{Gauge Dependence}
We can check the lack of Gauge Invariance by imposing a gauge transformation on $A_\mu$:
\begin{align*}
  A_\mu\to A'_\mu=A_\mu+\D_\mu\chi
\end{align*}
Which means the Field Strength Tensor transforms like:
\begin{align*}
  F^{\mu\nu}\to (F')^{\mu\nu}&=
  \D^\mu(A^\nu+\D^\nu\chi)-\D^\nu(A^\mu+\D^\mu\chi)\\
  &=F^{\mu\nu}+\D^\mu\D^\nu\chi-\D^\nu\D^\mu\chi
\end{align*}
However since mixed partial derivatives commute, this cancels out to something nice:
\begin{align*}
  F^{\mu\nu}\to (F')^{\mu\nu}&=F^{\mu\nu}
\end{align*}
This means the Energy Momentum tensor transforms like:
\begin{align*}
  T^{\mu\nu}\to (T')^{\mu\nu}&=F^{\mu\alpha}\D^\nu\qty(A_\alpha+\D_\alpha\chi)
  +\frac14\eta^{\mu\nu}F^{\alpha\beta}F_{\alpha\beta}\\
  &=T^{\mu\nu}+F^{\mu\alpha}\D^\nu\D_\alpha\chi
\end{align*}
Hence it is NOT gauge invariant.

\subsection{Modified Energy Momentum Tensor}
\subsubsection{Ensuring Index Symmetry}
First we should expand the derivative using the product rule in order to take advantage of the equations of motion:
\begin{align*}
  \D_\alpha(F^{\mu\alpha}A^\nu)=(\D_\alpha F^{\mu\alpha})A^\nu+
  F^{\mu\alpha}\D_\alpha A^\nu=F^{\mu\alpha}\D_\alpha A^\nu=-F^{\alpha\mu}
  \D_\alpha A^\nu
\end{align*}
If we add this to our energy momentum tensor we now have:
\begin{align*}
  \tilde{T}^{\mu\nu}=F^{\alpha\mu}(\D^\nu A_\alpha-\D_\alpha A^\nu)
  +\frac14\eta^{\mu\nu}F^{\alpha\beta}F_{\alpha\beta}
\end{align*}
We then calculate $\tilde{T}^{\mu\nu}$ and $\tilde{T}^{\nu\mu}$:
\begin{align*}
  \tilde{T}^{\mu\nu}&=
  \underbrace{\D_\alpha A^\nu\D^\mu A^\alpha}_1
  -\underbrace{\D^\mu A_\alpha\D^\nu A^\alpha}_2
  -\underbrace{\D_\alpha A^\mu\D^\alpha A^\nu}_3
  +\underbrace{\D^\mu A_\alpha\D^\alpha A^\nu}_4\\
  \tilde{T}^{\nu\mu}&=
  \underbrace{\D^\alpha A^\nu\D^\mu A_\alpha}_5
  -\underbrace{\D^\nu A^\alpha\D^\mu A_\alpha}_6
  -\underbrace{\D^\alpha A^\nu\D_\alpha A^\mu}_7
  +\underbrace{\D^\nu A^\alpha\D_\alpha A^\mu}_8
\end{align*}
Where we have swapped the upper and lower dummy indices by inserting a Minkowski metric.

We can see by commutativity of multiplication here that the following pairs will cancel when we subtract the two:
\begin{center}
  \begin{varwidth}{\textwidth}
    \begin{itemize}
    \item 1 and 8
    \item 2 and 6
    \item 3 and 7
    \item 4 and 5
    \end{itemize}
  \end{varwidth}
\end{center}
Therfore we can conclude that $\tilde{T}^{\mu\nu}$ is in fact symmetric:
\begin{align*}
  \boxed{\tilde{T}^{\mu\nu}=\tilde{T}^{\nu\mu}}
\end{align*}

\subsubsection{Verifying Gauge Invariance}

\subsection{Connecting to $E$ and $B$ Fields}

\subsection{Field Theoretic Form}

\section{Most General Lorentz-Invariant Langrangian}

\subsection{Field Redefinition}

\section{Proca Fields}

\subsection{Gauge Invariance}

\subsection{Classical Equation of Motion}

\section{Charge Lie Algebra}

\subsection{Noether Charges}

\section{Noether Charge for Boosts}

\end{document}