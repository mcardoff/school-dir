\documentclass[12pt]{article}

\title{\vspace{-3em}PHYS 202a HW 3}
\author{Michael Cardiff}
\date{\today}

%% science symbols 
\usepackage{amssymb,amsthm,bm,physics,slashed}

%% general pretty stuff
\usepackage{caption,enumitem,float,geometry,graphicx,tikz,varwidth}

% setups
\graphicspath{ {./figs/} }
\captionsetup{labelfont=bf}
\geometry{margin=1in}

% macros
\renewcommand{\L}{\mathcal{L}}
\newcommand{\D}{\partial}
\newcommand{\veps}{\varepsilon}
\newcommand{\circled}[1]{\tikz[baseline=(char.base)]{
    \node[shape=circle,draw,inner sep=2pt](char){#1};}}

\begin{document}
\maketitle

\section{Maxwell Energy Momentum Tensor}
The Lagrangian Density from the book with no source is given by:
\begin{align*}
  \L=-\frac14F_{\mu\nu}F^{\mu\nu}
\end{align*}
With $F_{\mu\nu}$ given by:
\begin{align*}
  F_{\mu\nu}=\D_\mu A_\nu-\D_\nu A_\mu
\end{align*}
So the contraction in the Lagrangian is given by:
\begin{align*}
  F_{\mu\nu}F^{\mu\nu}&=(\D_\mu A_\nu-\D_\nu A_\mu)(\D^\mu A^\nu-\D^\nu A^\mu)\\
  &=2(\D_\mu A_\nu\D^\mu A^\nu-\D_\mu A_\nu\D^\nu A^\mu)
\end{align*}
So the more useful Lagrangian for the purpose of doing derivatives is:
\begin{align*}
  \L=\frac12(\D_\mu A_\nu\D^\nu A^\mu-\D_\mu A_\nu\D^\mu A^\nu)
\end{align*}

\subsection{Form of $T_{\mu\nu}$}
The energy momentum tensor is given as:
\begin{align*}
  T^{\mu\nu}=\pdv{\L}{(\D_\mu A_\alpha)}\D^\nu A_\alpha-\eta^{\mu\nu}\L
\end{align*}
The derivative is given by:
\begin{align*}
  \pdv{\L}{(\D_\beta A_\alpha)}&=\frac12
  \pdv{(\D_\beta A_\alpha)}(\D_\mu A_\nu\D^\nu A^\mu-\D_\mu A_\nu\D^\mu A^\nu)\\
  &=\frac12\qty(2\D^\alpha A^\beta-2\D^\beta A^\alpha)\\
  &=\D^\alpha A^\beta-\D^\beta A^\alpha=F^{\alpha\beta}
\end{align*}
Hence what we have in the energy momentum tensor is:
\begin{align*}
  \pdv{\L}{(\D_\mu A_\alpha)}\D^\nu A_\alpha=F^{\alpha\mu}\D^\nu A_\alpha
\end{align*}
So the Energy Momentum tensor is:
\begin{align}
  \boxed{T^{\mu\nu}=F^{\alpha\mu}\D^\nu A_\alpha
    +\frac14\eta^{\mu\nu}F^{\alpha\beta}F_{\alpha\beta}}
\end{align}

\subsection{Conservation}
We can check it is conserved by finding the equations of motion for our Lagrangian $\L$:
\begin{align*}
  \D_\mu\qty(\pdv{\L}{(\D_\mu A_\nu)})-\pdv{\L}{A_\nu}=0
\end{align*}
Note we have already calculated the first derivative, and the second one is $0$:
\begin{align*}
  \D_\mu\qty(\pdv{\L}{(\D_\mu A_\nu)})=\D_\mu F^{\nu\mu}
\end{align*}
And then we have the equations of motion:
\begin{align*}
  \D_\mu F^{\mu\nu}=0
\end{align*}
Since the field strength tensor is antisymmetric in $\mu,\nu$. Apply the derivative to the energy momentum tensor. The derivative of the first term is zero due to the equations of motion above, the second term is $0$ since $\eta_{\mu\nu}$ is a constant with respect to position variables, since we have a flat spacetime. 
\begin{align}
  \boxed{\D_\mu T^{\mu\nu}=0}
\end{align}

\subsection{Bad Properties of $T_{\mu\nu}$}
\subsubsection{Non-Symmetric in Indices}
We can check that it is not symmetric by seeing:
\begin{align*}
  T^{\mu\nu}\neq T^{\nu\mu}
\end{align*}
From above we have:
\begin{align*}
  T^{\mu\nu}&=F^{\alpha\mu}\D^\nu A_\alpha
  +\frac14\eta^{\mu\nu}F^{\alpha\beta}F_{\alpha\beta}\\
  T^{\nu\mu}&=F^{\alpha\nu}\D^\mu A_\alpha
  +\frac14\eta^{\nu\mu}F^{\alpha\beta}F_{\alpha\beta}
\end{align*}
We should take the difference of these two to see if it is $0$, noting that the terms with the metric will cancel out since it is symmetric
\begin{align*}
  \Delta T\equiv T^{\mu\nu}-T^{\nu\mu}=
  (\D^\nu A^\alpha\D^\mu A_\alpha-\D^\alpha A^\nu\D^\mu A_\alpha)-
  (\D^\mu A^\alpha\D^\nu A_\alpha-\D^\alpha A^\mu\D^\nu A_\alpha)
\end{align*}
Note that the first two terms are equal by inserting a metric:
\begin{align*}
  \D^\mu A^\alpha\D^\nu A_\alpha=\D^\mu\eta^{\alpha\gamma}A_\gamma\D^\nu A_\alpha=
  \D^\mu A_\gamma\D^\nu\eta^{\alpha\gamma}A_\alpha=
  \D^\mu A_\gamma\D^\nu A^\gamma
\end{align*}
This is equal to the second term of the second parenthetical, since multiplication commutes. Therefore the difference is:
\begin{align*}
  \Delta T=\D^\alpha(A^\mu \D^\nu-A^\nu\D^\mu)A_\alpha
\end{align*}
Which is clearly not symmetrix in its indices, and therefore not 0, so $T^{\mu\nu}$ is NOT symmetric

\subsubsection{Gauge Dependence}
We can check the lack of Gauge Invariance by imposing a gauge transformation on $A_\mu$:
\begin{align*}
  A_\mu\to A'_\mu=A_\mu+\D_\mu\chi
\end{align*}
Which means the Field Strength Tensor transforms like:
\begin{align*}
  F^{\mu\nu}\to (F')^{\mu\nu}&=
  \D^\mu(A^\nu+\D^\nu\chi)-\D^\nu(A^\mu+\D^\mu\chi)\\
  &=F^{\mu\nu}+\D^\mu\D^\nu\chi-\D^\nu\D^\mu\chi
\end{align*}
However since mixed partial derivatives commute, this cancels out to something nice:
\begin{align*}
  F^{\mu\nu}\to (F')^{\mu\nu}&=F^{\mu\nu}
\end{align*}
This means the Energy Momentum tensor transforms like:
\begin{align*}
  T^{\mu\nu}\to (T')^{\mu\nu}&=F^{\mu\alpha}\D^\nu\qty(A_\alpha+\D_\alpha\chi)
  +\frac14\eta^{\mu\nu}F^{\alpha\beta}F_{\alpha\beta}\\
  &=T^{\mu\nu}+F^{\mu\alpha}\D^\nu\D_\alpha\chi
\end{align*}
Hence it is NOT gauge invariant.

\subsection{Modified Energy Momentum Tensor}
First we should expand the derivative using the product rule in order to take advantage of the equations of motion:
\begin{align*}
  \D_\alpha(F^{\mu\alpha}A^\nu)=(\D_\alpha F^{\mu\alpha})A^\nu+
  F^{\mu\alpha}\D_\alpha A^\nu=F^{\mu\alpha}\D_\alpha A^\nu=-F^{\alpha\mu}
  \D_\alpha A^\nu
\end{align*}
If we add this to our energy momentum tensor we now have:
\begin{align*}
  \tilde{T}^{\mu\nu}&=F^{\alpha\mu}(\D^\nu A_\alpha-\D_\alpha A^\nu)
  +\frac14\eta^{\mu\nu}F^{\alpha\beta}F_{\alpha\beta}\\
  &=F^{\alpha\mu}F^\nu_\alpha+\frac14\eta^{\mu\nu}F^{\alpha\beta}F_{\alpha\beta}
\end{align*}

\subsubsection{Ensuring Index Symmetry}
To ensure symmetry we then calculate $\tilde{T}^{\mu\nu}$ and $\tilde{T}^{\nu\mu}$:
\begin{align*}
  \tilde{T}^{\mu\nu}&=
  \underbrace{\D_\alpha A^\mu\D^\nu A^\alpha}_1
  -\underbrace{\D^\mu A_\alpha\D^\nu A^\alpha}_2
  -\underbrace{\D_\alpha A^\mu\D^\alpha A^\nu}_3
  +\underbrace{\D^\mu A_\alpha\D^\alpha A^\nu}_4\\
  \tilde{T}^{\nu\mu}&=
  \underbrace{\D^\alpha A^\nu\D^\mu A_\alpha}_5
  -\underbrace{\D^\nu A^\alpha\D^\mu A_\alpha}_6
  -\underbrace{\D^\alpha A^\nu\D_\alpha A^\mu}_7
  +\underbrace{\D^\nu A^\alpha\D_\alpha A^\mu}_8
\end{align*}
Where we have swapped the upper and lower dummy indices by inserting a Minkowski metric.

We can see by commutativity of multiplication here that the following pairs will cancel when we subtract the two:
\begin{center}
  \begin{varwidth}{\textwidth}
    \begin{itemize}
    \item 1 and 8
    \item 2 and 6
    \item 3 and 7
    \item 4 and 5
    \end{itemize}
  \end{varwidth}
\end{center}
Therfore we can conclude that $\tilde{T}^{\mu\nu}$ is in fact symmetric:
\begin{align*}
  \boxed{\tilde{T}^{\mu\nu}=\tilde{T}^{\nu\mu}}
\end{align*}

\subsubsection{Verifying Gauge Invariance}
We have already proven that $F^{\mu\nu}$ on its own is Gauge invariant, and we can see from simple index switching that $F^\mu_\nu$ is as well:
\begin{align*}
  F^\mu_\nu=\eta_{\nu\alpha}F^{\mu\alpha}
\end{align*}
And since the metric and $F^{\mu\nu}$ are already gauge invariant, we explicitly have $\tilde{T}^{\mu\nu}$ as gauge invariant

\subsection{Connecting to E and B Fields}
Recall the classical formulae for the $E$ and $B$ fields in terms of the scalar and vector potentials $\phi$ and $\vb{A}$:
\begin{align*}
  E_i&=-\grad_i\phi-\pdv{A_i}{t}\\
  B_i&=\veps_{ijk}\grad_jA_k
\end{align*}
We can write this in the usual relativistic notation:
\begin{align*}
  E_i&=-\D_iA^0-\D_0A^i\\
  &=\D^iA^0-\D^0A^i=F^{0i}\\
  B_i&=\veps_{ijk}\D_jA_k
\end{align*}
Since we found $\vb{E}$ in terms of $F$, it makes sense to write out the components of $F^{ij}$ in order to see if they relate to $\vb{B}$:
\begin{align*}
  F^{ij}&=\D^iA^j-\D^jA^i=\veps_{klm}\veps^{kij}\D^lA^m\\
  &=-\veps_{kij}(\curl{\vb{A}})^k=-\veps_{ijk}B^k
\end{align*}
Where in the first line we used the identity:
\begin{align*}
  \veps_{klm}\veps^{kij}=(\delta^i_l\delta^j_m-\delta^j_l\delta^i_m)
\end{align*}
We can then write the field strength tensor in matrix form in terms of the field, noting that the diagonal elements of an antisymmetric matrix is $0$:
\begin{align*}
  [F^{\mu\nu}]&=
  \pmqty{
    0  & -E^1& -E^2& -E^3\\
    E^1&  0  & -B^3&  B^2\\
    E^2&  B^3&  0  & -B^1\\
    E^3& -B^2&  B^1&  0
  }\\
  [F^{\mu\nu}]&=
  \pmqty{
     0  &  E^1&  E^2&  E^3\\
    -E^1&  0  & -B^3&  B^2\\
    -E^2&  B^3&  0  & -B^1\\
    -E^3& -B^2&  B^1&  0
  }
\end{align*}
We can then identify the Lagrangian term as:
\begin{align*}
  F^{\alpha\beta}F_{\alpha\beta}=-F^{\alpha\beta}F_{\beta\alpha}
\end{align*}
In this form we identify this as a trace over the product of the two matrices:
\begin{align*}
  -F^{\alpha\beta}F_{\beta\alpha}=2\qty(\vb{B}^2-\vb{E}^2)
\end{align*}
Then the Lagrangian is:
\begin{align*}
  \L=\frac12\qty(\vb{B}^2-\vb{E}^2)
\end{align*}
The other term in our stress-energy tensor can be identified as:
\begin{align*}
  F^{\alpha\mu}F^\nu_\alpha=-F^{\mu\alpha}F^\nu_\alpha
\end{align*}
Since we only need $T^{00}$, we can calculate the following dot product:
\begin{align*}
  -F^{0\alpha}F^0_\alpha=\vb{E}^2
\end{align*}
We calculated this in Mathematica, meaning the $00$ term of the Energy Momentum Tensor is:
\begin{align*}
  T^{00}=\vb{E}^2-\eta^{00}\L=\vb{E}^2-\frac12\qty(\vb{B}^2-\vb{E}^2)=
  \frac12\qty(\vb{E}^2+\vb{B}^2)
\end{align*}
Which is exactly the classical eletromagnetic Hamiltonian Density, the first equation in 3.81. So the Hamiltonian is:
\begin{align*}
  \boxed{H=\int\dd[3]{x}T^{00}=\int\dd[3]{x}\frac12\qty(\vb{E}^2+\vb{B}^2)}
\end{align*}
Which is the first equation in 3.81

\subsection{Field Theoretic Form}
The mode expansion we need to use is:
\begin{align*}
  A_\mu=\sum_\lambda\int\frac{\dd[3]{p}}{(2\pi)^3}\frac1{\sqrt{2\omega_p}}
  \qty[
  \hat{a}_{\vb{p},\lambda}\veps_\mu(\vu{p},\lambda)e^{-ip\vdot x}+
  \hat{a}^\dag_{\vb{p},\lambda}\veps^*_\mu(\vu{p},\lambda)e^{ip\vdot x}]
\end{align*}
The polarization vectors used are:
\begin{align*}
  \veps_\mu(p,\pm1)=\frac1{\sqrt{2}}\qty(0,1,\pm i,0)
\end{align*}
Which has the following properties:
\begin{gather*}
  \veps_\mu\veps^\mu=\veps^*_\mu\veps^{*\mu}=0\\
  \veps^*_\mu(\vu{p},\lambda)\veps^\mu(\vu{p},\lambda')
  =\veps_\mu(\vu{p},\lambda)\veps^{*\mu}(\vu{p},\lambda')=
  \delta_{\lambda,\lambda'}\\
  p^\mu\veps_\mu=0
\end{gather*}

Note the Explicit form of $T^{00}$:
\begin{align*}
  \tilde{T}^{00}&=
  F^{\alpha0}F^0_\alpha+\frac14\eta^{00}F^{\alpha\beta}F_{\alpha\beta}\\
  &=F^{\alpha0}F^0_\alpha+\frac14F^{\alpha\beta}F_{\alpha\beta}\\
  &=(\D^\alpha A^0-\D^0A^\alpha)(\D^0A_\alpha-\D_\alpha A^0)
  +\frac14
  \qty(\D^\alpha A^\beta-\D^\beta A^\alpha)
  \qty(\D_\alpha A_\beta-\D_\beta A_\alpha)\\
  &=(\D^\alpha A^0-\D^0A^\alpha)(\D^0A_\alpha-\D_\alpha A^0)
  +\frac12\qty(\D_\alpha A_\beta\D^\alpha A^\beta
  -\D_\alpha A_\beta\D^\beta A^\alpha)
\end{align*}
Note that since we are in the Radiation gauge, we have $A^0=0$, so the only surviving term from the contraction $F^{\alpha0}F_{\alpha}^0 is$:
\begin{align*}
  F^{\alpha0}F^0_\alpha&=-\D^0A^\alpha\D^0A_\alpha\\
  &=-\D^0A^0\D^0A_0+\D^0A^i\D^0A_i\\
  &=\D^0A^i\D^0A_i
\end{align*}
We can then expand out the Lagrangian term in space and time components in order to find a similar result, if we cancel terms with $A^0$ as they come along:
\begin{align*}
  \D_\alpha A_\beta\D^\alpha A^\beta-\D_\alpha A_\beta\D^\beta A^\alpha&=
  \D^0A^\beta\D_0A_\beta-\D^iA^\beta\D_iA_\beta
  -\D^0A^\beta\D_\beta A_0+\D^iA^\beta\D_\beta A_i\\
  &=\D^0A^0\D_0A_0-\D^0A^i\D_0A_i-\D^iA^0\D_iA_0\\
  &+\D^iA^j\D_iA_j-\D^iA^0\D_0A_i-\D^iA^j\D_jA_i\\
  &=-\D^0A^i\D_0A_i+\D^iA^j\D_iA_j-\D^iA^j\D_jA_i
\end{align*}
This combines quite well with our other surviving term:
\begin{align*}
  \tilde{T}^{00}&=\D^0A^i\D^0A_i
  +\frac12\qty(-\D^0A^i\D_0A_i+\D^iA^j\D_iA_j-\D^iA^j\D_jA_i)\\
  &=\frac12\D^0A^i\D_0A_i+\frac12\D^iA^j(\D_iA_j-\D_jA_i)
\end{align*}
We should then calculate these derivatives:
\begin{align*}
  \D^0A^i&=\sum_\lambda\int\frac{\dd[3]{p}}{(2\pi)^3}
  \frac1{\sqrt{2\omega_p}}
  \qty[\hat{a}_{\vb{p},\lambda}\veps^i(\vu{p},\lambda)(\D^0e^{-ip\vdot x})+
  \hat{a}^\dag_{\vb{p},\lambda}\veps^{*i}(\vu{p},\lambda)(\D^0e^{ip\vdot x})]\\
  &=\boxed{\sum_\lambda\int\frac{\dd[3]{p}}{(2\pi)^3}
  \frac{-i\omega_p}{\sqrt{2\omega_p}}
  \qty[\hat{a}_{\vb{p},\lambda}\veps^i(\vu{p},\lambda)e^{-ip\vdot x}-
  \hat{a}^\dag_{\vb{p},\lambda}\veps^{*i}(\vu{p},\lambda)e^{ip\vdot x}]}\\
  \D^iA^j&=\sum_\lambda\int\frac{\dd[3]{p}}{(2\pi)^3}
  \frac1{\sqrt{2\omega_p}}
  \qty[\hat{a}_{\vb{p},\lambda}\veps^j(\vu{p},\lambda)(\D^ie^{-ip\vdot x})+
  \hat{a}^\dag_{\vb{p},\lambda}\veps^{*j}(\vu{p},\lambda)(\D^ie^{ip\vdot x})]\\
  &=\boxed{\sum_\lambda\int\frac{\dd[3]{p}}{(2\pi)^3}
  \frac{-ip^i}{\sqrt{2\omega_p}}
  \qty[\hat{a}_{\vb{p},\lambda}\veps^j(\vu{p},\lambda)e^{-ip\vdot x}-
  \hat{a}^\dag_{\vb{p},\lambda}\veps^{*j}(\vu{p},\lambda)e^{ip\vdot x}]}
\end{align*}
First we should perform the product of the two $\D^0$s:
\begin{align*}
  \D^0A^i\D_0A_i=\sum_{\lambda',\lambda}\int\frac{\dd[3]{p}\dd[3]{p}'}{(2\pi)^6}
  \frac{-\omega_p\omega_{p'}}{2\sqrt{\omega_p\omega_{p'}}}
  &\qty[\hat{a}_{\vb{p},\lambda}\veps^i(\vu{p},\lambda)e^{-ip\vdot x}-
  \hat{a}^\dag_{\vb{p},\lambda}\veps^{*i}(\vu{p},\lambda)e^{ip\vdot x}]\\
  \times&\qty[\hat{a}_{\vb{p'},\lambda'}\veps_i(\vu{p}',\lambda')e^{-ip'\vdot x}-
  \hat{a}^\dag_{\vb{p'},\lambda'}\veps^*_i(\vu{p}',\lambda')e^{ip'\vdot x}]
\end{align*}
By expanding these products, We will find a few delta functions in 

\section{Most General Lorentz-Invariant Langrangian}

\subsection{Field Redefinition}

\section{Proca Fields}

\subsection{Gauge Invariance}

\subsection{Classical Equation of Motion}

\section{Charge Lie Algebra}

\subsection{Noether Charges}

\section{Noether Charge for Boosts}

\end{document}