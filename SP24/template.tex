%% Template for the semester

%% science symbols
\usepackage{amsmath}
\usepackage{amssymb}
\usepackage{amsthm}
\usepackage{bm}
\usepackage{cancel}
\usepackage{physics}
\usepackage{siunitx}
\usepackage{slashed}

%% general pretty stuff
\usepackage{float}
\usepackage{caption}
\usepackage{graphicx}
\usepackage{url}
\usepackage{enumitem}
\usepackage{hyperref}
\usepackage{tikz}
\usepackage{tikz-feynhand}

% setup options
\captionsetup{labelfont=bf}
\graphicspath{ {./figs/} }

% macros
\renewcommand{\L}{\mathcal{L}}
\renewcommand{\H}{\mathcal{H}}
\renewcommand{\l}{\ell}
\newcommand{\M}{\mathcal{M}}
\newcommand{\mcV}{\mathcal{V}}
\newcommand{\D}{\partial}
\newcommand{\veps}{\varepsilon}
\newcommand{\circled}[1]{\tikz[baseline=(char.base)]{
    \node[shape=circle,draw,inner sep=2pt](char){#1};}}

% mdframed environments
\usepackage[framemethod=TikZ]{mdframed}
\mdfsetup{skipabove=\topskip,skipbelow=\topskip}
\mdfdefinestyle{defstyle}{%
  linewidth=1pt,
  frametitlerule=true,
  frametitlebackgroundcolor=gray!40,
  backgroundcolor=gray!20,
  innertopmargin=\topskip
}

\mdtheorem[style=defstyle]{definition}{Definition}
\mdtheorem[style=defstyle]{theorem}{Theorem}
\mdtheorem[style=defstyle]{problem}{Problem}

\newenvironment{thebook}
{\begin{mdframed}[style=defstyle,frametitle={From the Book}]}{\end{mdframed}}
