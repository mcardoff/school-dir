\documentclass[12pt]{article}

\title{\vspace{-3em}PHYS 167b Exam 1}
\author{Michael Cardiff}
\date{\today}

%% science symbols
\usepackage{amsmath}
\usepackage{amssymb}
\usepackage{amsthm}
\usepackage{bm}
\usepackage{cancel}
\usepackage{physics}
\usepackage{siunitx}
\usepackage{slashed}

%% general pretty stuff
\usepackage{caption}
\usepackage{float}
\usepackage{graphicx}
\usepackage{url}
\usepackage{enumitem}
\usepackage{hyperref}
\usepackage{tikz}
\usepackage{tikz-feynhand}
\usepackage[margin=1in]{geometry}

% setup options
\captionsetup{labelfont=bf}
\graphicspath{ {./figs/} }

% macros
\renewcommand{\L}{\mathcal{L}}
\renewcommand{\H}{\mathcal{H}}
\renewcommand{\l}{\ell}
\newcommand{\M}{\mathcal{M}}
\newcommand{\mcV}{\mathcal{V}}
\newcommand{\D}{\partial}
\newcommand{\veps}{\varepsilon}
\newcommand{\circled}[1]{\tikz[baseline=(char.base)]{
    \node[shape=circle,draw,inner sep=2pt](char){#1};}}

% mdframed environments
\usepackage[framemethod=TikZ]{mdframed}
\mdfsetup{skipabove=\topskip,skipbelow=\topskip}
\mdfdefinestyle{defstyle}{%
  linewidth=1pt,
  frametitlerule=true,
  frametitlebackgroundcolor=gray!40,
  backgroundcolor=gray!20,
  innertopmargin=\topskip
}

\mdtheorem[style=defstyle]{definition}{Definition}
\mdtheorem[style=defstyle]{theorem}{Theorem}
\mdtheorem[style=defstyle]{problem}{Problem}

\newenvironment{thebook}
{\begin{mdframed}[style=defstyle,frametitle={From the Book}]}{\end{mdframed}}

\usepackage{fancyhdr}

\usepackage{siunitx}
\usepackage{textcomp}
\pagestyle{fancy}
\lhead{Michael Cardiff}
\rhead{PHYS 167b Exam 1}

\begin{document}
\maketitle
\section*{Task 1: Absorption Length and Cross Section}
\begin{problem}
  Consider a beam of intensity $I_0$ entering a long target. Let $z$ be the distance travelled by the beam in the target, measured from the entrance point. The interaction rate $R_i$is given by $R_i=\sigma nV\Phi$, where $\sigma$ is the cross section, $N=nV$ is the number of target particles encountered and $\Phi$ is the intensity per unit area.
  \begin{enumerate}
  \item What is the beam intensity $I(z)$ as a function of distance $z$?
  \item The ``absorption length'' $L$ is defined as the distance at which the beam intensity reduces by the factor $1/e$. Find L.
  \item The absorption length in lead is \SI{5}{cm}. Find the total neutron cross section of lead (Atomic mass number of about 200 and density of about \SI{10}{g.cm^{-3}}).
  \end{enumerate}
\end{problem}
Beam intensity is  related to the number of particles in the beam, so it will change based on number of interactions with the material $N$, so it is proportional to the interaction rate:
\begin{align*}
  \dd{I}=-\dd{R_i}
\end{align*}
We were given a form of $R_i$, so we can find this differential, assuming $\sigma$, $\phi$, and $n$ are constant:
\begin{align*}
  \dd{R_i}&=\dd{(\sigma N \phi)}=\sigma\phi\dd{N}\\
  \dd{N}&=n\dd{V}=nA\dd{z}
\end{align*}
The units for $\phi$ are intensity over area, so it can assume it has the form:
\begin{align*}
  \phi=\frac{I(z)}{A}
\end{align*}
Where $I(0)=I_0$. We can then form a differential equation for $I(z)$:
\begin{align*}
  \dd{I}=-\sigma\frac{I(z)}{A}nA\dd{z}=-\sigma nI(z)\dd{z}
\end{align*}
Where the solution is:
\begin{equation}
  \label{eq:p1a}
  \boxed{I(z)=I_0\exp{-n\sigma z}}
\end{equation}
We need to find $L$ such that $I(L)=I_0/e$:
\begin{align*}
  \frac{I_0}{e}=I_0e^{-n\sigma L}\implies e^{-1}=e^{-n\sigma L}
\end{align*}
Taking the log of both sides gives:
\begin{equation}
  \label{eq:p1b}
  \boxed{L=\frac{1}{n\sigma}}
\end{equation}
The atomic mass \# is 200, and $Z$ for lead is 82, so \# of neutrons is $N=200-82=118$:
\begin{align*}
  n=\frac{\text{neutron}}{V}
\end{align*}
Performing the dimensional analysis:
\begin{align*}
  n=\frac{\SI{10}{g}}{\SI{1}{cm^3}}\cdot
  \frac{\SI{1}{mol}}{\SI{207}{g}}\cdot
  \frac{N_A\text{ Pb atom}}{\SI{1}{mol}}\cdot
  \frac{118\text{ neutron}}{\text{Pb atom}}
\end{align*}
So the cross section is:
\begin{equation}
  \label{eq:p1c}
  \boxed{\sigma=\frac1{nL}=\SI{5.8e-26}{cm^{-2}}\approx0.06\text{ barn}}
\end{equation}
\newpage
\section*{Task 2: Isospin}
\begin{problem}
  \begin{enumerate}
  \item Compute the Clebsch-Gordan coefficients for the states with $\vb{J}=\vb{J}_1+\vb{J}_2$, $M=m_1+m_2$, where $J_1=1$, $J_2=1/2$, $J=3/2$ and $M=1/2$, for the various possible $m_1$ and $m_2$ values
  \item Consider the following strong interaction reactions (1) $\pi^+p\to\pi^+p$, (2) $\pi^-p\to\pi^-p$, (3) $\pi^-p\to\pi^0n$. Using the Clebsch0Gordan tables, write the isospin wavefunction of each state in terms of total isospin $I$ and $I_z$ basis.
  \item These isospin conserving reactions can occur in the isospin $I=3/2$ state ($\Delta$ resonance), or $I=1/2$ state ($N^*$ resonance). Using the result above calculate the ratio of these cross-sections ($\sigma_1:\sigma_2:\sigma_3$) for an exchange corresponding to a (1) $\Delta$ resonance, and toi (2) $N^*$ resonance. At a resonance energy you can neglect the effect due to the other isospin state. The cross sections are proportional to the isospin amplitudes and are independent of $I_z$
  \end{enumerate}
\end{problem}
Note the action of the ladder operators on a $\ket{j,m}$ state:
\begin{align*}
  J_+\ket{j,m}&=\sqrt{j(j+1)-m(m+1)}\ket{j,m+1}\\
  J_-\ket{j,m}&=\sqrt{j(j+1)-m(m-1)}\ket{j,m-1}
\end{align*}
To get $J=\frac32$, $M=\frac12$, we need to apply $J_-$ to the $J=\frac32$, $M=\frac32$ state, whhich only has one possibility from $1\oplus\frac12$:
\begin{align*}
  \ket{\tfrac32,\tfrac32}=\ket{1,1,\tfrac12,\tfrac12}
  \equiv\ket{1,1}\ket{\tfrac12,\tfrac12}
\end{align*}
Apply total $J_-$:
\begin{align*}
  J_-\ket{\tfrac32,\tfrac32}=\sqrt{\frac32\qty(\frac52)-\frac32\qty(\frac12)}
  \ket{\tfrac32,\tfrac12}=\sqrt{3}\ket{\tfrac32,\tfrac12}
\end{align*}
Apply the fact that $J_-=J_{1-}+J_{2-}$:
\begin{align*}
  J_-\ket{\tfrac32,\tfrac32}&=
  J_{1-}\ket{1,1,\tfrac12,\tfrac12}+J_{2-}\ket{1,1,\tfrac12,\tfrac12}\\
  &=\sqrt{1(2)-1(0)}\ket{1,0,\tfrac12,\tfrac12}+
  \sqrt{\tfrac12(\tfrac32)-\tfrac12(-\tfrac12)}\ket{1,1,\tfrac12,-\tfrac12}\\
  &=\sqrt{2}\ket{1,0,\tfrac12,\tfrac12}+\ket{1,1,\tfrac12,-\tfrac12}
\end{align*}
We then have:
\begin{align*}
  \sqrt{3}\ket{\tfrac32,\tfrac12}
  =\sqrt{2}\ket{1,0,\tfrac12,\tfrac12}+\ket{1,1,\tfrac12,-\tfrac12}
\end{align*}
Which gives our final answer, which matches the Clebsch-Gordan table:
\begin{equation}
  \label{eq:p2a}
  \boxed{\ket{\tfrac32,\tfrac12}
    =\sqrt{\frac23}\ket{1,0,\tfrac12,\tfrac12}
    +\sqrt{\frac13}\ket{1,1,\tfrac12,-\tfrac12}}
\end{equation}
The isospin of $\pi=1$, and the $p,n$ doublet is $\tfrac12$, so the corresponding states are:
\begin{gather*}
  \pi^+=\ket{1,1}\quad\pi^0=\ket{1,0}\quad\pi^-=\ket{1,-1}\\
  p=\ket{\tfrac12,\tfrac12}\quad n=\ket{\tfrac12,-\tfrac12}
\end{gather*}
So the states we see can be  found from the CG table as:

\begin{equation}
  \label{eq:p2b}
  \begin{aligned}
    \pi^+p&=\ket{1,1}\oplus\ket{\tfrac12,\tfrac12}
    =\boxed{\ket{\tfrac32,\tfrac32}}\\
    \pi^-p&=\ket{1,-1}\oplus\ket{\tfrac12,\tfrac12}
    =\boxed{
      \sqrt{\frac13}\ket{\tfrac32,-\tfrac12}-
      \sqrt{\frac23}\ket{\tfrac12,-\tfrac12}
    }\\
    \pi^0n&=\ket{1,0}\oplus\ket{\tfrac12,-\tfrac12}
    =\boxed{
      \sqrt{\frac23}\ket{\tfrac32,-\tfrac12}-
      \sqrt{\frac13}\ket{\tfrac12,-\tfrac12}
      }
  \end{aligned}
\end{equation}
Amplitudes will correspond to overlap of initial/final state:
\begin{align*}
  \pi^+p\to\pi^+p\equiv\ip{\pi^+p}&=1\ip{\tfrac32,\tfrac32}
  \equiv \sqrt{A_{3/2}}\\
  \pi^-p\to\pi^-p\equiv\ip{\pi^-p}&=
  \frac13\sqrt{A_{3/2}}+\frac23\sqrt{A_{1/2}}\\
  \pi^-p\to\pi^0n\equiv\ip{\pi^-p}{\pi^0n}&=
  \frac{\sqrt{2}}3\sqrt{A_{3/2}}-\frac{\sqrt{2}}3\sqrt{A_{1/2}}
\end{align*}
At $\Delta$ resonance, only count $A_{3/2}$:
\begin{align*}
  \sigma_1\sim 1,\quad \sigma_2\sim \frac19,\quad \sigma_3\sim\frac29
\end{align*}
So the ratios are:
\begin{equation}
  \label{eq:p2c1}
  \boxed{\sigma_1:\sigma_2:\sigma_3=9:1:2}
\end{equation}
And at $N^*$ resonance, only count $A_{1/2}$:
\begin{align*}
  \sigma_1\sim 0,\quad \sigma_2\sim \frac49,\quad \sigma_3\sim\frac29
\end{align*}
Giving the ratios:
\begin{equation}
  \label{eq:p2c2}
  \boxed{\sigma_1:\sigma_2:\sigma_3=0:2:1}
\end{equation}
\newpage
\section*{Task 3: Muon Decays}
\begin{problem}
  A pion at rest decays into a muon and a neutrino $(\pi^-\to\mu^-+\overline{\nu}_\mu)$
  \begin{enumerate}
  \item On the average, how far will the muon travel in vacuum before decaying? Give a numerical answer. 
  \item How much energy would a muon need to circumnavigate the earth with a fair chance of completing the journey, assuming that the earth's magnetic field is strong enough to keep it in orbit? Is the earth's field actually strong enough?
  \end{enumerate}
\end{problem}
The muon proper lifetime is $\sim2.2$ \textmu{s}, we can take the masses to be:
\begin{align*}
  m_\nu\approx0\quad
  m_\pi\approx\SI{135.0}{MeV}\quad
  m_\mu\approx\SI{105.7}{MeV}
\end{align*}
Conservation of energy for the system gives:
\begin{align*}
  E_\pi&=E_\mu+E_\nu\\
  m_\pi&=\sqrt{m_\mu^2+\abs{\vb{p}_\mu}^2}+\abs{\vb{p}_\nu}
\end{align*}
Where $E_\pi=m_\pi$ since we are in the $\pi$ rest frame, and since the neutrinos are kinematically massless its energy and momentum are the same. We
\newpage
\section*{Task 4: Spin-0 Photon}
\begin{problem}
  Let's imagine a parallel universe where the photon instead of being a massless vector (spin-1) particle, is a massive scalar (spin-0) particle. The QED vertex in the Feynman rules in this theory would be $-ig_e\bm{1}$, where $\bm{1}$ is the $4\times4$ unit matrix (to be compared to $-ig_e\gamma^\mu$ for the massless vector photon). There would also be no factors for the external photon lines as there is no photon polarization.
  \begin{enumerate}
  \item Assuming that this ``photon'' is heavy enough to decay, calculate the decay rate for $\gamma\to e^+e^-$ using the helicity spinors (in center of mass frame). Assume that the photon is heavy enough that the electron/positron mass can be neglected. 
  \item Is helicity ``conserved'' in high-energy interactions in this theory? Explain how is this different from QED?
  \item Carry out the calculation again including the electron/positron mass using the trace techniques
  \item If $m_\gamma=\SI{3}{GeV}$, find the lifetime of this photon in seconds.
  \end{enumerate}
\end{problem}

\end{document}