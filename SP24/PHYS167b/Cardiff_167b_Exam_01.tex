\documentclass[12pt]{article}

\title{\vspace{-3em}PHYS 167b Exam 1}
\author{Michael Cardiff}
\date{\today}

%% science symbols
\usepackage{amsmath}
\usepackage{amssymb}
\usepackage{amsthm}
\usepackage{bm}
\usepackage{cancel}
\usepackage{physics}
\usepackage{siunitx}
\usepackage{slashed}

%% general pretty stuff
\usepackage{caption}
\usepackage{float}
\usepackage{graphicx}
\usepackage{url}
\usepackage{enumitem}
\usepackage{hyperref}
\usepackage{tikz}
\usepackage{tikz-feynhand}
\usepackage[margin=1in]{geometry}

% setup options
\captionsetup{labelfont=bf}
\graphicspath{ {./figs/} }

% macros
\renewcommand{\L}{\mathcal{L}}
\renewcommand{\H}{\mathcal{H}}
\renewcommand{\l}{\ell}
\newcommand{\M}{\mathcal{M}}
\newcommand{\mcV}{\mathcal{V}}
\newcommand{\D}{\partial}
\newcommand{\veps}{\varepsilon}
\newcommand{\circled}[1]{\tikz[baseline=(char.base)]{
    \node[shape=circle,draw,inner sep=2pt](char){#1};}}

% mdframed environments
\usepackage[framemethod=TikZ]{mdframed}
\mdfsetup{skipabove=\topskip,skipbelow=\topskip}
\mdfdefinestyle{defstyle}{%
  linewidth=1pt,
  frametitlerule=true,
  frametitlebackgroundcolor=gray!40,
  backgroundcolor=gray!20,
  innertopmargin=\topskip
}

\mdtheorem[style=defstyle]{definition}{Definition}
\mdtheorem[style=defstyle]{theorem}{Theorem}
\mdtheorem[style=defstyle]{problem}{Problem}

\newenvironment{thebook}
{\begin{mdframed}[style=defstyle,frametitle={From the Book}]}{\end{mdframed}}

\usepackage{fancyhdr}
\usepackage{siunitx}
\pagestyle{fancy}
\lhead{Michael Cardiff}
\rhead{PHYS 167b Exam 1}

\begin{document}
\maketitle
\section*{Task 1: Absorption Length and Cross Section}
\begin{problem}
  Consider a beam of intensity $I_0$ entering a long target. Let $z$ be the distance travelled by the beam in the target, measured from the entrance point. The interaction rate $R_i$is given by $R_i=\sigma nV\Phi$, where $\sigma$ is the cross section, $N=nV$ is the number of target particles encountered and $\Phi$ is the intensity per unit area.
  \begin{enumerate}
  \item What is the beam intensity $I(z)$ as a function of distance $z$?
  \item The ``absorption length'' $L$ is defined as the distance at which the beam intensity reduces by the factor $1/e$. Find L.
  \item The absorption length in lead is \SI{5}{cm}. Find the total neutron cross section of lead (Atomic mass number of about 200 and density of about \SI{10}{g.cm^{-3}}).
  \end{enumerate}
\end{problem}

\newpage
\section*{Task 2: Isospin}
\begin{problem}
  \begin{enumerate}
  \item Compute the Clebsch-Gordan coefficients for the states with $\vb{J}=\vb{J}_1+\vb{J}_2$, $M=m_1+m_2$, where $J_1=1$, $J_2=1/2$, $J=3/2$ and $M=1/2$, for the various possible $m_1$ and $m_2$ values
  \item Consider the following strong interaction reactions (1) $\pi^+p\to\pi^+p$, (2) $\pi^-p\to\pi^-p$, (3) $\pi^-p\to\pi^0n$. Using the Clebsch0Gordan tables, write the isospin wavefunction of each state in terms of total isospin $I$ and $I_z$ basis.
  \item These isospin conserving reactions can occur in the isospin $I=3/2$ state ($\Delta$ resonance), or $I=1/2$ state ($N^*$ resonance). Using the result above calculate the ratio of these cross-sections ($\sigma_1:\sigma_2:\sigma_3$) for an exchange corresponding to a (1) $\Delta$ resonance, and toi (2) $N^*$ resonance. At a resonance energy you can neglect the effect due to the other isospin state. The cross sections are proportional to the isospin amplitudes and are independent of $I_z$
  \end{enumerate}
\end{problem}

\newpage
\section*{Task 3: Muon Decays}
\begin{problem}
  A pion at rest decays into a muon and a neutrino $(\pi^-\to\mu^-+\overline{\nu}_\mu)$
  \begin{enumerate}
  \item On the average, how far will the muon travel in vacuum before decaying? Give a numerical answer. 
  \item How much energy would a muon need to circumnavigate the earth with a fair chance of completing the journey, assuming that the earth's magnetic field is strong enough to keep it in orbit? Is the earth's field actually strong enough?
  \end{enumerate}
\end{problem}

\newpage
\section*{Task 4: Spin-0 Photon}
\begin{problem}
  Let's imagine a parallel universe where the photon instead of being a massless vector (spin-1) particle, is a massive scalar (spin-0) particle. The QED vertex in the Feynman rules in this theory would be $-ig_e\bm{1}$, where $\bm{1}$ is the $4\times4$ unit matrix (to be compared to $-ig_e\gamma^\mu$ for the massless vector photon). There would also be no factors for the external photon lines as there is no photon polarization.
  \begin{enumerate}
  \item Assuming that this ``photon'' is heavy enough to decay, calculate the decay rate for $\gamma\to e^+e^-$ using the helicity spinors (in center of mass frame). Assume that the photon is heavy enough that the electron/positron mass can be neglected. 
  \item Is helicity ``conserved'' in high-energy interactions in this theory? Explain how is this different from QED?
  \item Carry out the calculation again including the electron/positron mass using the trace techniques
  \item If $m_\gamma=\SI{3}{GeV}$, find the lifetime of this photon in seconds.
  \end{enumerate}
\end{problem}

\end{document}