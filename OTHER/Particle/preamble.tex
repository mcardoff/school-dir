\author{Michael Cardiff}
\date{\today}

%% science symbols
\usepackage{amsmath,amssymb,amsthm}  % ams
\usepackage{siunitx}
\usepackage{bm,cancel,mmacells}      % look nice
\usepackage{physics,slashed}         % phys specific

%% general pretty stuff
\usepackage{caption,float,graphicx,url,enumitem}
\usepackage{tikz,tikz-feynhand}
\usepackage{geometry}

% setup options
\captionsetup{labelfont=bf}
\geometry{margin=1in}

% macros
\renewcommand{\L}{\mathcal{L}}
\renewcommand{\H}{\mathcal{H}}
\renewcommand{\l}{\ell}
\newcommand{\id}{\bm{1}}
\newcommand{\mcV}{\mathcal{V}}
\newcommand{\D}{\partial}
\newcommand{\veps}{\varepsilon}
\newcommand{\circled}[1]{\tikz[baseline=(char.base)]{
    \node[shape=circle,draw,inner sep=2pt](char){#1};}}

% mdframed environments
\usepackage[framemethod=TikZ]{mdframed}
\mdfsetup{skipabove=\topskip,skipbelow=\topskip}
\mdfdefinestyle{defstyle}{%
  linewidth=1pt,
  frametitlerule=true,
  frametitlebackgroundcolor=gray!40,
  backgroundcolor=gray!20,
  innertopmargin=\topskip
}

\mdfdefinestyle{todostyle}{%
  linewidth=0pt,
  frametitlerule=false,
  frametitlebackgroundcolor=red!40,
  backgroundcolor=red!20,
  innertopmargin=\topskip
}

\mdtheorem[style=defstyle]{definition}{Definition}
\mdtheorem[style=defstyle]{theorem}{Theorem}
\mdtheorem[style=defstyle]{problem}{Problem}

\newenvironment{thebook}
{\begin{mdframed}[style=defstyle,frametitle={From the Book}]}{\end{mdframed}}

\newenvironment{remark}
{\begin{mdframed}[style=defstyle,frametitle={Remark}]}{\end{mdframed}}

\newenvironment{TODO}
{\begin{mdframed}[style=todostyle,frametitle={TO DO}]}{\end{mdframed}}

\theoremstyle{plain}
\newtheorem*{note}{Note}