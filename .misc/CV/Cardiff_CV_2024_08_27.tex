%------------------------
% Resume Template
% Author : Anubhav Singh
% Github : https://github.com/xprilion
% License : MIT
%------------------------

\documentclass[a4paper,20pt]{article}

\usepackage{latexsym}
\usepackage[margin=1in]{geometry}
\usepackage{titlesec}
\usepackage{marvosym}
\usepackage[usenames,dvipsnames]{color}
\usepackage{verbatim}
\usepackage{enumitem}
\usepackage[pdftex]{hyperref}
\usepackage{fancyhdr}

\pagestyle{fancy}
\fancyhf{} % clear all header and footer fields
\fancyfoot{}
\renewcommand{\headrulewidth}{0pt}
\renewcommand{\footrulewidth}{0pt}

% Adjust margins
\addtolength{\oddsidemargin}{-0.530in}
\addtolength{\evensidemargin}{-0.375in}
\addtolength{\textwidth}{1in}
\addtolength{\topmargin}{-.45in}
\addtolength{\textheight}{1in}

\urlstyle{rm}

\raggedbottom
\raggedright
\setlength{\tabcolsep}{0in}

% Sections formatting
\titleformat{\section}{
  \vspace{-10pt}\scshape\raggedright\large
}{}{0em}{}[\color{black}\titlerule \vspace{-6pt}]

%-------------------------
% Custom commands
\newcommand{\resumeItem}[2]{
  \item\small{
    \textbf{#1}{: #2 \vspace{-2pt}}
  }
}

\newcommand{\resumeItemWithoutTitle}[1]{
  \item\small{
    {\vspace{-2pt}}
  }
}

\newcommand{\resumeSubheading}[4]{
  \vspace{-1pt}\item
    \begin{tabular*}{0.97\textwidth}{l@{\extracolsep{\fill}}r}
      \textbf{#1} & #2 \\
      \textit{#3} & \textit{#4} \\
    \end{tabular*}\vspace{-5pt}
}


\newcommand{\resumeSubItem}[2]{\resumeItem{#1}{#2}\vspace{-3pt}}

\renewcommand{\labelitemii}{$\circ$}

\newcommand{\resumeSubHeadingListStart}{\begin{itemize}[leftmargin=*]}
\newcommand{\resumeSubHeadingListEnd}{\end{itemize}}
\newcommand{\resumeItemListStart}{\begin{itemize}}
\newcommand{\resumeItemListEnd}{\end{itemize}\vspace{-5pt}}

%-----------------------------
%%%%%%  CV STARTS HERE  %%%%%%

\begin{document}

%----------HEADING-----------------
\begin{tabular*}{\textwidth}{l@{\extracolsep{\fill}}r}
  \textbf{{\LARGE Michael Cardiff}} & Email: \href{mailto:}{mcardiff@brandeis.edu}\\
  \href{https://github.com/xprilion}{Github: ~~github.com/mcardoff} \\
\end{tabular*}

%-----------EDUCATION-----------------
\section{~~Education}
\resumeSubHeadingListStart
\resumeSubheading
{Brandeis University}{Chicago, Illinois}
{PhD - Physics;  GPA: 4.00}{August 2022 - 2027 (expected) }
{\scriptsize \textit{ \footnotesize{\newline{}\textbf{Courses:} Quantum Field Theory, Quantum Thermalization, Particle Physics}}}
\resumeSubheading
{Illinois Institute of Technology}{Chicago, Illinois}
{Bachelor of Science - Physics;  GPA: 3.97}{August 2018 - May 2022}
{\scriptsize \textit{ \footnotesize{\newline{}\textbf{Courses:} Particle Physics I,II, Graduate Math Methods II, Quantum Field Theory, General Relativity}}}
\resumeSubHeadingListEnd	    
\vspace{-5pt}
\section{Coding Skills Summary}
\resumeSubHeadingListStart
\resumeSubItem{Languages}{~~~~~~Python, C++, JAVA, Bash, Haskell}
\resumeSubItem{Tools}{~~~~~~~~~~~~~~Emacs, Git, Linux, Windows}
\resumeSubItem{Soft Skills}{~~~~~~~Leadership, Writing, Time Management}
\resumeSubHeadingListEnd
\vspace{-5pt}
\section{Research Experience}
\resumeSubHeadingListStart
\resumeSubheading{ATLAS Collaboration}{}
{Qualifying Project}{December 2022 - July 2024}
\resumeItemListStart
\resumeItem{Goals}{Evaluate performance of track selection criteria for ITk}
\resumeItem{Consistency}{Kept consistent communication with advisors in multiple time zones}
\resumeItem{Results}{Authorship in the ATLAS Collaboration, ATLAS internal note}\vspace{0.2cm}
\resumeItemListEnd
\vspace{-5pt}
\resumeSubheading{Dr John Zasadzinski}{}
{Research Assistant}{May 2020 - June 2022}
\resumeItemListStart
\resumeItem{Adaptation}{Worked with code written by a team, re-purposing previously written Python code to fit new needs}
\resumeItem{Consistency}{Meeting weekly with a team of five people over the internet and in person to present on progress}
\resumeItem{Communication}{In addition to weekly meetings, maintained a constant stream of communication with advisors to ensure work was getting one}
\resumeItem{Results}{Development of a paper currently in process of being published to The Springer Plasmonics Journal}
\vspace{0.2cm}
\resumeItemListEnd
\vspace{-5pt}
\resumeSubheading{Dr Zack Sullivan}{}
{Research Assistant}{August 2021 -  May 2022}
\resumeItemListStart
\resumeItem{Efficiency}{Worked to use modern GPU programming on a standard program used for numerous calculations in High Energy Physics}
\resumeItem{Consistency}{Meet up to twice per week with advisor to present current progress and set specific goals for the week}
\resumeItem{Sustainability}{Provide a sustainable long term project by focusing on creating a framework for porting these programs to GPU codes }
\resumeItemListEnd
\resumeSubHeadingListEnd

%-----------Relevant Courses-----------------
\vspace{-5pt}
\section{Relevant Courses}
\resumeSubHeadingListStart
\resumeSubItem{PHYS 545/546 (IIT)}{Particle Physics, focusing on a phenomenological understanding of QED, QCD, and electroweak physics. Topics include: QED: Spin-dependent cross sections, crossing symmetries, C/P/CP; QCD: Gluons, parton model, jets; Electroweak interactions: W, Z, and Higgs. Weak decays and production of weak bosons; Loop calculations: Running couplings, renormalization}
\vspace{2pt}
\resumeSubItem{PHYS 553 (IIT)}{Quantum Field Theory, Follow Fradkin's Quantum Field Theory: An Integrated Approach. Topics include: canonical quantization of fields; path integral quantizations of scalar, Dirac, and gauge theories; symmetries and conservation laws; perturbation theory and generating functionals; renormalization}
\vspace{2pt}
\resumeSubItem{PHYS 502 (IIT)}{Methods of Theoretical Physics II. Topics include Group theory, Discrete groups, elementary examples and properties. Lie groups, Lie algebras, generators. Their fundamental properties. Group representations. O(3), SU(2), SU(3). Complex variables: Cauchy-Riemann conditions, Complex variables integrals: Cauchy theorem, Cauchy formula.}
\vspace{2pt}
\resumeSubItem{PHYS 202a (Brandeis)}{Quantum Field Theory, through the lens of Donoghue \& Sorbo's A Prelude to Quantum Field Theory, topics include development of field theory, interactions, feynman rules, decay rates and cross sections, renormalization, path integrals, loop diagrams}
\vspace{2pt}
\resumeSubItem{PHYS 162b (Brandeis)}{Quantum Mechanics II, Includes a more advanced look into the various approximation methods used in QM, focusing on a more modern look into physics. Topics include Symmetries and conservation laws, time independent as well as dependent perturbation theory, variational methods, scattering theory, as well as relativistic quantum mechanics}
\vspace{2pt}
\resumeSubItem{PHYS 167b (Brandeis)}{Particle Phenomenology, The phenomenology of elementary particles and the strong, weak, and electromagnetic interactions. Properties of particles, kinematics of scattering and decay, phase space, quark model, unitary symmetries, and conservation laws.}
\resumeSubHeadingListEnd
\vspace{-5pt}
%-----------PROJECTS-----------------
\vspace{-5pt}
\section{Publications}
\resumeSubHeadingListStart
\resumeSubItem{Paper}{In Preprint: Superconducting Photocathode Quantum Efficiency enhancement with UV Plasmonics, publishing to the Springer Plas- monics Journal}
\resumeSubHeadingListEnd
\vspace{-5pt}
%-----------Awards-----------------
\section{Honors and Awards}
\begin{description}[font=$\bullet$]
\item {Honorable mention for a poster created for the Lewis College Virtual Undergraduate Research Day in 2021, created from work done with Dr. John Zasadzinski}
\vspace{-5pt}
\item {Dean's List for Fall 2018-2021, as well as Spring 2019-2022}
\vspace{-5pt}
\item {International Baccalaureate Diploma Holder}
\end{description}
\end{document}