\documentclass[12pt]{article}

\title{CS 485 Participation 7}
\author{Michael Cardiff}
\date{\today}

%% science symbols 
\usepackage{amsmath}
\usepackage{amssymb}
\usepackage{physics}

%% general pretty stuff
\usepackage{bm}
\usepackage{enumitem}
\usepackage{float}
\usepackage[margin=1in]{geometry}
\usepackage{graphicx}

% figures
\graphicspath{ {./figs/} }

\newcommand{\fig}[3]
{
  \begin{figure}[H]
    \centering
    \includegraphics[width=#1cm]{#2}
    \caption{#3}
  \end{figure}
}

\newcommand{\figref}[4]
{
  \begin{figure}[H]
    \centering
    \includegraphics[width=#1cm]{#2}
    \caption{#3}
    \label{#4}
  \end{figure}
}

\renewcommand{\L}{\mathcal{L}}

\begin{document}
\maketitle

\section*{Question 1}
I believe that this is the entire point of of intellectual property. However, the difference would be in the fact that in a typical claim of an intellectual property involves something like a patent, which can be argued to not be a direct claim of the idea. So it is a bit more difficult to claim ownership of an idea itself. I believe it is impossible to own an idea, due to the erratic nature of ideas, and the uncertainty of what they are.

A patent lays claim to an invention, which makes use of a new idea. An interesting example of a patent has to do with different brands of higher-order Rubik's cubes. The concept is that the larger your cube becomes (generally $5\times 5\times 5$ and above), the further each layer swings out. Eventually this will get to a point where an entire piece will hang out, and since they cannot float, they would fall out and the entire cube would simply fall apart. There are many ways to combat this, the company VCube solved it by placing the pieces on a sphere, which would keep the entire structure rigid and prevent it from falling apart (check US Patent No. US7600756). The VCube company owned the structure rather than the idea. What I want to bring up here is the fact that there are still other companies which are producing cubes, without VCube's structure. Clearly, VCube's original question was "How do we make a stable puzzle at more than 3 layers?" Other companies were faced with the same idea and found different solutions. What is interesting about the VCube structure however is that none of the VCube puzzles that you can purchase actually use the solution in the patent. Instead they use another company's idea where the copyright/patent laws are less clear, so they are free to use it. So what is the point of even owning their idea if they do not need to use it?

Next I want to talk about the concept of an idea. An idea can literally be any thought that enters our head. So owning an idea is a little hard when it comes to the claim that 'I am the first person to have this thought' which is just improbable given the sheer number of people on earth having thoughts and the fact that they're always having thoughts. So it is at least impractical to say you were the first person to come up with or have an idea, so what if you were the first person to claim that idea? There would need to be a sort of patent for the idea. The problem would arise when you make the requirements for this 'patent' as what information would you even provide as an applicant? It would have to be the idea, but not include anything where the idea would be applied, otherwise that would be a patent. What is left seems to be just a short story about a thought you had one time, so not really anything concrete or that could be enforced. 
\section*{Question 4}
The most interesting aspect of ownership that has arisen recently is in NFTs. From what I understand about NFTs, you are not necessarily owning the thing which is displayed in the NFT, but rather a unique ID assigned to that image. This is odd, as you do not own it in the sense that you can own a picture. This completely flips the concept of ownership on its head. Usually when you own something you are free to do to it as you please, but with and NFT-like ownership, you cannot do that at all.

However, it can be argued that this doesn't change ownership at all, as unlike an idea, song, or picture, there is not much TO change once you know what you own. In an NFT you simply own a link to an image or something similar, so changing that link would lead to a different item, which is not owned by you, so how could you change anything? The content of the link might want to be changed, but that is owned by someone else, and it is not what you paid for. When you buy a picture to edit, you technically should own that picture and are free to do what you want, though you only own that copy, not the original. So like owning a print, you only own the copy, and editing the digital copy is fair game.

I would not say that the technology age is destroying the concept of ownership, if anything it is simply refining it so as to accommodate the technology of today. When buying something like a copy of an image, you are simply buying a copy, so you own it of course. Much of the same applies to the NFT discussion as well. 
\end{document}