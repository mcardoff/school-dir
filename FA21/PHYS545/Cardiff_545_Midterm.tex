\documentclass[12pt]{article}

\title{\vspace{-3em}PHYS 545 Midterm}
\author{Michael Cardiff}
\date{\today}

%% science symbols 
\usepackage{amsmath}
\usepackage{amssymb}
\usepackage{physics}

%% general pretty stuff
\usepackage{bm}
\usepackage{enumitem}
\usepackage{float}
\usepackage[margin=1in]{geometry}
\usepackage{graphicx}

% figures
\graphicspath{ {./figs/} }

\newcommand{\fig}[3]
{
  \begin{figure}[H]
    \centering
    \includegraphics[width=#1cm]{#2}
    \caption{#3}
  \end{figure}
}

\newcommand{\figref}[4]
{
  \begin{figure}[H]
    \centering
    \includegraphics[width=#1cm]{#2}
    \caption{#3}
    \label{#4}
  \end{figure}
}

\renewcommand{\L}{\mathcal{L}}

\begin{document}
\maketitle
\section*{Question 1}
\section*{Question 2}
\section*{Question 3}
Pion isospin states:
\begin{align*}
  \pi^+&\equiv\ket{1,1}\\
  \pi^0&\equiv\ket{1,0}\\
  \pi^-&\equiv\ket{1,-1}
\end{align*}
The Quantities $B,C,D$ are all scattering into the same two objects so we need to take inner products with themselves. So we only need to calculate the following vector sums:
\begin{align*}
  B:\quad \pi^+\oplus\pi^+&=\ket{1,1}\oplus\ket{1,1} \\
  &=
  C:\quad \pi^+\oplus\pi^0&=\ket{1,1}\oplus\ket{1,0} \\
  D:\quad \pi^+\oplus\pi^-&=\ket{1,1}\oplus\ket{1,-1} 
\end{align*}
\section*{Question 4}
\section*{Question 5}
\section*{Question 6}
\end{document}