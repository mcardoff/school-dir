\documentclass[12pt]{article}

\title{\vspace{-3em}PHYS 545 HW 3B}
\author{Michael Cardiff}
\date{\today}

%% science symbols 
\usepackage{amsmath}
\usepackage{amssymb}
\usepackage{physics}

%% general pretty stuff
\usepackage{bm}
\usepackage{enumitem}
\usepackage{float}
\usepackage{graphicx}
\usepackage[margin=1in]{geometry}

% figures
\graphicspath{ {./figs/} }

\newcommand{\fig}[3]
{
  \begin{figure}[H]
    \centering
    \includegraphics[width=#1cm]{#2}
    \caption{#3}
  \end{figure}
}

\newcommand{\figref}[4]
{
  \begin{figure}[H]
    \centering
    \includegraphics[width=#1cm]{#2}
    \caption{#3}
    \label{#4}
  \end{figure}
}

\renewcommand{\L}{\mathcal{L}}

\begin{document}
\maketitle
\section*{Question 1}
We now have $\phi = a + ic$. First we write out the potential:
\begin{align*}
  V&=\lambda\qty(\phi^*\phi-\frac{v^2}{2})^2\\
  &=\lambda\qty(\qty(a-\frac{v}{\sqrt{2}}-ic)
  \qty(a-\frac{v}{\sqrt{2}}+ic)-\frac{v^2}{2})^2\\
  &=\lambda\qty(\qty(a-\frac{v}{\sqrt{2}})^2+c^2-\frac{v^2}{2})^2\\
  &=\lambda\qty(a^2+c^2-av\sqrt{2}+\frac{v^2}{2}-\frac{v^2}{2})^2\\
  &=\lambda\qty(a^2+c^2-av\sqrt{2})^2
\end{align*}
A mass term is given by $\frac{1}{2}m_\phi\phi^2$, so we have a couple terms when we distribute the $\lambda$ and expand:
\begin{align*}
  V&=\lambda\qty(a^4+c^4-2\sqrt{2}a^3v-2\sqrt{2}vac^2+2v^2a^2)\\
  &=\lambda\qty(a^4+c^4-2\sqrt{2}a^3v-2\sqrt{2}vac^2)+
  \boxed{\frac{1}{2}\qty(4\lambda v^2)a^2}\\
  &=\lambda\qty(a^4+c^4-2\sqrt{2}a^3v-2\sqrt{2}vac^2)+
  \frac{1}{2}m_a^2a^2
\end{align*}
We have found a single mass term, so the $c$ field should represent a goldstone boson. 
\section*{Question 2}
First we should work out the scalar product for $\phi$
\begin{align*}
  \phi^\dag\phi&=\pmqty{a'-ic & b-id}\pmqty{a'+ic\\b+id}\\
  &=(a'-ic)(a'+ic)+(b-id)(b+id)\\
  &=(a')^2+b^2+c^2+d^2
  =\qty(a-\frac{v}{\sqrt{2}})^2+b^2+c^2+d^2\\
  &=a^2+b^2+c^2+d^2-av\sqrt{2}+\frac{v^2}{2}
\end{align*}
The potential is then:
\begin{align*}
  V=&\lambda\qty(\phi^\dag\phi-\frac{v^2}{2})^2\\
  =&\lambda\qty(a^2+b^2+c^2+d^2-av\sqrt{2}+\frac{v^2}{2}-\frac{v^2}{2})^2\\
  =&\lambda\qty(a^2+b^2+c^2+d^2-av\sqrt{2})^2\\
  =&\lambda\left(a^4+b^4+c^4+d^4+2a^2v^2\right.\\
  &-2\sqrt{2}a^3v-2\sqrt{2}ab^2v-2\sqrt{2}ac^2v-2\sqrt{2}ad^2v\\
  &\left.+2a^2b^2+2a^2c^2+2a^2d^2+2b^2c^2+2b^2d^2+2c^2d^2\right)
\end{align*}
Once again the only terms quadratic in the field is the $a^2v^2$ term, so the $a$ field is the only one with mass, so $b,c,d$ are all goldstone bosons.
\section*{Question 3}
The symmetry of $SU(2)_{L/R}$ is in terms of matrices, so it might be better to write the Lagrangian in terms of matrices and the quark vector:
\begin{align*}
  q=\pmqty{u\\d}\implies \mathcal{L}=-\bar{q}\pmqty{m_u&\\&m_d}q
\end{align*}
In terms of $L$ and $R$ quarks this is:
\begin{align*}
  \mathcal{L}=-\bar{q}_L\pmqty{m_u&\\&m_d}q_R-\bar{q}_R\pmqty{m_u&\\&m_d}q_L
\end{align*}
For $SU(2)_L$, the quark vector gets transformed as:
\begin{align*}
  SU(2)_L:q_L\to U_Lq_L\quad\qty(\implies\bar{q}_L\to\bar{q}_LU^\dag_L)
\end{align*}
So the lagrangian will become:
\begin{align*}
  \mathcal{L}\to-\bar{q}_LU^\dag_L\pmqty{m_u&\\&m_d}q_R
  -\bar{q}_R\pmqty{m_u&\\&m_d}U_Lq_L
\end{align*}
However, since $m_u\neq m_d$, the matrix cannot be treated as a number times the identity, so $SU(2)_L$ is not a symmetry of this term. As for $SU(2)_R$, we have:
\begin{align*}
  SU(2)_R:q_R\to U_Rq_R\quad\qty(\implies\bar{q}_R\to\bar{q}_RU^\dag_R)
\end{align*}
For the lagrangian
\begin{align*}
  \mathcal{L}\to-\bar{q}_L\pmqty{m_u&\\&m_d}U_Rq_R
  -\bar{q}_RU^\dag_R\pmqty{m_u&\\&m_d}q_L
\end{align*}
Again, $SU(2)_R$ is not a symmetry

Let $m_u=m_d=m$, such that the matrix of masses now becomes $m\mathbb{I}$ where $\mathbb{I}$ is the identity matrix, and the lagrangrian becomes:
\begin{align*}
  \mathcal{L}=-m\qty(\bar{q}_Lq_R+\bar{q}_Rq_L)
\end{align*}
Isospin does not differentiate between quark helicity, so the transformation is:
\begin{align*}
  SU(2)_I:q_{L/R}\to Uq_{L/R}
\end{align*}
So the lagrangian becomes:
\begin{align*}
  \mathcal{L}&\to-m\qty(\bar{q}_LU^\dag Uq_R+\bar{q}_RU^\dag Uq_L)\\
  &\to-m\qty(\bar{q}_Lq_R+\bar{q}_Rq_L)=\mathcal{L}
\end{align*}
So Isospin is in fact a symmetry

The mass term of the pions should be:
\begin{align*}
  \mathcal{L}=-m_\pi^2\pi^i\pi^i
\end{align*}
We showed in class that an isospin transformation of the pion theory is a rotation:
\begin{align*}
  \text{Isospin}:\pi^i\to R^{ij}\pi^j
\end{align*}
The lagrangian becomes:
\begin{align*}
  \mathcal{L}\to-m_\pi^2R^{ij}\pi^jR^{ik}\pi^k=-m_\pi^2(R^T)^{ji}R^{ik}\pi^j\pi^k=
  -m_\pi^2\delta^{jk}\pi^j\pi^k=-m_\pi^2\pi^j\pi^j=\mathcal{L}
\end{align*}
So this term respects isospin
\end{document}
