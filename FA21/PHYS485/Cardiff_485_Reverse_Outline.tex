\documentclass[12pt]{article}

\title{\vspace{-3em}PHYS 485\\Reverse Outline Assignment}
\author{Michael Cardiff}
\date{\today}

%% science symbols 
\usepackage{amsmath}
\usepackage{amssymb}
\usepackage{physics}

%% general pretty stuff
\usepackage{bm}
\usepackage{enumitem}
\usepackage{float}
\usepackage[margin=1in]{geometry}
\usepackage{graphicx}

% figures
\graphicspath{ {./figs/} }

\newcommand{\fig}[3]
{
  \begin{figure}[H]
    \centering
    \includegraphics[width=#1cm]{#2}
    \caption{#3}
  \end{figure}
}

\newcommand{\figref}[4]
{
  \begin{figure}[H]
    \centering
    \includegraphics[width=#1cm]{#2}
    \caption{#3}
    \label{#4}
  \end{figure}
}

\renewcommand{\L}{\mathcal{L}}

\begin{document}
\maketitle
\section{Introduction}

Paragraph 1
\begin{itemize}
\item Real climate models vary a lot, and it is useful.
\item We cannot explain why they vary.
\end{itemize}
Paragraph 2
\begin{itemize}
\item Based on time scales, we cannot determine what is a proper input to the climate.
\item Previous work determined the statistical significance of these supposed inputs.
\end{itemize}
Paragraph 3
\begin{itemize}
\item Climate models we use are do not produce the same variability.
\item Climate variations treated as discrete yes/no transitions.
\end{itemize}
Paragraph 4
\begin{itemize}
\item Small scale randomness employed to get this spectrum.
\item Model is found to be consistent with recorded data, at least in terms of singularities.
\end{itemize}
Paragraph 5
\begin{itemize}
\item In order for the overall model to be 'stationary' a negative feedback is necessary to balance out the positive.
\item This counteracts the small scale randomness.
\end{itemize}
Paragraph 6\&7
\begin{itemize}
\item Develop a 'random walk' climate model, and use known solutions for it.
\item This method has been mentioned before, but nobody has really used it to its full extent.
\end{itemize}
\section{Relationship Between GCM's SDM's and Stochastic Forcing Models}
Paragraph 1
\begin{itemize}
\item Formalization of notation in order to properly discuss the model
\item Evolution of climate model is complex and most likely nonlinear
\end{itemize}
Paragraph 2\&3
\begin{itemize}
\item There are multiple components of the weather, fast and slow responding
\item Computers need to compute these DEs over the scale of the slow responding variable, often over months or years
\end{itemize}
Paragraph 4
\begin{itemize}
\item This computer thing is a problem (in Hasselmann's time) so we should reduce number of DEs
\item Do this by averaging over a long time, so only long time scale matters
\end{itemize}
Paragraph 5
\begin{itemize}
\item We can reduce this even further, the small scale dependences in this average can depend on the long scale variations
\end{itemize}
Paragraph 6
\begin{itemize}
\item Despite statistical principles being involved, the overall equation is still certain in its output
\item In weather predicting these random fluctuations are already found, but not in the case of an SDM
\item The normal explanation of the variability has been factors external to the math, but author argue it is internal to the math.
\end{itemize}
\end{document}