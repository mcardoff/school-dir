\documentclass[12pt, letterpaper]{article}

\title{How do Gender Norms Impact \\ Economic and Human Development?}
\author{Michael Cardiff}
\date{\today}

\usepackage{physics}
%% general pretty stuff
\usepackage{enumitem}
\usepackage[margin=1.2in]{geometry}
\usepackage{hyperref}
\usepackage{url}
\usepackage{setspace}

\doublespacing
\nocite{*}

\begin{document}
\begin{titlepage}
  \begin{center}
    \vspace*{7cm} \large
    Michael Cardiff \\
    \vspace*{0.5cm} \Large
    \textbf{How do Gender Norms Impact \\ Economic and Human Development?}\\
    \vspace{0.5cm} \large
    SSCI 380: International Development\\ \vspace{0.5cm}
    \today
    \vfill
  \end{center}
\end{titlepage}

\section{Introduction}
The goal of this paper is to examine the role that gender norms play in the development of countries. A focus is placed on comparing the effect of a specific Index, the Multidimensional Gender Inequality Index (MGII), and its effect on Economic and Human development.  There seems to be a sort of tug of war between these two sides when it comes to the development of women's rights specifically. When talking about this it is important to discuss the relation of the Capabilities approach to this, which discusses these sorts of issues of 'Human development' in a way that goes well with the discussion of the struggle for women's rights. More importantly, we wish to discuss not only the question proposed in the title but rather the question of how the MGII correlates with other indicators of economic and human development.

It is important to discuss the role that gender and women play in development. Many models of development do not take into account the social biases which are present in society and are not necessarily present in the models of development that consider only economics or human development. This paper specifically will address the concept of gender norms along with development. This concept is largely discussed socially, it is usually without rigid definition. This extends the purpose of this paper from the author's perspective, to get a greater understanding of the contribution of these societal concepts to development.

It should be clear in some aspects that gender equality and gender norms should have a net positive impact on at least human development. This is because gender equality should be a factor described by indicators of human development like the Human Development Index (HDI). Since human development has to do with significant changes in the quality of human life, there should be a significant correlation between human development and the MGII. As for the economic impact, from a purely theoretic standpoint, it is obvious. When gender equality is promoted, you have more of the population that is willing to participate in the economy, and some of the women may be more capable than men at these jobs, leading to a greater degree of productivity and net gain in something like GDP and GDP per capita. 

\section{Theories \& Concepts}
It is important to first and foremost explain the relevant terms and concepts which should appear often in the research related to this paper. The main concepts that require discussion could be, as mentioned before, what exactly gender norms are. How these gender norms arise and change over time is necessary to understand as well once we understand what exactly these norms are. The qualitative description of these concepts can only go so far, it is important to discuss the connection to more formal development theory. The capabilities approach is of particular interest here, but to a more specific degree than the general theory presented in class. We wish to apply it to a specific group rather than the entirety of human behavior. Multiple articles are discussed with various viewpoints on the subject of gender and development, but the author of \cite{ferrant, ferrant2} presents a method of particular interest, involving a more quantitative index, called the MGII, which should be explained further as well.

\subsection{Gender Norms}
Gender norms are very prevalent in modern conversation regarding the empowerment of women, especially having to do with the specific roles that men and women play traditionally in society. More specifically they are defined as the social conventions which constrain the choices of men and women \cite{elson}. The gender norms then produce specific behavior which is consistent with these constraints. These behaviors are what lead to the statistics that some authors call gender numbers, which can be anything from how long children live \footnotemark[1] to the representation of women in political systems \cite{bhatt, elson}. It is important to note that these norms are not static, they change (by necessity) when society changes, or when the behavior of either men or women changes. A further example of this change should be examined.

Times of crisis are often when gender norms are prone to change. A crisis can be a change to the economic or social status quo. A change in the economic status quo is an economic recession or depression, where it may be necessary for women to work to make enough money for the household. A change in the social status quo may be seen when there is a major war. Due to pre-existing gender norms, men have to go away as soldiers to fight for the country. This means that men who had families could no longer support themselves, as the primary earners of these families simply were not there anymore. This required a change in the norms of what women did, to support their families. This change in the family dynamic over such a long time frame as war causes a net change in gender norms. This in turn leads to a dramatic shift of the norms upon the return of men, where the jobs that the women took in the absence of men, wish to be filled by the same men once again. The dynamics of gender roles in these times of turmoil all contribute to how they affect development overall. This is where the arguments of the authors come in. 

This observed relationship between economic development and gender norms led many authors to say a change in gender norms to merely be a symptom of the economic development of the country. The authors examined here do not agree with this sentiment. Authors describing gender norms in greater detail will attribute some non-linearity in the relationship between changing gender norms and economic development. There are points where the overall goal of each of these roads of development will conflict with one another, so a positive change in gender norms does not necessarily lead to a positive change in some economic measure. The problem thus far is nothing to do with a qualitative analysis of these effects, but rather the discussion of quantity. In other words, how can we translate this purely qualitative discussion so far into a discussion that we can encapsulate the gender numbers? This is where the Multidimensional Gender Inequality Index comes in, to translate what has been discussed so far into a measurable statistic.

\subsection{The MGII}
The gender numbers are what are used normally to determine the dynamics of gender norms concerning another development indicator. The most common indicators examined by the authors are purely economic, such as GDP per capita. A more full and comprehensive discussion of the effect of gender norms on development should include quantitative data as well as this qualitative description we have given so far. The solution to this traditionally incorporates a plethora of 'dimensions' which various organizations such as the UNDP define as quantifying the current status of gender norms, or more specifically, for women's empowerment. There are different indicators introduced by multiple organizations, some include the Gender Empowerment Measure (GEM), the Gender Development Index (GDI), or the Gender Status Index (GSI). None of these are necessarily bad indicators of gender development, but they do come up short in terms of statistical analysis \cite{ferrant2}. There are some problems however with the indicators of specifically inequality, so these would less be measures of women's development, and both genders would be measured by some of these indicators. This is fixed in multiple ways by the MGII introduced by \cite{ferrant2}. The author outlines requirements that the MGII and other indices should meet, citing the following:
\begin{enumerate}
\item The index should take into account all relevant dimensions of gender inequality
\item The index should be a relative measure
\item The index should be appropriately weighted. 
\end{enumerate}
The first requirement very clearly tackles the problem of actually measuring gender inequality, there should not be some aspect of gender inequality or norms which is left out purposefully, all aspects should be accounted for. The relative aspect means it should be measured in terms of statistics like the standard deviation, mean, or range. The weights are important to note, as all of the dimensions of gender inequality do not contribute equally. These weights notably differ even between developed and non-developed countries. For example in a developed country, representation in politics is going to be much more important than in less developed countries, where there is little to no access to health, which is pretty much a given in developed countries \footnotemark[2]. There are a total of eight dimensions that the MGII covers, which comprehensively describe and quantify gender inequality according to the author. These dimensions are: Intra-family law, identity, access to healthcare, access to economic resources, access to education, the autonomy of the body, economic activity, and political representation \cite{ferrant2}. The calculation of the MGII is done by the following equation from \cite{ferrant}:
\begin{align*}
  \text{MGII}=
  &\,0.181\times\text{Family}^2+0.156\times\text{Identity}^2
  +0.156\times\text{Health}^2\\
  &+0.146\times\text{Economic resources}^2+0.118\times\text{Education}^2\\
  &+0.116\times\text{Autonomy of the body}^2\\
  &+0.068\times\text{Economic activity}^2\\
  &+0.06\times\text{Political representation}^2
\end{align*}

Some of the dimensions which are described are fairly obvious, such as access to education or healthcare. Some others are less obvious, such as identity, which is not immediately obvious as to what it could be. The least self-descriptive of these are Identity, Intra-family law, and Economic activity. The remainder is for the most part self-explanatory and not be discussed much further. Identity seems to encapsulate what we earlier called the social constraints that led to the creation of gender roles. The author described identity as "the social behavior conveyed by society and internalized by individuals in the process of socialization" \cite{ferrant2}. These behaviors then define the proper social and economic behaviors for men and women, just as seen in the example of war times. Despite this, the author ranks Identity very low in terms of importance in developing countries, which is an interesting prospect seeing how it is prevalent in times of drastic social change. The next ambiguous dimension would be intra-family laws, which are described as the legal right to divorce, decision rights, and inheritance. This concept can be overlooked very easily, simply grouped in with the Identity aspect, but it is important to differentiate it from identity, as it is specifically a legal aspect to the question, rather than purely social. The author of \cite{bhatt} sees this as well, even though they were only examining gender numbers, the effects are similar to those cited by \cite{ferrant2} citing the reduction of infant and child mortality in regions with better intra-family laws. Finally, we should discuss what exactly economic activity is, more importantly, how is it different from explicit access to economic resources. The key difference is that economic resources are related to land and housing, while economic activity relates to gender discrimination in the context of work. Think of economic resources as an opportunity to spend money, where economic activity would be opportunity to gain money. With a description of the MGII in hand now, we can now describe the capabilities approach to development.

\subsection{Capabilities}
The capabilities approach to development is an interesting lens to look at the problem of gender and development. The capabilities approach focuses on the opportunities provided to people, what they can do, and what they can be, guided by the sense of dignity that is inherent to being a human \cite{nuss}. A person's capabilities are described in multiple ways throughout literature, in general, the capabilities are described as the freedom to achieve some ideal goal, or more generally, some functioning combinations \cite{sen}. Depending on whether or not this freedom is truly valued, these combinations can have a varying value, leading to higher value combinations being desirable, rather than the capabilities themselves. This is an interesting concept when it comes to developing countries, as it requires there to be an outside perspective to evaluate the true capabilities of someone. Especially when a group of people is isolated, such as in rural India, there is no sense of being deprived of any opportunity, and simply accepting their place in society. However, this does not mean that capabilities are an inadequate tool for examining gender inequality it is useful in many ways. Many other ways of looking at the development are plagued by subjectivity, or opinions, regarding that women are weak or incapable of some tasks, where capabilities take an objective look at what society allows women to do, playing off the gender norms and inhibitions that society places. It is also important to note whether or not this concept of capabilities is covered in our discussion of the MGII as well. Quite clearly, the dimension of identity covers the capabilities of women very well. Both of these concepts as presented by their respective authors lead to the establishment of gender norms. Much of the identity discussion from Ferrant refers to capabilities without explicit reference to them, and vice versa for Sen and the concept of identity. Now that we have a more complete understanding of the concepts of gender inequality and the dimensions of the MGII, we can investigate various cases where the MGII can be linked to a gap is not only economic development, but also human development. 

\section{Case Study \& Analysis}
The previously accepted work from authors like \cite{bhatt,elson} saw that countries with lacking gender inequality saw deficits in other areas, such as in the GDP per capita or the human development index. Gender inequality directly can impact the economy from a theoretical standpoint, leading to a non-optimal distribution of certain resources, leading to a suffering of the overall capital that a country is bringing in. Gender inequality is more directly tied with human development, as increased access to amenities such as healthcare or greater political representation will ultimately lead to a greater human life for all women. The best way to examine the effect of this index is to sample a great number of countries over a long period. This is something I sadly do not have the time or length of this paper to do. Mostly data from \cite{ferrant} is used, which examines a multitude of countries across several backgrounds to determine a complete cross-sectional analysis of gender on development. However here we are only using the data from developing countries, those listed as non-OECD. These countries include Saudi Arabia, Chad, Burundi, Colombia, and Brazil, there is a full list in the cited paper.

\subsection{Human Development}
It is easiest to begin talking about the effect on the human development index. The HDI explicitly has components of health and education attached to it, so the dependence of both indices on these factors should be indicating a certain correlation. Across the board, in all effects, there is a clear indication of a negative correlation between inequality and human development \cite{ferrant}. This warrants some further explanation. The MGII measures inequality, so a negative correlation means that the presence of greater values of the MGII (more inequality), leads to lower values in the HDI (less human development). This overall means that if gender equality is to have a positive impact on human development we should see negative correlations across all dimensions. The author then breaks down into the various dimensions of the MGII, and almost all of them are negative as expected, except for political representation, which was slightly positive for country and year varying effects, country fixed with the year varying, as well as keeping both fixed. This is an interesting peculiarity which we should discuss further. The most important fact is that for the rest of these, we see net negative change.

This discussion aligns very well with the discussion of capabilities since the capabilities approach focuses on the human side of development. Gender inequality directly leads to fewer capabilities overall for women, both favorable and unfavorable functional outcomes are lessened with greater degrees of inequality. The effect on economic development should also be seen as a positive for the connection between the capabilities approach and gender inequality.

The human development index thus should include the MGII within it, as this complete correlation across nearly all factors of the MGII suggests. This complete correlation means that the MGII is a subset of the HDI, so the HDI should be modified to include such an explicit reference:
\begin{align*}
  \text{HDI}=a_0\text{MGII}+\sum_{i=1}^na_iX_i
\end{align*}
Where $X_i$ represents non-MGII related factors that are included in the HDI, such as trade, governance, rule, etc. \cite{ferrant}. This means the overall relationship between HDI and the MGII is a direct correlation, such a direct correlation that the MGII should be included as a dimension of the HDI. The story could be a bit different when it comes to economic development.

\subsection{Economic Development}
The system which \cite{ferrant} uses to properly describe the relationship between gender inequalities and economic development, specifically, there is a propagating relationship between them, instead of a permeating, consistent relationship between them. This is done purely for computational purposes, to make a computable model of the real-world relationships between these two factors. The 'flow' of the data is:
\begin{align*}
  \text{GDP}(t_1)\rightarrow\text{MGII}(t_1)\rightarrow\text{GDP}(t_2)
\end{align*}
So the GDP at one point in time determines the MGII at that same time, so the GDP is used as a predictor for the MGII. After that happens, the MGII at that time is used as a way to predict the future GDP. This is an untrue model, as seen in articles such as \cite{foho} which specifically mention a more direct continual relationship. Statistical methods aside, it is important to look at the results of the second step, to look at MGII as a predictor of GDP. The author's data shows a stronger negative correlation than human development. Unlike in the HDI, all eight of the dimensions of the MGII showed a negative correlation, across all combinations of fixed and varying effects, as well as across all the regression methods which the author used.

The numbers of these correlations are of important note as well. A change of one standard deviation in the log MGII increases per capita income by 3.4\% \cite{ferrant}. This means that in many countries in South \& East Asia as well as the Pacific, up to 10\% of the change in per capita income in the time of publication (the early 2010s) can be attributed to changes in gender inequality dating from the early 2000s. The most significant coefficients were from the Identity and Economic Activity. This suggests that promoting gender equality in these two areas could greatly improve the income per capita for a country. Compared to other predictors, the effect is comparable, being 28\% and 31\% for investment and education respectively. While not certainly on the same level, the promotion of gender equality aids not only in this economic aspect as touted by so many but also in a human way, as seen in the contribution to the HDI.

\section{Conclusion}
The MGII is a very powerful tool when examining gender inequality in the context of development. It is easily seen that the MGII correlates with both the economic and human development of nations, but it is more important to have seen which aspects specifically impact the development more. The specific dimensions of the MGII which impacted the economic factors the most were Identity and Economic activity. For human development, they were Intra-family laws and Access to Education. These correlations tell us what can be changed about developing countries to promote either one economic or human development. The greater implications of this analysis are found in the conclusion of \cite{ferrant}. Multidimensional analysis is important to cover a wide range of topics rather simply, but its greater use is in dissecting the dimensionality to find which of the dimensions have the biggest impact. The weighting of each dimension may present a problem when used for this. When using specific weights for each dimension, it is difficult to say if your result is indicative of what is seen in the individual analyses. For example, \cite{ferrant2} explicitly weights identity very high in developing countries, and correspondingly places it as having the greatest impact on economic development. However, this is mitigated by the second-largest factor, economic activity, which was ranked second-lowest in developing countries. Interestingly a similar effect is seen in the human development aspect, as Intra-family law is stated as having a large impact and is ranked highest in \cite{ferrant}'s definition of the MGII. This should warrant greater discussion, but this is beyond the scope of the paper.

In the overall research, there was a great deal on learning the broader concepts of the statistical analyses which were done to find the correlations discussed in the various articles. Much of this is not reflected in the present work as it was more for the morbidly curious physicist inside of me that wanted to know this. This paper was very interesting, and shed new light on the content covered to a lesser extent in class. 

\singlespacing
\footnotetext[1]{This  can be related to the health of a mother, and the mother is the only one who takes car of children in the gender norms of some developing countries.}
\footnotetext[2]{It is relevant to note in this case that while specific representation may not be a focus for developing countries, politics can still be a vehicle to aid the development}
\doublespacing

% \section{Bibliography}
\bibliographystyle{apalike}
\bibliography{sources}
\end{document}