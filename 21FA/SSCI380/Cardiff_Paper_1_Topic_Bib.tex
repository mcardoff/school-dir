\documentclass[12pt]{article}

\title{\vspace{-3em}Paper 1 Topic \& Bibliography}
\author{Michael Cardiff}
\date{\today}

%% science symbols 
\usepackage{amsmath}
\usepackage{amssymb}
\usepackage{physics}

%% general pretty stuff
\usepackage{bm}
\usepackage{enumitem}
\usepackage{float}
\usepackage[margin=1in]{geometry}
\usepackage{graphicx}

% figures
\graphicspath{ {./figs/} }

\newcommand{\fig}[3]
{
  \begin{figure}[H]
    \centering
    \includegraphics[width=#1cm]{#2}
    \caption{#3}
  \end{figure}
}

\newcommand{\figref}[4]
{
  \begin{figure}[H]
    \centering
    \includegraphics[width=#1cm]{#2}
    \caption{#3}
    \label{#4}
  \end{figure}
}

\renewcommand{\L}{\mathcal{L}}

\begin{document}
\maketitle
\section{Topic/Title}
\begin{itemize}
\item Gender equality and development:  How does gender equity (or lack thereof) affect development?
\item How does Gender Inequality Effect the Development of Countries? 
\end{itemize}
\section{Bibliography}
Here I provide a short description of each of my sources and following is the actual list of references
\begin{itemize}
\item \cite{bhatt} This article is a specific study in the country of India, however I call into question its recency due to it using data from almost 40 years ago at this point, but I believe the analysis is recent enough to be relevant. 
\item \cite{elson} This article looks less on the economic development of countries, but rather the response to economic crises in economically developing countries, and sets up a framework to analyze these responses in a gender-conscious way. 
\item \cite{foho} I found this article eye catching as it looked through the same lens of socialism that dependency theorists used, so it could be an interesting approach. 
\item \cite{ferrant} I thought this article would be useful as it makes use of multiple countries, and more directly addresses the questions at hand, specifically with the challenges of gender and development. 
\end{itemize}
\newpage
\bibliographystyle{apalike}
\bibliography{Cardiff_Paper_1_Topic_Bib} 

\end{document}