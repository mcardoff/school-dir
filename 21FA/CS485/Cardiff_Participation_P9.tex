\documentclass[12pt]{article}

\title{\vspace{-3em}CS 485 Participation 9}
\author{Michael Cardiff}
\date{\today}

%% science symbols 
\usepackage{amsmath}
\usepackage{amssymb}
\usepackage{physics}

%% general pretty stuff
\usepackage{bm}
\usepackage{enumitem}
\usepackage{float}
\usepackage[margin=1in]{geometry}
\usepackage{graphicx}

% figures
\graphicspath{ {./figs/} }

\newcommand{\fig}[3]
{
  \begin{figure}[H]
    \centering
    \includegraphics[width=#1cm]{#2}
    \caption{#3}
  \end{figure}
}

\newcommand{\figref}[4]
{
  \begin{figure}[H]
    \centering
    \includegraphics[width=#1cm]{#2}
    \caption{#3}
    \label{#4}
  \end{figure}
}

\renewcommand{\L}{\mathcal{L}}

\begin{document}
\maketitle

\section{What Causes Vulnerability? Who is Vulnerable?}
Before any of these questions can be answered we must discuss what vulnerability is in the first place. To be vulnerable to something is to be particularly susceptible to it, specifically the dangers of the vulnerability. So people who are vulnerable are going to be in danger if exposed to it. For example someone who is vulnerable to COVID-19 would probably die if exposed to it.

Now we can move on to who exactly is vulnerable in the first place. People are vulnerable when they are not immune to it. For example someone is vulnerable to peanuts has a peanut allergy, so vulnerability can be from an allergy which someone has no control over. On the other hand it can be caused by something that is completely controllable. Extreme cases include death, something like skydiving leaves someone extremely vulnerable to death. Skydiving is much more temporary than something like a disease, which someone is involuntarily vulnerable to by going outside. However, this involuntary nature is removed when something like a vaccine is introduced, then your vulnerability to the disease is completely voluntary and quite long term, and possibly permanent.

Then we can discuss who exactly is vulnerable, which was covered a bit before. People who are vulnerable are either involuntarily vulnerable, like someone with a peanut allergy, or voluntarily, someone who chooses not to get the COVID-19 vaccine. However there may be some rare cases where the vaccine is not a viable option for someone. In both cases however, there is no need for a long term vulnerability to COVID, as there is still the option to remain inside for an extended period of time. This is however excluding the need for money, a social need. So the vulnerability to COVID-19 the disease is definitely real to people who are not willing to get the vaccine, and even then people unwilling to get the vaccine who do not need money or to go out at all. As for a vulnerability to COVID-19-related disinformation, this is really everyone who is not directly involved in true COVID-19 statistics. This is simple, anyone who consumes any content, whether it be on the TV, internet, or even newspaper, there is the possibility for disinformation at any point at all. \newpage
\section{Technology and Vulnerability}
While talking about the disinformation, it is easy to say that technology most definitely exacerbated this vulnerability. The sheer number of sources of information, increases the chance of that information being disinformation. This chance is increased further if people recommend a specific source that is known for producing disinformation. There also seems to be a lower sense of quality in all of these sources. This could come from the sheer ease that technology has made it to make a similar source that looks insanely credible from say, a google search. This all increases vulnerability of some sort.

However, it is possible that technology can diminish vulnerability as well. One type of vulnerability we have not discussed is a physical disability. The inability to use one or more of your limbs is a struggle that would almost always result in someone either dying, or living a horrible life after. The whole concept of cybernetics and even prosthetics helps this type of vulnerability. This combined with advanced modern medicine have allowed people that do not have lower halves of their bodies to live relatively normal lives, except for the obvious. Technology has at lowest, reduced the vulnerability, and at its best, completely removed the vulnerability due to physical disability. 

\end{document}