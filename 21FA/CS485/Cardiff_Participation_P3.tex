\documentclass[12pt]{article}

\title{Participation 3}
\author{Michael Cardiff}
\date{\today}

%% science symbols 
\usepackage{amsmath}
\usepackage{amssymb}
\usepackage{physics}

%% general pretty stuff
\usepackage{bm}
\usepackage{enumitem}
\usepackage{float}
\usepackage{graphicx}

% figures
\graphicspath{ {./figs/} }

\newcommand{\fig}[3]
{
  \begin{figure}[H]
    \centering
    \includegraphics[width=#1cm]{#2}
    \caption{#3}
  \end{figure}
}

\newcommand{\figref}[4]
{
  \begin{figure}[H]
    \centering
    \includegraphics[width=#1cm]{#2}
    \caption{#3}
    \label{#4}
  \end{figure}
}


\begin{document}
\maketitle
\section*{Question 1}
If we allow the truth to be debatable, we will devalue facts. As I will explain later in question 3, facts are inherently connected to truth. So if a truth is debatable, the fact(s) that we connect with it lose value, since a fact should be valuable (this is a bit circular since it is connected to its truth). However, this can be contested with the existence of subjective truth. We can separate truth into two different views, objective and subjective truths. Objective truths would be the ones harmed by this, and tend to lead to more rigorous facts. Subjective truths are more view/opinion based. 
\section*{Question 2}
If truth is debatable, we essentially get all of modern politics, or any opinion. From that we get a sense of subjective truth, since the actual content of the truth is more based on the person (subject) to which the truth was formed. Once again the value in this is the very prospect in debate, would not exist if it werent for subjective truth, everyone would just agree! 
\section*{Question 3}
There is a difference, but the two are still connected. The way I have been taught, it seems that truth is a sense of right or wrong, whereas a fact is a statement which expresses truth in some way. A fact can express both a subjective and an objective truth. However, I would hesitate to call a fact that expresses a subjective truth a fact, it is more of an opinion. Hence why facts would get devalued if we allowed for them to be debated. If facts were debated, then they would be less factual, and more opinion. This all leads to an overall loss of objectivity and a wider focus on subjectivity. 
\section*{Question 4}
Virtual reality can only mimick reality, so in the sense of the actual content of something that is virtual reality, its 'realness' is very artificial. However, the technology that produces the images which lead to virtual reality are certainly real. So its 'realness' depends on what you actually call virtual reality. If you consider the actual images that make the videos/video games the virtual reality, then the realness comes only from the technology used. If you consider the virtual reality to be the technology then sure, it is real, and it produces images which are real, but the situations are not necessarily real. 
\section*{Question 5}
To answer this we only need look at the word, \textbf{real}ity. It is literally in the name! The distinction between this and VR is the lack of something virtual. Previously the virtual aspect was the screens and images, and actual reality has something \textbf{actual}ly real! There is not much more to debate about this, if reality is not real, then nothing can be real, since everything we observe is based on reality. 
\end{document}