\documentclass[12pt, letterpaper]{article}


%% science symbols 
\usepackage{amsmath}
\usepackage{amssymb}
\usepackage{physics}

%% general pretty stuff
\usepackage{bm}
\usepackage{enumitem}
\usepackage{float}
\usepackage[margin=1.0in]{geometry}
\usepackage{graphicx}
\usepackage{url}
\usepackage{hyperref}
\usepackage[labelfont=bf]{caption}

% figures
\graphicspath{ {./figs/} }

\newcommand{\fig}[3]
{
  \begin{figure}[H]
    \centering
    \includegraphics[width=#1cm]{#2}
    \caption{#3}
  \end{figure}
}

\newcommand{\figref}[4]
{
  \begin{figure}[H]
    \centering
    \includegraphics[width=#1cm]{#2}
    \caption{#3}
    \label{#4}
  \end{figure}
}

\renewcommand{\L}{\mathcal{L}}

\title{How Does Gender Inequality \\ Effect the Development of Countries?}
\author{Michael Cardiff}
\date{\today}

\begin{document}
\maketitle
\section*{Introduction}
The goal of the overall research paper is to understand how gender inequality effects the development of countries. In class this role was discussed to an extent, but not in depth with many examples. This literature review covers four articles all from different authors discussing different degrees of gender inequality along with other well known aspects of international development.

The first article discusses the concept of a woman's 'agency' in conversation with various economic impacts \cite{bhatt}. This study looks at the effect of a number of explanatory variables related to woman's agency. They are attempting to see changes in demographic measures, specifically in fertility, level of child survival, and a specific measure of the bias in whether or not there is a lower survival rate for female children. Specific demographics are taken from an example country of India, which allows for an inclusion of the discussion of a large number of minority groups split in multiple ways. This is an important article for the eventual discussion on a comparison between economic motives vs gender inequality motives.

The next article focuses on looking at gender and economics in a different way, looking specifically at economic crises. This article seeks to explain the 'gender dimensions' of a crisis which 'swept over developing countries since 2007' \cite{elson}. The author says that the second wave which some countries experience are the results of a few gendered economic processes. The article covers a few different developing countries and applies a framework for analyzing these problem(s) in the dimensional of gender. It is important to be able to identify and analyze the so called 'gender dimension' of even economic development, in order to later improve the development by tackling these gendered problems in a country's development.

Now the discussion shifts onto a different scale. The first two articles only either scratched the surface of a generalization of many countries, or went too deep into a single country. This article focuses on a select few developing countries, and notably, does not solely focus on economic development, but human development as well \cite{ferrant}. The author uses an index defined in a previous article of hers which demonstrates on a multidimensional level, the gender equality of a nation, similar to the national happiness index or even to a GDP measure. The main point demonstrates that there is a clear link between the equality index, and the inequality of women. 

This article is interesting to discussion with the article by Elson. It looks at a similar economic crisis as mentioned in \cite{elson}, however the origin is not from a stock market crash, but rather as the result of Socialism \cite{foho}. An interesting point made quite early in the journal claims that the countries which ar developing more rapidly and with an emphasis on foreign capital had a greater gender poverty gap. What is most important about this article is that it studies the converse of what everything else thus far has. Instead of evaluating the effect of gender inequality on certain statistics or indices, it evaluates to what extent the gender inequality is a result of other factors. 
\section*{What are the Authors Explaining?}
Many of the authors in these articles used similar definitions for different terms. In all of these articles a sense of either 'agency' or 'empowerment' is either explicitly defined or referred to \cite{bhatt,ferrant}. There is also a discussion of the multidimensionality of this type of study, i.e. the need to not only study one aspect of gender in light of international development, but we need to discuss more than one aspect. Overall, the authors are attempting to find a way to link gender inequality, defined by the agency or empowerment of woman, in a multidimensional sense to the development of nations. 

When analyzing anything in any science, it is important to have rigorous definitions of specialized terms. This is not only good for your scientific communication skills, but for the clarity of your essay to your peers. The first definition in \cite{bhatt} gives for agency, which seems much more robotic, and as if women were more philosophical tools than real people. The difference between the definitions is very slight, with only the real term differing between all the documents. 

The need for multidimensionality is seen in \cite[202]{elson} very well in a 'matrix' of analysis. Comparing not only the gender 'numbers' which are more in line with demographics and statistics, but also the gender 'norms'. There is a very need to compare gender norms to consider development in a more complete sense. Just as GDP per capita does not fully describe how well a country is developed, gender statistics cannot either. 
\section*{Summary of Arguments}
% \newpage
% \subsection{outline}
% \begin{itemize}
% \item \cite{bhatt}
%   \begin{itemize}
%   \item Purely statistical analysis
%   \item Analysis over large cross section in terms of geography, social status, religion and even literacy
%   \item Measured total fertility rate, child mortality rate, and then this stat split by male and female
%   \end{itemize}
% \item \cite{elson}
%   \begin{itemize}
%   \item Analyzes in terms of numbers and norms over various stages of the financial crisis. 
%   \item Gender numbers: stats
%   \item Gender norms: 'qualitative' aspects of gender
%   \item Cross sectional in the sense that it covers multiple impacted sectors
%   \item Explains how gender norms may be transformed, including stimulus packages
%   \item Example of Latin America defying and transforming these norms
%   \end{itemize}
% \item \cite{ferrant}
%   \begin{itemize}
%   \item Direct analysis of how inequality may result from x, where x is productivity, governance, and the next generation
%   \item Concludes that gender inequality has a net negative effect on development
%   \item Use of MGII indicator as an explanatory variable.
%   \item Compares use of MGII indicator to use of others that may conflict. 
%   \item Overall, gender inequality reflects social shortcomings
%   \item Examples as far out as women from the black plague. 
%   \end{itemize}
% \item \cite{foho}
%   \begin{itemize}
%   \item Pure economic analysis. 
%   \item Development is an intended change in the volume of GDP
%   \item Analyzing probability/risk of poverty in the 2008 financial crisis
%   \item EU countries that were socialist
%   \item Try to explain the gender poverty gap through macroeconomic policies
%   \end{itemize}
% \end{itemize}
% \subsection{final}
The first article reviewed is by Bhattacharya, and is a purely statistical analysis of what the author refers to as women's agency and its affect on development. The eras of discussion are specifically the years 1981 and 1991, and comments are made based on results from the census studies of those years. The author looks as various statistics such as fertility rate of women, and child mortality rates. These are very interesting statistics to see when considering development, as they are very social in nature as opposed to economic. A country can only develop so much if its people are not living long enough. Where the author brings in a differentiation by gender is by whether or not the children which die are female. From here the author discusses a few more statistics including literacy in females. Where this article shines is its attention to detail. The author highlights a specific past study and its shortcomings and corrects for them. The author corrects for these errors by introducing a comparison by not only from the GDP standpoint, but also from the standpoint of various social causes as well. The article goes on to argue that the more generally agreed upon viewpoint, that women's agency is what determines these statistics such as fertility or literacy, is not true, and that there are other factors more prominent which do in fact cause these problems \cite{bhatt}.

The article from Elson looks at the role of gender in situations such as economic crises. The article makes a notable distinction between statistical and social reasons for gender inequality. Specifically, gender norms are described as the social conventions which in turn constrain choices of men and women, and form behaviors consistent with these norms. This leads then to a discussion of how gender roles change in times of crisis. The presence of economic crisis would lead to many so-called 'breadwinner men' to lose their jobs, causing a shift in the gender norms. The overall result of this article is a framework of talking about specifically economic crisis in light of dynamic gender roles, with many developing countries challenging more traditional gender roles in the face of an economic crisis \cite{elson}.

Ferrant's article is perhaps the most relevant to the actual course material, as specific indices for development are mentioned that the author created in previous studies. The author makes use of what is called the Multidimensional Gender Inequality Index (MGII). The author relates the increase of this index to increase of per capita income and the human development index by 3.4 and 4.6\% respectively. The author looks at the dynamics of this index with the presence of other well known indices, so its effectiveness could be dependent on these factors as well. The author specifically looks at three different 'results' of gender inequality, productivity, governance, and the next generation. Several example from African countries as well as Latin American countries which aid in demonstrating the author's point statistically. The author moves to discuss a different viewpoint that has been seen before, other authors have described poor economic development as a symptom of gender inequality, but Ferrant here discusses the possibility of economic development as a promoter of gender equality. The MGII is more of a summarizing index of various factors described in \cite{bhatt}, but aggregated into one index. The author concludes that gender equality will come at a cost of some economic development, it is necessary, as having gender equality does lead to a net positive gain for GDP per capita as well as other measures \cite{ferrant}.

The final article is from two authors, Fodor and Horn. In contrast to the previous articles this analysis is almost purely from an economics standpoint, going as far to define development in general as a relative increase of GDP (per capita). This article uses examples from numerous EU countries to describe statistics such poverty rate as a result of percent of GDP spent on pensions. An important idea from this article is that the countries studied are very fast paced in terms of their growth, so their target is, for example, a greater GDP per capita as opposed to social aspects like gender equality. The reason for these countries to have to 'catch up' in a sense is due to the fact that they are all former socialist countries. The countries studied are an extreme to say the least, access to paid jobs by women are severely less than that of men. The author mainly looks at what is described as the gender poverty gap, which is the difference in poverty levels for men and women. The conclusion reached however is that it is simply not enough to just look at this aspect, as there is much more which could be compared in the domain of gender inequality alone \cite{foho}. 
\section*{Comparison}
% btwn bhatt and elson, elson talks about gender norms as opposed to fertility rates.
% discuss elson and fohorn as foils to similar situations
These articles all talk about relatively the same topic, so it is clear that they must be similar in that aspect. Despite this, each author makes a very different argument so it is important to notice the differences in their arguments. For example, there is a clear divide between the Bhattacharya and all the other articles, as Bhattacharya makes almost no reference to the social gender norms that are mentioned in both Ferrant, Elson, and Fodor/Horn. There is definitely a unique aspect to each of these articles which distinguishes one from the other.

The most glaring comparison that was already made is between the Bhattacharya article and for example Elson \cite{bhatt,elson}. The analysis made by Bhattacharya is completely statistical, looking at differences between the two census periods' data. The framework put forth by Elson accommodates for this in the gender numbers aspect, but where this article fails is the relation between the gender numbers and gender norms. It is artificially linked in some discussion, but not explicitly mentioned. This may appeal more to the strength of Elson's framework, as Bhattacharya was making use of it more than four years prior to Elson put pen to paper. The last part of Bhattacharya's conclusion appeals to a change of gender norms, which may lead to a different type of crisis than that which Elson discusses.

The articles from Elson and Fodor/Horn both decide to focus in on the economic aspect of gender inequality. Even more specifically, they focus on economic crises, specifically from 2008 \cite{elson,foho}. The difference from Bhattacharya returns once again, as there is a lesser degree of talk of gender norms, but there is definitely a discussion of this in on Fodor \& Horn, though it is through an economic lens. Fodor \& Horn talk about poverty rates, which is in principle an economic concept, but its consequences are seen in gender norms \cite{foho}. This is also seen in Ferrant's discussion of MGII, since the increase/decrease of this index can result in a change of economic factors \cite{ferrant}. The examples that these articles bring up are drastically different. Specifically, Elson discusses countries in Latin America, which is politically, economically and socially different to countries that are formerly socialist in the EU, which are discussed by Fodor \& Horn.

Overall, even though the discussions facilitated by each of these authors is similar in the aspect of talking about gender inequality, they differ in many aspects. The authors make similar points even though they talk about different countries. The aspects in which they differ are more to do with what the authors originally attempted to argue. For example, Elson and Bhattacharya are looking through two different spheres of influence, with Bhattacharya discussing a lesser developed country and the progression of gender inequality, while Elson wants to discuss a specific crisis and how a more developed country has responded.
\newpage
\section*{Conclusion}
All of these articles were interesting in their arguments of how gender inequality effects, either countries that are already developed or developing countries. The aspects that seem richer in this category are in developing countries, since the discussion of developed countries in \cite{elson} seemed less social and more political. Each of the articles had their own strengths and weaknesses, which I would like to discuss.

Bhattacharya's article is very important in the discussion for developing countries, showing what the trajectory of developing countries looks like in terms of gender numbers \cite{bhatt}. The weakness here is that there is little to no mention of the norms which are essential to discuss. A lack of this is supported by the framework provided by Elson. The ability to not only analyze the numbers but also the social norms is essential in gauging the trajectory of a country's development in terms of gender inequality \cite{elson}. The weakness in Elson is more the lack of examples, more attention is given to the developed countries, which are less helpful in this case. Regions are mentioned (e.g. Latin America) but very few countries specifically. This is where the Ferrant article helps, as it provides not specific countries to look at, but rather aspects to look at, in the MGII \cite{ferrant}. The weakness here is once again a focus on the numbers rather than the norms, just as in Bhattacharya's article. Fodor \& Horn's article is very useful in terms of comparisons to its subject (post socialism EU nations) to countries which experience similar levels of gender inequality in Latin America and Asia \cite{foho}. The clear weakness in this article is mentioned in the conclusion, the lack of consideration of multidimensionality, as the authors decided to focus on a single aspect, there were many holes in the argument which would be filled by another dimension of gender inequality. This weakness once again is solved by the Elson framework. 

To conclude, a more specific research question should be created. First off, a focus should be placed on developing countries, as they should be the most susceptible to have a lack of equality. In addition, special attention should be brought onto the gender norms as opposed to just focusing on the numbers, as they help show a trajectory of the development away from gender inequality. Despite this, an investigation into the economic should still be considered, as it works as an indicator of development in some aspects, but this should be used with another indicator that is not related to the GDP. Thus, a question can be formed: How do Gender Norms Impact the Economic and Human Development of [a Country]. Where a country should be determined during the research process if necessary to make this a more specific case study. 
% case study of a country?
\newpage
\bibliographystyle{apalike}
\bibliography{sources}
\end{document}