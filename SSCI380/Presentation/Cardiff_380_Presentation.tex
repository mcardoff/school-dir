\documentclass{beamer}

\usepackage{hyperref}
\usepackage{url}

\newenvironment{itemframe}[1]{\begin{frame}{#1}\begin{itemize}}   {\end{itemize}\end{frame}}

\title{Gender \& Development}
\author{Michael Cardiff}
% \logo{\large \LaTeX{}}
\subtitle{SSCI 380}

% Changes style of actual slides
\usetheme{Dresden}
% Changes color of slides
\usecolortheme{spruce}
% removes controls at bottom right side
\usenavigationsymbolstemplate{}
\begin{document}

\begin{frame}
  \titlepage
\end{frame}


% Outline frame
% \begin{frame}{Outline}
  % \tableofcontents
% \end{frame}
\section{Gender Norms}
\begin{frame}{Gender Norms}{What are they?}
  \begin{itemize}
    \item Restrain choices of men and women in their roles
    \item Do we need to abandon these in order to advance development?
  \item Overarching connection to development: More developed countries have more 'forward-thinking' gender norms
  \end{itemize}
\end{frame}

\begin{frame}{Gender Norms}{Impact on Development?}
  \begin{itemize}
  \item In lesser developed countries, women have it worse off.
  \item If women have less 'agency' then half of the population is not contributing to development \cite{bhatt}.
  \item Many articles cite impact of economic development on Gender Norms \cite{elson,ferrant,foho}
  \item I want to focus on the opposite, how these norms impact Human \& Economic Development
  \end{itemize}
\end{frame}

\section{Economic Affect}
\begin{frame}{How Does the Economy Affect Gender Norms?}
  An example to look at from developed countries: War \cite{elson}
  \begin{itemize}
  \item Men go off to war
  \item Family homes lose their breadwinner
  \item Jobs become open to the people who did not go
  \item Only people who remain are women
  \item Women take those jobs
  \item Net change in the gender norms
  \item The 'normal' behaviors for women have changed
  \end{itemize}
\end{frame}

\begin{frame}{A Side Effect}
  \begin{itemize}
  \item Change in gender norms is seen as a consequence of economic development
  \item Some authors challenge this outright, saying instead it is a promoter of economic development \cite{ferrant}.
  \item Looking at this requires quantification of gender inequality.
  \item How do we do this?
  \end{itemize}
\end{frame}

\section{Describing Inequality}
\begin{frame}{Quantifying Inequality}
  \begin{itemize}
  \item GDP per capita \textbf{quantifies} economic development
  \item How can we quantify status of gender norms and inequality?
  \item The \textbf{M}ultidimensional \textbf{G}ender \textbf{I}nequality \textbf{I}ndex (\textbf{MGII}) \cite{ferrant}
  \item Aggregates factors like political representation, access to health care, education, etc. into a single measure
  \end{itemize}
\end{frame}

\begin{frame}{What Does the MGII Tell Us?}
  \begin{itemize}
  \item Direct quantification of inequality in developing countries
  \item Increase of MGII correlates with increase in:
    \begin{itemize}
    \item Per capita income
    \item Human development index
    \end{itemize}
  \item Higher value of MGII has an impact on productivity, governance and even the next generation of MGII values
  \end{itemize}
\end{frame}

\nocite{*}

\section{Conclusion}
\begin{frame}{What is next?}
  \begin{itemize}
  \item Study affect of gender norms on economic development
  \item Read more into MGII index factors
  \item Connection to Capabilities approach? Theory in general?
  \end{itemize}
\end{frame}
\begin{frame}{Questions}
  \fontsize{20pt}{7.2}\selectfont
  \begin{center}
    Any Questions?
  \end{center}
\end{frame}

\begin{frame}{References}
  \fontsize{6pt}{7.2}\selectfont
  \bibliographystyle{abbrv}
  \bibliography{sources}
\end{frame}

\end{document}
