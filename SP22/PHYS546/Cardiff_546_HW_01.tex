\documentclass[12pt]{article}

\title{\vspace{-3em}PHYS 546 HW 1}
\author{Michael Cardiff}
\date{\today}

%% science symbols 
\usepackage{amsmath}
\usepackage{amssymb}
\usepackage{physics}

%% general pretty stuff
\usepackage{bm}
\usepackage{enumitem}
\usepackage{float}
\usepackage{graphicx}
\usepackage[margin=1in]{geometry}

% figures
\graphicspath{ {./figs/} }

\newcommand{\fig}[3]
{
  \begin{figure}[H]
    \centering
    \includegraphics[width=#1cm]{#2}
    \caption{#3}
  \end{figure}
}

\newcommand{\figref}[4]
{
  \begin{figure}[H]
    \centering
    \includegraphics[width=#1cm]{#2}
    \caption{#3}
    \label{#4}
  \end{figure}
}

\renewcommand{\L}{\mathcal{L}}
\newcommand{\D}{\partial}
\newcommand{\munu}{{\mu\nu}}

\begin{document}
\maketitle

\section*{Problem 1}
A mass term in a lagrangian will look like the following:
\begin{align*}
  \L_{mass}=\frac{1}{2}m^2\phi^2
\end{align*}
This is for an example field $\phi$ which is its own antiparticle. So in order for the Maxwell-Dirac Lagrangian to have a mass term for the photon field we must see something like:
\begin{align*}
  \frac{1}{2}m_\gamma^2A^\mu A_\mu
\end{align*}
Where $m_\gamma$ would be the mass of the photon. However this is not the only way a mass term can come, since out of all of these there are no explicit second order terms of the $A$ field. It can also come from the breaking of the local $U(1)$ symmetry of the theory, similar to how pions have mass. The transformations of this theory are the following:
\begin{gather*}
  \psi\to e^{iQ\theta}\psi\qquad\bar{\psi}\to \bar{\psi}e^{-iQ\theta}\\
  A_\mu\to A_\mu-\frac{1}{e}\D_\mu\theta
\end{gather*}
So if the term is invariant under these transformations, it can still produce massless photons.
\begin{enumerate}
\item By symmetry of indices, if $A^\mu A_\mu$ is invariant, so will $A^\nu A_\nu$:
  \begin{align*}
    A^\mu A_\mu&\to\qty(A^\mu-\frac{1}{e}\D^\mu\theta)
    \qty(A_\mu-\frac{1}{e}\D_\mu\theta)\\
    &=A^\mu A_\mu-\frac{1}{e}A^\mu\D_\mu\theta-\frac{1}{e}\D^\mu\theta A_\mu
    +\frac{1}{e^2}\qty(\D^\mu\theta)\qty(\D_\mu\theta)
  \end{align*}
  Multiply with the $\nu$ term:
  \begin{align*}
    A^\mu A_\mu A^\nu A_\nu=&
    \qty(A^\mu A_\mu-\frac{1}{e}A^\mu\D_\mu\theta-\frac{1}{e}\D^\mu\theta A_\mu
    +\frac{1}{e^2}\qty(\D^\mu\theta)\qty(\D_\mu\theta))\\\times&
    \qty(A^\nu A_\nu-\frac{1}{e}A^\nu\D_\nu\theta-\frac{1}{e}\D^\nu\theta A_\nu
    +\frac{1}{e^2}\qty(\D^\nu\theta)\qty(\D_\nu\theta))
  \end{align*}
  Due to the distinct lack of opposing signs in the terms containing $\D^\mu\theta$, there is no reason why any of this should cancel. So it would not describe massless photons. 
\item This term will transform as:
  \begin{align*}
    A^\mu A^\nu\D_\mu A_\nu&\to
    \qty(A^\mu-\frac{1}{e}\D^\mu\theta)
    \qty(A^\nu-\frac{1}{e}\D^\nu\theta)
    \D_\mu\qty(A_\nu-\frac{1}{e}\D_\nu\theta)\\
    &=\qty(A^\mu A^\nu-\frac{1}{e}A^\mu\D^\nu\theta-\frac{1}{e}A^\nu\D^\mu\theta
    +\frac{1}{e^2}\D^\mu\theta\D^\nu\theta)\\
    &\times\qty(\D_\mu A_\nu-\frac{1}{e}\D_\mu\D_\nu\theta)\\
    &=A^\mu A^\nu\D_\mu A_\nu-\frac{1}{e}A^\mu A^\nu\D_\mu\D_\nu\theta
    -\frac{1}{e}A^\mu\D^\nu\theta\D_\mu A_\nu
    +\frac{1}{e^2}A^\mu\D^\nu\theta\D_\mu\D_\nu\theta\\
    &-\frac{1}{e}A^\nu\D^\mu\theta\D_\mu A_\nu
    +\frac{1}{e^2}A^\nu\D^\mu\theta\D_\mu\D_\nu\theta
    +\frac{1}{e^2}\D^\mu\theta\D^\nu\theta\D_\mu A_\nu
    -\frac{1}{e^3}\D^\mu\theta\D^\nu\theta\D_\mu\D_\nu\theta
  \end{align*}
  Even if we could find a way where all the $\frac{1}{e}$ or $\frac{1}{e^2}$ terms could cancel, there is no term for the $\frac{1}{e^3}$ to cancel with, so this term would not describe massless photons.
\item The field strength tensor is defined as:
  \begin{align*}
    F_\munu=\D_\mu A_\nu-\D_\nu A_\mu
  \end{align*}
  So it is antisymmetric in its indices.
  \begin{align*}
    F^\mu_\nu&\to\D^\mu (A_\nu-\frac{1}{e}\D_\nu\theta)-\D_\nu
    (A^\mu-\frac{1}{e}\D^\mu\theta)\\
    &=\D^\mu A_\nu-\frac{1}{e}\D^\mu\D_\nu\theta+\frac{1}{e}\D_\nu\D^\mu\theta
    -\D_\nu A^\mu\\
    &=\D^\mu A_\nu-\D_\nu A^\mu=\boxed{F^\mu_\nu}
  \end{align*}
  Since one strength tensor individually transforms as itself, we can assume the others will, so the entire product will be invariant, and hence would still describe massless photons.

  Another idea I had for this problem: we know that the trace of an antisymmetric quantity is 0, and this product is identical to a trace. So we simply need to find if this product is if the product is antisymmetric. However I am not sure how that exactly would be done. I believe it is true that it should be antisymmetric in this case, I am not exactly sure how it would be proven. 
\item This is the square of the contraction we see in the Maxwell-Dirac Lagrangian. So it should be invariant under $U(1)$ and thus should describe massless photons. 
\item With the commutator written out this term becomes:
  \begin{align*}
    \bar{\psi}\sigma^\munu\psi F_\munu=
    \bar{\psi}\qty(\gamma^\mu\gamma^\nu-\gamma^\nu\gamma^\mu)\psi
    \qty(\D_\mu A_\nu-\D_\nu A_\mu)
  \end{align*}
  We already know $F_\munu$ transforms as itself, and the rest will transform as:
  \begin{align*}
    \bar{\psi}\sigma^\munu\psi F_\munu\to\bar{\psi}e^{-iQ\theta}
    \sigma^\munu e^{iQ\theta}\psi F_\munu
  \end{align*}
  However, since we are dealing with a scalar field, we need not worry about interchanging the sigma and exponential, giving:
  \begin{align*}
    \bar{\psi}\sigma^\munu\psi F_\munu\to\bar{\psi}
    \sigma^\munu \psi F_\munu
  \end{align*}
  So it could describe massless photons. 
\item This term should be able to describe massless photons since not only does it not contain the $A$ field at all, but also it is simply the square of the conserved dirac current, so it should be invariant under $U(1)$ anyways. 
\end{enumerate}
\section*{Problem 2}
The Lagrangian is:
\begin{align*}
  \L=\qty[(\D_\mu-ieQA_\mu)\phi^*]\qty[(\D^\mu+ieQA^\mu)\phi]
  -m^2\phi^*\phi-\frac{1}{4}F^\munu F_\munu
\end{align*}
\begin{enumerate}[label=\alph*)]
\item The Lagrangian will be gauge invariant if the terms remain the same if we introduce the following transformations to $\phi$ and $A_\mu$:
  \begin{gather*}
    \phi\to e^{iQ\theta}\phi\qquad\phi^*\to \phi^*e^{-iQ\theta} \\
    A_\mu\to A_\mu-\frac{1}{e}\D_\mu\theta
  \end{gather*}
  We can start with the mass term:
  \begin{align*}
    m^2\phi^*\phi\to\phi^*e^{-iQ\theta}e^{iQ\theta}\phi=\boxed{m^2\phi^*\phi}
  \end{align*}
  So this term is invariant. Next we should do the strength tensor term:
  \begin{align*}
    F_{\mu\nu}F^{\mu\nu}=& (\D_\mu A_\nu-\D_\nu A_\mu)(\D^\mu A^\nu-\D^\nu A^\mu)\\
    =& \qty(\D_\mu\qty(A_\nu-\frac{1}{e}\D_\nu\theta)
    -\D_\nu\qty(A_\mu-\frac{1}{e}\D_\mu\theta))\\
    \times& \qty(\D^\mu\qty(A^\nu-\frac{1}{e}\D^\nu\theta)
    -\D^\nu\qty(A^\mu-\frac{1}{e}\D^\mu\theta))\\
    =& \qty(\D_\mu A_\nu-\frac{1}{e}\D_\mu\D_\nu\theta
    -\D_\nu A_\mu-\frac{1}{e}\D_\nu\D_\mu\theta)\\
    \times& \qty(\D^\mu A^\nu-\frac{1}{e}\D^\mu\D^\nu\theta
    -\D^\nu A^\mu-\frac{1}{e}\D^\nu\D^\mu\theta)
  \end{align*}
  Notice how in both terms that there is a second derivative of $\theta$, and since second derivatives are interchangeable then we can simple cancel them out in both cases, leaving us with:
  \begin{align*}
    F_{\mu\nu}F^{\mu\nu}\to\qty(\D_\mu A_\nu-\D_\nu A_\mu)
    \qty(\D^\mu A^\nu-\D^\nu A^\mu)=\boxed{F_{\mu\nu}F^{\mu\nu}}
  \end{align*}
  All that remains is the covariant derivative term:
  \begin{align*}
    (\D^\mu+ieQA^\mu)\phi&\to\qty(\D^\mu+ieQ\qty[A^\mu-\frac{1}{e}\D^\mu\theta])
    e^{iQ\theta}\phi\\
    &=e^{iQ\theta}(\D^\mu\phi)+\qty(iQe^{iQ\theta}\D^\mu\theta)\phi+
    ieQA^\mu e^{iQ\theta}\phi-(iQe^{iQ\theta}\D^\mu\theta)\phi\\
    &=e^{iQ\theta}\qty(\D^\mu+ieQA^\mu)\phi
  \end{align*}
  So just like when we did the mass term, the complex conjugate term should cancel out since we have a $(D_\mu\phi)^*D^\mu\phi$ term. Therefore the entire Lagrangian is invariant under this gauge transoformation. 
\item The field for $\phi$ is obtained by using the Euler-Lagrange equations and varying with respect to the $\phi^*$ field:
  \begin{align*}
    \pdv{\L}{\phi^*}=\D_\mu\pdv{\L}{(\D_\mu\phi^*)}
  \end{align*}
  The first term only affects two terms, one in the covariant derivative and then the mass term:
  \begin{align*}
    \pdv{\L}{\phi^*}&=-ieA_\mu\qty(\D^\mu+ieQA^\mu)\phi-m^2\phi\\
    &=-ieA_\mu\D^\mu\phi+e^2QA_\mu A^\mu\phi-m^2\phi
  \end{align*}
  Next with respect to the spacetime derivatives:
  \begin{align*}
    \pdv{\L}{(\D_\mu\phi^*)}&=\qty(\D^\mu+ieQA^\mu)\phi\\
    &=\D^\mu\phi+ieQA^\mu\phi\\
    \D_\mu\pdv{\L}{(\D_\mu\phi^*)}&=\D^2\phi+\D_\mu\qty(ieQA^\mu\phi)
  \end{align*}
  Equating them will give the equations of motion:
  \begin{align*}
    \D^2\phi+\D_\mu(ieQA^\mu\phi)=-ieA_\mu\D^\mu\phi+e^2QA_\mu A^\mu\phi-m^2\phi
  \end{align*}
  This does in fact include a massive Klein-Gordon Equation:
  \begin{align*}
    \D^2\phi+m^2\phi=
    -\D_\mu\qty(ieQA^\mu\phi)-ieA_\mu\D^\mu\phi+e^2QA_\mu A^\mu\phi
  \end{align*}
\item For the $A^\mu$ field, we only need to consider the $A^\mu$ field:
  \begin{align*}
    \pdv{\L}{A_\mu}=\D_\nu\pdv{\L}{(\D_\nu A_\mu)}
  \end{align*}
  The left hand side:
  \begin{align*}
    \pdv{\L}{A_\mu}&=-ieQ\phi^*\qty(\D^\mu+ieQA^\mu)\phi+
    \pdv{A_\mu}\qty(ieQ\D_\mu\phi^*A^\mu\phi)\\
    &=-ieQ\phi^*D^\mu\phi+ieQ(\D_\mu\phi^*)\phi
  \end{align*}
  This is because $F_\munu$ only has derivatives of $A_\mu$, so the right hand side is:
  \begin{align*}
    \pdv{\L}{(\D_\nu A_\mu)}&=\frac{1}{4}\pdv{(\D_\nu A_\mu)}
    \qty(\D^\mu A^\nu-\D^\nu A^\mu)(\D_\mu A_\nu-\D_\nu A_\mu)\\
    &=-\frac{1}{4}\qty(\D^\mu A^\nu-\D^\nu A^\mu)\\
    \D_\nu\pdv{\L}{(\D_\nu A_\mu)}&=
    -\frac{1}{4}\qty(\D_\nu\D^\mu A^\nu-\D_\nu\D^\nu A^\mu)\\
    &=-\D_\nu F^\munu
  \end{align*}
  Setting them equal:
  \begin{align*}
    \D_\nu F^\munu=ieQ\D_\mu(\phi^*\phi)-e^2Q^2\phi^*A^\mu\phi
  \end{align*}
\item This is a local $U(1)$ so our variations are given by:
  \begin{align*}
    \delta\phi=iQ\theta\phi\quad
    \delta A_\mu=-\frac{1}{e}\D_\mu\theta
  \end{align*}
  The variation in the Lagrangian density:
  \begin{align*}
    \delta\L&=\D_\mu\qty(\pdv{\L}{(\D_\mu\phi)}\delta\phi)+
    \D_\mu\qty(\pdv{\L}{(\D_\mu\phi^*)}\delta\phi^*)
    +\D_\mu\qty(\pdv{\L}{(\D_\mu A_\nu)}\delta A_\nu)\\
    &=\D_\mu\qty(D_\mu\phi^*iQ\theta\phi-\phi^*D_\mu\phi iQ\theta
    -\frac{1}{e}\D_\mu F^\munu\D_\nu\theta)
  \end{align*}
  The last term should cancel out, as it is a symmetric term, hence we get
  \begin{align*}
    \delta\L=iQ\theta\D_\mu\qty(\phi D_\mu\phi-\phi^* D_\mu\phi)
  \end{align*}
  Which is our Noether current:
  \begin{align*}
    J_\mu=iQ\theta\qty(\phi D_\mu\phi-\phi^* D_\mu\phi)
  \end{align*}
\item Given that variations of the lagrangian should be $0$, the above should be:
  \begin{align*}
    \delta\L=iQ\theta\D_\mu\qty(\phi D_\mu\phi-\phi^* D_\mu\phi)=0
  \end{align*}
  Which is:
  \begin{align*}
    \D_\mu J^\mu=0
  \end{align*}
  Which means our Noether current is conserved.
\end{enumerate}
\section*{Problem 3}
The Maxwell-Dirac Lagrangian is:
\begin{align*}
  \L=i\bar{\psi}\gamma^\mu\qty(\D_\mu+ieQA_\mu)\psi-
  m\bar{\psi}\psi-\frac{1}{4}F_\munu F^\munu
\end{align*}
\begin{enumerate}[label=\alph*)]
\item If we are finding the field for $\bar{\psi}$ we need to vary with respect to $\psi$ in our Lagrangian, so we can ignore the field strength terms:
  \begin{align*}
    \pdv{\L}{\psi}=-i\bar{\psi}\gamma^\mu ieQA_\mu+m\bar{\psi}
  \end{align*}
  Note that the following derivative is not as expected:
  \begin{align*}
    \pdv{\psi}\qty(\bar{\psi}\psi)=-\bar{\psi}
  \end{align*}
  Since we have to interchange the derivative with the fermionic term $\bar{\psi}$, which requires an anticommutator and thus an extra minus sign, the same is done to the term from the covariant derivative.

  Next the derivative term:
  \begin{align*}
    \D_\mu\pdv{\L}{(\D_\mu\psi)}=\D_{\mu}\qty(-i\bar{\psi}\gamma^\mu)
  \end{align*}
  Setting them equal:
  \begin{align*}
    i\D_\mu\bar{\psi}\gamma^\mu&=-ieQ\bar{\psi}i\gamma^\mu A_\mu+m\bar{\psi}\\
    m\bar{\psi}&=\bar{\psi}\qty(i\D_\mu\gamma^\mu+ieQi\gamma^\mu A_\mu)\\
    &=\bar{\psi}(\D_\mu-ieQA_\mu)i\gamma^\mu
  \end{align*}
  Which certainly looks like the Dirac equation:
  \begin{align*}
    \bar{\psi}(-\D_\mu+ieQA_\mu)i\gamma^\mu=m\bar{\psi}
  \end{align*}
\item Taking the conjugate would invert the sign of $i$s and give us:
  \begin{align*}
    i\gamma^\mu(\D_\mu+ieQA_\mu)\psi=m\psi
  \end{align*}
  Which is the equation of motion which we derived in class.
\item Rearranging our field equations gives us a set of two equations of the following form:
  \begin{align*}
    \gamma^\mu\D_\mu\psi&=-im\psi-ieQ\gamma^\mu A_\mu\psi\\
    \D_\mu\bar{\psi}\gamma^\mu&=-im\bar{\psi}-ieQ\bar{\psi}A_\mu\gamma^\mu \\
  \end{align*}
  And current conservation entails that:
  \begin{align*}
    \D_\mu J^\mu=Q\qty((\D_\mu\bar{\psi})\gamma^\mu\psi
    +\bar{\psi}\gamma^\mu\qty(\D_\mu\psi))
  \end{align*}
  I assume that since there is a $\gamma^0$ from $\bar{\psi}$ as well as a $\gamma^\mu$ we do not need to flip signs on the second term.

  We can then sub in our relation to get the following:
  \begin{align*}
    \D_\mu J^\mu=Q\qty(-im\bar{\psi}\psi-ieQ\bar{\psi}A_\mu\gamma^\mu\psi
    -im\bar{\psi}\psi-ieQ\bar{\psi}\gamma^\mu A_\mu\psi)
  \end{align*}
  Interchanging everything ends up with the right amount of minus signs to cancel everything out, leading to current conservation.
  \begin{align*}
    \boxed{\D_\mu J^\mu=0}
  \end{align*}
\end{enumerate}
\end{document}