\documentclass[12pt]{article}

\title{\vspace{-3em}PHYS 546 HW 1}
\author{Michael Cardiff}
\date{\today}

%% science symbols 
\usepackage{amsmath}
\usepackage{amssymb}
\usepackage{physics}

%% general pretty stuff
\usepackage{bm}
\usepackage{enumitem}
\usepackage{float}
\usepackage{graphicx}
\usepackage[margin=1in]{geometry}

% figures
\graphicspath{ {./figs/} }

\newcommand{\fig}[3]
{
  \begin{figure}[H]
    \centering
    \includegraphics[width=#1cm]{#2}
    \caption{#3}
  \end{figure}
}

\newcommand{\figref}[4]
{
  \begin{figure}[H]
    \centering
    \includegraphics[width=#1cm]{#2}
    \caption{#3}
    \label{#4}
  \end{figure}
}

\renewcommand{\L}{\mathcal{L}}
\newcommand{\D}{\partial}
\newcommand{\munu}{{\mu\nu}}

\begin{document}
\maketitle

\section*{Problem 1}
A mass term in a lagrangian will look like the following:
\begin{align*}
  \L_{mass}=\frac{1}{2}m^2\phi^2
\end{align*}
This is for an example field $\phi$ which is its own antiparticle. So in order for the Maxwell-Dirac Lagrangian to have a mass term for the photon field we must see something like:
\begin{align*}
  \frac{1}{2}m_\gamma^2A^\mu A_\mu
\end{align*}
Where $m_\gamma$ would be the mass of the photon. However this is not the only way a mass term can come, since out of all of these there are no explicit second order terms of the $A$ field. It can also come from the breaking of the local $U(1)$ symmetry of the theory, similar to how pions have mass. The transformations of this theory are the following:
\begin{gather*}
  \psi\to e^{iQ\theta}\psi\qquad\bar{\psi}\to \bar{\psi}e^{-iQ\theta}\\
  A_\mu\to A_\mu-\frac{1}{e}\D_\mu\theta
\end{gather*}
So if the term is invariant under these transformations, it can still produce massless photons.
\begin{enumerate}[label=\alph*)]
\item By symmetry of indices, if $A^\mu A_\mu$ is invariant, so will $A^\nu A_\nu$:
  \begin{align*}
    A^\mu A_\mu&\to\qty(A^\mu-\frac{1}{e}\D^\mu\theta)
    \qty(A_\mu-\frac{1}{e}\D_\mu\theta)\\
    &=A^\mu A_\mu-\frac{1}{e}A^\mu\D_\mu\theta-\frac{1}{e}\D^\mu\theta A_\mu
    +\frac{1}{e^2}\qty(\D^\mu\theta)\qty(\D_\mu\theta)
  \end{align*}
\end{enumerate}
\section*{Problem 2}
The Lagrangian is:
\begin{align*}
  \L=\qty[(\D_\mu-ieQA_\mu)\phi^*]\qty[(\D^\mu+ieQA^\mu)\phi]-m^2\phi^*\phi-\frac{1}{4}F^\munu F_\munu
\end{align*}
\begin{enumerate}[label=\alph*)]
\item The Lagrangian will be gauge invariant if the terms remain the same if we introduce the following transformations to $\phi$ and $A_\mu$:
  \begin{gather*}
    \phi\to e^{iQ\theta}\phi\qquad\phi^*\to \phi^*e^{-iQ\theta} \\
    A_\mu\to A_\mu-\frac{1}{e}\D_\mu\theta
  \end{gather*}
  We can start with the mass term:
  \begin{align*}
    m^2\phi^*\phi\to\phi^*e^{-iQ\theta}e^{iQ\theta}\phi=\boxed{m^2\phi^*\phi}
  \end{align*}
  So this term is invariant. Next we should do the strength tensor term:
  \begin{align*}
    F_{\mu\nu}F^{\mu\nu}=& (\D_\mu A_\nu-\D_\nu A_\mu)(\D^\mu A^\nu-\D^\nu A^\mu)\\
    =& \qty(\D_\mu\qty(A_\nu-\frac{1}{e}\D_\nu\theta)
    -\D_\nu\qty(A_\mu-\frac{1}{e}\D_\mu\theta))\\
    \times& \qty(\D^\mu\qty(A^\nu-\frac{1}{e}\D^\nu\theta)
    -\D^\nu\qty(A^\mu-\frac{1}{e}\D^\mu\theta))\\
    =& \qty(\D_\mu A_\nu-\frac{1}{e}\D_\mu\D_\nu\theta
    -\D_\nu A_\mu-\frac{1}{e}\D_\nu\D_\mu\theta)\\
    \times& \qty(\D^\mu A^\nu-\frac{1}{e}\D^\mu\D^\nu\theta
    -\D^\nu A^\mu-\frac{1}{e}\D^\nu\D^\mu\theta)
  \end{align*}
  Notice how in both terms that there is a second derivative of $\theta$, and since second derivatives are interchangeable then we can simple cancel them out in both cases, leaving us with:
  \begin{align*}
    F_{\mu\nu}F^{\mu\nu}\to\qty(\D_\mu A_\nu-\D_\nu A_\mu)
    \qty(\D^\mu A^\nu-\D^\nu A^\mu)=\boxed{F_{\mu\nu}F^{\mu\nu}}
  \end{align*}
  All that remains is the covariant derivative term:
  \begin{align*}
    (\D^\mu+ieQA^\mu)\phi&\to\qty(\D^\mu+ieQ\qty[A^\mu-\frac{1}{e}\D^\mu\theta])
    e^{iQ\theta}\phi\\
    &=e^{iQ\theta}(\D^\mu\phi)+\qty(iQe^{iQ\theta}\D^\mu\theta)\phi+
    ieQA^\mu e^{iQ\theta}\phi-(iQe^{iQ\theta}\D^\mu\theta)\phi\\
    &=e^{iQ\theta}\qty(\D^\mu+ieQA^\mu)\phi
  \end{align*}
  So just like when we did the mass term, the complex conjugate term should cancel out since we have a $(D_\mu\phi)^*D^\mu\phi$ term. Therefore the entire Lagrangian is invariant under this gauge transoformation. 
\item The field for $\phi$ is obtained by using the Euler-Lagrange equations and varying with respect to the $\phi^*$ field:
  \begin{align*}
    \pdv{\L}{\phi^*}=\D_\mu\pdv{\L}{(\D_\mu\phi^*)}
  \end{align*}
  The first term only affects two terms, one in the covariant derivative and then the mass term:
  \begin{align*}
    \pdv{\L}{\phi^*}&=-ieA_\mu\qty(\D^\mu+ieQA^\mu)\phi-m^2\phi\\
    &=-ieA_\mu\D^\mu\phi+e^2QA_\mu A^\mu\phi-m^2\phi
  \end{align*}
  Next with respect to the spacetime derivatives:
  \begin{align*}
    \pdv{\L}{(\D_\mu\phi^*)}&=\qty(\D^\mu+ieQA^\mu)\phi\\
    &=\D^\mu\phi+ieQA^\mu\phi\\
    \D_\mu\pdv{\L}{(\D_\mu\phi^*)}&=\D^2\phi+\D_\mu\qty(ieQA^\mu\phi)
  \end{align*}
  Equating them will give the equations of motion:
  \begin{align*}
    \D^2\phi+\D_\mu(ieQA^\mu\phi)=-ieA_\mu\D^\mu\phi+e^2QA_\mu A^\mu\phi-m^2\phi
  \end{align*}
\end{enumerate}
\section*{Problem 3}

\end{document}