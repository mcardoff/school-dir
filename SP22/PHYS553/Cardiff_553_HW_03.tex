\documentclass[12pt]{article}

\title{\vspace{-3em}PHYS 553 HW 3}
\author{Michael Cardiff}
\date{\today}

%% science symbols 
\usepackage{amsmath}
\usepackage{amssymb}
\usepackage{physics}

%% general pretty stuff
\usepackage{bm}
\usepackage{enumitem}
\usepackage{float}
\usepackage{graphicx}
\usepackage[margin=1in]{geometry}

% figures
\graphicspath{ {./figs/} }

\newcommand{\fig}[3]{\begin{figure}[H]\centering\includegraphics[width=#1cm]{#2}\caption{#3}\end{figure}}

\newcommand{\figref}[4]
{\begin{figure}[H]\centering\includegraphics[width=#1cm]{#2}\caption{#3}\label{#4}\end{figure}}

\renewcommand{\L}{\mathcal{L}}
\newcommand{\cD}{\mathcal{D}}
\newcommand{\D}{\partial}
\newcommand{\q}{\vb{q}}
\newcommand{\p}{\vb{p}}
\newcommand{\phis}{\phi^*}
\newcommand{\bphis}{\bar{\phi}^*}
\newcommand{\bphi}{\bar{\phi}}

\begin{document}
\maketitle

\section{Charged Particle Path Integral}
Note the book uses $\vb{r}$ where here $\q$ is used.

As a reminder, here is the Hamiltonian as well as the field $\vb{A}$:
\begin{gather*}
  H(\q,\p)=\frac{1}{2m}\qty(\p+\frac{e}{c}\vb{A}(\q))^2\\
  A_1(\q)=-\frac{B}{2}q_2\\ A_2(\q)=\frac{B}{2}q_1
\end{gather*}

\subsection{Transition Amplitude for $\vb{q}_i=\vb{q}_f$}
Our aim is the following:
\begin{align*}
  \ip{\q_0,t_f}{\q_0,t_i}
\end{align*}
In class we derived the following formula for this inner product:
\begin{align*}
  \ip{\q_0,t_f}{\q_0,t_i}&=\int\cD q_1\cD q_2\cD p_1\cD p_2
  \exp{\frac{i}{\hbar}S}\\
  &=\int\cD\q\cD\p\exp{\frac{i}{\hbar}S}
\end{align*}
The action in this case is simple:
\begin{align*}
  S&=\int\dd{t}L
\end{align*}
The Lagrangian is given in terms of the Hamiltonian:
\begin{align*}
  L=\p\vdot\dot{\q}-H
\end{align*}
Expanding the Hamiltonian:
\begin{align*}
  H&=\frac{1}{2m}\qty(\p\vdot\p+\frac{e^2}{c^2}\vb{A\vdot A}+
  2\frac{e}{c}\p\vdot\vb{A})\\
  &=\frac{1}{2m}\qty(p_1^2+p_2^2+
  \frac{e^2}{c^2}\qty(\frac{B^2}{4}x_2^2+\frac{B^2}{4}q_1^2)+
  2\frac{e}{c}\qty(-\frac{B}{2}p_1q_2+\frac{B}{2}p_2q_1))\\
  &=\frac{1}{2m}\qty(p_1^2+p_2^2+\qty(\frac{eB}{2c})^2\qty(q_1^2+q_2^2)
  +\frac{eB}{c}\qty(p_2q_1-p_1q_2))\\
  &=\frac{1}{2m}\qty(p_1^2+p_2^2)
  +\frac{1}{2m}\qty(\frac{eB}{2c})^2\qty(q_1^2+q_2^2)
  +\frac{eB}{2mc}\qty(p_2q_1-p_1q_2)
\end{align*}
The transformation part is given by:
\begin{align*}
  \p\vdot\q=p_1\dot{q}_1+p_2\dot{q}_2
\end{align*}
We ultimately want to integrate out the $\p$ variables, so we should organize in terms of powers of $p$:
\begin{align*}
  \p\vdot\q-H=-\frac{p_1^2}{2m}+p_1\qty(\dot{q}_1+\frac{eB}{2mc}q_2)
  -\frac{p_2^2}{2m}+p_2\qty(\dot{q}_2-\frac{eB}{2mc}q_1)
  -\frac{1}{2m}\qty(\frac{eB}{2c})^2\qty(q_1^2+q_2^2)
\end{align*}
We should look at one of these terms, and see how we can write it as a complete square:
\begin{align*}
  -\frac{p_1^2}{2m}+p_1\qty(\dot{q}_1+\frac{eB}{2mc}q_2)=
  -\frac{1}{2m}\qty[p_1^2-p_1\qty(2m\dot{q}_1+\frac{eB}{c}q_2)]
\end{align*}

\paragraph{Completing the Square}
To complete the square, we need the following form:
\begin{align*}
  (p+a)^2=p^2+2pa+a^2
\end{align*}
In our case, we identify the constant $2a$ with the coefficient of the linear term:
\begin{align*}
  p_1^2-p_1\qty(2m\dot{q}_1+\frac{eB}{c}q_2)
  =\qty(p_1-\qty(m\dot{q}_1+\frac{eB}{2c}q_2))^2
  -\qty(m\dot{q}_1+\frac{eB}{2c}q_2)^2
\end{align*}
The overall term is then:
\begin{align*}
  -\frac{1}{2m}
  &\left[
    \qty(p_1-\qty(m\dot{q}_1+\frac{eB}{2c}q_2))^2+
    \qty(p_2-\qty(m\dot{q}_2-\frac{eB}{2c}q_1))^2
  \right. \\
  &\left.
    -\qty(m\dot{q}_1+\frac{eB}{2c}q_2)^2
    -\qty(m\dot{q}_2-\frac{eB}{2c}q_1)^2 \right]
\end{align*}
Where we have taken into account the opposite signs of the two components of $\vb{A}$. After Integrating out the $p$ variables (Which we can do because the argument of all this is quadratic in either $p_1$ or $p_2$) we can absorb the result into a normalization, leaving us with:
\begin{align*}
  &-\frac{1}{2m}\qty(\frac{eB}{2c})^2(q_1^2+q_2^2)
  +\frac{1}{2m}\qty(m\dot{q}_1+\frac{eB}{2c}q_2)^2
  +\frac{1}{2m}\qty(m\dot{q}_2-\frac{eB}{2c}q_1)^2\\
  =&-\frac{1}{2m}\qty(\frac{eB}{2c})^2(q_1^2+q_2^2)
  +\frac{1}{2}m\dot{q}_1^2+\frac{e^2B^2}{8mc^2}q_2^2+\frac{eB}{4c}\dot{q}_1q_2
  +\frac{1}{2}m\dot{q}_2^2+\frac{e^2B^2}{8mc^2}q_1^2-\frac{eB}{4c}\dot{q}_2q_1\\
  =&\frac{1}{2}m\qty(\dot{q}_1^2+\dot{q}_2^2)
  -\frac{eB}{4c}\qty(\dot{q}_2q_1-\dot{q}_1q_2)\\
  =&\frac{1}{2}m\dot{\q}^2-\frac{e}{2c}\dot{\q}\vdot\vb{A}
\end{align*}
So that our path integral becomes:
\begin{align*}
  \boxed{\ip{\q_0,t_f}{\q_0,t_i}=\int\cD{\q}
  \exp{\frac{i}{\hbar}\int_{t_i}^{t_f}\dd{t}
    \qty[\frac{1}{2}m\dot{\q}^2-\frac{e}{2c}\dot{\q}\vdot\vb{A}]}}
\end{align*}
\subsection{Ultra-Quantum Limit}
Luckily with our action, we were able to eliminate all terms proportional to one over the mass, so the action becomes:
\begin{align*}
  \lim_{m\to0}S&=-\int\dd{t}\frac{e}{2c}\dot{\q}\vdot\vb{A}\\
  &=-\frac{e}{2c}\int\dd{t}\dv{\q}{t}\vdot\vb{A}\\
  &=-\frac{e}{2c}\int\dd{\q}\vdot\vb{A}
\end{align*}
Since $\q$ is a path, this is a line integral, but it is over a closed loop, so it defines the boundary of a surface, so we can use the following form of Stokes' theorem:
\begin{align*}
  \oint_{\partial\Sigma}\dd{\vb{r}}\vdot\vb{V}=
  \int_\Sigma\dd{\vb{S}}\vdot\qty(\curl{\vb{V}})
\end{align*}
So our integral becomes:
\begin{align*}
  -\frac{e}{2c}\oint_{\partial\Sigma}\dd{\q}\vdot\vb{A}
  &=-\frac{e}{2c}\int_\Sigma\dd{\vb{S}}\vdot\vb{B}
  =-\frac{e}{2c}\int_\Sigma\dd{\vb{S}}\vdot(B\vu{n})\\
  &=-\frac{eB}{2c}\int_\Sigma\dd{S_n}=\mp\frac{eB}{2c}A(\Sigma)
\end{align*}
Where $\vb{B}$ is perpendicular as opposed to along $z$ because of the next part, and the surface can either be oriented along or against the surface:
\begin{align*}
  \boxed{\lim_{m\to0}S=\mp\frac{eB}{2c}A(\Sigma)}
\end{align*}
\subsection{Particle on a Torus}
In the flat case, we have the normal axis as just the $\vu{z}$ unit vector, if the trajectory and surface have opposite signs, so we have a positive action. The amplitude is:
\begin{align*}
  e^{iS/\hbar}=\exp{\frac{eBi}{2\hbar c}A\qty(\Sigma)}
\end{align*}
From solid state, the constant can be simplified using the quantized magnetic flux $\Phi$:
\begin{align*}
  \frac{e}{2c}=\frac{h}{2e}=BA=\Phi
\end{align*}
Rewriting our amplitude:
\begin{align*}
  \exp{\frac{eBi}{2\hbar c}A\qty(\Sigma)}=\exp{\frac{2\pi i}{hc/e}B A(\Sigma)}
\end{align*}
Note that since we are considering the trajectory, the $A$ is the area inside the trajectory.

If we consider this instead to be a Torus, we have the following boundary conditions for our space:
\begin{align*}
  q_i+L=q_i
\end{align*}
\section{Scalar Theory Path Integral}
The action we are working with is:
\begin{align*}
  S=\int\dd[4]{x}\qty(\D_\mu\phi^*\D^\mu\phi-m^2\phi^*\phi-J^*\phi-J\phi^*)
\end{align*}
\subsection{Source Vacuum Persistence}
The source vacuum persistence in Minkowski space is simply:
\begin{align*}
  _J\ip{0}{0}_J=Z_M[J,J^*]&=\int\cD\phi\cD\phi^*\exp{\frac{i}{\hbar}S}\\
  &=\boxed{\int\cD\phi\cD\phi^*\exp{\frac{i}{\hbar}
    \int\dd[4]{x}\qty(\D_\mu\phi^*\D^\mu\phi-m^2\phi^*\phi-J^*\phi-J\phi^*)}}
\end{align*}
To get the Euclidean persistence $Z_E$, we replace $x^0\to -ix^4$, and define the new gradient $\grad_\mu$ with $\mu=1,2,3,4$ such that the action is:
\begin{align*}
  iS=-\int\dd[4]{x}\qty(\grad_\mu\phis\grad_\mu\phi+m^2\phis\phi+J^*\phi+J\phis)
\end{align*}
Which gives the Euclidean vacuum persistence as:
\begin{align*}
  Z_E[J,J^*]=&=\boxed{\int\cD\phi\cD\phi^*\exp{-\frac{1}{\hbar}
    \int\dd[4]{x}\qty(\grad_\mu\phis\grad_\mu\phi+m^2\phis\phi+J^*\phi+J\phis)}}
\end{align*}
\subsubsection{Evaluating The Above}
For Minkowski space, we separate each of the equations into a classical solution plus a perturbation $\eta$:
\begin{align*}
  \phi=\bphi+\eta\qquad \phis=\bphis+\eta^*
\end{align*}
The Minkowski Lagrangian is:
\begin{align*}
  \L=&\,\D_\mu\bphis\D^\mu\bphi+\D_\mu\bphi\D^\mu\eta^*+\D_\mu\bphis\D^\mu\eta
  + \D_\mu\eta^*\D^\mu\eta\\
  &-m^2\bphis\bphi-m^2\bphi\eta^*-m^2\bphis\eta-m^2\eta^*\eta\\
  &-J^*(\bphi+\eta)-J(\bphis+\eta^*)
\end{align*}
We want to integrate this out in order to separate into the individual klein gordon parts, and many of the parts with $\eta$ will vanish due to the boundary conditions of $\eta$:
\begin{align*}
  \L=\,&-\frac{1}{2}\bphis\qty(\D^2+m^2)\bphi
  -\frac{1}{2}\bphi\qty(\D^2+m^2)\bphis\\
  &-\eta\qty(\D^2+m^2)\bphis-\eta^*\qty(\D^2+m^2)\bphi\\
  &-\frac{1}{2}\eta^*\qty(\D^2+m^2)\eta-\frac{1}{2}\eta\qty(\D^2+m^2)\eta^*\\
  &-J^*\phi-J\phis
\end{align*}
The barred variables should obey the source equations when we act on the Klein-Gordon operator:
\begin{align*}
  \hat{A}\bphi=\qty(\D^2+m^2)\bphi&=-J\\
  \hat{A}\bphis=\qty(\D^2+m^2)\bphis&=-J^*
\end{align*}
We then have the Lagrangian as:
\begin{align*}
  \L=\,&-\frac{1}{2}\eta^*\hat{A}\eta-\frac{1}{2}\eta\hat{A}\eta^*\\
  &-\frac{1}{2}\qty(J^*\bphi+J\bphis)
\end{align*}
Note that the functions $\bphi/bphis$ define the Green function of $\hat{A}$ for a delta function source:
\begin{equation*}
  \qty(\D^2+m^2)G_0^{\mathcal{M}}(x-y)=\delta(x-y)
  \begin{aligned}
    &\implies\bphi=-\int\dd[4]{y}G_0^{\mathcal{M}}(x-y)J(y)\\
    &\implies\bphis=-\int\dd[4]{y}G_0^{\mathcal{M}}(x-y)J^*(y)
  \end{aligned}
\end{equation*}
We can then identify the Green function as a sort of inverse Laplace operator:
\begin{align*}
  G_0^{\mathcal{M}}(x-y)=\mel{x}{\frac{1}{\D^2+m^2}}{y}
\end{align*}
The Minkowski partition functional will be identical to the real scalar field one, but squared, and the source term will be a bit different:
\begin{align*}
  Z[J,J^*]&=\boxed{\mathcal{N}\qty(\det\hat{A})^{-1}
    \exp{\frac{i}{\hbar}\int\dd[4]{y}\dd[4]{x}J^*(x)G_0^{\mathcal{M}}(x-y)J(y)}}
\end{align*}
Note, we have the determinant of $A$ to the minus one instead of $\frac{1}{2}$ since we have the extra degree of freedom from $\phis$. This is also up to some normalization, which could be absorbed into $Z[0,0]$.

Analytically ccontinuing time would turn the $\D^2$ into a $\grad_\mu\grad_\mu$ instead of $\D_\mu\D^\mu$, call it $-\D^2$, and we would need a new Green function due to this:
\begin{align*}
  Z_E[J,J^*]&=\boxed{\mathcal{N}\qty(\det\qty(-\D^2+m^2))^{-1}
    \exp{\frac{i}{\hbar}\int\dd[4]{y}\dd[4]{x}J^*(x)G_0^E(x-y)J(y)}}\\
  G_0^E(x-y)&=\mel{x}{\frac{1}{-\D^2+m^2}}{y}
\end{align*}
\subsection{Two Point Functions}
\subsubsection{Minkowski}
The two point function is given as a variation depending on the ordering.
\begin{align*}
  G_2(x-y)&=\frac{1}{i^2}\frac{1}{Z[0,0]}
  \eval{\frac{\delta Z[J,J^*]}{\delta J^*(x)\delta J(y)}}_{J=0}
\end{align*}
Note that the normalization and the determinant will cancel out from our functions, leaving only the arguments of the exponentials:
\begin{align*}
  G_2=\frac{1}{i}\fdv{J^*(x)}\eval{
  \qty(\int\dd[4]{z}J^*(z)G_0^{\mathcal{M}}(z-y))
  \exp{i\int\dd[4]{z}\dd[4]{z'}J^*(y)G_0^{\mathcal{M}}(z-z')J(z')}}_{J=0}
\end{align*}
Differentiating with respect $J^*$ will leave only the Green function, with the $x$ variable in front due to symmetry:
\begin{align*}
  \boxed{G_2(x-y)=-iG_0^{\mathcal{M}}(x-y)}
\end{align*}
The next function is done similarly, except we reverse the order of the functional derivatives, which will end up adding in a reversal of the variables in the ening Green function:
\begin{align*}
  \boxed{G_2^*(x-y)=-iG_0^{\mathcal{M}}(y-x)}
\end{align*}
For these next two, the first functional derivative eliminates most of the $J^*$ except for in the exponentials, but they should cancel out when we take the second functional derivative, hence:
\begin{align*}
  \boxed{G_2^*(x-y)=G_2^{\prime*}(x-y)=0}
\end{align*}
\subsubsection{Euclidean}
The Euclidean functions should simply not have the factors of $i$:
\begin{equation*}
  \boxed{
    \begin{aligned}
      G_2^E(x-y)&=G_0^E(x-y)\\
      G_2^{E*}(x-y)&=G_0^E(y-x)\\
      G_2^{\prime E}(x-y)&=G_2^{\prime E*}(x-y)=0
    \end{aligned}
  }
\end{equation*}
\subsubsection{Equation Satisfying the Propogator}
The equation satisfied is that with a delta function source:
\begin{gather*}
  (\D_x^2+m^2)G_0^{\mathcal{M}}(x-y)=\delta(x-y)\\
  (-\D_x^2+m^2)G_0^{E}(x-y)=\delta(x-y)
\end{gather*}
Solving the Euclidean case is fairly simply done with a Fourier transform:
\begin{align*}
  (-\D^2_x+m^2)\int\frac{\dd[4]{p}}{(2\pi)^4}G_0^E(p)e^{ip\vdot(x-y)}=
  \int\frac{\dd[4]{p}}{(2\pi)^4}e^{ip\vdot(x-y)}
\end{align*}
The transform of the equation turns the derivative into $ip$s:
\begin{align*}
  \int\frac{\dd[4]{p}}{(2\pi)^4}(p^2+m^2)G_0^E(p)e^{ip\vdot(x-y)}=
  \int\frac{\dd[4]{p}}{(2\pi)^4}e^{ip\vdot(x-y)}
\end{align*}
Matching terms allows us to see that:
\begin{align*}
  G_0^E(p)=\frac{1}{p^2+m^2}
\end{align*}
Which means the Euclidean Green function is given by:
\begin{align*}
  G_0^E(x-y)=\int\frac{\dd[4]{p}}{(2\pi)^4}\frac{e^{ip\vdot(x-y)}}{p^2+m^2}
\end{align*}
This can be manipulated to a modified Bessel function if we introduce an integration parameter, but I am not sure how we end up there...
Mathematica gives the integral in terms of the Exponential integral $\mathrm{Ei}(x)$, which can certainly be turned into the combination required for the exponential which we need. 
\begin{align*}
  G_0^E(x-y)&=\frac{1}{4\pi^2}\qty(\frac{m}{\abs{x-y}})K_1(m\abs{x-y})\\
  K_\nu(z)&=\frac{1}{2}\int\dd{t}t^{\nu-1}\exp{-\frac{z}{2}\qty(t+\frac{1}{t})}
\end{align*}
For long distances, the modified Bessel function becomes:
\begin{align*}
  K_\nu(z)\approx\sqrt{\frac{\pi}{2z}}e^{-z}\implies
  \boxed{G_0^E\approx\frac{\sqrt{\pi}}{4\sqrt{2}\pi^2}\frac{m^2}{m^{3/2}}
  \frac{e^{-m\abs{x-y}}}{\abs{x-y}^{3/2}}}
\end{align*}
Short distances have the modified Bessel function as:
\begin{align*}
  K_\nu(z)\approx\frac{\Gamma(\nu)}{2}\qty(\frac{2}{z})^\nu
\end{align*}
Giving the much simpler Green function of:
\begin{align*}
  \boxed{G_0^E\approx\frac{1}{4\pi^2\abs{x-y}^2}}
\end{align*}
Analytically continuing back to Minkowski space:
\begin{align*}
  G_0^{\mathcal{M}}(x-y)=\frac{i}{4\pi^2}\frac{m}{\sqrt{-s^2}}K_1(m\sqrt{-s^2})
\end{align*}
We then go to spacelike vs timelike separations, spacelike separations will act the same as euclidean but with the usual oddities, for long separations:
\begin{align*}
  \boxed{G_0^{\mathcal{M}}\approx i\frac{\sqrt{\pi}}{4\sqrt{2}\pi^2}
    \frac{m^2}{m^{3/2}}\frac{e^{-m\sqrt{-s^2}}}{\sqrt{-s^2}^{3/2}}}
\end{align*}
For short ones:
\begin{align*}
  \boxed{G_0^{\mathcal{M}}\approx\frac{i}{4\pi^2(-s^2)}}
\end{align*}
For timelike separations we should get a Hankel function, and get very similar functions, but with complex exponentials as opposed to real ones.
\subsection{Four Point Functions}
Wick's theorem says we can just use two point functions to create the four point ones:
\begin{align*}
  
\end{align*}
\end{document}