\documentclass[12pt]{article}

\title{\vspace{-3em}PHYS 553 HW 3}
\author{Michael Cardiff}
\date{\today}

%% science symbols 
\usepackage{amsmath}
\usepackage{amssymb}
\usepackage{physics}

%% general pretty stuff
\usepackage{bm}
\usepackage{enumitem}
\usepackage{float}
\usepackage{graphicx}
\usepackage[margin=1in]{geometry}

% figures
\graphicspath{ {./figs/} }

\newcommand{\fig}[3]{\begin{figure}[H]\centering\includegraphics[width=#1cm]{#2}\caption{#3}\end{figure}}

\newcommand{\figref}[4]
{\begin{figure}[H]\centering\includegraphics[width=#1cm]{#2}\caption{#3}\label{#4}\end{figure}}

\renewcommand{\L}{\mathcal{L}}
\newcommand{\cD}{\mathcal{D}}
\newcommand{\q}{\vb{q}}
\newcommand{\p}{\vb{p}}

\begin{document}
\maketitle

\section{Charged Particle Path Integral}
Note the book uses $\vb{r}$ where here $\q$ is used.

As a reminder, here is the Hamiltonian as well as the field $\vb{A}$:
\begin{gather*}
  H(\q,\p)=\frac{1}{2m}\qty(\p+\frac{e}{c}\vb{A}(\q))^2\\
  A_1(\q)=-\frac{B}{2}q_2\\ A_2(\q)=\frac{B}{2}q_1
\end{gather*}

\subsection{Transition Amplitude for $\vb{q}_i=\vb{q}_f$}
Our aim is the following:
\begin{align*}
  \ip{\q_0,t_f}{\q_0,t_i}
\end{align*}
In class we derived the following formula for this inner product:
\begin{align*}
  \ip{\q_0,t_f}{\q_0,t_i}&=\int\cD q_1\cD q_2\cD p_1\cD p_2
  \exp{\frac{i}{\hbar}S}\\
  &=\int\cD\q\cD\p\exp{\frac{i}{\hbar}S}
\end{align*}
The action in this case is simple:
\begin{align*}
  S&=\int\dd{t}L
\end{align*}
The Lagrangian is given in terms of the Hamiltonian:
\begin{align*}
  L=\p\vdot\dot{\q}-H
\end{align*}
Expanding the Hamiltonian:
\begin{align*}
  H&=\frac{1}{2m}\qty(\p\vdot\p+\frac{e^2}{c^2}\vb{A\vdot A}+
  2\frac{e}{c}\p\vdot\vb{A})\\
  &=\frac{1}{2m}\qty(p_1^2+p_2^2+
  \frac{e^2}{c^2}\qty(\frac{B^2}{4}x_2^2+\frac{B^2}{4}x_1^2)+
  2\frac{e}{c}\qty(-\frac{B}{2}p_1q_2+\frac{B}{2}p_2q_1))\\
  &=\frac{1}{2m}\qty(p_1^2+p_2^2+\qty(\frac{eB}{2c})^2\qty(x_1^2+x_2^2)
  +\frac{eB}{c}\qty(p_2q_1-p_1q_2))\\
  &=\frac{1}{2m}\qty(p_1^2+p_2^2)
  +\frac{1}{2m}\qty(\frac{eB}{2c})^2\qty(x_1^2+x_2^2)
  +\frac{eB}{2mc}\qty(p_2q_1-p_1q_2)
\end{align*}
The transformation part is given by:
\begin{align*}
  \p\vdot\q=p_1\dot{q}_1+p_2\dot{q}_2
\end{align*}
We ultimately want to integrate out the $\p$ variables, so we should organize in terms of powers of $p$:
\begin{align*}
  \p\vdot\q-H=-\frac{p_1^2}{2m}+p_1\qty(\dot{q}_1+\frac{eB}{2mc}q_2)
  -\frac{p_2^2}{2m}+p_2\qty(\dot{q}_2-\frac{eB}{2mc}q_1)
  -\frac{1}{2m}\qty(\frac{eB}{2c})^2\qty(x_1^2+x_2^2)
\end{align*}
We should look at one of these terms, and see how we can write it as a complete square:
\begin{align*}
  -\frac{p_1^2}{2m}+p_1\qty(\dot{q}_1+\frac{eB}{2mc}q_2)=
  -\frac{1}{2m}\qty[p_1^2-p_1\qty(2m\dot{q}_1+\frac{eB}{c}q_2)]
\end{align*}

\paragraph{Completing the Square}
To complete the square, we need the following form:
\begin{align*}
  (p+a)^2=p^2+2pa+a^2
\end{align*}
In our case, we identify the constant $2a$ with the coefficient of the linear term:
\begin{align*}
  -\frac{1}{2m}\qty[p_1^2-p_1\qty(2m\dot{q}_1+\frac{eB}{c}q_2)]
  =-\frac{1}{2m}\qty[\qty(p_1-\qty(m\dot{q}_1+\frac{eB}{2c}q))^2
  -\qty(m\dot{q}_1+\frac{eB}{2c})]
\end{align*}
\subsection{Ultra-Quantum Limit}

\subsection{Particle on a Torus}

\section{Scalar Theory Path Integral}

\subsection{Source Vacuum Persistence}

\subsubsection{Evaluating The Above}

\subsection{Two Point Functions}

\subsection{Propagator}

\subsubsection{Euclidean}

\subsubsection{Minkowski}

\subsection{Four Point Functions}



\end{document}