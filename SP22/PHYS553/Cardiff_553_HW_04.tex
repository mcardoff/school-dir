\documentclass[12pt]{article}

\title{\vspace{-3em}PHYS 553 HW 4}
\author{Michael Cardiff\\Collaborated with Shayarneel Kundu}
\date{\today}

%% science symbols 
\usepackage{amsmath}
\usepackage{amssymb}
\usepackage{physics}

%% slash notation
\usepackage{slashed}

%% general pretty stuff
\usepackage{bm}
\usepackage{enumitem}
\usepackage{float}
\usepackage{graphicx}
\usepackage[margin=1in]{geometry}

% figures
\graphicspath{ {./figs/} }

\newcommand{\fig}[3]
{
  \begin{figure}[H]
    \centering
    \includegraphics[width=#1cm]{#2}
    \caption{#3}
  \end{figure}
}

\newcommand{\figref}[4]
{
  \begin{figure}[H]
    \centering
    \includegraphics[width=#1cm]{#2}
    \caption{#3}
    \label{#4}
  \end{figure}
}

\renewcommand{\L}{\mathcal{L}}
% integration measure
\newcommand{\cD}{\mathcal{D}}
% spacetime deriv
\newcommand{\D}{\partial}
% p,q as vecs
\newcommand{\q}{\vb{q}}
\newcommand{\p}{\vb{p}}
% fields
\newcommand{\phis}{\phi^*}
\newcommand{\bphis}{\bar{\phi}^*}
\newcommand{\bphi}{\bar{\phi}}
\newcommand{\psib}{\bar{\psi}}
\newcommand{\etab}{\bar{\eta}}
\newcommand{\xib}{\bar{\xi}}
\newcommand{\A}{\hat{A}}
\renewcommand{\d}{\delta}
\renewcommand{\a}{\hat{a}}

\begin{document}
\maketitle
\section{Grassman Variables}
The notation in the question is a bit confusing with respect to proper complex conjugates, the convention I used for the function is:
\begin{align*}
  f(a)\equiv f_0+f_1a
  f(a^*)^*\equiv \bar{f_0}+\bar{f_1}a
\end{align*}
\subsection{Inner Product}
The inner product makes use of the product of these two functions:
\begin{align*}
  f(a^*)^*g(a^*)=\qty(\bar{f_0}+\bar{f_1}a)(g_0+g_1a^*)
\end{align*}
Remembering that $f_0,f_1,g_0,g_1$ are all just complex numbers, we can write this product as:
\begin{align*}
  f(a^*)^*g(a^*)=\bar{f_0}g_0+\bar{f_0}g_1a^*+\bar{f_1}g_0a+\bar{f_1}g_1aa^*
\end{align*}
In the integrand we also have an exponential, which is defined by its power series, which truncates after two terms:
\begin{align*}
  e^{-a^*a}=\sum_{n=0}^\infty-\frac{a^*a}{n!}=1-a^*a
\end{align*}
Distributing this to the rest of the integrand:
\begin{align*}
  e^{-a^*a}f(a^*)^*g(a^*)&=(1-a^*a)\qty(
  \bar{f_0}g_0+\bar{f_0}g_1a^*+\bar{f_1}g_0a+\bar{f_1}g_1aa^*)\\
  &=\bar{f_0}g_0+\bar{f_0}g_1a^*+\bar{f_1}g_0a+\bar{f_1}g_1aa^*
  -\bar{f_0}g_0a^*a-\bar{f_0}g_1a^*aa^*-\bar{f_1}g_0a^*aa-\bar{f_1}g_1a^*aaa^*\\
\end{align*}
The last two terms are immediately killed due to having powers of $a$ or $a^*$ that are greater than one, but the term with $a^*aa^*$ requires we use the following identity:
\begin{align*}
  \acomm{a}{a^*}=aa^*-a^*a\equiv0
\end{align*}
This means we can simply swap them to pick up a minus sign, but the square is $0$ anyways, so it gets killed without reprecussion:
\begin{align*}
  e^{-a^*a}f(a^*)^*g(a^*)&=\bar{f_0}g_0+\bar{f_0}g_1a^*+\bar{f_1}g_0a
  +\bar{f_1}g_1aa^*-\bar{f_0}g_0a^*a\\
  &=\bar{f_0}g_0+\bar{f_0}g_1a^*+\bar{f_1}g_0a+\qty(\bar{f_1}g_1+\bar{f_0}g_0)aa^*
\end{align*}
When we integrate over $a$, anything which is a constant in $a$ will go to $0$, this will kill the terms linear in $a^*$ as well:
\begin{align*}
  \int\dd{a}e^{-a^*a}f(a^*)^*g(a^*)=\bar{f_1}g_0+\qty(\bar{f_1}g_1+\bar{f_0}g_0)a^*
\end{align*}
Integrating then through $a^*$ we are left with only the terms which have an $a^*$:
\begin{align*}
  \int\dd{a^*}\dd{a}e^{-a^*a}f(a^*)^*g(a^*)=\bar{f_1}g_1+\bar{f_0}g_0
\end{align*}
Rearranging makes this look like a normal (expected) inner product:
\begin{align*}
  \boxed{
    \ip{f}{g}=\int\dd{a^*}\dd{a}e^{-a^*a}f(a^*)^*g(a^*)=\bar{f_0}g_0+\bar{f_1}g_1}
\end{align*}
\subsection{Matrix Identities}
\subsubsection{Matrix Product on a Function}
We can identify the 'vector' on the left hand side of the matrix equation with the function $f$:
\begin{align*}
  f\equiv\mqty(f_0\\f_1)
\end{align*}
The most general form of a function of two Grassman variables is given by by 4 coefficients, so we can in fact arrange them in a matrix $A$:
\begin{align*}
  A(a^*,b)=A_{00}+A_{01}b+A_{10}a^*+A_{11}a^*b
\end{align*}
Since this itself is a function in Grassman variables, we can compute its inner product with $f$:
\begin{align*}
  Af(a^*)=\int\dd{b^*}\dd{b} A(a^*,b)f(b^*)e^{-b^*b}
\end{align*}
The only terms in the product that will have $b^*b$:
\begin{align*}
  A(a^*,b)f(b^*)e^{-b^*b}=\qty(A_{00}f_0+A_{01}f_1+\qty(A_{10}f_0+A_{11}f_1)a^*)bb^*
\end{align*}
Finishing this off, we see that we simply take the coefficient of $bb^*$:
\begin{align*}
  Af(a^*)&=A_{00}f_0+A_{01}f_1+\qty(A_{10}f_0+A_{11}f_1)a^*=
  \mqty(A_{00}f_0+A_{01}f_1\\A_{10}f_0+A_{11}f_1)\\
  &=\boxed{
    \mqty(A_{00}&A_{10}\\A_{10}&A_{11})\mqty(f_0\\f_1)\equiv \mqty(g_0\\g_1)=g(a^*)}
\end{align*}

\subsubsection{Matrix Multiplication}
For Matrix multiplication, we require a function $B$ of two grassman variables, but this time with $b^*$ and $a$ instead of $a^*,b$:
\begin{align*}
  AB(a^*,a)&\equiv\int\dd{b^*}\dd{b}e^{-b^*b}A(-a^*,b)B(b^*,a)\\
  e^{-b^*b}A(-a^*,b)B(b^*,a)&=(1-b^*b)(A_{00}+A_{01}b+A_{01}a^*+A_{11}a^*b)
  (B_{00}+B_{01}a+B_{10}b^*+B_{11}b^*a)\\
  &=\qty[T(1)+T(a)a+T(a^*)a^*+T(a^*a)a^*a]bb^*
\end{align*}
Where the quantities $T$ are:
\begin{align*}
  T(1)&=A_{00}B_{00}+A_{01}B_{10}\\
  T(a)&=A_{01}B_{11}+A_{00}B_{01}\\
  T(a^*)&=A_{11}B_{10}+A_{10}B_{00}\\
  T(a^*a)&=A_{11}B_{11}+A_{10}B_{01}
\end{align*}
Note that these are the entries of a matrix, whose entries are:
\begin{align*}
  \mqty(T(1)&T(a)\\T(a^*)&T(a^*a))
\end{align*}
Up to a possible transpose which I misdid, this is our $C$:
\begin{align*}
  \boxed{AB(a^*,a)=C(a^*a)}
\end{align*}
\subsection{Anticommutation Identities}
Start with $\hat{a}^{*2}=0$:
\begin{align*}
  \a^*\a^*f(a^*)=\a^*(\a^*f(a^*))=a^*\a^*f(a^*)=a^*a^*f(a^*)=0
\end{align*}
Since $a^*a^*=0$

Next is $\a^2=0$:
\begin{align*}
  \a(\a f(a^*))=\a\qty(\dv{a^*}f(a^*))
\end{align*}
The derivative of a function $f$ will result in the constant coefficient of $a^*$ in $f$, so its second derivative must be $0$:
\begin{align*}
  \a\a = 0
\end{align*}
Now the final one:
\begin{align*}
  \a^*(\a f(a^*))&=\a^*\qty(\dv{a^*}f(a^*))=\a^*f_1=a^*f_1\\
  \a(\a^*f(a^*))&=\a\qty(a^*f(a^*))=\dv{a^*}\qty(a^*f_0+a^*f_1a^*)=f_0
\end{align*}
Putting these together gives the anticommutator operated on a function
\begin{align*}
  \acomm{\a^*}{\a}f(a^*)=\a^*\a f(a^*)+\a\a^*f(a^*)=f_0+f_1a^*=f(a^*)
\end{align*}
Getting rid of the function leaves us with the identity:
\begin{align*}
  \boxed{\acomm{\a^*}{\a}=1}
\end{align*}
\section{Path Integral for Fermions}
The Lagrangian:
\begin{align*}
  \L=\psib(i\slashed{\D}-m)\psi
\end{align*}
With sources:
\begin{align*}
  \L_\eta=\psib(i\slashed{\D}-m)\psi+\psib\eta+\etab\psi
\end{align*}
\subsection{Generating Function}
We know that the partition functional will be:
\begin{align*}
  Z[\etab,\eta]=N\int\cD\psib\cD\psi \exp{iS}
\end{align*}
Where $N$ is the normalization constant, note factors of $\hbar$ are ignored. The action is given by:
\begin{align*}
  S(\psib,\psi)=\int\dd[4]{x}\L_\eta
\end{align*}
The operator we (eventually) need to find the Green function of is $\A\equiv i\slashed{\D}-m$, write the solution in terms of the classical solution and its perturbation:
\begin{align*}
  \psi(x)=\psi_0(x)+\xi(x)
\end{align*}
A similar equation exists for $\psib$ as well. The Lagrangian will have the form:
\begin{align*}
  \L_\eta=\psib_0\A\psi_0+\xib\A\xi+
  \psi_0\etab+\psib_0\eta+\xi\etab+\xib\eta+
  \xib\A\psi_0+\psib_0\A\xi
\end{align*}
The definition of our Green function is:
\begin{align*}
  \A G(x-y)=\delta(x-y)
\end{align*}
Giving the form of $\psi_0\psib_0$
\begin{align*}
  \psi_0&=-\int\dd[4]{y}G(x-y)\eta(y)\\
  \psib_0&=-\int\dd[4]{y}G(x-y)\etab(y)
\end{align*}
We have in both cases $\A$ acting on $\psi_0,\psib_0$:
\begin{align*}
  \A\psi_0&=-\int\dd[4]{y}\A G(x-y)\eta(y)
  =-\int\dd[4]{y}\delta(x-y)\eta(y)=-\eta(x)\\
  \A\psib_0&=-\int\dd[4]{y}\A G(x-y)\etab(y)
  =-\int\dd[4]{y}\delta(x-y)\etab(y)=-\etab(x)
\end{align*}
So the mixed $\psi_0$ and $\xi$ terms cancel out. Putting all of this back into the Lagrangian gives:
\begin{align*}
  \L=\xib\A\xi-\int\dd[4]{y}\etab(x)G(x-y)\eta(y)
\end{align*}
The same we had as in class. The partition functional is:
\begin{align*}
  Z[\etab,\eta]=\qty(\int\cD\xib\cD\xi\exp{iS_\xi})
  \exp{i\int\dd[4]{x}\dd[4]{y}\etab(x)G(x-y)\eta(y)}
\end{align*}
With the action of $\xi$ is:
\begin{align*}
  S_\xi=\int\dd[4]{x}\xib\A\xi
\end{align*}
This is the structure which gave us the determinant in class, we also must add in the spin indices which were dropped since they did not contribute to the content of the integral at all:
\begin{align*}
  \boxed{Z[\etab,\eta]=\det(\A)
  \exp{i\int\dd[4]{x}\dd[4]{y}\etab_a(x)G_{ab}(x-y)\eta_b(y)}}
\end{align*}
Instead of greek indices I used Latin but it should not matter in this case. 
\subsection{Feynman Propagator}
The normal partition functional is given in terms of the fields $\psib,\psi$:
\begin{align*}
  Z[\etab,\eta]=\int\cD\psib\cD\psi\exp{i\int\dd[4]{x}\L_\eta}
\end{align*}
We get the two-point function by varying the partition functional with respect to the two sources:
\begin{align*}
  \eval{\frac{\delta^2Z[\etab,\eta]}{\delta\etab_a(x)\delta\eta_b(y)}}_
  {\eta,\etab=0}
\end{align*}
Th
\begin{align*}
  \fdv{Z[\etab,\eta]}{\eta_b(y)}=\int\cD\psib\cD\psi\int\dd[4]{x'}
  \fdv{\psib_c(x')\eta_c(x')}{\eta_b(y)}e^{iS}
\end{align*}
We need to commute the $\psib$ and the $\eta$ to take the derivative, picking up a minus sign:
\begin{align*}
  \fdv{Z[\etab,\eta]}{\eta_b(y)}=-\int\cD\psib\cD\psi\psib_be^{iS}
\end{align*}
The next one will have similar results, but since we have to commute the functional derivative past another Grassman variable $\psi$, we pick up an overall one more minus sign, cancelling out the original, but we will get another from the derivative:
\begin{align*}
  \frac{\delta^2Z[\etab,\eta]}{\delta\etab_a(x)\delta\eta_b(y)}=
  -\int\cD\psib\cD\psi\psib_b\psi_ae^{iS}
\end{align*}
Commuting the fields on the inside one last time makes things positive again:
\begin{align*}
  \frac{\delta^2Z[\etab,\eta]}{\delta\etab_a(x)\delta\eta_b(y)}=
  \int\cD\psib\cD\psi\psi_a(x)\psib_b(y)e^{iS}
\end{align*}
We then get:
\begin{align*}
  \eval{\frac{\delta^2Z[\etab,\eta]}{\delta\etab_a(x)\delta\eta_b(y)}}_
  {\eta,\etab=0}=\mel{0}{T\psi_a(x)\psi_b(y)}{0}
\end{align*}
If we were to do this in the scheme of the previous part, the derivative would give us something proportional to:
\begin{align*}
  \fdv{Z}{\eta_b(y)}=Z\int\dd[4]{x'}\dd[4]{y'}\fdv{\eta_b(y)}
  \qty(\etab_d(x')G_{dc}(x'-y')\eta_c(y'))
\end{align*}
When we take the functional derivative, we should replace $y'$ with $y$\footnote{Equivalent to a $\delta(y'-y)$}, replace the $c$ indices with $b$ indices\footnote{Equivalent to a $\delta_{bc}$} finally, we need to add in a minus sign from commuting the functional derivative with the first Grassman source:
\begin{align*}
  \fdv{Z}{\eta_b(y)}=-Z\int\dd[4]{x'}
  \etab_d(x')G_{db}(x'-y)
\end{align*}
Taking the next derivative is exactly the same, so we can immediately set everything to $0$:
\begin{align*}
  \eval{\frac{\delta^2Z}{\delta\etab_a(x)\eta_b(y)}}_{\eta,\etab=0}=
  iG_{ab}(x-y)Z[0,0]
\end{align*}
The propagator will get rid of the $Z[0,0]$ from the normalization, and also gets rid of the $i$:
\begin{align*}
  \boxed{S^{ab}_F(x-y)=G_{ab}(x-y)=\mel{x,a}{\frac{1}{i\slashed{\D}-m}}{y,b}}
\end{align*}
\subsection{Four-Point Function}
The Four-Point function is given by a fourth variation of the partition functional:
\begin{align*}
  S_{F,abcd}^4=\frac{1}{Z[0,0]}\eval{\frac{\delta^4Z[\etab,\eta]}{\delta \etab_a(x_1)\delta\etab(x_2)\delta\eta_c(x_3)\delta\eta_d(x_4)}}_{\eta,\etab=0}
\end{align*}
Due to the exponential like before, these will all be proportional to $Z$ or a variation of it:
\begin{align*}
  \fdv{Z[\etab,\eta]}{\eta_d(x_4)}&=
  -\qty[\int\dd[4]{z}\etab_f(z)F_{fd}(z-x_4)]Z[\etab,\eta]\\
  \frac{\delta^2Z[\etab,\eta]}{\delta\eta_c(x_3)\delta\eta_d(x_4)}=
  \fdv{\eta_c(x_3)}\fdv{Z[\etab,\eta]}{\eta_d(x_4)}&=
  -\qty[\int\dd[4]{z}\etab_f(z)G_{fd}(z-x_4)]
  \qty[\int\dd[4]{z}\etab_f(z)G_{fc}(z-x_3)]Z[\etab,\eta]\\
  \frac{\d^3Z[\etab,\eta]}{\d\etab_b(x_2)\d\eta_c(x_3)\d\eta_c(x_3)\d\eta_d(x_4)}
  &=\int\dd[4]{z}\qty[G_{bc}(x_2-x_3)\etab_fG_{fd}(z-x_4)-
  G_{bd}(x_2-x_4)\etab_fG_{fc}(z-x_3)]Z[\etab,\eta]
\end{align*}
The $Z$ in the last term should really be of $[0,0]$ since I am implicitly ignoring a derivative of $Z$, but it would give an overall term of a source, which we will set to $0$ inevitably.

The final form out of this will in fact be a combination of 4-pt functions as expected from Wick's Theorem:
\begin{align*}
  \boxed{S^4_{F,abcd}=G_{bd}(x_2-x_4)G_{ac}(x_1-x_3)
    -G_{bc}(x_2-x_3)G_{ad}(x_1-x_4)}
\end{align*}
\end{document}