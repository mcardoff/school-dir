\documentclass[12pt]{article}

\title{\vspace{-3em}PHYS 553 Final Exam}
\author{Michael Cardiff}
\date{\today}

%% science symbols 
\usepackage{amsmath}
\usepackage{amssymb}
\usepackage{physics}
\usepackage{slashed}

%% general pretty stuff
\usepackage{bm}
\usepackage{setspace}
\usepackage{enumitem}
\usepackage{float}
\usepackage{graphicx}
\usepackage[margin=1in]{geometry}
\usepackage{tikz}
\usepackage{tikz-feynman}

% figures
\graphicspath{ {./figs/} }

\newcommand{\fig}[3]
{
  \begin{figure}[H]
    \centering
    \includegraphics[width=#1cm]{#2}
    \caption{#3}
  \end{figure}
}

\newcommand{\figref}[4]
{
  \begin{figure}[H]
    \centering
    \includegraphics[width=#1cm]{#2}
    \caption{#3}
    \label{#4}
  \end{figure}
}

\renewcommand{\L}{\mathcal{L}}
% integration measure
\newcommand{\cD}{\mathcal{D}}
% spacetime deriv
\newcommand{\D}{\partial}
% fields
\newcommand{\phis}{\phi^*}
\newcommand{\bphis}{\bar{\phi}^*}
\newcommand{\bphi}{\bar{\phi}}
\newcommand{\psib}{\bar{\psi}}
\newcommand{\etab}{\bar{\eta}}
\newcommand{\xib}{\bar{\xi}}
\newcommand{\A}{\hat{A}}
\renewcommand{\d}{\delta}
\renewcommand{\a}{\hat{a}}
\newcommand{\sla}[1]{\slashed{#1}}
\newcommand{\munu}{{\mu\nu}}
\renewcommand{\sp}{\slashed{p}}
\renewcommand{\xib}{\bar\xi}
\newcommand{\zt}{\zeta}
\newcommand{\ztb}{\bar\zeta}
\newcommand{\g}{\gamma}

\begin{document}
\maketitle

\section{Casimir Effect}
We are considering the following Lagrangian density in 1+1 spacetime dimensions, so $\mu=0,1$:
\begin{align*}
  \L=\frac12\D_\mu\phi\D^\mu\phi-\frac12m^2\phi^2
\end{align*}
\subsection{Classical Ground State Energy}
The classical ground state energy is found by minimizing the classical Hamiltonian:
\begin{align*}
  \H&=\Pi\D_0\phi-\L
\end{align*}
The momentum is given by:
\begin{align*}
  \Pi=\D_0\phi
\end{align*}
So the Hamiltonian Density is:
\begin{align*}
  \H=\frac12\qty[\Pi^2+(\grad\phi)^2+m^2\phi^2]
\end{align*}
The ground state of a system should have zero momentum and since the gradient term is only positive its lowest value is $0$, otherwise the minimizing value would be $\phi=0$, so the ground state energy is $0$.
\subsection{Ground State Energy Density in Path Integral Formalism}
We start with the partition functional with our test sources:
\begin{align*}
  Z[J]&\propto\int\cD\phi\exp{iS}\\
  S&=\int\dd[2]x\qty(\frac12(\D_\mu\phi)^2-\frac12m^2\phi^2+J\phi)
\end{align*}
Wick rotate, turn $x_0\to ix_4$:
\begin{align*}
  Z_E[J]&\propto\int\cD\phi\exp{-S_E}\\
  S_E&=\int\dd[2]x\qty(\frac12(\D_\mu\phi)(\D_\mu\phi)+\frac12m^2\phi^2-J\phi)
\end{align*}
We can get this into a form which is bilinear in fields by integrating by parts, putting an extra derivative on a $\D_\mu\phi$:
\begin{align*}
  \L_E\to\frac12\phi\qty(-\D_\mu\D_\mu+m^2)\phi-J\phi
\end{align*}
Set $\D_\mu\D_\mu=\laplacian$, we can then rewrite out partition functional in terms of the Euclidean Green Function $G_E(x-x')$ by integrating through all $x,x'$:
\begin{align*}
  Z_E[J]&=Z_E[0]\exp{\frac{1}{2}\int\dd[2]x\dd[2]{x'}J(x)G_E(x-x')J(x')}\\
  Z_E[0]&=\int\cD\delta\exp{-\frac12
    \int\dd[2]\delta(x)\qty[-\laplacian+m^2]\delta(x)}
\end{align*}
We identified this $Z_E[0]$ with a determinant:
\begin{align*}
  Z_E[0]=\det[-\laplacian+m^2]^{-1/2}
\end{align*}
The full partition functional is then:
\begin{align*}
  Z_E[J]=\det[-\laplacian+m^2]^{-1/2}
  \exp{\frac{1}{2}\int\dd[2]x\dd[2]{x'}J(x)G_E(x-x')J(x')}
\end{align*}
We wrote the ground state energy as a log of our partition functional:
\begin{align*}
  E_G=-\lim_{\beta\to\infty}\frac{\ln Z_E[J]}{\beta}
\end{align*}
We want to expand our field to act like a simple harmonic oscillator, the leading order term in the classical + quantum correction expansion has the form:
\begin{align*}
  \L=\L[\phi_c]+\frac12\phi\qty[-\D^2
  -\eval{\dv[2]{V(\phi)}{x}}_{\phi=\phi_c}]\phi
\end{align*}
Where the value for the potential is simply $m^2$ in our case. Wick rotating THIS case we get:
\begin{align*}
  Z=\qty[\det(-\laplacian+m^2)]^{-1/2}\exp{-S[\phi_c]}
\end{align*}
We can identify the classical action with the $\beta$ parameter and the free field ground state energy:
\begin{align*}
  S[\phi_c]=-\beta E_0
\end{align*}
So our ground state energy at leading order is:
\begin{align*}
  \boxed{E_G=E_0+\frac{1}{2\beta}\ln\det(-\laplacian+m^2)}
\end{align*}
\subsection{Quantum Correction in Massless Limit}
Using the generalized $\zeta$ function, we can find the determinant of an operator $\A$:
\begin{align*}
  -\ln\det\A=\lim_{s\to0^+}\dv{\zeta_A}{s}
\end{align*}
To do this we need to solve the generalized heat kernel:
\begin{align*}
  G_A(x,y,t)=\sum_n\exp{-a_nt}f_n(x)f_n^*(y)
\end{align*}
Where $f_n$ are eigenfunctions of $\A$ with $a_n$ as the corresponding eigenvalue, meaning we can write this as:
\begin{align*}
  G_A(x,y,t)=\mel{x}{\exp{-t\A}}{y}
\end{align*}
The condition that this must conform to is:
\begin{align*}
  \lim_{t\to0^+}G_A(x,y,t)=\delta(x-y)
\end{align*}
Relate it to the generalized $\zeta$:
\begin{align*}
  \int\dd[2]x\lim_{y\to x}G_A(x,y,t)=\sum_n\exp{-a_nt}\int\dd[2]xf_nf_n^*
  =\sum_n\exp{-a_nt}\equiv\Tr\exp{-t\A}
\end{align*}
We have from class:
\begin{align*}
  \frac{\Gamma(s)}{a_n^s}=\int_0^\infty\dd{z}z^{s-1}e^{-a_nz}
\end{align*}
Summing this over $n$:
\begin{align*}
  \sum_n\frac{\Gamma(s)}{a_n^s}=\int_0^\infty\dd{t}z^{s-1}\sum_ne^{-a_nt}
\end{align*}
Which we can then again relate to our $\zeta_A$:
\begin{align*}
  \zeta_A(s)=\frac{1}{\Gamma(s)}\int\dd{t}t^{s-1}\int\dd[2]x
  \lim_{y\to x}G_A(x,y,t)
\end{align*}
The eigenvalues of $\A$ using the periodic boundary conditions will be given by some propagator pole:
\begin{align*}
  a_n=k^2+m^2
\end{align*}
Where $k$ is a four-vector $k=\qty(\frac{2\pi n}{L},\frac{2\pi m}{\beta})$. Note two indices since we have 2 total dimensions:
\begin{align*}
  \zeta_A&=\frac{1}{\Gamma(s)}\sum_{m,n}\int_0^\infty\dd{t}t^{s-1}e^{-a_{mn}t}\\
  &=\frac{\beta}{\Gamma(s)}\sum_{n}\int_0^\infty\int_{-\infty}^\infty
  \frac{\dd{t}\dd\omega}{2\pi}t^{s-1}e^{-t(\omega^2+k_n^2+m^2)}
\end{align*}
Where in the second step we replace the sum with a fourier integral, notice the $\omega$ integral however is a Gaussian:
\begin{align*}
  \zeta_A(s)=\frac{\beta}{\Gamma(s)}\sum_{n}\int_0^\infty
  \frac{\dd{t}}{2\sqrt\pi}t^{s-3/2}e^{-t(k_n^2+m^2)}
\end{align*}
We can then use the formula given:
\begin{align*}
  \zeta_A(s)=\frac{\beta}{\Gamma(s)}\sum_m\int_{-\infty}^\infty\dd{y}
  \int_0^\infty\frac\dd{t}{2\sqrt\pi}e^{2\pi imy}t^{s-3/2}
  e^{-t((\frac{2\pi}{L})^2+m_0^2)}
\end{align*}
Where in this case the mass is $m_0$ to not be confused with an index $m$:
\begin{align*}
  \zeta_A(s)=\frac{\beta L}{4\pi\Gamma(s)}\sum_n  \int_0^\infty
  \dd{t}t^{s-2} e^{-m^2t-(nL)^2/4t}
\end{align*}
The massless limit tells us:
\begin{align*}
  \zeta_A(s)=\frac{\beta L}{4\pi\Gamma(s)}\sum_n  \int_0^\infty
  \dd{t}t^{s-2} e^{-(nL)^2/4t}
\end{align*}
Right now we are summing over all $n\in\mathbb{Z}$, but we have a quadratic $n$ term, so we can consider the individual $n=0$ and then sum from $1\to\infty$:
\begin{align*}
  \zeta_A(s)=\frac{\beta L}{4\pi\Gamma(s)}
  \qty[\int_0^\infty\dd{t}t^{s-2}e^{-m^2t}+2\sum_{n=1}  \int_0^\infty
  \dd{t}t^{s-2} e^{-(nL)^2/4t}]
\end{align*}
There would not be a $\Gamma$ if we got rid of the mass in the $n=0$ term, so we need it:
\begin{align*}
  \int_0^\infty\dd{t}t^{s-2}e^{-m^2t}=\frac{\Gamma(s-1)}{(m^2)^{s-1}}
\end{align*}
The other term (with a change of variables) is a product of a $\zeta$ function and a $\Gamma$ function, we can then see:
\begin{align*}
  \zeta_A(s)=\frac{\beta L}{4\pi\Gamma(s)}
  \qty[\frac{\Gamma(s-1)}{(m^2)^{s-1}}+2\qty(\frac2L)^{2-2s}
  \zeta(2-2s)\Gamma(1-s)]
\end{align*}
Using the defining properties of the $\Gamma$ function can simplify the first term:
\begin{align*}
  \frac{\Gamma(s-1)}{\Gamma(s)}=\frac1{s-1}
\end{align*}
This is about as far as I am going to go without Mathematica, I am going to not only differentiate but also take the limit. This is the code I used and the final algebraic simplification:
\begin{doublespace}
\noindent\(\pmb{\text{$\zeta $A}=\frac{\beta  L}{4\pi }\left(\frac{1}{\left(m^2\right)^{s-1}(s-1)}+2\left(\frac{2}{L}\right)^{2-2s}\text{Zeta}[2-2s]\frac{\text{Gamma}[1-s]}{\text{Gamma}[s]}\right)}\)
\end{doublespace}

\begin{doublespace}
\noindent\(\frac{L \beta  \left(\frac{\left(m^2\right)^{1-s}}{-1+s}+\frac{2^{3-2 s} \left(\frac{1}{L}\right)^{2-2 s} \text{Gamma}[1-s] \text{Zeta}[2-2
s]}{\text{Gamma}[s]}\right)}{4 \pi }\)
\end{doublespace}

\begin{doublespace}
\noindent\(\pmb{\text{Limit}[D[\text{$\zeta $A},s],s\text{-$>$}0]\text{//}\text{FullSimplify}}\)
\end{doublespace}

\begin{doublespace}
\noindent\(\frac{\beta  \left(-3 L^2 m^2+4 \pi ^2+3 L^2 m^2 \text{Log}\left[m^2\right]\right)}{12 L \pi }\)
\end{doublespace}

\begin{doublespace}
\noindent\(\pmb{-\frac{\beta  L}{4\pi }\left(-m^2\left(1-\text{Log}\left[m^2\right]\right)+\frac{4 \pi ^2}{3L^2}\right)}\)
\end{doublespace}
The Final expression we get for our functional determinant:
\begin{align*}
  \ln\det\A=-\lim_{s\to0}\dv{\zeta_A}{s}=-\frac{\beta L}{3\pi}
  \qty[-m^2(1-\ln(m^2))+\frac{4\pi^2}{3L^2}]
\end{align*}
However we need the log to be dimensionless, so we have the scale $\mu^2$ introduced:
\begin{align*}
  \ln\det\A=-\frac{\beta L}{3\pi}
  \qty[-m^2(1-\ln(m^2/\mu^2))+\frac{4\pi^2}{3L^2}]
\end{align*}
We then correspond this with our energy fluctuations $E_G-E_0$:
\begin{align*}
  \boxed{E_G-E_0=\frac1{2\beta}\ln\det\A=\frac{m^2L}{8\pi}\qty(1-\ln(m^2/\mu^2))
  -\frac{\pi}{6L}}
\end{align*}
Which is of the form which is mentioned in the problem
\subsection{Zero-Point Fluctuations}
We can associate the divergent part of energy with the 'outside' energy, which would be practically infinite. This means the term that disappears when we enlarge $L$ should be the internal one, So our force should be this term over the length of the box:
\begin{align*}
  \boxed{F_{C}=-\frac{\pi}{6L^2}}
\end{align*}
This is an attractive force!
\section{Perturbation Theory for $SU(2)$}
The Lagrangian with the gauge-covariant derivative written out is:
\begin{align*}
  \L=\psib i\qty(\sla{\D}+ig\sla{A})\psi-
  \frac12\Tr[F_\munu F^\munu]+\frac\lambda2\Tr(\D_\mu A^\mu)^2
  -\etab_a\D_\mu D^\mu_{ab}\eta_b
\end{align*}
This should be fully written out:
\begin{align*}
  \L&=\psib i\qty(\sla{\D}+ig\sla{A})\psi\\
  &\,-\frac12\Tr[t^a\qty(\D_\mu A^a_\nu-\D_\nu A^a_\mu+gf^{abc}A_\mu^bA_\nu^c)
  t^b\qty(\D^\mu A^{b\nu}-\D^\nu A^{b\mu}+gf^{beg}A^{e\mu}A^{g\nu})]\\
  &\,+\frac\lambda2\Tr[\D_\mu A^{a\mu}\D_\mu A^{a\mu}]\\
  &\,-\etab_a\qty(\D^2\delta_{ab}+gf_{abc}\D_\mu  A^\mu_c)\eta_b
\end{align*}

\subsection{Feynman Propagators in Momentum Space}
The kinetic terms are the ones that give propagators:
\begin{align*}
  \L_{K}&=i\psib\sla{\D}\psi\\
  &\,-\frac12\Tr[t^at^b\{\qty(\D_\mu A^a_\nu-\D_\nu A^a_\mu)
  \qty(\D^\mu A^{b\nu}-\D^\nu A^{b\mu})]\\
  &\,+\frac\lambda2\Tr[\D_\mu A^{a\mu}\D_\mu A^{a\mu}\}]\\
  &\,-\etab_a\D^2\delta_{ab}\eta_b
\end{align*}
The propagators are given by the following rules, we have derived at least the gauge field and fermion in class, and the ghosts should act like a scalar would, and we have derived their propagators as well:
\begin{align*}
  \feynmandiagram[horizontal=a to b]{
    a 
    -- [fermion, momentum=\(p\)]
    b 
  };&=\frac{i}{\sp} \\
  \feynmandiagram[horizontal=a to b]{
    a [particle=\(a\text{,}\mu\)]
    -- [gluon, momentum=\(p\)]
    b [particle=\(b\text{,}\nu\)]
  };&=\frac{i}{p^2}\qty(-g_\munu+\frac{p_\mu p_\nu}{p^2})\delta_{ab} \\
  \feynmandiagram[horizontal=a to b]{
    a [particle=\(a\)]
    -- [ghost, momentum=\(p\)]
    b [particle=\(b\)]
  };&=-\frac{i}{p^2}\delta_{ab}
\end{align*}
\subsection{Feynman Rules in Position Space}
I am not entirely sure how to go about this. The diagrams are a term in a perturbative series. We need to expand the following:
\begin{align*}
  Z[\eta,\etab,J^a_\mu]&=\exp{-\int\dd[4]{x}
    \L_\text{int}
    \qty[-i\fdv{\xib},-i\fdv{\xi},-i\fdv{J^a_\mu},-i\fdv{\ztb},-i\fdv{\zt}]}\\
  &\quad \times Z_\text{'Maxwell'}[J_\mu^a] Z_\text{Dirac}[\xi,\xib]
  Z_\text{Ghost}[\zt,\ztb]
\end{align*}
The corresponding partition Functionals we need are:
\begin{align*}
  Z_\text{'Maxwell'}[J_\mu^a]&=\int\cD{A_\mu}\exp{i\int\dd[4]{x}
    \qty(-\frac12\Tr[F_\munu F^\munu]+\frac\lambda2\Tr[(\D_\mu A^\mu)^2])
    +J_\mu^aA^{a\mu}}\\
  Z_\text{Dirac}[\xi,\xib]&=\int\cD\psi\cD\psib\exp{i\int\dd[4]{x}
    \qty(\psib(i\sla{\D}-m)\psi+\xib\psi+\psib\xi)}\\
  Z_\text{Ghost}[\zt,\ztb]&=\int\cD\eta\cD\etab\exp{i\int\dd[4]{x}
    \qty(\etab(-i\sla{\D}-m)\eta+\ztb\eta+\etab\zt)}
\end{align*}
The tricky part now is identifying our interacting Lagrangian, which is formally:
\begin{align*}
  \L_\text{int}=\L-\L_K
\end{align*}
Mostly fully written out:
\begin{align*}
  \L_\text{int}&=-g\psib\sla{A}\psi\\
  &\,-\frac12\Tr\hspace{0pt}[t^at^b\{
  g(\D_\mu A^a_\nu-\D_\nu A^a_\mu)f^{beg}A^{e\mu}A^{g\nu} \\
  &\qquad\qquad+g f^{abc}A_\mu^b A_\nu^c(\D^\mu A^{b\nu}-\D^\nu A^{b\mu})\\
  &\qquad\qquad+g^2f^{abc}f^{beg}A_\mu^a A_\nu^cA^{e\mu}A^{g\nu}\}]\\
  &\,-g\etab_af_{abc}\D_\mu A^\mu_c\eta_b
\end{align*}
All of the position space Feynman rules are then:
\begin{gather*}
  \begin{gathered}
    \feynmandiagram [small, horizontal=d to b]{
      a -- [anti fermion] b -- [fermion] c,
      b -- [gluon] d,
    }; 
  \end{gathered}
  = \boxed{-ig\g^\mu} \qquad
  \begin{gathered}
    \feynmandiagram [small, horizontal=d to b]{
      a -- [ghost] b -- [ghost] c,
      b -- [gluon] d,
    };
  \end{gathered}
  = \boxed{-igf^{abc}\D_\mu}\\
  \begin{gathered}
    \feynmandiagram [small, horizontal=d to b]{
      a -- [gluon] b -- [gluon] c,
      b -- [gluon] d,
    };
  \end{gathered}
  = \boxed{-i\frac{g}{2}t^at^bf^{abc}\D_\mu}\qquad
  \begin{gathered}
    \feynmandiagram [small, horizontal=i1 to f2] {
      {i1, i2} -- [gluon] c -- [gluon] {f1, f2},
    };
  \end{gathered}
  = \boxed{-i\frac{g^2}{2}t^at^bf^{abc}f^{beg}}
\end{gather*}
\subsection{One Loop Corrections}
I am going to stop after writing the integral with the two propagators in it
\subsubsection{Fermion Propagator}
The correction to the fermion propagator would involve the following diagram:
\begin{figure}[H]
  \centering
  \feynmandiagram [tree layout, horizontal=a to c] {
    a [particle=\(\psi\)] -- [fermion, momentum=\(p\)] b
    -- [gluon, half left, looseness=1.8,
    rmomentum=\(k\)] c
    -- [anti fermion, half left, looseness=1.8,
    rmomentum=\(p+k\)] b,
    c -- [fermion, momentum=\(p\)] d [particle=\(\psi\)]
  };  
  \caption{One Loop Fermion Propagator}
\end{figure}
The loop integral would be:
\begin{align*}
  \int\frac{\dd[4]k}{(2\pi)^4}\frac{i}{\sla{\D}+\sla{k}}
  \frac{i}{\D^2}\qty(-g_\munu+\frac{\D_\mu\D_\nu}{\D^2})
\end{align*}
Except for most likely some factors of the coupling constant
\subsubsection{Gauge Propagator}
The correction to the gauge propagator would involve a couple diagrams:
\begin{figure}[H]
  \centering
  \feynmandiagram [tree layout, horizontal=a to c] {
    a -- [gluon, momentum=\(p\)] b
    -- [anti fermion, half left, looseness=1.8,
    rmomentum=\(k\)] c
    -- [anti fermion, half left, looseness=1.8,
    rmomentum=\(p+k\)] b,
    c -- [gluon, momentum=\(p\)] d
  };  
  \caption{One Loop Gauge Propagator (Fermion)}
\end{figure}
\begin{figure}[H]
  \centering
  \feynmandiagram [tree layout, horizontal=a to c] {
    a -- [gluon, momentum=\(p\)] b
    -- [gluon, half left, looseness=1.8,
    rmomentum=\(k\)] c
    -- [gluon, half left, looseness=1.8,
    rmomentum=\(p+k\)] b,
    c -- [gluon, momentum=\(p\)] d
  };  
  \caption{One Loop Gauge Propagator (Gauge)}
\end{figure}
\begin{figure}[H]
  \centering
  \feynmandiagram [tree layout, horizontal=a to c] {
    a -- [gluon, momentum=\(p\)] b
    -- [ghost, half left, looseness=1.8,
    rmomentum=\(k\)] c
    -- [ghost, half left, looseness=1.8,
    rmomentum=\(p+k\)] b,
    c -- [gluon, momentum=\(p\)] d
  };  
  \caption{One Loop Gauge Propagator (Ghost)}
\end{figure}
Each of these have the same form, except with the proper propagators inserted, in the first case its two fermion propagators, then two gauge propagators, and two ghost propagators:
\begin{gather*}
  -\int\frac{\dd[4]k}{(2\pi)^4}\frac{i}{\sla{\D}+\sla{k}}\frac{i}{\sla{k}}\\
  \int\frac{\dd[4]k}{(2\pi)^4}\frac{i}{(\D+k)^2}
  \qty(-g_\munu+\frac{(\D+k)_\mu(\D+k)_\nu}{(\D+k)^2})
  \frac{i}{\D^2}\qty(-g^\munu+\frac{\D^\mu\D^\nu}{\D^2})\\
  \int\frac{\dd[4]k}{(2\pi)^4}\frac{i}{(\D+k)^2}\frac{i}{\D^2}
\end{gather*}
Which is probably not right
\end{document}