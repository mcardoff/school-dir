\documentclass[12pt]{article}

\title{\vspace{-3em}PHYS 553 Final Exam}
\author{Michael Cardiff}
\date{\today}

%% science symbols 
\usepackage{amsmath}
\usepackage{amssymb}
\usepackage{physics}

%% general pretty stuff
\usepackage{bm}
\usepackage{enumitem}
\usepackage{float}
\usepackage{graphicx}
\usepackage[margin=1in]{geometry}

% figures
\graphicspath{ {./figs/} }

\newcommand{\fig}[3]
{
  \begin{figure}[H]
    \centering
    \includegraphics[width=#1cm]{#2}
    \caption{#3}
  \end{figure}
}

\newcommand{\figref}[4]
{
  \begin{figure}[H]
    \centering
    \includegraphics[width=#1cm]{#2}
    \caption{#3}
    \label{#4}
  \end{figure}
}

\renewcommand{\L}{\mathcal{L}}
% integration measure
\newcommand{\cD}{\mathcal{D}}
% spacetime deriv
\newcommand{\D}{\partial}
% fields
\newcommand{\phis}{\phi^*}
\newcommand{\bphis}{\bar{\phi}^*}
\newcommand{\bphi}{\bar{\phi}}
\newcommand{\psib}{\bar{\psi}}
\newcommand{\etab}{\bar{\eta}}
\newcommand{\xib}{\bar{\xi}}
\newcommand{\A}{\hat{A}}
\renewcommand{\d}{\delta}
\renewcommand{\a}{\hat{a}}

\begin{document}
\maketitle

\section{Casimir Effect}
We are considering the following Lagrangian density in 1+1 spacetime dimensions, so $\mu=0,1$:
\begin{align*}
  \L=\frac12\D_\mu\phi\D^\mu\phi-\frac12m^2\phi^2
\end{align*}
\subsection{Classical Ground State Energy}
The classical ground state energy is found by minimizing the classical Hamiltonian:
\begin{align*}
  \H&=\Pi\D_0\phi-\L
\end{align*}
The momentum is given by:
\begin{align*}
  \Pi=\D_0\phi
\end{align*}
So the Hamiltonian Density is:
\begin{align*}
  \H=\frac12\qty[\Pi^2+(\grad\phi)^2+m^2\phi^2]
\end{align*}
The ground state of a system should have zero momentum and since the gradient term is only positive its lowest value is $0$, otherwise the minimizing value would be $\phi=0$, so the ground state energy is $0$.
\subsection{Ground State Energy Density in Path Integral Formalism}
We start with the partition functional with our test sources:
\begin{align*}
  Z[J]&\propto\int\cD\phi\exp{iS}\\
  S&=\int\dd[2]x\qty(\frac12(\D_\mu\phi)^2-\frac12m^2\phi^2+J\phi)
\end{align*}
Wick rotate, turn $x_0\to ix_4$:
\begin{align*}
  Z_E[J]&\propto\int\cD\phi\exp{-S_E}\\
  S_E&=\int\dd[2]x\qty(\frac12(\D_\mu\phi)(\D_\mu\phi)+\frac12m^2\phi^2-J\phi)
\end{align*}
We can get this into a form which is bilinear in fields by integrating by parts, putting an extra derivative on a $\D_\mu\phi$:
\begin{align*}
  \L_E\to\frac12\phi\qty(-\D_\mu\D_\mu+m^2)\phi-J\phi
\end{align*}
Set $\D_\mu\D_\mu=\laplacian$, we can then rewrite out partition functional in terms of the Euclidean Green Function $G_E(x-x')$ by integrating through all $x,x'$:
\begin{align*}
  Z_E[J]&=Z_E[0]\exp{\frac{1}{2}\int\dd[2]x\dd[2]{x'}J(x)G_E(x-x')J(x')}\\
  Z_E[0]&=\int\cD\delta\exp{-\frac12
    \int\dd[2]\delta(x)\qty[-\laplacian+m^2]\delta(x)}
\end{align*}
We identified this $Z_E[0]$ with a determinant:
\begin{align*}
  Z_E[0]=\det[-\laplacian+m^2]^{-1/2}
\end{align*}
The full partition functional is then:
\begin{align*}
  Z_E[J]=\det[-\laplacian+m^2]^{-1/2}
  \exp{\frac{1}{2}\int\dd[2]x\dd[2]{x'}J(x)G_E(x-x')J(x')}
\end{align*}
We wrote the ground state energy as a log of our partition functional:
\begin{align*}
  E_G=-\lim_{\beta\to\infty}\frac{\ln Z_E[J]}{\beta}
\end{align*}
We want to expand our field to act like a simple harmonic oscillator, the leading order term in the classical + quantum correction expansion has the form:
\begin{align*}
  \L=\L[\phi_c]+\frac12\phi\qty[-\D^2
  -\eval{\dv[2]{V(\phi)}{x}}_{\phi=\phi_c}]\phi
\end{align*}
Where the value for the potential is simply $m^2$ in our case. Wick rotating THIS case we get:
\begin{align*}
  Z=\qty[\det(-\laplacian+m^2)]^{-1/2}\exp{-S[\phi_c]}
\end{align*}
We can identify the classical action with the $\beta$ parameter and the free field ground state energy:
\begin{align*}
  S[\phi_c]=-\beta E_0
\end{align*}
So our ground state energy at leading order is:
\begin{align*}
  \boxed{E_G=E_0+\frac{1}{2\beta}\ln\det(-\laplacian+m^2)}
\end{align*}
\subsection{Quantum Correction in Massless Limit}
Using the generalized $\zeta$ function, we can find the determinant of an operator $\A$:
\begin{align*}
  -\ln\det\A=\lim_{s\to0^+}\dv{\zeta_A}{s}
\end{align*}
To do this we need to solve the generalized heat kernel:
\begin{align*}
  G_A(x,y,t)=\sum_n\exp{-a_nt}f_n(x)f_n^*(y)
\end{align*}
Where $f_n$ are eigenfunctions of $\A$ with $a_n$ as the corresponding eigenvalue, meaning we can write this as:
\begin{align*}
  G_A(x,y,t)=\mel{x}{\exp{-t\A}}{y}
\end{align*}
The condition that this must conform to is:
\begin{align*}
  \lim_{t\to0^+}G_A(x,y,t)=\delta(x-y)
\end{align*}
Relate it to the generalized $\zeta$:
\begin{align*}
  \int\dd[2]x\lim_{y\to x}G_A(x,y,t)=\sum_n\exp{-a_nt}\int\dd[2]xf_nf_n^*
  =\sum_n\exp{-a_nt}\equiv\Tr\exp{-t\A}
\end{align*}
We have from class:
\begin{align*}
  \frac{\Gamma(s)}{a_n^s}=\int_0^\infty\dd{z}z^{s-1}e^{-a_nz}
\end{align*}
Summing this over $n$:
\begin{align*}
  \sum_n\frac{\Gamma(s)}{a_n^s}=\int_0^\infty\dd{t}z^{s-1}\sum_ne^{-a_nt}
\end{align*}
Which we can then again relate to our $\zeta_A$:
\begin{align*}
  \zeta_A(s)=\frac{1}{\Gamma(s)}\int\dd{t}t^{s-1}\int\dd[2]x
  \lim_{y\to x}G_A(x,y,t)
\end{align*}
The eigenvalues of $\A$ using the periodic boundary conditions will be given by some propagator pole:
\begin{align*}
  a_n=k^2+m^2
\end{align*}
Where $k$ is a four-vector $k=\qty(\frac{2\pi n}{L},\frac{2\pi m}{\beta})$. Note two indices since we have 2 total dimensions:
\begin{align*}
  \zeta_A&=\frac{1}{\Gamma(s)}\sum_{m,n}\int_0^\infty\dd{t}t^{s-1}e^{-a_{mn}t}\\
  &=\frac{\beta}{\Gamma(s)}\sum_{n}\int_0^\infty\int_{-\infty}^\infty
  \frac{\dd{t}\dd\omega}{2\pi}t^{s-1}e^{-t(\omega^2+k_n^2+m^2)}
\end{align*}
Where in the second step we replace the sum with a fourier integral, notice the $\omega$ integral however is a Gaussian:
\begin{align*}
  \zeta_A(s)=\frac{\beta}{\Gamma(s)}\sum_{n}\int_0^\infty
  \frac{\dd{t}}{2\sqrt\pi}t^{s-3/2}e^{-t(k_n^2+m^2)}
\end{align*}
We can then use the formula given:
\begin{align*}
  \zeta_A(s)=\frac{\beta}{\Gamma(s)}\sum_m\int_{-\infty}^\infty\dd{y}
  \int_0^\infty\frac\dd{t}{2\sqrt\pi}e^{2\pi imy}t^{s-3/2}
  e^{-t((\frac{2\pi}{L})^2+m_0^2)}
\end{align*}
Where in this case the mass is $m_0$ to not be confused with an index $m$:
\begin{align*}
  \zeta_A(s)=\frac{\beta L}{4\pi\Gamma(s)}\sum_n  \int_0^\infty
  \dd{t}t^{s-2} e^{-m^2t-(nL)^2/4t}
\end{align*}
I am not sure where to go from here
\subsection{Zero-Point Fluctuations}

\section{Perturbation Theory for $SU(2)$}

\subsection{Feynman Propagators in Momentum Space}

\subsection{Feynman Rules in Position Space}

\subsection{One Loop Corrections}

\subsubsection{Fermion Propagator}

\subsubsection{Gauge Propagator}

\subsubsection{Fermion-Gauge Vertex}

\subsubsection{Three/Four Gauge Field Vertices}

\end{document}