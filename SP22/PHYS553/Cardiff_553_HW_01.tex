\documentclass[12pt]{article}

\title{\vspace{-3em}PHYS 553 HW 1}
\author{Michael Cardiff}
\date{\today}

%% science symbols 
\usepackage{amsmath}
\usepackage{amssymb}
\usepackage{physics}

%% general pretty stuff
\usepackage{bm}
\usepackage{enumitem}
\usepackage{float}
\usepackage{graphicx}
\usepackage[margin=1in]{geometry}

% figures
\graphicspath{ {./figs/} }

\newcommand{\fig}[3]
{
  \begin{figure}[H]
    \centering
    \includegraphics[width=#1cm]{#2}
    \caption{#3}
  \end{figure}
}

\newcommand{\figref}[4]
{
  \begin{figure}[H]
    \centering
    \includegraphics[width=#1cm]{#2}
    \caption{#3}
    \label{#4}
  \end{figure}
}

\renewcommand{\L}{\mathcal{L}}
\renewcommand{\H}{\mathcal{H}}

\newcommand{\D}{\partial}
\newcommand{\g}{\gamma}
\newcommand{\psib}{\bar{\psi}}
\newcommand{\psid}{\psi^\dag}

\begin{document}
\maketitle
\section*{Question 1}
We want to verify the Dirac bilinears:
\begin{enumerate}
\item The scalar bilinear is fairly simple if we find a transformation law for $\psib$:
  \begin{align*}
    \psib'(x')=(\psi')^\dag\g^0=(S\psi)^\dag\g^0=\psid S^\dag\g^0
  \end{align*}
  Next we should compute the following commutator:
  \begin{align*}
    \comm{S^\dag}{\g^0}\approx
    \comm{I+\frac{i}{4}\sigma_{\mu\nu}\omega^{\mu\nu}}{\g^0}
  \end{align*}
  Clearly, $\g^0$ commutes with the identity:
  \begin{align*}
    \comm{\frac{i}{4}\sigma_{\mu\nu}\omega^{\mu\nu}}{\g^0}=
    \frac{i}{4}\comm{\sigma_{\mu\nu}\omega^{\mu\nu}}{\g^0}=
    \frac{i}{4}\comm{\sigma_{\mu\nu}}{\g^0}\omega^{\mu\nu}
  \end{align*}
  We can then use Eq (2.119) of Fradkin:
  \begin{align*}
    \comm{\sigma_{\mu\nu}}{\g^0}=-\comm{\g^0}{\sigma_{\mu\nu}}=
    -2i\qty(g^0_\mu\gamma_\nu-g^0_\nu\gamma_\mu)
  \end{align*}
  For $\mu=\nu$ this disappears, and for $\mu,\nu=1,2,3$ the metric is $0$, so we can conclude:
  \begin{align*}
    \comm{S^\dag}{\g^0}=0
  \end{align*}
  We can then say:
  \begin{align*}
    \psib'(x)=\psib S^\dag
  \end{align*}
  Since $S$ is a unitary matrix it follows that:
  \begin{align*}
    \psib'(x')\psi'(x')=\psib(x)S^\dag S\psi(x)=\psib\psi
  \end{align*}
\item The pseudoscalar bilinear relies on the following identity:
  \begin{align*}
    S^\dag\g_5S=\det\Lambda\g_5
  \end{align*}
  The bilinear gives us:
  \begin{align*}
    \psib'\g_5\psi'=\psib S^\dag\g_5 S\psi=\psib\qty(\det\Lambda\g_5)\psi
  \end{align*}
  Since the determinant is just a number, it can be moved to the front:
  \begin{align*}
    \psib'\g_5\psi'=\det\Lambda\,\psib\g_5\psi
  \end{align*}
\item The vector bilinear takes advantage of this identity:
  \begin{align*}
    S\g^\mu S^\dag=\qty(\Lambda^{-1})^\mu_\nu\g^\nu=\Lambda^\nu_\mu\g^\nu
  \end{align*}
  So the transformation becomes:
  \begin{align*}
    \psib'\g^\mu\psi'=\psib S^\dag\g^\mu S\psi
  \end{align*}
  I am going to make quite a bold claim, mainly that it does not matter that the interchange of $S^\dag$ and $S$ does not matter, so we can simply use the same identity which we have above:
  \begin{align*}
    \psib'\g^\mu\psi'=\Lambda^\mu_\nu\psib\g^\mu\psi
  \end{align*}
\item The axial vector identity simply comes from applying the previous two identities one after the other, resulting in both of their side effects:
  \begin{align*}
    \psib'\g_5\g^\mu\psi'=\det(\Lambda)\Lambda^\mu_\nu\psib\g_5\g^\mu\psi
  \end{align*}
\item The same goes for the tensor, especially this one, as it is a combination of two vectors we have seen before:
  \begin{align*}
    \psib'\sigma^{\mu\nu}\psi'=\Lambda^\mu_\alpha\Lambda^\nu_\beta\psib
    \sigma^{\alpha\beta}\psi
  \end{align*}
\end{enumerate}

\section*{Question 2}
\begin{enumerate}
\item The saddle point equations are given by:
  \begin{align*}
    -\fdv{F}{\phi(\vb{x})}+\grad_j\qty(\fdv{F}{\grad_j\phi(\vb{x})})
  \end{align*}
  First the derivative with respect to the field:
  \begin{align*}
    -\fdv{F}{\phi}=-\qty(m_0^2\phi
    +\frac{\abs{\lambda_4}}{3!}\phi^3
    +\frac{\lambda_6}{5!}\phi^5)
  \end{align*}
  And with respect to the gradient:
  \begin{align*}
    \grad_j\fdv{F}{\grad_j\phi}=\grad_j\qty(\grad_j\phi)=\laplacian{\phi}
  \end{align*}
  So the Landau-Ginzburg equation of motion is given as:
  \begin{align*}
    \laplacian{\phi}-m_0^2+\frac{\abs{\lambda_4}}{3!}\phi^3
    -\frac{\lambda_6}{5!}\phi^5=0
  \end{align*}
  The saddle point $\phi_c$ minimizes equations of motion, so it will have no 'momentum' so the laplacian goes away:
  \begin{align*}
    m_0^2\phi_c-\frac{\abs{\lambda_4}}{3!}\phi^3+\frac{\lambda_6}{5!}\phi^5=0
  \end{align*}
\item The potential is given by:
  \begin{align*}
    U(\phi)-\frac{a}{2}(T-T_0)\phi^2+\frac{\lambda_4}{4!}\phi^4
    +\frac{\lambda_6}{6!}\phi^6
  \end{align*}
  Choosing a $\lambda_4$ which is negative and $\lambda_6$ which is positive can give us the following plot:
  \fig{9.0}{potentialhw1}{The topmost represents the highest temperature}
  Notice how slowly as the temperature increases there develops a minimal $\ev{\phi}\neq0$, we can see this more clearly in the next plot:
  \fig{9.0}{tempvariation}{Temperature variation of $\ev{\phi}$}
  This was done by numerically computing the minima at various temperatures and at a specific $\lambda_4,\lambda_6$. This is clearly not a continuous function, as a phase transition occurs before and after the specified $T_0$. Above the $T^*$ the potential should be $0$, meaning the energy is minimized, coming only from the kinetic energy.
\item The plot is the following:
  \fig{9.0}{plusl4}{The same process as the previous figure but with a positive $\lambda_4$}
  Since this is a numerical computation, the region around $T_c$ does not seem to be continuous, but I believe it should be, especially when more and computations are done. This should also be a second order phase transition although I do not know why. 
\end{enumerate}
\section*{Question 3}
The Lagrangian for convenience:
\begin{align*}
  \L=\qty(\D_\mu-ieA_\mu)\phi^*\qty(\D^\mu+ieA^\mu)\phi
  -m_0^2\phi^*\phi-\frac{\lambda}{2}(\phi^*\phi)^2
  -\frac{1}{4}\qty(\D^\mu A^\nu-\D^\nu A^\mu)\qty(\D_\mu A_\nu-\D_\nu A_\mu)
\end{align*}
\begin{enumerate}
\item Going term by term starting with the terms proportional to $\phi^*\phi$
  \begin{align*}
    \phi^*\phi\to\phi^*\exp{ie\Lambda}\exp{-ie\Lambda}\phi=\phi^*\phi
  \end{align*}
  Since the mass term and the $\lambda$ term in the Lagrangian both do not have anything else, so they are invariant. The next term would be the covariant derivative $D_\mu$:
  \begin{align*}
    (\D_\mu+ieA_\mu)\phi&\to(\D_\mu+ie\qty(A_\mu+\D_\mu\Lambda))
    \phi\exp{-ie\Lambda}\\
    &=(\D_\mu-ie\D_\mu\Lambda+ieA_\mu+ie\D_\mu\Lambda)\phi\exp{-ie\Lambda}\\
    &=(\D_\mu+ieA_\mu)\phi\exp{-ie\Lambda}\\
    &=\exp{-ie\Lambda}D_\mu\phi
  \end{align*}
  What we have is a product of $D_\mu$ terms, so we have:
  \begin{align*}
    (D_\mu\phi)^*(D^\mu\phi)\to\exp{ie\Lambda}\exp{-ie\Lambda}
    (D_\mu\phi)^*(D^\mu\phi)=(D_\mu\phi)^*(D^\mu\phi)
  \end{align*}
  So the term in the Lagrangian is in fact invariant. By construction the field strength tensor should be invariant, but we should check anyways!
  \begin{align*}
    F_{\mu\nu}&=\D_\mu A_\nu-\D_\nu A_\mu\to
    \D_\mu\qty(A_\nu+\D_\nu\Lambda)-\D_\nu\qty(A_\mu+\D_\mu\Lambda)\\
    &=F_{\mu\nu}+\D_\mu\Lambda-\D_\nu\Lambda=F_{\mu\nu}
  \end{align*}
  Hence, two separate field strength tensors should be invariant. Thus, under this gauge transformation, the Lagrangian is invariant:
  \begin{align*}
    \boxed{\L\to\L}
  \end{align*}
\item There are three total field equations, first for $\phi^*$:
  \begin{align*}
    \fdv{\L}{\phi^*}&=-ieA_\mu\D^\mu\phi-m_0^2\phi-\lambda\phi^*\phi^2\\
    \fdv{\L}{(\D_\mu\phi^*)}&=D^\mu\phi\\
    \D_\mu\fdv{\L}{(\D_\mu\phi^*)}&=\D_\mu D^\mu\phi\\
  \end{align*}
  Notice that when we set these terms equal to each other, we get the following:
  \begin{align*}
    \fdv{\L}{\phi^*}=\D_\mu\fdv{\L}{(\D_\mu\phi^*)}\implies
    (D_\mu D^\mu+m_0^2)\phi=\lambda\phi^*\phi^2
  \end{align*}
  Now for $\phi$:
  \begin{align*}
    \fdv{\L}{\phi}&=ieD_\mu\phi^*A^\mu-m_0^2\phi^*-\lambda(\phi^*)^2\phi\\
    \fdv{\L}{(\D_\mu\phi)}&=(D^\mu\phi)^*\\
    \D_\mu\fdv{\L}{(\D_\mu\phi^*)}&=\D_\mu(D^\mu\phi)^*\\
  \end{align*}
  So the field equation will look like:
  \begin{align*}
    D_\mu \qty(D^\mu\phi)^*+m_0^2\phi^*=\lambda(\phi^*)^2\phi
  \end{align*}
  Finally for $A$:
  \begin{align*}
    \fdv{\L}{A_\mu}&=\frac{1}{2}\qty[-ie\phi^*D^\phi+ie\phi(D^\mu\phi)^*]\\
    &=-\frac{ie}{2}\qty[\phi^*D^\mu\phi-\phi(D^\mu\phi)^*]\\
    \fdv{\L}{(\D_\nu A_\mu)}&=-\fdv{(\D_\nu A_\mu)}
    -\frac{1}{2}\qty(\D_\mu A_\nu\D^\mu A^\nu-\D_\nu A_\mu\D^\mu A^\nu)\\
    &=-\frac{1}{2}\fdv{(\D_\nu A_\mu)}\qty(\D_\mu A_\nu F^{\mu\nu})\\
    &=-\frac{1}{2}F^{\mu\nu}\\
    \D_\mu\fdv{\L}{(\D_\nu A_\mu)}&=-\frac{1}{2}\D_\mu F^{\mu\nu}
  \end{align*}
  So our field equations will look like current conservation:
  \begin{align*}
    \D_\mu F^{\mu\nu}=ie\qty[\phi^*D^\mu\phi-\phi(D^\mu\phi)^*]
  \end{align*}
\item The Hamiltonian density is given by the following:
  \begin{align*}
    \H=\sum_{\psi=\phi,\phi^*,A_\mu}\Pi_\psi\D_0\psi-\L
  \end{align*}
  We need to compute the conjugate momentum for each of the fields:
  \begin{align*}
    \Pi_{\phi^*}&=D^0\phi\\
    \Pi_\phi&=(D^0\phi)^*\\
    \Pi_{A_\mu}&=-\frac{1}{2}\D_0F^{\mu0}
  \end{align*}
  We can rewrite $\D_0\phi$ and $\D_0\phi^*$ as:
  \begin{align*}
    \D_0\phi&=D_0\phi-ieA_0\\
    \D_0\phi^*&=D_0\phi^*+ieA_0
  \end{align*}
  So the products are:
  \begin{align*}
    \Pi_{\phi^*}\D_0\phi^*&=D^0\phi\qty(D_0\phi^*+ieA_0)\\
    \Pi_{\phi}\D_0\phi&=(D^0\phi)^*\qty(D_0\phi-ieA_0)\\
    \Pi_{A_\mu}&=-\frac{1}{2}\D_0F^{\mu0}\D_0A_\mu
  \end{align*}
  Combining all of these and rewriting the lagrangian in terms of conjugate momentum will get you the Hamiltonian density $\H$
\item We now will replace $\phi$ with:
  \begin{align*}
    \phi=\rho e^{i\theta}
  \end{align*}
  The terms with $\phi^*\phi$ simply turn into $\abs{\rho}^2$, the only significant changes come from derivatives:
  \begin{align*}
    D_\mu\phi&=\qty(\D_\mu+ieA_\mu)\rho e^{i\theta}\\
    (D_\mu\phi)^*D^\mu\phi&=
    \mqty{\D_\mu\rho\D^\mu\rho+i\rho\D_\mu\rho\D^\mu\theta+ieA^\mu\rho\D_\mu\rho\\
      -i\rho\D_\mu\theta\D^\mu\rho+\rho^2\D_\mu\theta\D^\mu\theta+
      eA^\mu\rho^2\D_\mu\theta\\-ieA_\mu\rho\D^\mu\rho+e\rho^2A_\mu\D^\mu\theta+
      e^2\rho^2A_\mu A^\mu}\\
    &=\D_\mu\rho\D^\mu\rho+(\D_\mu\theta\D^\mu\theta)\rho^2+e^2\rho^2A_\mu A^\mu
    +2e\rho^2A_\mu\D^\mu\theta\\
    &=\D_\mu\rho\D^\mu\rho+\rho^2\qty(\D_\mu\theta\D^\mu\theta+
    e^2A_\mu A^\mu+2eA_\mu\D^\mu\theta)
  \end{align*}
  There is no change with the field strength term. The Lagrangian is:
  \begin{align*}
    \L=\D_\mu\rho\D^\mu\rho+\rho^2\qty(\D_\mu\theta\D^\mu\theta+
    e^2A_\mu A^\mu+2eA_\mu\D^\mu\theta)
    -m_0^2\abs{\rho}^2-\frac{\lambda}{2}\abs{\rho}^4
    -\frac{1}{4}F^{\mu\nu}F_{\mu\nu}
  \end{align*}
  If we choose the London gauge all terms with $\theta$ disappear:
  \begin{align*}
    \L=\D_\mu\rho\D^\mu\rho+\rho^2e^2A_\mu A^\mu
    -m_0^2\abs{\rho}^2-\frac{\lambda}{2}\abs{\rho}^4
    -\frac{1}{4}F^{\mu\nu}F_{\mu\nu}
  \end{align*}
  The equations of motion for $\rho$ are then:
  \begin{align*}
    \fdv{\L}{\rho}&=2\rho e^2A_\mu A^\mu-2m_0^2\rho-2\lambda\rho^3\\
    \fdv{\L}{\D_\mu\rho}&=\D^\mu\rho\\
    \D_\mu\fdv{\L}{\D_\mu\rho}&=\D_\mu\D^\mu\rho
  \end{align*}
  So the equations of motion are:
  \begin{align*}
    \frac{1}{2}\D^2\rho=\rho e^2A_\mu A^\mu-m_0^2\rho-\lambda\rho^3
  \end{align*}
\item If we have a constant term in $\rho$ we may have the following term:
  \begin{align*}
    \L_{eff}=\overline{\rho}^2e^2A_\mu A^\mu
  \end{align*}
  Which could be seen as a photon mass, as it is quadratic in $A$ and so in our equations of motion we would get an effective Klein-Gordon equation that looks like:
  \begin{align*}
    \D^2A_\mu+m_\gamma^2A_\mu=???
  \end{align*}
  Which would represent a photon with mass as opposed to the normal field equation which simply has:
  \begin{align*}
    \D^2A_\mu=0
  \end{align*}
\end{enumerate}
\end{document}