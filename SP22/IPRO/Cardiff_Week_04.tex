\documentclass[12pt]{article}

\title{Week 4 Report}
\author{Michael Cardiff}
\date{\today}

%% science symbols 
\usepackage{amsmath}
\usepackage{amssymb}
\usepackage{physics}

%% general pretty stuff
\usepackage{bm}
\usepackage{enumitem}
\usepackage{float}
\usepackage{graphicx}
\usepackage[margin=1in]{geometry}

% figures
\graphicspath{ {./figs/} }

\newcommand{\fig}[3]
{
  \begin{figure}[H]
    \centering
    \includegraphics[width=#1cm]{#2}
    \caption{#3}
  \end{figure}
}

\newcommand{\figref}[4]
{
  \begin{figure}[H]
    \centering
    \includegraphics[width=#1cm]{#2}
    \caption{#3}
    \label{#4}
  \end{figure}
}

\renewcommand{\L}{\mathcal{L}}

\begin{document}
\maketitle

This week I worked on a bit of research for our presentation, and I had to do some catching up since I had to leave before meeting with the professor last Friday. My research was mainly focused on convincing myself that this approach would be worth doing. This approach being the use of smartphones or related technologies during surgery specifically to do margin analysis. The article I found saw promising results when using an iPad for margin analysis, but the results seem to be limiting in the fact that most people have very little experience in the use of the mobile device, but this can change in time.

Our team meeting this week was very productive, we not only started our presentation for this Friday, but also agreed on our plans going forward. We decided to commit to doing Cassies's idea of the small lens items for our fluorescence microscope. The overall equipment required would be fairly cheap and even at that, the lenses are reuseable. We also divided ourselves into different teams based on what is going to be needed in the future. We created three total teams: Wet Lab, Materials, and Surgeon Outreach. I volunteered along with Eric and Cassie to be in the Materials group, which some work has been done already. The future looks promising for our group!
\end{document}