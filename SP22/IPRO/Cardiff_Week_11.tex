\documentclass[12pt]{article}

\title{\vspace{-3em}Week 11 Report}
\author{Michael Cardiff}
\date{\today}

%% science symbols 
\usepackage{amsmath}
\usepackage{amssymb}
\usepackage{physics}

%% general pretty stuff
\usepackage{bm}
\usepackage{enumitem}
\usepackage{float}
\usepackage{graphicx}
\usepackage[margin=1in]{geometry}

% figures
\graphicspath{ {./figs/} }

\newcommand{\fig}[3]
{
  \begin{figure}[H]
    \centering
    \includegraphics[width=#1cm]{#2}
    \caption{#3}
  \end{figure}
}

\newcommand{\figref}[4]
{
  \begin{figure}[H]
    \centering
    \includegraphics[width=#1cm]{#2}
    \caption{#3}
    \label{#4}
  \end{figure}
}

\renewcommand{\L}{\mathcal{L}}

\begin{document}
\maketitle
This week I fell a bit behind schedule but I finally think I am ready to produce physical results for my team. I finished familiarizing myself with the kinetic modelling code, everything from its input and output to each of the intermediate steps taken. I liked how many of the components corresponded to what was in the reading, even if it was a bit abstract. I am getting used to this type of abstraction especially in my computational physics class so it certainly is nothing new.

I sadly have not been able to meet much with my group lately as I have been busy with grad school visits, so I look forward to moving on further with them to see what progress has or has not been made on the project. I believe we will ge tto make a lot more progress in these coming weeks. 
\end{document}
