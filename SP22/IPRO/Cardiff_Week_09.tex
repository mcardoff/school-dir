\documentclass[12pt]{article}

\title{\vspace{-3em}Week 9 Report}
\author{Michael Cardiff}
\date{\today}

%% science symbols 
\usepackage{amsmath}
\usepackage{amssymb}
\usepackage{physics}

%% general pretty stuff
\usepackage{bm}
\usepackage{enumitem}
\usepackage{float}
\usepackage{graphicx}
\usepackage[margin=1in]{geometry}

% figures
\graphicspath{ {./figs/} }

\newcommand{\fig}[3]
{
  \begin{figure}[H]
    \centering
    \includegraphics[width=#1cm]{#2}
    \caption{#3}
  \end{figure}
}

\newcommand{\figref}[4]
{
  \begin{figure}[H]
    \centering
    \includegraphics[width=#1cm]{#2}
    \caption{#3}
    \label{#4}
  \end{figure}
}

\renewcommand{\L}{\mathcal{L}}

\begin{document}
\maketitle
This week presented a bit more clarity especially with respect to what we will be doing for the rest of the semester. We decided we will be focusing on the paired agent imaging. This is in order to ensure we have a clear goal throughout the remainder of the semseter. One of my team members, Cassie, came up with a tentative schedule for the rest of the semester which I believe is attainable despite the short time we have left in the semester. The plan involves me just creating models with different optical properties in order to test tumor heterogeneity. While this is not what we planned from the beginning I am glad we finally have something to focus on for the rest of the semester.

As for what I have been doing, I have ben attempting to catch up on what I missed from the class last week in terms of the kinetic modeling demo. I was mostly able to understand the slides provided to us as it was really consistent with the paper provided which had most of the same information on kinetic modelling. Otherwise the code is not that difficult to understand either, but since I am not as familiar with MATLAB as I am other languages, it may take a bit of time to adjust to the language over spring break. 
\end{document}