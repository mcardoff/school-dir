\documentclass[12pt]{article}

\title{Week 10 Report}
\author{Michael Cardiff}
\date{\today}

%% science symbols 
\usepackage{amsmath}
\usepackage{amssymb}
\usepackage{physics}

%% general pretty stuff
\usepackage{bm}
\usepackage{enumitem}
\usepackage{float}
\usepackage{graphicx}
\usepackage[margin=1in]{geometry}

% figures
\graphicspath{ {./figs/} }

\newcommand{\fig}[3]
{
  \begin{figure}[H]
    \centering
    \includegraphics[width=#1cm]{#2}
    \caption{#3}
  \end{figure}
}

\newcommand{\figref}[4]
{
  \begin{figure}[H]
    \centering
    \includegraphics[width=#1cm]{#2}
    \caption{#3}
    \label{#4}
  \end{figure}
}

\renewcommand{\L}{\mathcal{L}}

\begin{document}
\maketitle
This week and last were not very productive since I was busy over both spring break and the week back, but I did find enough time to finish looking through the kinetic modeling lecture in order to do what I need for the remainder of the semester, which will mostly be providing kinetic models to then feed forward in the process. The next few weeks will most likely be very busy for me as well in order to make up for what I have missed, but I am willing to put in the work.

As for the team, everyone has pretty much been working on what they need to. I was not able to attend our team meeting during class time this week but I am hoping to get an update from everyone so I am all caught up with what we are doing. However since we are at a decent point, I hope the volatility will calm down and we will be able to completely focus on the paired agent imaging. This leads to the team name, which I did not include in the previous write-up. We decided on Team 007, to go with the paired 'agent' imaging.
\end{document}