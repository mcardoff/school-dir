\documentclass[12pt]{article}

\title{\vspace{-3em}PHYS 437 HW 2}
\author{Michael Cardiff}
\date{\today}

%% science symbols 
\usepackage{amsmath}
\usepackage{amssymb}
\usepackage{physics}

%% general pretty stuff
\usepackage{bm}
\usepackage{enumitem}
\usepackage{float}
\usepackage{graphicx}
\usepackage[margin=1in]{geometry}

% figures
\graphicspath{ {./figs/} }

\newcommand{\fig}[3]
{
  \begin{figure}[H]
    \centering
    \includegraphics[width=#1cm]{#2}
    \caption{#3}
  \end{figure}
}

\newcommand{\figref}[4]
{
  \begin{figure}[H]
    \centering
    \includegraphics[width=#1cm]{#2}
    \caption{#3}
    \label{#4}
  \end{figure}
}

\renewcommand{\L}{\mathcal{L}}

\begin{document}
\maketitle
\section*{Question 1}
The angles $\theta$ are given by the Bragg's law:
\begin{align*}
  \sin\theta=\frac{n\lambda}{2d_{hk\ell}}
\end{align*}
The first three allowed peaks would be at the first three $(hk\ell)$ values allowed by a bcc lattice. The selection rules for bcc require that $h+k+\ell=2n$. Thus the first three that are $(1,1,0)$, $(2,0,0)$ and $(2,1,1)$. So the corresponding $d$ vals are:
\begin{align*}
  d_{110}&=\frac{a}{\sqrt{1^2+1^2+0^2}}=\frac{a}{\sqrt{2}}\\
  d_{200}&=\frac{a}{\sqrt{2^2+0^2+0^2}}=\frac{a}{2}\\
  d_{211}&=\frac{a}{\sqrt{2^2+1^2+1^2}}=\frac{a}{\sqrt{6}}
\end{align*}
The first three are such that $n=1$, and we are given $\lambda$ and $a$:
\begin{align*}
  \sin\theta_{hk\ell}=\frac{1.54}{2*d_{hk\ell}}
\end{align*}
So we can just take the inverse sine of the right hand side in order to get the angle:
\begin{align*}
  \sin\theta_{110}&=\frac{\sqrt{2}*1.54}{2*3.30}
  \implies\theta_{110}\approx\boxed{19.27^\circ}\\
  \sin\theta_{200}&=\frac{2*1.54}{2*3.30}
  \implies\theta_{200}\approx\boxed{27.82^\circ}\\
  \sin\theta_{211}&=\frac{\sqrt{6}*1.54}{2*3.30}
  \implies\theta_{211}\approx\boxed{34.86^\circ}
\end{align*}
\section*{Question 2}
We can once again use the Bragg law in order to find the lattice parameter $a$:
\begin{align*}
  a=\frac{n\lambda}{2\sin\theta}\sqrt{h^2+k^2+\ell^2}
\end{align*}
Once again we can assume $n=1$:
\begin{align*}
  a=\frac{\lambda}{2\sin\theta}\sqrt{h^2+k^2+\ell^2}
\end{align*}
We have the $h,k,\ell$ values, $(1,1,1)$ and $(2,0,0)$.
\begin{align*}
  a_{111}&=\frac{\lambda}{2\sin\theta}\sqrt{3}\\
  a_{200}&=\frac{\lambda}{2\sin\theta}2
\end{align*}
The values specifically are:
\begin{align*}
  a_{111}&\approx\boxed{6.55}\\
  a_{200}&\approx\boxed{6.59}
\end{align*}
\section*{Question 3}
The primitive translation vectors should be:
\begin{equation*}
  \boxed{
    \begin{gathered}
      \vb{a}_1=a\vu{x}\\
      \vb{a}_2=a\vu{y}\\
      \vb{a}_3=3a\vu{z}
    \end{gathered}
  }
\end{equation*}
As long as the $z$ axis is oriented in the direction of the tall part of the cuboid.

The reciprocal space vectors require the volume element:
\begin{align*}
  \vb{a}_1\vdot\qty(\vb{a}_2\times\vb{a}_3)=a\vu{x}\vdot3a\vu{x}=3a^3
\end{align*}
Now we can find each of the vectors:
\begin{align*}
  \vb{b}_1&=\frac{2\pi}{3a^3}3a^2\vu{y}\times\vu{z}=\frac{2\pi}{a}\vu{x}\\
  \vb{b}_2&=\frac{2\pi}{3a^3}3a^2\vu{z}\times\vu{x}=\frac{2\pi}{a}\vu{y}\\
  \vb{b}_3&=\frac{2\pi}{3a^3}a^2\vu{x}\times\vu{y}=\frac{2\pi}{3a}\vu{z}
\end{align*}
The first Brillouin zone will be the shape of the Wigner-Seitz cell of the reciprocal lattice, since the reciprocal lattice has the same structure as the real space lattice, so it will look something like this:
\fig{5.0}{hw2bz}{First Brillouin zone of this lattice, z pointing up}
\section*{Question 4}
\begin{enumerate}[label=\alph*)]
\item Vectors in the plane are defined by the translation vectors:
  \begin{align*}
    \vb{T}=\frac{\vb{a}_1}{h}+\frac{\vb{a}_2}{k}+\frac{\vb{a}_3}{\ell}
  \end{align*}
  So some arbitrary vectors in the plane are:
  \begin{align*}
    \vb{v}_1=\frac{\vb{a}_1}{h}-\frac{\vb{a}_2}{k}\quad
    \vb{v}_2=\frac{\vb{a}_2}{k}-\frac{\vb{a}_3}{\ell}
  \end{align*}
  Since these vectors lie in the plane, if we dot them with the vector $\vb{G}$, we will get $0$ if it is perpendicular to the plane:
  \begin{align*}
    \vb{v}_1\vdot\vb{G}&=\qty(\frac{\vb{a}_1}{h}-\frac{\vb{a}_2}{k})\vdot
    \qty(h\vb{b}_1+k\vb{b}_2+\ell\vb{b}_3)\\
    &=\vb{a}_1\vdot\vb{b}_1-\vb{a}_2\vdot\vb{b}_2
  \end{align*}
  The $b$ vectors without the same index as the $a$ vectors should be $0$ since they contain a cross product, which is perpendicular to the one you're dotting it with. The remaining ones should have the form:
  \begin{align*}
    \vb{a}_1\vdot\vb{b}_1=\vb{a}_1\vdot\qty(2\pi\frac{\vb{a}_2\times\vb{a}_3}
    {\vb{a}_1\vdot(\vb{a}_2\times\vb{a}_3)})
  \end{align*}
  We realize that the numerator and denominator are exactly the same, even in order of the vectors, so they cancel:
  \begin{align*}
    \vb{a}_1\vdot\vb{b}_1=2\pi
  \end{align*}
  We get something similar with the 2 product:
  \begin{align*}
    \vb{a}_2\vdot\vb{b}_2=2\pi\frac{\vb{a}_2\vdot(\vb{a}_3\times\vb{a}_1)}
    {\vb{a}_1\vdot(\vb{a}_2\times\vb{a}_3)}
  \end{align*}
  This still cancels out as the triple product is the same as long as the order of the indices is cyclic, and the reciprocal space vectors are defined in a way such that they all will be in this case. Hence our dot product is:
  \begin{align*}
    \vb{a}_2\vdot\vb{b}_2=2\pi
  \end{align*}
  And for our vector $\vb{v}_1$:
  \begin{align*}
    \vb{v}_1\vdot\vb{G}=\vb{a}_1\vdot\vb{b}_1-\vb{a}_2\vdot\vb{b}_2
    =\boxed{2\pi-2\pi=0}
  \end{align*}
  An entirely similar argument can be made for $\vb{v}_2$ giving the same result:
  \begin{align*}
    \boxed{\vb{v}_2\vdot\vb{G}=0}
  \end{align*}
  Hence $\vb{G}$ is perpendicular to the plane defined by $\vb{T}$
\item The unit normal vector to a plane is given by:
  \begin{align*}
    \vu{n}=\frac{\vb{G}}{\abs{\vb{G}}}
  \end{align*}
  This is the vector is normal to the plane, the interplanar spacing will be the projection onto this vector of the distance to the next plane, for example $\vb{a}_1/h$, dotting with the same $\vb{G}$ defined earlier, we simply get:
  \begin{align*}
    \frac{1}{\abs{\vb{G}}}\qty(\vb{a}_1\vdot\vb{b}_1)=\frac{2\pi}{\abs{\vb{G}}}
  \end{align*}
  Which is what we expect:
  \begin{align*}
    \boxed{d_{hk\ell}=\frac{2\pi}{\abs{\vb{G}}}}
  \end{align*}
\item We can arrive at this by dotting $\vb{G}$ with itself:
  \begin{align*}
    \vb{G}\vdot\vb{G}&=
    h^2\abs{\vb{b}_1}^2
    +k^2\abs{\vb{b}_2}^2
    +\ell^2\abs{\vb{b}_3}^2\\
    &=\qty(\frac{2\pi}{a})^2\qty(h^2+k^2+\ell^2)
  \end{align*}
  The second step is found using the reciprocal lattice vectors for a simple cubic lattice:
  \begin{align*}
    \vb{b}_1=\frac{2\pi}{a}\vu{x}\quad
    \vb{b}_2=\frac{2\pi}{a}\vu{y}\quad
    \vb{b}_3=\frac{2\pi}{a}\vu{z}
  \end{align*}
  We can finished by using the fact that $d^2$ is equal to:
  \begin{align*}
    d^2=\frac{(2\pi)^2}{\vb{G}\vdot\vb{G}}=\boxed{\frac{a^2}{h^2+k^2+\ell^2}}
  \end{align*}
\end{enumerate}
\section*{Question 5}
For diamond, the basis is two identical $fcc$ lattices, one offset from the other by $\frac{1}{4}$ in each direction. For the first lattice:
\begin{gather*}
  \vb{d}_1=\vb{0}=(0,0,0)\\
  F^{(1)}_{hk\ell}=f\qty[1+e^{i\pi(k+\ell)}+e^{i\pi(h+\ell)}+e^{i\pi(h+k)}]
\end{gather*}
For the second one
\begin{gather*}
  \vb{d}_2=\qty(\frac{1}{4},\frac{1}{4},\frac{1}{4})\\
  F^{(2)}_{hk\ell}=f\qty[1+e^{i\pi(k+\ell)}+e^{i\pi(h+\ell)}+e^{i\pi(h+k)}]
  e^{i\pi(k+k+\ell)/2}
\end{gather*}
So the full structure factor is the sum of these two:
\begin{align*}
  F_{hk\ell}=F^{(1)}+F^{(2)}=f\qty(1+e^{i\pi(k+\ell)}+e^{i\pi(k+\ell)}+e^{i\pi(h+k)})
  \qty(1+e^{i\pi(h+k+l)/2})
\end{align*}
So not only do $(h,k,\ell)$ all need to be even or all need to be odd, they will take the following values:
\begin{align*}
  F_{hk\ell}=
  \begin{cases}
    8f, & h+k+\ell=4N\\
    4(1\pm i)f, & h+k+\ell=2N+1\\
    0, & h+k+\ell=4N+2
  \end{cases}
\end{align*}
\section*{Question 6}
We are given $n$ for ground state hydrogen:
\begin{align*}
  \frac{1}{\pi a_0^3}e^{-2r/a_0}
\end{align*}
And the form factor is given by:
\begin{align*}
  f&=4\pi\int_0^\infty\dd{r} n(r)r^2\frac{\sin(Gr)}{Gr}\\
  &=\frac{4}{a_0^3}\int_0^\infty\dd{r}e^{-2r/a_0}\frac{\sin(Gr)}{Gr}\\
  &=\frac{4}{Ga_0}\int_0^\infty\dd{x}e^{-2x}\sin(Gxa_0)x
\end{align*}
When this is evaluated in Mathematica, we get:
\begin{align*}
  f=\frac{16}{\qty(4+G^2a_0^2)^2}
\end{align*}
Which is the expected result
\end{document}