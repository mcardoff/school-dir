\documentclass[12pt]{article}

\title{\vspace{-4em}PHYS 437 HW 4}
\author{Michael Cardiff}
\date{\today}

%% science symbols 
\usepackage{amsmath}
\usepackage{amssymb}
\usepackage{physics}

%% general pretty stuff
\usepackage{bm}
\usepackage{enumitem}
\usepackage{float}
\usepackage{graphicx}
\usepackage[margin=1in]{geometry}

% figures
\graphicspath{ {./figs/} }

\newcommand{\fig}[3]
{
  \begin{figure}[H]
    \centering
    \includegraphics[width=#1cm]{#2}
    \caption{#3}
  \end{figure}
}

\newcommand{\figref}[4]
{
  \begin{figure}[H]
    \centering
    \includegraphics[width=#1cm]{#2}
    \caption{#3}
    \label{#4}
  \end{figure}
}

\renewcommand{\L}{\mathcal{L}}
\newcommand{\D}{\mathcal{D}}

\begin{document}
\maketitle
\section*{Question 1}
The equation for the Debye temperature is:
\begin{align*}
  \theta=\frac{\hbar\omega_D}{k_B}
\end{align*}
The flattening region of the plot happens around $\nu_D=3.5\times10^{12}$\,Hz. This gives $\omega_D$ as:
\begin{align*}
  \omega_D=2\pi\nu_D=7\pi\times10^{12}
\end{align*}
The values of $k_B$ and $\hbar$ are known, so our temperature $\theta$ is:
\begin{align*}
  \theta\approx\frac{\qty(1.055\times10^{-34})\qty(7\pi\times10^{12})}
  {\qty1.38\times10^{-23}}\approx168\,\text{K}
\end{align*}
\section*{Question 2}
\subsection*{Density of Modes in 3-D}
We integrate up until the Debye frequency $\omega_D$:
\begin{align*}
  I=\int_0^{\omega_D}\D(\omega)\dd{\omega}
\end{align*}
The density of states in 3-D:
\begin{align*}
  \D(\omega)=\frac{V\omega^2}{2\pi^2v^3}
\end{align*}
And the corresponding Debye frequency:
\begin{align*}
  \omega_D=\qty(\frac{6\pi^2v^3N}{V})^{1/3}
\end{align*}
The integral then becomes
\begin{align*}
  \int_0^{\omega_D}\D(\omega)\dd{\omega}
  =\frac{V}{2\pi^2v^3}\int_0^{\omega_D}\omega^2\dd{\omega}
  =\frac{V}{6\pi^2v^3}\eval{\omega^3}_0^{\omega_D}
  =\frac{V}{6\pi^2v^3}\omega_D^3
\end{align*}
Simplifying the Debye frequency:
\begin{align*}
  \frac{V}{6\pi^2v^3}\omega_D^3=\frac{V}{6\pi^2v^3}\qty(\frac{6\pi^2v^3N}{V})
  =\frac{V}{V}\frac{6\pi^2v^3}{6\pi^2v^3}N=N
\end{align*}
Which is exactly the expected result:
\begin{align*}
  \boxed{\int_0^{\omega_D}\D(\omega)\dd{\omega}=N}
\end{align*}
For 3-D
\subsection*{Heat Capacity}
The heat capacity is given by:
\begin{align*}
  C_V=\dv{U}{T}
\end{align*}
We have an expression for $U$:
\begin{align*}
  U=\int\dd{\omega}\D(\omega)\ev{n(\omega)}\hbar\omega
  =\int_0^{\omega_D}\dd{\omega}\qty(\frac{V\omega^2}{2\pi^2v^3})
  \qty(\frac{\hbar\omega}{e^{\hbar\omega/\tau}-1})
\end{align*}
The only thing dependent on temperature is $\tau=k_BT$, and the Debye frequency $\omega_D$ is not dependent on temperature either, so without any punishment, we can move the derivative into the integrand, turning it into a partial derivative:
\begin{align*}
  C_V=\dv{T}\int_0^{\omega_D}\dd{\omega}\qty(\frac{V\omega^2}{2\pi^2v^3})
  \qty(\frac{\hbar\omega}{e^{\hbar\omega/k_BT}-1})
  =\frac{3V\hbar}{2\pi^2v^3}\int_0^{\omega_D}\dd{\omega}\omega^3\pdv{T}
  \qty(\frac{1}{e^{\hbar\omega/k_BT}-1})
\end{align*}
The derivative is:
\begin{align*}
  \pdv{T}\qty(\frac{1}{e^{\hbar\omega/k_BT}-1})=\frac{\hbar\omega}{k_BT^2}
  \frac{e^{\hbar\omega/k_BT}}{(e^{\hbar\omega/k_BT}-1)^2}
\end{align*}
Subbing into our integral again:
\begin{align*}
  C_V=\frac{3V\hbar^2}{2\pi^2v^3k_BT^2}\int_0^{\omega_D}\dd{\omega}
  \frac{\omega^4e^{\hbar\omega/k_BT}}{(e^{\hbar\omega/k_BT}-1)^2}
\end{align*}
Which is equation 30, at least the side that is still in terms of $\omega$
\iffalse % ignore from here
We can factor out a few constant terms:
\begin{align*}
  U=\frac{V\hbar}{2\pi^2v^3}\int_0^{\omega_D}\dd{\omega}\frac{\omega^3}
  {e^{\hbar\omega/\tau}-1}
\end{align*}
Introducing a change in variables, we get the following:
\begin{gather*}
  x=\frac{\hbar\omega}{k_BT}\\
  \dd{x}=\frac{\hbar}{k_BT}\dd{\omega}\\
  x_D=\frac{\hbar\omega_D}{k_BT}\\
  U=\frac{3Vk^4_BT^4}{2\pi^2v^3\hbar^3}\int_0^{x_D}\dd{x}\frac{x^3}{e^x-1}
\end{gather*}
We can then introduce the Debye temperature:
\begin{gather*}
  x_D=\frac{\theta}{T}\\
  \theta=\frac{\hbar v}{k_B}\qty(\frac{6\pi^2N}{V})^{1/3}
\end{gather*}
Since we have a $V$ as well as $(\hbar v)^3$ we can extract a $(T/\theta)^3$:
\begin{align*}
  \qty(\frac{T}{\theta})^3&=\frac{k_B^3V}{\hbar^3v^36\pi^2N}\\
  Nk_BT\qty(\frac{T}{\theta})^3&=\frac{k_B^4VT}{\hbar^3v^36\pi^2}
\end{align*}
And we can then see that our expression is:
\begin{align*}
  U=9Nk_BT\qty(\frac{T}{\theta})^3\int_0^{x_D}\dd{x}\frac{x^3}{e^{x}-1}
\end{align*}
\fi % to here
\subsection*{The Limit}
The integral for $U$ in terms of $x$ and with constants rearranged:
\begin{align*}
  U=9Nk_BT\qty(\frac{T}{\theta})^3\int_0^{x_D}\frac{x^3}{e^x-1}
\end{align*}
In our limit, $x$ is small, so we can expand the exponential in terms of a taylor series:
\begin{align*}
  e^x\approx 1+x
\end{align*}
So our integral becomes:
\begin{align*}
  \int_0^{x_D}\dd{x}\frac{x^3}{1+x-1}=\int_0^{x_D}\dd{x}x^2
\end{align*}
Evaluating this is simple:
\begin{align*}
  I=\eval{\frac{x^3}{3}}_0^{x_D}=\frac{x_D^3}{3}=\frac{1}{3}\frac{\theta^3}{T^3}
\end{align*}
In the coefficient of our energy we had:
\begin{align*}
  U=9Nk_BT\frac{T^3}{\theta^3}\frac{1}{3}\frac{\theta^3}{T^3}=3Nk_BT
\end{align*}
Then $C_V$ would be:
\begin{align*}
  C_V=\dv{U}{T}=3Nk_B
\end{align*}
\section*{Question 3}
\subsection*{Density of Modes}
The dispersion relation is exactly given by:
\begin{align*}
  \omega=\omega_m\abs{\sin\frac{Ka}{2}}
\end{align*}
Solving for $K$ in terms of $\omega$ to find $\dv{K}{\omega}$:
\begin{align*}
  K=\frac{2}{a}\arcsin\frac{\omega}{\omega_m}
\end{align*}
The derivative:
\begin{align*}
  \dv{K}{\omega}=\frac{2}{a}\frac{1}{(\omega_m^2-\omega^2)^{1/2}}
\end{align*}
Simplifying with the formula for density of states we get:
\begin{align*}
  \D(\omega)=\frac{2N}{\pi}\frac{1}{\qty(\omega_m^2-\omega^2)^{1/2}}
\end{align*}
The expected result
\subsection*{Optical Branch}
In $K$ space the volume of a sphere is:
\begin{align*}
  V=\frac{4}{3}\pi K^3=\frac{4}{3}\qty((\omega_0-\omega)A)^{3/2}
\end{align*}
The frequency density is then the derivative of this with respect to $\omega$ times a constant:
\begin{align*}
  \D(\omega)=\frac{L^3}{(2\pi)^3}(2\pi/A^{3/2})\qty(\omega_0-\omega)^{1/2}
\end{align*}
Near the minimum the density vanishes!
\section*{Question 4}
\subsection*{Low Temperature Heat Capacity}
If it is restricted to a single plane, it is essentially 2-D like graphene. The density in this case is:
\begin{align*}
  \frac{L^2}{(2\pi)^2}
\end{align*}
Allowed $k$ vals per unit area in $k$ space. We use a ring of radius $k$ to find $N$:
\begin{align*}
  N=\frac{L^2}{(2\pi)^2}\pi k^2
\end{align*}
Using the Debye approximation we can find it in terms of $\omega$:
\begin{align*}
  N=\frac{L^2\omega^2}{4\pi v^2}
\end{align*}
Then the density is:
\begin{align*}
  \D(\omega)=\frac{L^2\omega}{2\pi v^2}
\end{align*}
The energy is given by:
\begin{align*}
  U=2\int_0^{\omega_D}\D(\omega)\ev{n(\omega)}\hbar\omega\dd{\omega}
  =\frac{2L^2k_B^3T^3}{2\pi v^2\hbar^2}\int_0^?\frac{x^2}{e^{x}-1}
\end{align*}
We can take the limit $x_D\to\infty$ since it is proportional to $T^{-1}$ and $T$ is small, leaving us with:
\begin{align*}
  U=\frac{2L^2k_B^3T^3}{2\pi v^2\hbar^2}\int_0^\infty\frac{x^2}{e^{x}-1}
\end{align*}
Thus the integral will have no temperature dependence, so it is just a number, and we will get a $C_V\propto T^2$ when differentiating.
\subsection*{What If They Interacted?}
If the layers were coupled together, then we can assume that they would act as a sort of chain where the basic unit was an entire sheet, and they together would form a mode of vibration. This would be a 1-D chain in essence, so we would expect the $C_V$ to be proportional to $T$ at low temperatures.
\end{document}
