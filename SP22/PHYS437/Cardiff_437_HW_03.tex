\documentclass[12pt]{article}

\title{\vspace{-3em}PHYS 437 HW 3}
\author{Michael Cardiff}
\date{\today}

%% science symbols 
\usepackage{amsmath}
\usepackage{amssymb}
\usepackage{physics}

%% general pretty stuff
\usepackage{bm}
\usepackage{enumitem}
\usepackage{float}
\usepackage{graphicx}
\usepackage[margin=1in]{geometry}

% figures
\graphicspath{ {./figs/} }

\newcommand{\fig}[3]
{
  \begin{figure}[H]
    \centering
    \includegraphics[width=#1cm]{#2}
    \caption{#3}
  \end{figure}
}

\newcommand{\figref}[4]
{
  \begin{figure}[H]
    \centering
    \includegraphics[width=#1cm]{#2}
    \caption{#3}
    \label{#4}
  \end{figure}
}

\renewcommand{\L}{\mathcal{L}}

\begin{document}
\maketitle
\section*{Question 1}
\begin{enumerate}[label=\alph*)]
\item There are not weird effects going on, so we can assume the total energy is:
  \begin{align*}
    E=T+V
  \end{align*}
  Further we can expand this by finding the energy per atom, then summing over all the atoms in the lattice:
  \begin{align*}
    E_s=T_s+V_s\qquad E=\sum_s E_s
  \end{align*}
  The kinetic energy for a single particle is given by:
  \begin{align*}
    T_s=\frac{1}{2}Mv_s^2=\frac{1}{2}M\qty(\dv{u_s}{t})^2
  \end{align*}
  So the total kinetic energy is the first term in the energy given:
  \begin{align*}
    T=\frac{1}{2}M\sum_s\qty(\dv{u_s}{t})^2
  \end{align*}
  For a single spring, the potential energy is:
  \begin{align*}
    V_{spring}=\frac{1}{2}Cx^2
  \end{align*}
  Without any loss of information, the stretching of the spring can be determined by its current displacement, minus any displacement of the one on the other side, hence:
  \begin{align*}
    \delta{x}_s=u_s-u_{s+1}
  \end{align*}
  So the total potential is:
  \begin{align*}
    V=\frac{1}{2}C\sum_s(u_s-u_{s+1})^2
  \end{align*}
  And the total energy:
  \begin{align*}
    E=T+V=\boxed{\frac{1}{2}M\sum_s\qty(\dv{u_s}{t})^2+
    \frac{1}{2}C\sum_s(u_s-u_{s+1})^2}
  \end{align*}
\item The time derivative of $u_s$ with respect to time is:
  \begin{align*}
    \dv{u_s}{t}=-u\omega\sin(\omega t - sKa)
  \end{align*}
  Squaring this and time averaging yields:
  \begin{align*}
    \ev{\qty(\dv{u_s}{t})^2}=\ev{u^2\omega^2\sin^2(\omega t- sKa)}
    =\frac{u^2\omega^2}{2}
  \end{align*}
  Hence the average kinetic energy per atom:
  \begin{align*}
    \ev{T_s}=\frac{1}{4}Mu^2\omega^2
  \end{align*}
  Which is exactly the first term.

  The potential is:
  \begin{align*}
    V_s&=\frac{C}{2}u^2\qty(\cos(\omega t+sKa)-\cos(\omega t+ska+ka))^2\\
       &=\frac{C}{2}u^2\qty(\cos(\omega t +sKa)(1-\cos(Ka))+
         \sin(\omega t+ sKa)\sin(k a))^2\\
       &=\frac{C}{2}u^2\qty(\cos^2(\omega t+sKa))(1-\cos(Ka))^2+
         \sin(\omega t+ sKa)\sin^2(k a)
  \end{align*}
  
\end{enumerate}
\section*{Question 2}
In a uniform lattice, since there is only one spring constant, periodic conditions can be imposed, that is:
\begin{align*}
  u_{s+1}=u_s(x+a)\quad u_{s-1}=u_s(x-a)
\end{align*}
Where $a$ is the lattice spacing. We can then taylor expand each of these to relate it to $u_s$:
\begin{align*}
  u_{s+1}&=u_s+a\dv{u}{x}+a^2+\dv[2]{u}{x}+\cdots\\
  u_{s-1}&=u_s-a\dv{u}{x}+a^2+\dv[2]{u}{x}+\cdots
\end{align*}
The equations of motion can be rewritten to make our results a bit more useful:
\begin{align*}
  M\dv[2]{u_s}{t}&=C\qty(u_{s+1}+u_{s-1}-2u_s)\\
  &=C\qty((u_{s+1}- u_s) - (u_{s}-u_{s-1}))
\end{align*}
We can then compute these differences in our series:
\begin{align*}
  u_{s+1}-u_s&=a\dv{u}{x}+a^2+\dv[2]{u}{x}+\cdots\\
  u_{s-1}-u_s&=-a\dv{u}{x}+a^2+\dv[2]{u}{x}+\cdots
\end{align*}
If we keep up through the second order terms, the term we need in the equation of motion:
\begin{align*}
  (u_{s+1}- u_s) - (u_{s}-u_{s-1})=2a^2\dv[2]{u_s}{x}
\end{align*}
Substitute and get it into the form of a wave equation:
\begin{align*}
  M\dv[2]{u_s}{t}&=2a^2C\dv[2]{u_s}{x}\\
  \dv[2]{u_s}{t}&=2\frac{a^2C}{M}\dv[2]{u_s}{x}
\end{align*}
Define $v_s$ in the following way:
\begin{align*}
  v_s^2=\frac{2a^2C}{M}
\end{align*}
So that we get the wave equation:
\begin{align*}
  \dv[2]{u_s}{t}&=v_2^2\dv[2]{u_s}{x}
\end{align*}
\section*{Question 3}
\section*{Question 4}
\end{document}