\documentclass[12pt]{article}

\title{\vspace{-3em}PHYS 437 HW 3}
\author{Michael Cardiff}
\date{\today}

%% science symbols 
\usepackage{amsmath}
\usepackage{amssymb}
\usepackage{physics}

%% general pretty stuff
\usepackage{bm}
\usepackage{enumitem}
\usepackage{float}
\usepackage{graphicx}
\usepackage[margin=1in]{geometry}

% figures
\graphicspath{ {./figs/} }

\newcommand{\fig}[3]
{
  \begin{figure}[H]
    \centering
    \includegraphics[width=#1cm]{#2}
    \caption{#3}
  \end{figure}
}

\newcommand{\figref}[4]
{
  \begin{figure}[H]
    \centering
    \includegraphics[width=#1cm]{#2}
    \caption{#3}
    \label{#4}
  \end{figure}
}

\renewcommand{\L}{\mathcal{L}}

\begin{document}
\maketitle
\section*{Question 1}
\begin{enumerate}[label=\alph*)]
\item There are not weird effects going on, so we can assume the total energy is:
  \begin{align*}
    E=T+V
  \end{align*}
  Further we can expand this by finding the energy per atom, then summing over all the atoms in the lattice:
  \begin{align*}
    E_s=T_s+V_s\qquad E=\sum_s E_s
  \end{align*}
  The kinetic energy for a single particle is given by:
  \begin{align*}
    T_s=\frac{1}{2}Mv_s^2=\frac{1}{2}M\qty(\dv{u_s}{t})^2
  \end{align*}
  So the total kinetic energy is the first term in the energy given:
  \begin{align*}
    T=\frac{1}{2}M\sum_s\qty(\dv{u_s}{t})^2
  \end{align*}
  For a single spring, the potential energy is:
  \begin{align*}
    V_{spring}=\frac{1}{2}Cx^2
  \end{align*}
  Without any loss of information, the stretching of the spring can be determined by its current displacement, minus any displacement of the one on the other side, hence:
  \begin{align*}
    \delta{x}_s=u_s-u_{s+1}
  \end{align*}
  So the total potential is:
  \begin{align*}
    V=\frac{1}{2}C\sum_s(u_s-u_{s+1})^2
  \end{align*}
  And the total energy:
  \begin{align*}
    E=T+V=\boxed{\frac{1}{2}M\sum_s\qty(\dv{u_s}{t})^2+
    \frac{1}{2}C\sum_s(u_s-u_{s+1})^2}
  \end{align*}
\item The time derivative of $u_s$ with respect to time is:
  \begin{align*}
    \dv{u_s}{t}=-u\omega\sin(\omega t - sKa)
  \end{align*}
  Squaring this and time averaging yields:
  \begin{align*}
    \ev{\qty(\dv{u_s}{t})^2}=\ev{u^2\omega^2\sin^2(\omega t- sKa)}
    =\frac{u^2\omega^2}{2}
  \end{align*}
  Hence the average kinetic energy per atom:
  \begin{align*}
    \ev{T_s}=\frac{1}{4}Mu^2\omega^2
  \end{align*}
  Which is exactly the first term.

  The potential is:
  \begin{align*}
    V_s&=\frac{C}{2}u^2\qty(\cos(\omega t+sKa)-\cos(\omega t+ska+ka))^2\\
    &=\frac{C}{2}u^2\qty(\cos(\omega t +sKa)(1-\cos(Ka))+
    \sin(\omega t+ sKa)\sin(k a))^2\\
    &=\frac{C}{2}u^2\qty(\cos^2(\omega t+sKa))(1-\cos(Ka))^2+
    \sin(\omega t+ sKa)\sin^2(K a)
  \end{align*}
  Since sine and cosine are orthogonal, the average of their product is zero, and once again the average of their square is $\frac{1}{2}$. We end up with:
  \begin{align*}
    \ev{V_s}=\frac{C}{2}u^2\qty(1-\cos(Ka))
  \end{align*}
  So the averaged energy is:
  \begin{align*}
    \ev{E_s}=\boxed{\frac{1}{4}M\omega^2u^2+\frac{1}{2}C\qty(1-\cos Ka)u^2}
  \end{align*}
  Using the dispersion relation:
  \begin{align*}
    \omega^2&=\frac{2C}{M}\qty(1-\cos Ka)\\
    \implies\frac{M\omega^2}{2C}&=1-\cos Ka
  \end{align*}
  Hence we get our final answer of:
  \begin{align*}
    \boxed{\ev{E_s}=\frac{1}{2}M\omega^2u^2}
  \end{align*}
\end{enumerate}
\section*{Question 2}
In a uniform lattice, since there is only one spring constant, periodic conditions can be imposed, that is:
\begin{align*}
  u_{s+1}=u_s(x+a)\quad u_{s-1}=u_s(x-a)
\end{align*}
Where $a$ is the lattice spacing. We can then taylor expand each of these to relate it to $u_s$:
\begin{align*}
  u_{s+1}&=u_s+a\dv{u}{x}+a^2+\dv[2]{u}{x}+\cdots\\
  u_{s-1}&=u_s-a\dv{u}{x}+a^2+\dv[2]{u}{x}+\cdots
\end{align*}
The equations of motion can be rewritten to make our results a bit more useful:
\begin{align*}
  M\dv[2]{u_s}{t}&=C\qty(u_{s+1}+u_{s-1}-2u_s)\\
  &=C\qty((u_{s+1}- u_s) - (u_{s}-u_{s-1}))
\end{align*}
We can then compute these differences in our series:
\begin{align*}
  u_{s+1}-u_s&=a\dv{u}{x}+a^2+\dv[2]{u}{x}+\cdots\\
  u_{s-1}-u_s&=-a\dv{u}{x}+a^2+\dv[2]{u}{x}+\cdots
\end{align*}
If we keep up through the second order terms, the term we need in the equation of motion:
\begin{align*}
  (u_{s+1}- u_s) - (u_{s}-u_{s-1})=2a^2\dv[2]{u_s}{x}
\end{align*}
Substitute and get it into the form of a wave equation:
\begin{align*}
  M\dv[2]{u_s}{t}&=2a^2C\dv[2]{u_s}{x}\\
  \dv[2]{u_s}{t}&=2\frac{a^2C}{M}\dv[2]{u_s}{x}
\end{align*}
Define $v_s$ in the following way:
\begin{align*}
  v_s^2=\frac{2a^2C}{M}
\end{align*}
So that we get the wave equation:
\begin{align*}
  \dv[2]{u_s}{t}&=v_2^2\dv[2]{u_s}{x}
\end{align*}
\section*{Question 3}
Our equations of motion are:
\begin{align*}
  m_1\dv[2]{u_s}{t}&=C\qty(v_s+v_{s-1}-2u_s)\\
  m_2\dv[2]{v_s}{t}&=C\qty(u_{s+1}+u_{s}-2v_s)\\
\end{align*}
Using the following guessed solution we can find the properties of the various modes:
\begin{align*}
  u_s&=u\exp{iska}\exp{-i\omega t}\\
  v_s&=v\exp{iska}\exp{-i\omega t}
\end{align*}
Plugging these into the equations of motion:
\begin{align*}
  -\omega^2m_1u&=Cv(1+\exp{-ika})-2Cu\\
  -\omega^2m_2v&=Cu(1+\exp{-ika})-2Cv
\end{align*}
We automatically assume the value of $k$ to be $K_{max}=\frac{\pi}{a}$. Both terms have an exponential of $ka$, so we end up with:
\begin{align*}
  \exp{-ik_{max}a}=\exp{-i\pi}=-1
\end{align*}
This means that our equations now look like:
\begin{align*}
  -\omega^2m_1u&=-2Cu\\
  -\omega^2m_2v&=-2Cv
\end{align*}
Then we divide to get rid of constants:
\begin{align*}
  \frac{m_1}{m_2}\frac{u}{v}&=\frac{u}{v}\\
  \implies\frac{u}{v}\qty(\frac{m_1}{m_2}-1)&=0
\end{align*}
Since we explicitly have $m_1\neq m_2$, we have:
\begin{align*}
  \frac{u}{v}=0\quad\text{or}\quad\frac{v}{u}=0
\end{align*}
So one of the two is not moving, and they are decoupled from one another, the one in the numerator is at rest while the denominator oscillates. 
\section*{Question 5}
Our equations of motion would be:
\begin{align*}
  m\dv[2]{u_s}{t}&=C\qty(v_s-u_s-10u_s-v_{s-1})=C\qty(v_s+10v_{s-1}-11u_s)\\
  m\dv[2]{v_s}{t}&=C\qty(10u_{s+1}+u_s-11v_s)
\end{align*}
Once again assume solutions of the form:
\begin{align*}
  u_s&=u\exp{iska}\exp{-i\omega t}\\
  v_s&=v\exp{iska}\exp{-i\omega t}
\end{align*}
The equations become:
\begin{align*}
  m\dv[2]{u_s}{t}&=mu\exp{iska}\omega^2\exp{-i\omega t}=-m\omega^2u_s\\
  &=Cv_s\qty(1+10\exp{-ika})-11Cu_s\\
  m\dv[2]{v_s}{t}&=mv\exp{iska}\omega^2\exp{-i\omega t}=-m\omega^2v_s\\
  &=Cu_s\qty(10\exp{ika}+1)-11Cv_s
\end{align*}
We end up with:
\begin{align*}
  \qty(-m\omega^2+11C)u_s-C\qty(1+10\exp{-ika})v_s&=0\\
  -C\qty(10\exp{-ika}+1)u_s+\qty(-m\omega^2+11C)v_s&=0
\end{align*}
We can write this as a matrix equation:
\begin{align*}
  \pmqty{-m\omega^2+11C&-C\qty(1+10\exp{-ika})\\
    -C\qty(10\exp{-ika}+1)&-m\omega^2+11C}
  \pmqty{u_s\\v_s}=\pmqty{0\\0}
\end{align*}
Using basic linear algebra, we know that this determinant must be 0:
\begin{align*}
  \vmqty{-m\omega^2+11C&-C\qty(1+10\exp{-ika})\\
    -C\qty(10\exp{-ika}+1)&-m\omega^2+11C}=\qty(-m\omega^2+11C)^2
  -C^2\qty(1+10\exp{-ika})^2=0
\end{align*}
Now we can break it up by $k$ value. First we do $k=0$, where the exponentials become $1$:
\begin{align*}
  &\qty(11C-m\omega^2)^2-C^2(11)(11)=0\\
  &\implies11C-m\omega^2=\pm 11C\\
  &\implies \omega^2=0,22\frac{C}{m}
\end{align*}
For $k=k_{max}$, the exponentials are $-1$:
\begin{align*}
  \qty(11C-m\omega^2)^2=C^281\\
  \implies 11C-m\omega^2=\pm 9C\\
  \implies \omega^2=2\frac{C}{m},20\frac{C}{m}
\end{align*}
The dispersion relation is then:
\fig{10.0}{hw4disp}{The dispersion relation, $\omega$ vs $k$}
\end{document}