\documentclass[12pt]{article}

\title{\vspace{-3em}PHYS 437 Review}
\author{Michael Cardiff}
\date{\today}

%% science symbols 
\usepackage{amsmath}
\usepackage{amssymb}
\usepackage{physics}

%% general pretty stuff
\usepackage{bm}
\usepackage{enumitem}
\usepackage{float}
\usepackage{graphicx}
\usepackage[margin=1in]{geometry}

% figures
\graphicspath{ {./figs/} }

\newcommand{\fig}[3]
{
  \begin{figure}[H]
    \centering
    \includegraphics[width=#1cm]{#2}
    \caption{#3}
  \end{figure}
}

\newcommand{\figref}[4]
{
  \begin{figure}[H]
    \centering
    \includegraphics[width=#1cm]{#2}
    \caption{#3}
    \label{#4}
  \end{figure}
}

\renewcommand{\L}{\mathcal{L}}
\newcommand{\vbb}[1]{\bm\vb{#1}}

\begin{document}
\maketitle
\section{Crystal Structure}
Any lattice point in a crystal structure can be found by knowing the basis and the translation vectors $\{\vb{a}_i\}$, in 3-D this looks like:
\begin{align*}
  \vb{T}=\Delta{\vb{r}}=u_1\vb{a}_1+u_2\vb{a}_2+u_3\vb{a}_3
\end{align*}

Some notable notations are:
\begin{itemize}
\item \emph{Conventional cell}: The first cell that comes to mind when you think of a crystal structure. 
\item \emph{Primitive cell}: The cell that has only one lattice point per cell.
\end{itemize}

The basic types of lattice we will study are:
\begin{itemize}
\item \emph{Simple Cubic} (SC) Conventional cell has an atom in each corner of the cube
\item \emph{Face Centered Cubic} (FCC) Conventional cell has an atom in each corner of the cube as well as one in the \textbf{center} of each \textbf{face}
\item \emph{Body Centered Cubic} (BCC) Conventional cell has an atom in each corner of the cube as well as one in the \textbf{center} of the \textbf{body}
\end{itemize}
From these description alone many of the properties in Kittel's table 2 can be calculated.

Some notable other structures are diamond, which is FCC with some caveats, NaCl, which is FCC but with a basis of one Na$^+$ and one Cl$^-$, CsCl which is actuall two simple subic structures, and Hexagonal close packing.
\section{Wave Diffraction and the Reciprocal Lattice}
Bragg's Law is the important equation here, though we can rewrite it in other terms, it is useful in the 'path difference' form here:
\begin{align*}
  2d\sin\theta=n\lambda
\end{align*}

Extinction Rules:
\begin{table}[H]
  \centering
  \begin{tabular}{c|c}
    Lattice & Allowed vals \\ \hline
    SC & All \\
    BCC & $h+k+\ell=2n$ \\
    FCC & $h,k,\ell$ all odd or all even \\
    Diamond & FCC +$h+k+\ell=4n$
  \end{tabular}
  \caption{Extinction Rules}
\end{table}

\subsection{Fourier Analysis}
Crystals are invariant under translations, they will look the same, so we have some periodic condition of electron number density, $n(\vb{r})$:
\begin{align*}
  n(\vb{r+T})=n(\vb{r})
\end{align*}
It might be useful to write this as a Fourier series:
\begin{align*}
  n(x)=n_0+\sum_{p>0}\qty(C_p\cos(2\pi px/a)+S_p\sin(2\pi px/a))
\end{align*}
This form satisfies the periodic condition explicitly.

We can write this in a single sum using a complex exponential and summing over all $p$ instead of just $p>0$:
\begin{align*}
  n(x)=\sum_pn_p\exp{2\pi ipx/a}
\end{align*}
Due to our formulation of the coefficients we find that:
\begin{align*}
  n_p=n^*_{-p}
\end{align*}

For multiple dimensions we replace the quantity $p$ with a vector $\vb{G}$ such that:
\begin{align*}
  n(\vb{r})=\sum_{\vb{G}}n_{\vb{G}}\exp{-\vb{G}\vdot\vb{r}}
\end{align*}

We can find the coefficient by taking an integral:
\begin{align*}
  n_p=\frac{1}{a}\int_0^a\dd{x}n(x)\exp{-i2\pi px/a}
\end{align*}
Using our sum definition for $n(x)$ we get:
\begin{align*}
  n_p=\frac{1}{a}\sum_{p'}n_{p'}\int_0^a\dd{x}\exp{-2\pi(p'-p)x/a}
\end{align*}
So long as $p\neq p'$ we can evaluate the integral:
\begin{align*}
  \frac{a}{i2\pi(p'-p)}\qty(\exp{-2\pi(p-p')}-1)=0
\end{align*}

In 3-D:
\begin{align*}
  n_{\vb{G}}=V_c^{-1}\int_{cell}\dd{V}n(\vb{r})\exp{-i\vb{G}\vdot\vb{r}}
\end{align*}

The reciprocal lattice vectors describe the spacing in (essentially) momentum space, they are the primitive vectors of the reciprocal lattice. 
\begin{align*}
  \vb{b}_1=2\pi\frac{\vb{a}_2\times\vb{a}_3}
  {\vb{a}_1\vdot\vb{a}_2\times\vb{a}_3}\qquad
  \vb{b}_2=2\pi\frac{\vb{a}_3\times\vb{a}_1}
  {\vb{a}_1\vdot\vb{a}_2\times\vb{a}_3}\qquad
  \vb{b}_3=2\pi\frac{\vb{a}_1\times\vb{a}_2}
  {\vb{a}_1\vdot\vb{a}_2\times\vb{a}_3}
\end{align*}
Due to their definition with a cross product, the following property holds:
\begin{align*}
  \vb{b}_i\vdot\vb{a}_j=2\pi\delta_{ij}
\end{align*}
So points in reciprocal are mapped by the vector $\vb{G}$:
\begin{align*}
  \vb{G}=v_1\vb{b}_1+v_2\vb{b}_2+v_3\vb{b}_3
\end{align*}

Different scattering conditions can be found using various properties of waves:
\begin{align*}
  \Delta{\vb{k}}=\vb{G}
\end{align*}
We can also write a diffraction condition, equivalent to Bragg's law:
\begin{align*}
  2\vb{k}\vdot\vb{G}=G^2
\end{align*}
Which reduces to the more familiar Bragg's law:
\begin{align*}
  2d_{hkl}\sin\theta=\lambda
\end{align*}

We define a Brillouin zone based on the following reformulation of the diffraction condition:
\begin{align*}
  \vb{k}\vdot\qty(\frac{1}{2}\vb{G})=\qty(\frac{1}{2}G)^2
\end{align*}
Here is the process for finding Brillouin zones:
\begin{enumerate}
\item Select a vector $\vb{G}$ from the origin to a reciprocal lattice point.
\item Construct a plane normal to this vector at its midpoint
\item This plane is the boundary of the Brillouin zone.
\end{enumerate}

\subsection{Fourier Analysis of the Basis}
When the diffraction $\Delta{\vb{k}}=\vb{G}$ condition is met, we can write the scattering amplitude for a crystal of $N$ cells can be written:
\begin{align*}
  F_{\vb{G}}=N\int\dd{V}n(\vb{r})\exp{-i\vb{G}\vdot\vb{r}}=NS_{\vb{G}}
\end{align*}
Where $S_{\vb{G}}$ is the structure factor is defined as an integral over a single cell, with $\vb{r}=0$ at one corner.

We can write the total concentration as a sum over individual electron concentrations $n_j$ associated with each atom $j$ of the cell:
\begin{align*}
  n(\vb{r})=\sum_{j-1}^sn_j(\vb{r-r}_j)
\end{align*}
We can then write the structure factor defined earlier may be written over interals over the atoms:
\begin{align*}
  S_{\vb{G}}&=\sum_j\int\dd{V}n_j(\vb{r-r}_j)\exp{-i\vb{G\vdot r}}\\
  &=\sum_j\exp{-i\vb{G}\vdot\vb{r}_j}\int\dd{V}n_j(\vb{\bm{\rho}})
  \exp{-i\vb{G}\vdot\vb{\bm{\rho}}}
\end{align*}
We then define the atomic form factor as:
\begin{align*}
  f_j=\int\dd{V}n_j(\vb{\bm{\rho}})\exp{-i\vb{G}\vdot\vb{\bm{\rho}}}
\end{align*}
So that the structure factor is:
\begin{align*}
  S_{\vb{G}}=\sum_jf_j\exp{-i\vb{G}\vdot\vb{r}_j}
\end{align*}
This can be written in terms of the position of the atoms $(x_i,y_i,z_i)$ and the standard $\vb{G}$ vector:
\begin{align*}
  \vb{G}\vdot\vb{r}_j=2\pi\qty(v_1x_j+v_2y_j+v_3z_j)
\end{align*}
The structure factor is then:
\begin{align*}
  S_{\vb{G}}=\sum_jf_j\exp{-2\pi i(v_1x_j+v_2y_j+v_3z_j)}
\end{align*}
There are certain assumptions you can make to reduce the integral for $f_j$, you would end up with:
\begin{align*}
  f_j=4\pi\int\dd{r}n_j(r)r^2\frac{\sin(Gr)}{Gr}
\end{align*}
Where $G=\norm{\vb{G}}$
\section{Phonons}

\subsection{Crystal Vibrations}
If we describe an infinite chain of atoms connected in 1-D, the force on atom $s$ will be:
\begin{align*}
  F_s=C(u_{s+1}-u_s)+C(u_{s-1}-u_s)
\end{align*}
Where $C$ is the force constant between each of the atoms, for these purposes the constant is the same for each atom, this is not necessarily true in all cases.

We find that the solutions have the relationship:
\begin{align*}
  u_{s\pm1}=u\exp{isKa}\exp{\pm iKa}
\end{align*}
From this we can derive a dispersion relation:
\begin{align*}
  \omega^2=\frac{2C}{M}\qty(1-\cos Ka)
\end{align*}
At the first Brillouin zone, the derivative $\dv{\omega}{K}$ should be $0$:
\begin{align*}
  \dv{\omega^2}{K}=\frac{2Ca}{M}\sin(Ka)=0
\end{align*}
Therefore $K=\pm\pi/a$. At not the boundary of the Brillouin zone, we define the derivative as the group velocity:
\begin{align*}
  v_g=\dv{\omega}{K}
\end{align*}
Having two or mode per the primitive cell introduces acoustic and optical branches, specifically, if there are $p$ atoms in the cell, there will be 3 acoustic and 3p-3 optical branches. With $p$ atoms in the primitive cell, and $N$ primitive cells, there are $pN$ atoms. Each atom has three degrees of freedom, so $3pN$ degrees of freedom in the crystal total. The number of $K$ values allowed in a branch is $N$, defined by the Brillouin zone. So all of the acoustic branches account for $3N$ degrees of freedom (2 transverse and one longitudinal), the remaining $3(p-1)N$ are from optical branches.

For a crystal of say $2$ atoms, we get a system of equations:
\begin{align*}
  M_1\dv[2]{u_s}{t}&=C(v_s+v_{s-1}-2u_s)\\
  M_2\dv[2]{v_s}{t}&=C(u_{s+1}+u_s-2v_s)
\end{align*}

We examine two cases for the dispersion relation:
\begin{itemize}
\item[$Ka\ll1$)]For the optical branch the dispersion relation is given by:
  \begin{align*}
    \omega^2\approx2C\qty(\frac{1}{M_1}+\frac{1}{M_2})
  \end{align*}
  And the acoustic:
  \begin{align*}
    \omega^2\approx2C\qty(\frac{1}{M_1}+\frac{1}{M_2})
  \end{align*}
\item[$Ka=\pm\pi$)] The roots are:
  \begin{align*}
    \omega^2=\frac{2C}{M_1}\qquad\frac{2C}{M_2}
  \end{align*}
\end{itemize}

The energy of these lattice vibrations are quantized:
\begin{align*}
  E=\qty(n+\frac{1}{2})\hbar\omega
\end{align*}
\subsection{Thermal Properties}

\section{Free Electron Fermi Gas}


\end{document}