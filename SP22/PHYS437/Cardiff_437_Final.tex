\documentclass[12pt]{article}

\title{\vspace{-3em}PHYS 437 Final Exam}
\author{Michael Cardiff}
\date{\today}

%% science symbols 
\usepackage{amsmath}
\usepackage{amssymb}
\usepackage{physics}
\usepackage{siunitx}

%% general pretty stuff
\usepackage{bm}
\usepackage{enumitem}
\usepackage{float}
\usepackage{graphicx}
\usepackage{tikz}
\usepackage[margin=1in]{geometry}

\def\checkmark{\tikz\fill[scale=0.4](0,.35) -- (.25,0) -- (1,.7) -- (.25,.15) -- cycle;}

% figures
\graphicspath{ {./figs/} }

\newcommand{\fig}[3]
{
  \begin{figure}[H]
    \centering
    \includegraphics[width=#1cm]{#2}
    \caption{#3}
  \end{figure}
}

\newcommand{\figref}[4]
{
  \begin{figure}[H]
    \centering
    \includegraphics[width=#1cm]{#2}
    \caption{#3}
    \label{#4}
  \end{figure}
}

\renewcommand{\L}{\mathcal{L}}

\begin{document}
\maketitle

\section{Magnetic Field Penetration in a Plate}

\subsection{$B(x)$ in a superconducting plate}
This is a linear problem, so the equation becomes a lot simpler:
\begin{align*}
  \dv[2]{B}{x}=\frac{1}{\lambda}B
\end{align*}
The boundary condition should be that $B(\pm\delta/2)=B_a$, plugging in these values we get:
\begin{align*}
  B\qty(\frac{\delta}{2})=B_a\frac{\cosh(\delta/2\lambda)}
  {\cosh(\delta/2\lambda)}=B_a\qquad&\text{\checkmark}\\
  B\qty(-\frac{\delta}{2})=B_a\frac{\cosh(-\delta/2\lambda)}
  {\cosh(\delta/2\lambda)}=B_a\qquad&\text{\checkmark}
\end{align*}
In the second line I used the evenness of the $\cosh$ function to cancel it with the denominator. Since the coefficient is just a constant we can recognize that the second derivative of $\cosh$ is $\cosh$ and see that from that second derivative two factors of $\lambda^{-1}$ will come out and be canceled by our equation. 
\subsection{Effective Magnetization}
If we Taylor expand both of our $\cosh$ functions (since none appear in the final result), we get:
\begin{align*}
  \cosh(\frac{x}{\lambda})\approx 1+\frac{1}{2}\qty(\frac{x}{\lambda})^2\\
  \cosh(\frac{\delta}{2\lambda})\approx
  1+\frac{1}{2}\qty(\frac{\delta}{2\lambda})^2
\end{align*}
The magnetic field is then:
\begin{align*}
  B(x)=B_a\qty(\frac{1+x^2/2\lambda^2}{1+\delta^2/8\lambda^2})
\end{align*}
Using the geometric series we can approximate this series as a simple linear term since $\delta\ll\lambda$:
\begin{align*}
  B(x)=B_a-B_a\frac{\delta^2-4x^2}{8\lambda^2}
\end{align*}
This means that:
\begin{align*}
  \boxed{B(x)-B_a\equiv 4\pi M(x)=-B_a\frac{\delta^2-4x^2}{8\lambda^2}}
\end{align*}
\subsection{Current Density}
????
\section{Structure of a Vortex}
\subsection{Solution \& Boundary Condition}
Since the system we are considering is cylindrically symmetric, we only need to find the $\rho$ dependence:
\begin{align*}
  B-\lambda^2\qty(\pdv[2]{B}{\rho}+\frac1\rho\pdv{B}{\rho})=0
\end{align*}
Plugging this into Mathematica tells us that the solution is one of the modified Bessel functions:
\begin{align*}
  B(\rho)=CK_0(\rho/\lambda)
\end{align*}
If we integrate to get a flux, we get a $2\pi$ from the $\phi$ integration in polar coordinates, and the rest is:
\begin{align*}
  2\pi\int_0^\infty\rho B(\rho)\dd{\rho}&=2\pi C
  \int_0^\infty\dd{\rho}\rho K_0(\rho/\lambda)\\
  \phi_0&=2\pi C\lambda^2\int_0^\infty\dd{x}xK_0(x)
\end{align*}
The integral given is 1 from chucking it into Mathematica, giving us the constant $C$:
\begin{align*}
  \phi_0=2\pi C\lambda^2\implies \boxed{C=\frac{\phi_0}{2\pi\lambda^2}}
\end{align*}
Thus our solution is:
\begin{align*}
  \boxed{B(\rho)=\frac{\phi_0}{2\pi\lambda^2}K_0\qty(\frac{\rho}{\lambda})}
\end{align*}
\subsection{Limits}
If we have that $\lambda\gg\rho$, the second term in our differential equation dominates, giving us a log solution. If $\rho\gg\lambda$ then we can drop the second term in the laplacian, giving an exponential:
\begin{align*}
  \boxed{
    \mqty{(\rho\ll\lambda)&B(\rho)=(\phi_0/2\pi\lambda^2)\ln(\lambda/\rho)\\\\
    (\rho\gg\lambda)&B(\rho)=(\phi_0/2\pi\lambda^2)(\pi\lambda/2\rho)^{1/2}
    e^{-\rho/\lambda}}}
\end{align*}
The constant can be found by a series expansion about infinity, just something cool I found from playing in Mathematica. 
\section{BCS Coupling Strength for Various Materials}
Subbing in the approximation we are given leaves the following formula for BCS $T_C$:
\begin{align*}
  T_c\approx1.14\theta_De^{-1/N(0)V}
\end{align*}
From here it is simple to solve for the argument of the exponential:
\begin{align*}
  \boxed{N(0)V=-\ln\qty(\frac{T_c}{1.14\times\theta_D})}
\end{align*}
\subsection{Niobium}
For Nb, the $T_c=\SI{9.3}{\K}$ and $\theta_D=\SI{276}{\K}$
\begin{align*}
  N(0)V=-\ln\qty(\frac{\SI{9.3}{\K}}{1.14\times\SI{276}{\K}})=\boxed{3.5}
\end{align*}
\subsection{Lead}
For Pb, the $T_c=\SI{7.2}{\K}$ and $\theta_D=\SI{105}{\K}$
\begin{align*}
  N(0)V=-\ln\qty(\frac{\SI{7.2}{\K}}{1.14\times\SI{105}{\K}})=\boxed{2.8}
\end{align*}
\subsection{Aluminum}
For Al, the $T_c=\SI{1.2}{\K}$ and $\theta_D=\SI{433}{\K}$
\begin{align*}
  N(0)V=-\ln\qty(\frac{\SI{1.2}{\K}}{1.14\times\SI{433}{\K}})=\boxed{6.0}
\end{align*}
\section{Diffraction Effect of Josephson Junction}
Call $x$ the coordiate in the plane of the junction, such that $x$ is between $\pm w/2$. We can write the current in terms of a flux:
\begin{align*}
  \dd{j}\propto\cos\phi(x)\dd{x}
\end{align*}
The rectangle we are measuring the flux through is of width $2x$ and has thickness $T$, the flux $\phi=2xT B$ where $B$ is some magnetic field, giving the differential current density as:
\begin{align*}
  \dd{j}&=\frac{j_0}{w}\cos(\frac{e\phi(x)}{\hbar c})\\
  &=\frac{j_0}{w}\cos(\frac{2exTB}{\hbar c})
\end{align*}
Integrating this:
\begin{align*}
  j=\int_{-\infty}^\infty\dd{j}
\end{align*}
However, our 'infinity' is cut off, as there is not current density outside of the box:
\begin{align*}
  j=\frac{j_0}{w}\int_{-w/2}^{w/2}\cos(\frac{2exTB}{\hbar c})\dd{x}
\end{align*}
Since this is an even function we are integrating we can reduce this to just an integral over half the distance but doubled:
\begin{align*}
  j&=\frac{2j_0}{w}\int_{0}^{w/2}\cos(2\frac{exTB}{\hbar c})\dd{x}\\
  &=\frac{2j_0}{w}\eval(\frac{\sin(2exTB/\hbar c)}{2eTB/\hbar c}|_0^{w/2}\\
  &=\boxed{j_0\frac{\sin(weTB/\hbar c)}{weTB/\hbar c}}
\end{align*}
\section{BCS Coherence Length}
\subsection{Non-Dirty Materials}
We simply need the following formula:
\begin{align*}
  \xi_{BCS}=\frac{\hbar v_f}{\pi\Delta}
\end{align*}
\subsubsection{Niobium}
The fermi velocity for Ni is $\SI{2.03e6}{\m\per\s}$ and the band gap is: $\SI{3.05e-3}{\eV}$
\begin{align*}
  \xi_{BCS}&=\frac{(\SI{6.582e-16}{\eV\s})(\SI{2.03e6}{\m\per\s})}
  {\pi(\SI{3.05e-3}{\eV})}=\boxed{\SI{1.39e-7}{\m}}
\end{align*}
\subsubsection{Aluminum}
The fermi velocity for Al is $\SI{1.37e6}{\m\per\s}$ and the band gap is: $\SI{3.4e-4}{\eV}$
\begin{align*}
  \xi_{BCS}&=\frac{(\SI{6.582e-16}{\eV\s})(\SI{1.37e6}{\m\per\s})}
  {\pi(\SI{3.4e-4}{\eV})}=\boxed{\SI{8.44e-7}{\m}}
\end{align*}

\subsection{Concentration of Alloy}
The formula given for $H_{c2}$ is given from Kittel as:
\begin{align*}
  H_{c2}=\frac{\lambda}{\xi}H_c=\frac{\lambda}{\sqrt{\xi_0\lambda}}H_c
  =\sqrt\frac{\lambda}{\xi_0}H_c
\end{align*}
The critical field for Nb is $\SI{15}{\tesla}=\num{1.5e6}\text{G}$, I am not going to assume any value for $\lambda$ here and simply solve for $\xi$:
\begin{align*}
  \qty(\frac{H_{c2}}{H_c})^2&=\frac{\lambda}{\xi_0}\\
  \xi_0&=\lambda\qty(\frac{H_c}{H_{c2}})^2
\end{align*}
The value is then:
\begin{align*}
  \boxed{\xi_0=\lambda\qty(\frac{\num{1.5e6}}{10^5})^2=225\lambda}
\end{align*}
\section{Josephson Junction in the Regetti Qubit}
\subsection{Capacitance}
The capacitor used in this qubit is a parallel plate capacitor, for which the capacitance is given by:
\begin{align*}
  C=\varepsilon_r\varepsilon_0\frac{A}{d}
\end{align*}
Where $A$ is the area of the plates, $d$ is the separation between the plates, $\varepsilon_0$ is the permitivity of free space and $\varepsilon_r$ is the relative permitivity. Plugging these values in for the capacitance we get:
\begin{align*}
  C=2.0\cdot\SI{8.854e-12}{\farad\per\meter}
  \frac{\qty(\SI{1.0e-6}{\m})^2}{\SI{2.0e-9}{\m}}
  =\boxed{\SI{8.854e-15}{\F}}
\end{align*}
\subsection{Voltage \& Stored Energy}
The capacitance is defined in terms of the voltage and charge, giving:
\begin{align*}
  q=CV
\end{align*}
Rearranging this gives the formula for the voltage across the capacitor:
\begin{align*}
  V=\frac{q}{C}
\end{align*}
With $q=\SI{1.6e-19}{\coulomb}$ and the value of the capacitance from the previous problem we get:
\begin{align*}
  V=\frac{\SI{1.6e-19}{\coulomb}}{\SI{8.854e-15}{\F}}=\boxed{\SI{1.807e-5}{\V}}
\end{align*}
The enery stored is given by:
\begin{align*}
  U=\frac{1}{2}\frac{q^2}{C}
\end{align*}
Using the values we have seen already:
\begin{align*}
  U=\frac{1}{2}\frac{(\SI{1.6e-19}{\coulomb})^2}{\SI{8.854e-15}{\F}}
  =\SI{1.446e-24}{\J}
\end{align*}
To convert from Joule to eV, we divide through by the electric charge $e$:
\begin{align*}
  \boxed{U=\SI{9.03e-6}{\eV}}
\end{align*}
\subsection{Kinetic Inductance}
The formula for kinetic inductance is given by:
\begin{align*}
  L_k=\frac{\phi_0}{2\pi I_C\cos\phi}
\end{align*}
We need to make the following insertions:
\begin{align*}
  \cos\phi\approx 1\qquad I_C=\frac{\pi\Delta}{2R_ne}
\end{align*}
This means our formula is now:
\begin{align*}
  L_k=\frac{\phi_0R_ne}{\pi^2\Delta}
\end{align*}
The value for the flux quantum is $\SI{2.068e-15}{\weber}$, we are given the normal resistance, and the band gap for Aluminum is $\SI{3.4}{\eV}=\SI{5.4e-19}{\J}$. But dividing out the $e$ in the numerator means we can use the value in eV:
\begin{align*}
  L_k=\frac{\phi_0R_n}{\pi^2\Delta[\si{\eV}]}
\end{align*}
thus the kinetic inductance is:
\begin{align*}
  L_k=\frac{(\SI{2.068e-15}{\weber})(\SI{e4}{\ohm})}{
  \pi^2(\SI{3.4}{\eV})}=\boxed{\SI{6.16e-13}{\H}}
\end{align*}
\subsection{Shunt Capacitance}
I am not sure about this one, but I am going to assume it has to do with the frequency $\omega$:
\begin{align*}
  \omega=\frac{1}{\sqrt{LC}}
\end{align*}
This is an angular frequency, so we need to divide by $2\pi$ for our frequency, then solve for $C$:
\begin{align*}
  (2\pi)^2f^2=\frac{1}{LC}\implies C=\frac{1}{L(2\pi f)^2}
\end{align*}
So our shunt capacitance is:
\begin{align*}
  \boxed{C=\SI{3.56e-9}{\F}}
\end{align*}
\section{AC Resistivity}

\section{Nb Examples}
We are assuming that there is a linear fit, we know the resistivity is related to the resistance in the following form:
\begin{align*}
  R=\rho\frac{l}{A}
\end{align*}
\subsection{Resistance Near $T_c$}
The value seen at $\SI{300}{K}$ is about $\SI{0.16}{\micro\ohm\m}$, and using the definition of RRR:
\begin{align*}
  RRR\equiv\frac{\rho_{\SI{300}{\K}}}{\rho_{\SI{0}{\K}}}\implies\rho_{\SI{0}{\K}}
  =\frac{0.16}{1500}=\SI{1.07e-4}{\micro\ohm\meter}
\end{align*}
We know $L=\SI{1}{\m}$ and the 'wire' is a circle of diameter $\SI{2.0e-4}{\m}$:
\begin{align*}
  R&=\SI{1.07e-4}{\micro\ohm\m}\frac{\SI{1.0}{\m}}{\pi\qty(\SI{1.0e-4}{\m})^2}\\
  &=\boxed{\SI{3.4e-2}{\ohm}}
\end{align*}
\subsection{Circuit Properties}
Ohm's law says:
\begin{align*}
  V=I(R+r)
\end{align*}
The room temperature resistance is:
\begin{align*}
  \rho_{\SI{300}{K}}=\SI{0.16}{\micro\ohm\meter}\implies
  R_{\SI{300}{\K}}=\SI{0.16}{\micro\ohm\meter}
  \frac{\SI{1.0}{\m}}{\pi\qty(\SI{1.0e-4}{\m})^2}=\boxed{\SI{5.1}{\ohm}}
\end{align*}
Then the current is:
\begin{align*}
  I=\frac{\SI{10}{\V}}{\SI{5.1}{\ohm}+\SI{0.1}{\ohm}}=\boxed{\SI{1.9}{\A}}
\end{align*}
If we assume the current is uniform across the wire, then the current density is the current over the cross-sectional area:
\begin{align*}
  J=\frac{\SI{1.9}{\A}}{\pi\qty(\SI{1.0e-4}{\m})^2}=
  \boxed{\SI{6.0e7}{\A\per\square\m}}
\end{align*}
The electric field is given by the field version of Ohm's Law:
\begin{align*}
  E=\rho J
\end{align*}
Assuming this is still at $\SI{300}{\K}$ we get:
\begin{align*}
  E=\SI{1.6e-5}{\ohm\meter}\cdot\SI{6.0e7}{\A\per\square\meter}=
  \boxed{\SI{960}{\V\per\m}}
\end{align*}
Since we were careful about our units, and the wire is a meter long, we get:
\begin{align*}
  \boxed{V=\SI{960}{\V}}
\end{align*}
The current at the surface of a wire of radius $R$ is:
\begin{align*}
  B=\frac{\mu_0I}{2\pi R}=\boxed{\SI{3.8e-3}{\tesla}}
\end{align*}
\subsection{Superconducting Circuit Properties}
The resistance is now $\SI{3.4e-2}{\ohm}$, the current starting off is:
\begin{align*}
  I=\frac{\SI{10}{\V}}{\SI{3.4e-2}{\ohm}+\SI{0.1}{\ohm}}=\boxed{\SI{74.6}{\A}}
\end{align*}
The current density is now:
\begin{align*}
  J=\frac{\SI{74.6}{\A}}{\pi\qty(\SI{1.0e-4}{\m})^2}=
  \boxed{\SI{2.4e9}{\A\per\square\m}}
\end{align*}
The field:
\begin{align*}
  E=\SI{1.07e-4}{\micro\ohm\meter}\cdot\SI{2.4e9}{\A\per\square\meter}=
  \boxed{\SI{0.26}{\V\per\m}}
\end{align*}
The voltage:
\begin{align*}
  \boxed{V=\SI{0.26}{\V}}
\end{align*}
\subsection{Maximum Magnetic Field}
The critical current in terms of the critical field is:
\begin{align*}
  I_c=\frac{2\pi R}{\mu_0}B_c
\end{align*}
From what I have looked up the critical field for Nb is $\SI{15}{\tesla}$:
\begin{align*}
  \boxed{I_c=\frac{2\pi\times10^{-4}\si{\m}}{4\pi\times10^{-7}\si{\H\per\m}}=
  \SI{500}{\A}}
\end{align*}
\end{document}