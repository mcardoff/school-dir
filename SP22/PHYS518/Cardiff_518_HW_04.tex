\documentclass[12pt]{article}

\title{\vspace{-3em}PHYS 518 HW 4}
\author{Michael Cardiff}
\date{\today}

%% science symbols 
\usepackage{amsmath}
\usepackage{amssymb}
\usepackage{physics}

%% general pretty stuff
\usepackage{bm}
\usepackage{enumitem}
\usepackage{float}
\usepackage{graphicx}
\usepackage[margin=1in]{geometry}

% figures
\graphicspath{ {./figs/} }

\newcommand{\fig}[3]
{
  \begin{figure}[H]
    \centering
    \includegraphics[width=#1cm]{#2}
    \caption{#3}
  \end{figure}
}

\newcommand{\figref}[4]
{
  \begin{figure}[H]
    \centering
    \includegraphics[width=#1cm]{#2}
    \caption{#3}
    \label{#4}
  \end{figure}
}

\renewcommand{\L}{\mathcal{L}}
\newcommand{\D}{\partial}
\newcommand{\h}{\phi}
\newcommand{\s}{\psi}
\newcommand{\munu}{{\mu\nu}}

\begin{document}
\maketitle

\section{Chapter 9.9}
The chapter starts off by introducing the following equation and constants:
\begin{align*}
  \dv[2]{u}{\phi}+u=\frac{GM}{h^2}+\frac{3GM}{c^2}u^2
\end{align*}
Where $u=\frac{1}{r}$, $h=r^2\dot{phi}$, but we can rewrite it as:
\begin{align*}
  \qty(\dv{r}{t})^2+\frac{h^2}{r^2}\qty(1-\frac{2GM}{c^2r})-\frac{2GM}{r}=
  c^2(k^2-1)
\end{align*}
Where $k=E/(m_0c^2)$. The author introduces the constant $\mu\equiv GM/c^2$:
\begin{align*}
  \qty(\dv{r}{t})^2+\frac{h^2}{r^2}\qty(1-\frac{2\mu}{r})-\frac{2\mu c^2}{r}=
  c^2(k^2-1)
\end{align*}
We want to manipulate this a bit further so it matches this equation:
\begin{align*}
  \frac{1}{2}\qty(\dv{r}{t})^2+V_{eff}(r)=E
\end{align*}
This is done simply by dividing through by 2:
\begin{align*}
  \frac{1}{2}\qty(\dv{r}{t})^2
  +\frac{h^2}{2r^2}\qty(1-\frac{2\mu}{r})
  -\frac{\mu c^2}{r}=\frac{c^2}{2}(k^2-1)
\end{align*}
So we can identify the effective potential as:
\begin{align*}
  V_{eff}(r)=-\frac{\mu c^2}{r}+\frac{h^2}{2r^2}-\frac{\mu h^2}{r^3}
\end{align*}
Differentiating it is fairly simple:
\begin{align*}
  \dv{V_{eff}}{r}=\frac{\mu c^2}{r^2}-\frac{h^2}{r^3}+\frac{4\mu h^2}{r^4}
\end{align*}
\section{Chapter 9.12 \& 9.13}

\section{Chapter 10.3}

\section{$f(R)$ Theory}


\end{document}