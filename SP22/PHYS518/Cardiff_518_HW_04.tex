\documentclass[12pt]{article}

\title{\vspace{-3em}PHYS 518 HW 4}
\author{Michael Cardiff}
\date{\today}

%% science symbols 
\usepackage{amsmath}
\usepackage{amssymb}
\usepackage{physics}

%% general pretty stuff
\usepackage{bm}
\usepackage{enumitem}
\usepackage{float}
\usepackage{graphicx}
\usepackage[margin=1in]{geometry}

% figures
\graphicspath{ {./figs/} }

\newcommand{\fig}[3]
{
  \begin{figure}[H]
    \centering
    \includegraphics[width=#1cm]{#2}
    \caption{#3}
  \end{figure}
}

\newcommand{\figref}[4]
{
  \begin{figure}[H]
    \centering
    \includegraphics[width=#1cm]{#2}
    \caption{#3}
    \label{#4}
  \end{figure}
}

\renewcommand{\L}{\mathcal{L}}
\newcommand{\D}{\partial}
\newcommand{\h}{\phi}
\newcommand{\s}{\psi}
\newcommand{\munu}{{\mu\nu}}

\begin{document}
\maketitle

\section{Chapter 9.9}
The chapter starts off by introducing the following equation and constants:
\begin{align*}
  \dv[2]{u}{\phi}+u=\frac{GM}{h^2}+\frac{3GM}{c^2}u^2
\end{align*}
Where $u=\frac{1}{r}$, $h=r^2\dot{phi}$, but we can rewrite it as:
\begin{align*}
  \qty(\dv{r}{t})^2+\frac{h^2}{r^2}\qty(1-\frac{2GM}{c^2r})-\frac{2GM}{r}=
  c^2(k^2-1)
\end{align*}
Where $k=E/(m_0c^2)$. The author introduces the constant $\mu\equiv GM/c^2$:
\begin{align*}
  \qty(\dv{r}{t})^2+\frac{h^2}{r^2}\qty(1-\frac{2\mu}{r})-\frac{2\mu c^2}{r}=
  c^2(k^2-1)
\end{align*}
We want to manipulate this a bit further so it matches this equation:
\begin{align*}
  \frac{1}{2}\qty(\dv{r}{t})^2+V_{eff}(r)=E
\end{align*}
This is done simply by dividing through by 2:
\begin{align*}
  \frac{1}{2}\qty(\dv{r}{t})^2
  +\frac{h^2}{2r^2}\qty(1-\frac{2\mu}{r})
  -\frac{\mu c^2}{r}=\frac{c^2}{2}(k^2-1)
\end{align*}
So we can identify the effective potential as:
\begin{align*}
  V_{eff}(r)=-\frac{\mu c^2}{r}+\frac{h^2}{2r^2}-\frac{\mu h^2}{r^3}
\end{align*}
Differentiating it is fairly simple:
\begin{align*}
  \dv{V_{eff}}{r}=\frac{\mu c^2}{r^2}-\frac{h^2}{r^3}+\frac{3\mu h^2}{r^4}
\end{align*}
Finding the minimum:
\begin{align*}
  \dv{V_{eff}}{r}=0\implies
  0&=\frac{\mu c^2}{r^2}-\frac{h^2}{r^3}+\frac{3\mu h^2}{r^4}\\
  &=\mu c^2-\frac{h^2}{r}+\frac{3\mu h^2}{r^2}\\
  &=\mu c^2r^2-h^2r+3\mu h^2\\
  0&=\boxed{r^2-\frac{h^2}{\mu c^2}r+\frac{3h^2}{c^2}}
\end{align*}
Solving for $r$ is just the quadratic formula:
\begin{align*}
  r&=\frac{h^2}{2\mu c^2}\pm
  \frac{1}{2}\sqrt{\frac{h^4}{\mu^2c^4}-\frac{12h^2}{c^2}}\\
  &=\frac{h^2}{2\mu c^2}\pm\frac{h}{2c}\sqrt{\qty(\frac{h}{\mu c})^2-12}\\
  &=\frac{h^2}{2\mu c^2}\pm\frac{h}{2\mu c^2}\sqrt{h^2-12\mu^2c^2}\\
  &=\boxed{\frac{h}{2\mu c^2}\qty(h\pm\sqrt{h^2-12\mu^2c^2})}
\end{align*}
The value of $h=\sqrt{12}\mu c$ then there is one extremum:
\begin{align*}
  r&=\frac{\sqrt{12}\mu c}{2\mu c^2}\sqrt{12}\mu c\\
  &=\frac{12\mu^2c^2}{2\mu c^2}\\
  &=6\mu
\end{align*}
Lets check that this satisfies $\dv[2]{V_{eff}}{r}=0$ as well:
\begin{align*}
  \dv[2]{V_{eff}}{r}=-\frac{2\mu c^2}{r^3}+\frac{3h^2}{r^4}-\frac{12\mu h^2}{r^5}
\end{align*}
The minimum condition can be found in a similar way:
\begin{align*}
  \boxed{-2\mu c^2r^2+3h^2r-12\mu h^2=0}
\end{align*}
Setting $h=\sqrt{12}\mu c$:
\begin{align*}
  0&=-2\mu c^2r^2+36\mu cr-144\mu c\\
  &=-2\mu c^2(36\mu)+36\mu c(6\mu)-144\mu c\\
  &=0
\end{align*}
We did the remainder in class, notably the relativistic correction to the newtonian orbit.
\section{Chapter 9.12 \& 9.13}
The photon trajectory is:
\begin{align*}
  \dv[2]{u}{\phi}+u=\frac{3GM}{c^2}u^2
\end{align*}
A circular photon orbit occurs when $r=\text{const}$ so all derivatives disappear:
\begin{align*}
  u=\frac{3GM}{c^2}u^2
\end{align*}
Transforming $u\to\frac{1}{r}$:
\begin{align*}
  \frac{1}{r}=\frac{3GM}{c^2}\frac{1}{r^2}\implies r=\frac{3GM}{c^2}
\end{align*}
The energy equation is:
\begin{align*}
  \frac{\dot{r}^2}{h^2}+\frac{1}{r^2}\qty(1-\frac{2\mu}{r})=\frac{c^2k^2}{h^2}
\end{align*}
Where we define $b=h/(ck)$ as the constant on the right hand side, the term only in $\mu,r$ is the effective potential. The $h$ dependence can be ignored. The we can rewrite this with the geodesic equation:
\begin{align*}
  k=\qty(1-\frac{2\mu}{r})\dot{t}
\end{align*}
This gives the $r$ deriv of $\phi$:
\begin{align*}
  \dv{\phi}{r}=\frac{1}{r^2}
  \qty[\frac{1}{b^2}-\frac{1}{r^2}\qty(1-\frac{2\mu}{r})]^{-1/2}
\end{align*}
Take the limit as $r\to\infty:$
\begin{align*}
  \lim_{r\to\infty}r^2\dv{\phi}{r}=\qty[\frac{1}{b^2}-\frac{1}{r^2}\{\}]^{-1/2}
\end{align*}
Where the curly bracket term does not matter since it is $0$:
\begin{align*}
  \lim_{r\to\infty}r^2\dv{\phi}{r}=\qty[\frac{1}{b^2}]^{-1/2}=\pm b
\end{align*}
The solution can be found by separation of varibles
\begin{align*}
  r^2\dd{\phi}=\pm b\dd{r}\implies\dd{\phi}=\pm b\frac{\dd{r}}{r^2}
\end{align*}
Integrating will give a linear approximation of the solution, extending it gives:
\begin{align*}
  r=\pm\frac{b}{\sin\phi}
\end{align*}
\section{Chapter 10.3}

\section{$f(R)$ Theory}


\end{document}