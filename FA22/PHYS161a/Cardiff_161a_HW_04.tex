\documentclass[12pt]{article}

\title{\vspace{-3em}PHYS 161a HW 04}
\author{Michael Cardiff}
\date{\today}

%% science symbols 
\usepackage{amsmath}
\usepackage{amssymb}
\usepackage{physics}
\usepackage{slashed}

%% general pretty stuff
\usepackage{bm}
\usepackage{enumitem}
\usepackage{float}
\usepackage{graphicx}
\usepackage[margin=1in]{geometry}
\usepackage[labelfont=bf]{caption}
\usepackage{tikz}

% figures
\graphicspath{ {./figs/} }

\renewcommand{\L}{\mathcal{L}}

\newcommand{\D}{\partial}
\newcommand{\munu}{{\mu\nu}}
\newcommand{\sla}[1]{\slashed{#1}}
\newcommand*\circled[1]{\tikz[baseline=(char.base)]{
    \node[shape=circle,draw,inner sep=2pt] (char) {#1};}}

\begin{document}
\maketitle

\section{Dipole Moment Practice}
\subsection{Charged Ring}
The charge density of a ring must reduce the integral by two, so we get a length instead of a volume multiplying the charge density. By analysis and from discussions in class, we get:
\begin{align*}
  \sigma(r,\theta)=\frac1{r\sin\theta}\delta\qty(\theta-\frac\pi2)\delta(r-R)
\end{align*}
Where the ring is a radius $R$. This is the surface charge density associated with the line density $\lambda$, with units such that:
\begin{align*}
  \rho(\vb{r})=\lambda(\phi)\sigma(r,\theta)
\end{align*}
The dipole moment is a vector $\vb{p}$, defined by:
\begin{align*}
  \vb{p}=\int\dd[3]{r}\vb{r}\rho(\vb{r})
\end{align*}
Where $\vb{r}$ is the standard position vector in terms $\vu{x},\vu{y},\vu{z}$:
\begin{align*}
  \vb{r}=
  \qty(\sin\theta\cos\phi)\vu{x}+
  \qty(\sin\theta\sin\phi)\vu{y}+
  \qty(\cos\theta)\vu{z}
\end{align*}
Note that $p_z$ has the following integral:
\begin{align*}
  p_z\propto\int_0^\pi\dd{\theta}\delta\qty(\theta-\frac\pi2)\cos\theta
\end{align*}
By the definition of the $\delta$, we have $\cos\pi/2$, which is $0$, so $p_z=0$
\begin{align*}
  \vb{p}\vdot\vu{z}=0
\end{align*}
The integral for $y$ requires that we sub in the charge density:
\begin{align*}
  \rho(\vb{r})=\frac{\lambda_0\cos\phi}{r\sin\theta}
  \delta\qty(\theta-\frac\pi2)\delta(r-R)
\end{align*}
The $\phi$ integral for $p_y$ is:
\begin{align*}
  p_y\propto\int_0^{2\pi}\dd{\phi}\cos\phi\sin\phi
\end{align*}
Since $\sin$ and $\cos$ are orthogonal functions over a single period, this integral is 0, and so is $p_y$:
\begin{align*}
  \vb{p}\vdot\vu{y}=0
\end{align*}
This leaves only $p_x$:
\begin{align*}
  p_x=\int_0^{2\pi}\dd{\phi}\int_0^\pi\dd{\theta}\sin\theta\int_0^\infty\dd{r}r^2
  \qty[r\sin\theta\cos\phi]\qty[\frac{\lambda_0\cos\phi}{r\sin\theta}
  \delta\qty(\theta-\frac\pi2)\delta(r-R)]
\end{align*}
Simplifying a bit:
\begin{align*}
  p_x=\lambda_0\int_0^{2\pi}\dd{\phi}\cos^2\phi
  \int_0^\pi\dd{\theta}\sin\theta\delta\qty(\theta-\frac\pi2)
  \int_0^\infty\dd{r}r^2\delta(r-R)
\end{align*}
The second two integrals are trivial by definition of the $\delta$, and the first we can use the properties of $\cos$ as a complete set to get that it is simply $\pi$, leaving us with:
\begin{align*}
  p_x=\lambda_0(\pi)\qty(\sin(\frac\pi2))R^2=\lambda_0\pi R^2
\end{align*}
Thus the dipole moment is:
\begin{align}
  \boxed{\vb{p}=\lambda_0\pi R^2\vu{x}}
\end{align}
\subsection{Charged Sphere}
Now we have a surface charge density $\sigma(\theta)$. We only need the result:
\begin{align*}
  Q=4\pi R^2\sigma=\int\dd{\Omega}\int\dd{r}r^2\delta(r-R)\sigma
\end{align*}
Thus, we need 
\begin{align*}
  \rho(\vb{r})=\sigma(\theta)\delta(r-R)=\sigma_0\cos\theta\delta(r-R)
\end{align*}
Thus the dipole moment is:
\begin{align*}
  \vb{p}=\int\dd[3]{r}\vb{r}\rho(\vb{r})
\end{align*}
With the $\vb{r}$ the same as in the previous part. Since our charge density has a $\cos\theta$ in it, immediately, the $\vu{x}$ and $\vu{y}$ integrals must be $0$, since they contain $\sin\theta$, which is orthogonal to $\cos\theta$, hence:
\begin{align*}
  \vb{p}\vdot\vu{x}&=0\\
  \vb{p}\vdot\vu{y}&=0
\end{align*}
The $z$ integral is the only non-zero one then:
\begin{align*}
  p_z&=\int_0^{2\pi}\dd{\phi}\int_0^\pi\dd{\theta}\sin\theta\int_0^\infty\dd{r}r^2
  \qty(r\cos\theta)\qty[\sigma_0\cos\theta\delta(r-R)]\\
  &=\sigma_0\int_0^{2\pi}\dd{\phi}
  \int_0^\pi\dd{\theta}\sin\theta\cos^2\theta
  \int_0^\infty\dd{r}r^3\delta(r-R)
\end{align*}
The trivial integrals are:
\begin{align*}
  \int_0^{2\pi}\dd{\phi}&=2\pi\\
  \int_0^\infty\dd{r}r^3\delta(r-R)&=R^3
\end{align*}
The only remaining one is:
\begin{align*}
  p_z&=2\pi R^3\sigma_0\int_0^\pi\dd{\theta}\sin\theta\cos^2\theta\\
  u&=\cos\theta\quad\dd{u}=-\sin\theta\dd{\theta}\\
  p_z&=2\pi R^3\sigma_0\int_{-1}^{1}\dd{u}u^2\\
  &=\frac43\pi R^3\sigma_0
\end{align*}
Thus the dipole moment is:
\begin{align}
  \boxed{\vb{p}=\frac43\pi R^3\sigma_0\vu{z}}
\end{align}
\section{Cartesian Multipole Practice}
\subsection{Charges in a square}
The total charge is the 'multipole moment', since there are 2 plus charges and 2 minus charges, the total is $0$:
\begin{align}
  \boxed{Q=0}
\end{align}
The dipole moment from the plus charges would cancel out that of the minus charges, so the dipole moment is also $0$:
\begin{align}
  \boxed{\vb{P}=\bm{0}}
\end{align}
The quadrupole moment must be calculated by the charge density:
\begin{align*}
  Q_{ij}=\frac12\int\dd[3]{r}\qty(r_ir_j)\rho(\vb{r})
\end{align*}
There are 4 total charges, at each corner of a square, the density is:
\begin{align*}
  \rho(\vb{r})&=q\delta(z)(\delta(x-a)\delta(y-a)+\delta(x+a)\delta(y+a)
  -\delta(x-a)\delta(y+a)-\delta(x+a)\delta(y-a))\\
  &=q\delta(z)\qty(
  \delta^{(2)}(a,a)+
  \delta^{(2)}(-a,-a)-
  \delta^{(2)}(a,-a)-
  \delta^{(2)}(-a,a))
\end{align*}
All terms with a $z$ will go away because of $\delta(z)$ which means:
\begin{align*}
  Q_{iz}=Q_{zi}=Q_{zz}=0
\end{align*}
Thus all that is left is $Q_{xx},Q_{yy},Q_{yx},Q_{xy}$:
\begin{align*}
  Q_{ii}=\frac12\int\dd{x}\dd{y}\dd{z}r_i^2\rho(\vb{r})
\end{align*}
Notice for each of the diagonal terms, we will get the same for all 4 terms, except the second 2 will be opposite sign of the first two, so it will all go to $0$:
\begin{align*}
  Q_{ii}=0
\end{align*}
The last one is $Q_{xy}=Q_{yx}$ by symmetry:
\begin{align*}
  Q_{yx}=Q_{xy}=\frac12\int\dd{x}\dd{y}\dd{z}xy^2\rho(\vb{r})
\end{align*}
The $z$ integral is immediately $1$:
\begin{align*}
  Q_{xy}&=\frac12\int\dd{x}x\int\dd{y}y\rho(\vb{r})\\
  &=q\frac12\int\dd{x}x\int\dd{y}y\qty(\delta^{(2)}(a,a)+
  \delta^{(2)}(-a,-a)-
  \delta^{(2)}(a,-a)-
  \delta^{(2)}(-a,a))
\end{align*}
Hence we get a reduction of these $\delta$:
\begin{align*}
  Q_{xy}&=q\frac12\int\dd{x}x
  \qty(a\delta(x-a)-a\delta(x+a)+a\delta(x-a)-a\delta(x+a))\\
  &=qa\int\dd{x}x\qty(\delta(x-a)-\delta(x+a))\\
  &=qa\qty(a-(-a))=2qa^2
\end{align*}
Thus our quadrupole moment tensor is:
\begin{align}
  \boxed{
    \va{\vb{Q}}=2qa^2
    \begin{bmatrix}
      0 & 1 & 0 \\ 1 & 0 & 0 \\ 0 & 0 & 0
    \end{bmatrix}
  }
\end{align}
Where I have noted the tensor object as bold with an arrow above it.
\subsection{Finite rod of charge}
The rod has a charge density $\lambda$ and length $2\ell$, so the monopole moment $Q$ is:
\begin{align*}
  \boxed{Q=2\lambda\ell}
\end{align*}
The distribution is symmetric about the origin, and the multiplier for the dipole moment is odd, so the integral for the dipole moment is $0$:
\begin{align}
  \boxed{\vb{P}=\bm{0}}
\end{align}
Thus the quadrupole moment is the only nonzero. The rod only exists in the $z$ axis, so it can only be spread out in the $z$ axis, which is what the quadrupole moment measures, like the variance of a probability distribution. Thus the only nonzero one is:
\begin{align*}
  Q_{zz}=\frac12\int\dd[3]{r}z^2\rho(\vb{r})
\end{align*}
The $x,y$ integrals go away, and we get:
\begin{align*}
  Q_{zz}=\frac12\lambda\int_{-\ell}^\ell\dd{z}z^2=\frac\lambda3\ell^3
\end{align*}
So the quadrupole moment tensor is:
\begin{align}
  \boxed{
    \va{\vb{Q}}=\frac\lambda3\ell^3
    \begin{bmatrix}
      0 & 0 & 0 \\ 0 & 0 & 0 \\ 0 & 0 & 1
    \end{bmatrix}
  }
\end{align}
\subsection{Charged Ring}
First we must for Our charge density. It is in the $x-y$ plane, with $z=0$, so we need a $\delta(z)$. Since we have circular symmetry in the plane, we should use cylindrical coordinates, with $r=R$ being the radius of the ring. Then we have:
\begin{align*}
  \rho(r,\phi,z)=\lambda\delta(z)\delta(r-R)
\end{align*}
Just like the first one, since the charge is in the $x-y$ plane, all of the $Q_{iz}=Q_{zi}=Q_{zz}=0$, leaving $Q_{xx},Q_{yy},Q_{xy}$. Recall that $x=r\cos\phi$ and $y=r\sin\phi$ in cylindrical coordinates. We also must remember the Jacobian for cylindrical coordinates, which is $\dd{x}\dd{y}\dd{z}=r\dd{r}\dd{\phi}\dd{z}$:
\begin{align*}
  Q_{xy}&=\frac\lambda2\int_{-\infty}^\infty\dd{z}
  \int_0^{2\pi}\dd{\phi}\int_0^{\infty}\dd{r}
  (r\cos\phi)(r\sin\phi)\delta(z)\delta(r-R)\\
  &=\frac\lambda2\int_{-\infty}^\infty\dd{z}\delta(z)
  \int_0^{2\pi}\dd{\phi}\cos\phi\sin\phi
  \int_0^\infty\dd{r}r^3\delta(r-R)
\end{align*}
Which is $0$ by orthogonality of $\sin,\cos$:
\begin{align*}
  Q_{xy}=Q_{yx}=0
\end{align*}
We can recognize that $Q_{xx}$ and $Q_{yy}$ will be the same except for one will have a sine and one will have cosine. However since:
\begin{align*}
  \int_0^{2\pi}\dd{\theta}\cos^2\theta=\int_0^{2\pi}\dd{\theta}\sin^2\theta=\pi
\end{align*}
We are then left with 2 identical $r$ integrals:
\begin{align*}
  Q_{xx}=Q_{yy}=\frac\lambda2\pi\int_0^\infty\dd{r}r^3\delta(r-R)=
  \frac12\lambda\pi R^3
\end{align*}
Hence the quadrupole moment tensor is:
\begin{align}
  \boxed{
    \va{\vb{Q}}=\frac12\lambda\pi R^3
    \begin{bmatrix}
      1 & 0 & 0 \\ 0 & 1 & 0 \\ 0 & 0 & 0
    \end{bmatrix}
  }
\end{align}
\section{Spherical Multipole Moment Example}
Notice that the spherical harmonics are orthogonal with respect to each other, such that:
\begin{align*}
  \int_0^{2\pi}\dd{\phi}\int_0^\pi\dd{\theta}\sin\theta
  \qty(Y^*_{\ell m}(\theta,\phi)Y_{\ell'm'}(\theta,\phi))
  =\delta_{\ell\ell'}\delta_{mm'}
\end{align*}
There are 2 different types of multipole moments for spherical expansions, first the interior, and then the exterior ones. This is important since the charge distribution is:
\begin{align*}
  \rho(\vb{r})=
  \begin{cases}
    \alpha(R-r)\qty[1-\cos\theta]^2 & \abs{\vb{r}}\leq R\\
    0 & \abs{\vb{r}}\geq R
  \end{cases}
\end{align*}
The interior spherical multipole moments are given by:
\begin{align*}
  B_{\ell m}=\frac{4\pi}{2\ell+1}\int\dd[3]{r'}
  \frac{\rho(\vb{r'})}{(r^{\prime\ell+1})}Y_{\ell m}(\theta',\phi')
\end{align*}
This is strictly in a regime where $r<r'$, so we have:
\begin{align*}
  B_{\ell m}=\frac{4\pi}{2\ell+1}\int_0^{2\pi}\dd{\phi'}
  \int_0^\pi\dd{\theta'}\sin\theta'
  \int_0^\infty\dd{r'}\alpha(R-r')(r')^{1-\ell}(1-\cos\theta')^2
  Y_{\ell m}(\theta',\phi')
\end{align*}
However we can notice immediately that the $\theta$ dependence of the charge density we are limited on which moments we have to consider. First notice the lack of $\phi$ dependence anywhere, this immediately gives us the hint that $m=0$, since $m$ is what controls whether or not we have dependence on $\phi$:
\begin{align*}
  B_{\ell0}=\frac{4\pi}{2\ell+1}\int_0^{2\pi}\dd{\phi'}
  \int_0^\pi\dd{\theta'}\sin\theta'(1-\cos\theta')^2Y_{\ell0}(\theta',\phi')
  \int_0^\infty\dd{r'}\alpha(R-r')(r')^{1-\ell}
\end{align*}
We know the angular integrals are what will choose $\ell,m$, so we see:
\begin{align*}
  \int_0^{2\pi}\dd{\phi'}\int_0^\pi\dd{\theta'}\sin\theta'
  (1-\cos\theta')^2Y_{\ell0}(\theta',\phi')
\end{align*}
We notice that there will be $3$ terms in this integral which contribute:
\begin{align*}
  (1-\cos\theta')^2=1-2\cos\theta'+\cos^2\theta'
\end{align*}
Note that $Y_{00}$ is a constant, $Y_{10}$ is proportional to $\cos\theta$, and $Y_{20}$ is proportional to $\cos^2\theta$, so the only contributing interior multipole moments are $\ell=0,1,2$ and $m=0$. Thus the angular integrals will mostly disappear due to the normalization of the spherical harmonics:
\begin{align*}
  B_{00}&=4\pi\int_0^{2\pi}\dd{\phi'}\int_0^\pi\dd{\theta'}\sin\theta'
  Y_{00}(1-\cos\theta')^2
  \int_0^s\dd{r'}\alpha(R-r')(r')^{1-0}\\
  &=\frac{32\pi^{3/2}}3\int_0^s\dd{r'}\alpha(R-r')r'\\
  &=\frac{16\pi^{3/2}}9s^2(3R-2s)\\
  B_{10}&=\frac{4\pi}3\int_0^{2\pi}\dd{\phi'}\int_0^\pi\dd{\theta'}\sin\theta'
  Y_{10}(1-\cos\theta')^2
  \int_0^s\dd{r'}\alpha(R-r')(r')^{1-1}\\
  &=-\frac{16\pi^{3/2}}{3\sqrt{3}}\int_0^s\dd{r'}\alpha(R-r')\\
  &=-\frac{8\pi^{3/2}}{3\sqrt{3}}s(2R-s)\\
  B_{20}&=\frac{4\pi}5\int_0^{2\pi}\dd{\phi'}\int_0^\pi\dd{\theta'}\sin\theta'
  Y_{20}(1-\cos\theta')^2
  \int_0^s\dd{r'}\alpha(R-r')(r')^{1-2}\\
  &=\frac{16\pi^{3/2}}{15\sqrt{5}}\int_0^s\dd{r'}\alpha\frac{(R-r')}{r'}
\end{align*}
Recall these are only valid for $s\leq R$, and I am not sure how we should go about the third one...

Therefore the only interior multipole moments are:
\begin{equation}
  \boxed{
  \begin{aligned}
    B_{00}&=\frac{\alpha16\pi^{3/2}}9s^2(3R-2s)\\
    B_{10}&=-\frac{\alpha8\pi^{3/2}}{3\sqrt{3}}s(2R-s)\\
    B_{20}&=\frac{\alpha16\pi^{3/2}}{15\sqrt{5}}\int_0^s\dd{r'}\frac{(R-r')}{r'}
  \end{aligned}
  }
\end{equation}
This leaves the exterior moments:
\begin{align*}
  A_{\ell m}=\frac{4\pi}{2\ell+1}\int\dd[3]{r'}
  \rho(\vb{r}')(r')^{\ell}Y_{\ell m}^*(\theta',\phi')
\end{align*}
The integrals separate to:
\begin{align*}
  A_{\ell m}&=\frac{4\pi}{2\ell+1}
  \int_0^{2\pi}\dd{\phi'}
  \int_0^\pi\dd{\theta'}\sin\theta'
  \int_0^R\dd{r'}
  \rho(\vb{r}')(r')^{2+\ell}Y_{\ell m}^*(\theta',\phi')\\
  &=\frac{4\pi}{2\ell+1}
  \int_0^{2\pi}\dd{\phi'}
  \int_0^\pi\dd{\theta'}\sin\theta'(1-\cos\theta')^2Y_{\ell m}^*(\theta',\phi')
  \int_0^R\dd{r'}\alpha(R-r')(r')^{2+\ell}
\end{align*}
Once again the only valid values of $\ell$ and $m$ are $0,1,2$ and $0$, the only different thing is the $r'$ integrals:
\begin{align*}
  A_{00}&=4\pi\int_0^{2\pi}\dd{\phi'}\int_0^\pi\dd{\theta'}\sin\theta'
  Y_{00}(1-\cos\theta')^2\int_0^R\dd{r'}\alpha(R-r')(r')^{2+0}\\
  &=\frac{32\pi^{3/2}}3\int_0^R\dd{r'}\alpha(R-r')(r')^{2}\\
  &=\frac89\pi^{3/2}\alpha R^4\\
  A_{10}&=\frac{4\pi}3\int_0^{2\pi}\dd{\phi'}\int_0^\pi\dd{\theta'}\sin\theta'
  Y_{10}(1-\cos\theta')^2\int_0^R\dd{r'}\alpha(R-r')(r')^{2+1}\\
  &=-\frac{16\pi^{3/2}}{3\sqrt{3}}\int_0^R\dd{r'}\alpha(R-r')(r')^{3}\\
  &=-\frac{4}{15\sqrt{3}}\pi^{3/2}\alpha R^5\\
  A_{20}&=\frac{4\pi}5\int_0^{2\pi}\dd{\phi'}\int_0^\pi\dd{\theta'}\sin\theta'
  Y_{20}(1-\cos\theta')^2\int_0^R\dd{r'}\alpha(R-r')(r')^{2+2}\\
  &=\frac{16\pi^{3/2}}{15\sqrt{5}}\int_0^R\dd{r'}\alpha(R-r')(r')^4\\
  &=\frac{8}{225\sqrt{5}}\pi^{3/2}\alpha R^6
\end{align*}
Thus the exterior moments are:
\begin{equation}
  \boxed{
    \begin{aligned}
      A_{00}&=\frac89\pi^{3/2}\alpha R^4\\
      A_{10}&=-\frac{4}{15\sqrt{3}}\pi^{3/2}\alpha R^5\\
      A_{20}&=\frac{8}{225\sqrt{5}}\pi^{3/2}\alpha R^6
    \end{aligned}
  }
\end{equation}
The potential exterior to the distribution is then given by:
\begin{align*}
  \varphi(\vb{r})&=\frac1{4\pi\varepsilon_0}\sum_{\ell=0}^2
  A_{\ell0}\frac{Y_{\ell0}}{r^{\ell+1}}\\
  &=\frac1{4\pi\varepsilon_0}
  \qty(\frac49\pi\alpha R^4
  -\frac2{15}\pi\alpha R^5\cos\theta
  +\frac2{225}\pi\alpha R^6(2\cos^2\theta-1))\\
  &=\frac{2\pi\alpha R^4}{4\pi\varepsilon_0}
  \qty(\frac29-\frac{R}{15}\cos\theta-
  \frac{R^2}{225}+\frac{R^2}{75}\cos^2\theta)\\
  &=\frac{\alpha R^4}{2\varepsilon_0}
  \qty(\frac29-\frac{R}{15}\cos\theta-
  \frac{R^2}{225}+\frac{R^2}{75}\cos^2\theta)
\end{align*}
Hence the potential outside of the sphere is:
\begin{align}
  \boxed{\varphi(\vb{r})=\frac{\alpha R^4}{2\varepsilon_0}
  \qty(\frac29-\frac{R}{15}\cos\theta-
  R^2\qty(\frac1{225}-\frac1{75}\cos^2\theta))}
\end{align}
\end{document}
