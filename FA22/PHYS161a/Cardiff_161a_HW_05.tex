\documentclass[12pt]{article}

\title{\vspace{-3em}PHYS 161a HW 05}
\author{Michael Cardiff}
\date{\today}

% science symbols 
\usepackage{amssymb,amsthm,bm,physics,slashed}

% general pretty stuff
\usepackage{caption,enumitem,float,geometry,graphicx,tikz}

% setups
\graphicspath{ {./figs/} }
\captionsetup{labelfont=bf}
\geometry{margin=1in}

% macros
\renewcommand{\L}{\mathcal{L}}
\newcommand{\D}{\partial}
\newcommand{\circled}[1]{\tikz[baseline=(char.base)]{
    \node[shape=circle,draw,inner sep=2pt](char){#1};}}

\begin{document}
\maketitle

\section{Point Charge Between two Conducting Plates}
\subsection{Reduced Green Function}
We start by assuming the following form of the Green Function:
\begin{align*}
  G(\vb{r,r}')=\int\frac{\dd{k_{\perp}}}{(2\pi)^2}
  e^{-i\vb{k}_\perp\vdot(\vb{r-r}')_\perp}g(z,z',k_\perp)
\end{align*}
We want to use the defining differential equation of the Green function:
\begin{align*}
  \laplacian_{\vb{r}}G(\vb{r,r}')=-4\pi\delta^3\qty(\vb{r-r}')
\end{align*}
Since the form of the $\delta$ is:
\begin{align*}
  \delta^3\qty(\vb{r-r}')=\delta(x-x')\delta(y-y')\delta(z-z')
\end{align*}
We can use the Laplacian directly on the Green Function:
\begin{align*}
  \laplacian_{\vb{r}}G(\vb{r,r}')=\int\frac{\dd{k_{\perp}}}{(2\pi)^2}
  e^{-i\vb{k}_\perp\vdot(\vb{r-r}')_\perp}\qty(-k_\perp^2+\pdv[2]{z})g(z,z',k_\perp)
\end{align*}
Since the only dependence of the other $\vb{r}$ variables are in the exponential. We can recognize the other products as $\delta$ functions in $x,y$ as long as the following is true:
\begin{align*}
  \qty(-k_\perp^2+\pdv[2]{z})g(z,z',k_\perp)=-\delta(z-z')
\end{align*}
Then we have the following boundary conditions:
\begin{gather*}
  g(0,z',k_\perp)=g(a,z',k_\perp)=0\\
  \text{Continuity of $g$ at $z=z'$}\\
  \text{Continuity of $\pdv{g}{z}$ at $z=z'$}
\end{gather*}
The equation we need to solve for $z\neq z'$
\begin{align*}
  \pdv[2]{g}{z}\qty(z,z',k_\perp) =k_\perp^2g(z,z',k_\perp)
\end{align*}
The solution to this equation in general is typically written as a sum of exponentials with opposite sign arguments, however we can also write it as:
\begin{align*}
  g(z,z',k_\perp)=A(z')\sinh(k_\perp z)+B(z')\cosh(k_\perp z)
\end{align*}
The first boundary condition gives:
\begin{align*}
  g(0,z',k_\perp)&=A(z')\sinh(0)+B(z')\cosh(0)=0\\
  &\implies B(z')=0
\end{align*}
The continuity of the first 
\subsection{Induced Charge}

\subsection{Taking Limits}

\subsubsection{Right Plate to $\infty$}

\subsubsection{Both Plates Far Away}

\subsubsection{Justification}

\section{Potential Inside a Sphere}

\end{document}