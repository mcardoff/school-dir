\documentclass[12pt]{article}

\title{\vspace{-3em}PHYS 162a HW 1}
\author{Michael Cardiff}
\date{\today}

%% science symbols 
\usepackage{amsmath}
\usepackage{amssymb}
\usepackage{physics}
\usepackage{slashed}

%% general pretty stuff
\usepackage{bm}
\usepackage{enumitem}
\usepackage{float}
\usepackage{graphicx}
\usepackage[margin=1in]{geometry}
\usepackage[labelfont=bf]{caption}

% figures
\graphicspath{ {./figs/} }

\newcommand{\fig}[3]
{
  \begin{figure}[H]
    \centering
    \includegraphics[width=#1cm]{#2}
    \caption{#3}
  \end{figure}
}

\newcommand{\figref}[4]
{
  \begin{figure}[H]
    \centering
    \includegraphics[width=#1cm]{#2}
    \caption{#3}
    \label{#4}
  \end{figure}
}

\renewcommand{\L}{\mathcal{L}}
\newcommand{\D}{\partial}
\newcommand{\munu}{{\mu\nu}}
\newcommand{\sla}[1]{\slashed{#1}}

\begin{document}
\maketitle

\section{Bead Constrained to a Hoop}
First we should derive a form for the kinetic energy term in spherical coordinates, with some help from Mathematica:
\begin{align*}
  \dot{x}&=\dot{r}\sin\theta\cos\phi
  +r\dot\theta\cos\theta\cos\phi
  -r\dot\phi\sin\theta\sin\phi\\
  \dot{y}&=\dot{r}\sin\theta\sin\phi
  +r\dot\theta\cos\theta\sin\phi
  +r\dot\phi\sin\theta\cos\phi\\
  \dot{z}&=\dot{r}\cos\theta-r\dot\theta\sin\theta
\end{align*}
Adding the sum of their squares leaves us with the following:
\begin{align*}
  \dot{x}^2+\dot{y}^2+\dot{z}^2=\dot{r}^2+r^2\dot\theta+\sin^2\theta\dot\phi^2
\end{align*}
We can immediately recognize that we do not need to use a full spherical coordinate system, as the bead will always be kept at $r=a$ where the hoop is. We also know that $\dot{\phi}$ is fixed, at $\omega$. Leaving the $\theta$ velocity as undetermined, giving the kinetic energy as:
\begin{align*}
  T=\frac12ma^2\qty(\sin^2\theta\,\omega^2+\dot{\theta}^2)
\end{align*}
The only potential this bead is subject to is local gravity:
\begin{align*}
  V=-mga\cos\theta
\end{align*}
If we wanted to be more strict, we can add Lagrange multipliers $\lambda_i$ to impose constraints on our lagrangian as well:
\begin{align*}
  L_{c1}=\lambda_1(r-a)\qquad L_{c2}=\lambda_2(\dot{\phi}-\omega)
\end{align*}
These accomplish the same thing as simply imposing the constraints at the level of the potential/kinetic energies, however this is arbitrary.

Thus the lagrangian is:
\begin{align*}
  L=\frac12ma^2\qty(\sin^2\theta\,\omega^2+\dot{\theta}^2)+mga\cos\theta
\end{align*}
We clearly will only have an equation of motion for $\theta$ which is non-trivial. The equations for $r,\phi$ are simple:
\begin{align*}
  r(t)=a\qquad \phi(t)=\phi_0+\omega t
\end{align*}
Where $\phi_0$ is whatever the initial angle is

However we need to perform the Euler-Lagrange equations to get the $\theta$ equation of motion:
\begin{align*}
  \pdv{L}{\theta}&=m\omega^2a^2\sin\theta\cos\theta-mga\sin\theta\\
  \pdv{L}{\dot{\theta}}&=ma^2\dot{\theta}\implies
  \dv{t}\pdv{L}{\dot{\theta}}=ma^2\ddot{\theta}
\end{align*}
The last equation of motion for $\theta$ is hence:
\begin{align*}
  ma^2\ddot{\theta}&=m\omega^2a^2\sin\theta\cos\theta-mga\sin\theta
\end{align*}
More succinctly:
\begin{align*}
  \ddot{\theta}=\omega^2\sin\theta\cos\theta-\frac{g}{a}\sin\theta
\end{align*}
The full set of equations of motion are thus given by:
\begin{equation}
  \boxed{
    \begin{gathered}
      r(t)=a\\
      \phi(t)=\phi_0+\omega t\\
      \ddot{\theta}=\omega^2\sin\theta\cos\theta-\frac{g}{a}\sin\theta
    \end{gathered}
  }
\end{equation}
\newpage
\section{Einsteinian Lagrangian}
\subsection{Generalized Momenta and Forces}
The generalized Momenta are:
\begin{align*}
  p_\delta=\pdv{L}{\dot{q}^\delta}
\end{align*}
There will be 2 sets of derivatives, where $\delta=\alpha$ and where $\delta=\beta$, however due to index shifting they will be the same, resulting in something like:
\begin{align*}
  p_\delta=\pdv{L}{\dot{q}^\delta}=g_{\delta\beta}\dot{q}^\beta
\end{align*}
The generalized forces come only from the metric, as it is the only thing which explicitly depends on $q$:
\begin{align*}
  F_\delta=\pdv{L}{q^\delta}&=\frac12\pdv{g_{\alpha\beta}}{q^\delta}
  \dot{q}^\alpha\dot{q}^\beta\\
  &\equiv\frac12g_{\alpha\beta,\delta}\dot{q}^\alpha\dot{q}^\beta
\end{align*}
Thus each of the generalized momenta and forces are:
\begin{align}
  \boxed{
      p_\delta=g_{\delta\beta}\dot{q}^\beta\qquad
      F_\delta=\frac12g_{\alpha\beta,\delta}\dot{q}^\alpha\dot{q}^\beta
  }
\end{align}
\subsection{The Geodesic Equation}
In order to show equivalence, we derive the Geodesic equation from Euler-Lagrange. Differentiating the momenta with respect to time gives:
\begin{align*}
  \dv{t}\qty(g_{\alpha\delta}(q)\dot{x}^\alpha)=g_{\alpha\delta,\gamma}
  \dot{q}^\alpha\dot{q}^{\gamma}+g_{\alpha\delta}\ddot{q}^\alpha
\end{align*}
We need to do some index manipulation on the first term to get a Christoffel symbol, by switching the indices $\alpha,\gamma$ since they are silent.
\begin{align*}
  g_{\alpha\delta,\gamma}\dot{q}^\alpha\dot{q}^{\gamma}=
  \frac12\qty(g_{\alpha\delta,\gamma}+g_{\gamma\delta,\alpha})
  \dot{q}^\alpha\dot{q}^\gamma
\end{align*}
Combining with the generalized force we have:
\begin{align*}
  g_{\alpha\delta}\ddot{q}^\alpha+
  \frac12\qty(g_{\alpha\delta,\gamma}+g_{\gamma\delta,\alpha}
  -g_{\alpha\beta,\delta})\dot{q}^\alpha\dot{q}^\gamma=0
\end{align*}
We can then contract with the inverse metric to get our desired equation
\begin{align*}
  \ddot{q}^\sigma+\frac12g^{\sigma\delta}
  \qty(g_{\alpha\delta,\gamma}+g_{\gamma\delta,\alpha}
  -g_{\alpha\beta,\delta})\dot{q}^\alpha\dot{q}^\gamma=0
\end{align*}
After some index rearranging we can get the exact desired equation, indices and all:
\begin{align}
  \boxed{
    \ddot{q}^\alpha+\Gamma^{\alpha}_{\beta\gamma}\dot{q}^\beta\dot{q}^\gamma=0
  }
\end{align}
Since we used the Euler-Lagrange equations to derive the geodesic equation, they are in fact equivalent!
\end{document}