\documentclass[12pt]{article}

\title{\vspace{-3em}PHYS 19a Lab 5 Notes}
\author{Michael Cardiff}
\date{\today}

%% science symbols 
\usepackage{amssymb,amsthm,bm,physics,slashed}

%% general pretty stuff
\usepackage{caption,enumitem,float,geometry,graphicx,tikz}

% setups
\graphicspath{ {./figs/} }
\captionsetup{labelfont=bf}
\geometry{margin=1in}

% macros
\renewcommand{\L}{\mathcal{L}}
\newcommand{\D}{\partial}
\newcommand{\circled}[1]{\tikz[baseline=(char.base)]{
    \node[shape=circle,draw,inner sep=2pt](char){#1};}}

\begin{document}
\maketitle

\section{Notes for Lab}
\begin{itemize}
\item Finding $l_{cm}$:
  \begin{itemize}
  \item Scales to measure mass of the ball
  \item Mass of pendulum on the launcher
  \item Rulers throughout to measure lengths
  \item Center of mass of pendulum arm with ball should be marked on the arm
  \end{itemize}
\item Calculate Period of Pendulum
  \begin{itemize}
  \item Can count number of swings using phone
  \item Use LoggerPro to do it
  \end{itemize}
\item Calculate Theta from LoggerPro
  \begin{itemize}
  \item Open LoggerPro
  \item Click icon under file
  \item On CH1, choose raw voltage (0-5)V
  \item You can modify scales using Right Click menu $\to$ AutoScale
  \item Measure theta 10 times
  \item Note: Due to the circular nature of the potential device, you may encounter a discontinuity where it jumps from $\sim0$ to the max of about $3.5$, this can be fixed by properly rotating the measurement device
  \end{itemize}
\item BEFORE LAUNCHING: Check tightness/looseness of bolt holding Pendulum arm to the pivot
\item For range measurement, find distance to paper then paper to impact crater
  \begin{itemize}
  \item Note, error propagation will be large for this.
  \end{itemize}
\end{itemize}
\section{Equations}
$\Delta H$ from $\theta$:
\begin{align*}
  \Delta H= l_{cm}(1-\cos\theta)
\end{align*}
Determining $\theta$ from $V_{min}$ and $V_0$:
\begin{align*}
  V_{min}=V_0-\frac\theta{100}
\end{align*}
\end{document}