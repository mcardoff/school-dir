\documentclass[12pt]{article}

\title{\vspace{-3em}PHYS 163a HW 5}
\author{Michael Cardiff}
\date{\today}

%% science symbols 
\usepackage{amssymb,amsthm,bm,physics,slashed}

%% general pretty stuff
\usepackage{caption,enumitem,float,geometry,graphicx,tikz}

% setups
\graphicspath{ {./figs/} }
\captionsetup{labelfont=bf}
\geometry{margin=1in}

% macros
\renewcommand{\L}{\mathcal{L}}
\newcommand{\absk}{\abs{\vb{k}}}
\newcommand{\D}{\partial}
\newcommand{\circled}[1]{\tikz[baseline=(char.base)]{
    \node[shape=circle,draw,inner sep=2pt](char){#1};}}

\begin{document}
\maketitle

\section{Phonons Revisited}
\subsection{Isotropic vs Anisotropic Oscillator}
We have the following Hamiltonian for each case:
\begin{align*}
  H_{is}&=\sum_{i=1}^N\qty(\frac{p_i^2}{2m}+\frac12m\omega q_i^2)\\
  H_{an}&=\sum_{i=1}^N\qty(\frac{p_i^2}{2m}+\frac12m\omega_iq_i^2)
\end{align*}
The heat capacity is found using:
\begin{align*}
  C_V=\eval{\pdv{\ev{E}}{T}}_V
\end{align*}
Where we can find $\ev{E}$ with:
\begin{align*}
  \ev{E}=-\pdv{\ln Z}{\beta}
\end{align*}
The normal derivation for the partition function goes as:
\begin{align*}
  Z_{is}&=\frac1{N!h^3}\int\prod_{i=1}^N\dd{\vb{p}_i}\dd{\vb{q}_i}
  e^{-\beta H_{is}}=\frac1{N!h^3}\int\prod_{j=1}^N\dd{\vb{p}_j}\dd{\vb{q}_j}
  \exp{-\beta\sum_{i=1}^N\qty(\frac{p_i^2}{2m}+\frac12m\omega q_i^2)}\\
  Z_{an}&=\frac1{N!h^3}\int\prod_{i=1}^N\dd{\vb{p}_i}\dd{\vb{q}_i}
  e^{-\beta H_{an}}=\frac1{N!h^3}\int\prod_{j=1}^N\dd{\vb{p}_j}\dd{\vb{q}_j}
  \exp{-\beta\sum_{i=1}^N\qty(\frac{p_i^2}{2m}+\frac12m\omega_iq_i^2)}
\end{align*}
Clearly the momentum integrals are all the same:
\begin{align*}
  \int\prod_j\dd{\vb{p}_j}\exp{-\beta\sum_i\frac{p_i^2}{2m}}
  &=\qty(\frac{2m\pi}{\beta})^{3N/2}
\end{align*}
The position integrals in the case where all $\omega_i=\omega$ is given by:
\begin{align*}
  \int\prod_j\dd{\vb{q}_j}\exp{-\beta\sum_i\frac{m\omega q_i^2}{2}}
  &=\qty(\frac{2\pi}{m\beta\omega})^{3N/2}
\end{align*}
However, for the case where they are different, we have:
\begin{align*}
  \int\prod_j\dd{\vb{q}_j}\exp{-\beta\sum_i\frac{m\omega_iq_i^2}{2}}
  &=\int\prod_j\dd{\vb{q}_j}\prod_ie^{-\frac{\beta m\omega_iq_i^2}{2}}
\end{align*}
We can exchange the product and the integral seeing as everything behaves well:
\begin{align*}
  \int\prod_j\dd{\vb{q}_j}\prod_ie^{-\frac{\beta m\omega_iq_i^2}{2}}=
  \prod_i\int\prod_j\dd{\vb{q}_j}e^{-\frac{\beta m\omega_iq_i^2}{2}}
\end{align*}
So for a single integral we get the usual result of:
\begin{align*}
  \int\dd{\vb{q}_j}e^{-\frac{\beta m\omega_iq_i^2}{2}}=
  \qty(\frac{2\pi}{m\beta\omega_i})^{3/2}
\end{align*}
And the total product is:
\begin{align*}
  \prod_i\int\prod_j\dd{\vb{q}_j}e^{-\frac{\beta m\omega_iq_i^2}{2}}=
  \prod_{i=1}^N\qty(\frac{2\pi}{m\beta\omega_i})^{3/2}
\end{align*}
However notice the only thing which depends on $i$ is the $\omega_i$, so all of the other factors just get carried along in the process:
\begin{align*}
  \prod_{i=1}^N\qty(\frac{2\pi}{m\beta\omega_i})^{3/2}=
  \qty(\frac{2\pi}{m\beta})^{3N/2}\prod_{i=1}^N\frac1{\omega_i}
\end{align*}
Now we only need to notice that the only thing which is dependent on temperature, $\beta$, have the same powers in each case, hence when we take the $\beta$ derivative of $\ln Z$ and after the $T$ derivative of that, we will get the exact same heat capacity.

For a more explicit calculation, note the full partition functions:
\begin{align*}
  Z_{is}&=\qty(\frac{2m\pi}{\beta})^{3N/2}
  \qty(\frac{2\pi}{m\beta\omega})^{3N/2}\\
  &=\qty(\frac{2\pi}{\beta})^{3N}\omega^{-3N/2}\\
  Z_{an}&=\qty(\frac{2m\pi}{\beta})^{3N/2}
  \qty(\frac{2\pi}{m\beta})^{3N/2}\prod_i\frac1{\omega_i}\\
  &=\qty(\frac{2\pi}{\beta})^{3N}\prod_i\frac1{\omega_i}
\end{align*}
Expanding out the Log:
\begin{align*}
  \log Z_{is}&=3N\log(2\pi)-3N\log\beta-\frac{3N}{2}\log\omega\\
  \log Z_{an}&=3N\log(2\pi)-3N\log\beta-\sum_i\log\omega_i
\end{align*}
Differentiating with respect to $\beta$:
\begin{align*}
  \ev{H_{is}}&=-\pdv{\log Z_{is}}{\beta}=\frac{3N}{\beta}=3Nk_BT\\
  \ev{H_{an}}&=-\pdv{\log Z_{an}}{\beta}=\frac{3N}{\beta}=3Nk_BT
\end{align*}
Giving the exact same heat capacity:
\begin{equation}
  \boxed{
    \begin{aligned}
      (C_V)_{is}&=\pdv{\ev{H_{is}}}{T}=3Nk_B\\
      (C_V)_{an}&=\pdv{\ev{H_{an}}}{T}=3Nk_B
    \end{aligned}
  }
\end{equation}
\subsection{Dulong-Petit From High Temperature Limit}
The Einstein Law for the heat capacity of solids is:
\begin{align*}
  C_V=3Nk_B\qty(\frac{T_E}{T})^2\frac{e^{-T_E/T}}{(1-e^{-T_E/T})^2}
\end{align*}
When $T$ is large, $T^{-1}$ is small, so we can use the standard Taylor series for the exponential:
\begin{align*}
  e^x\approx 1+x\implies e^{-T_E/T}\approx 1-\frac{T_E}{T}
\end{align*}
Plugging this in for both exponentials we have:
\begin{align*}
  C_V&\approx3Nk_B\qty(\frac{T_E}T)^2\frac{1-\frac{T_E}{T}}{(1-(1-T_E/T))^2}\\
  &=3Nk_B\qty(\frac{T_E}T)^2\frac{1-\frac{T_E}{T}}{(T_E/T)^2}\\
  &=3Nk_B\qty(1-\frac{T_E}{T})
\end{align*}
And if $T$ is large enough, $T^{-1}$ is just 0, hence:
\begin{align}
  \boxed{C_V\approx 3Nk_B}
\end{align}
For large $T$ in the Einstein law, we find the Dulong-Petit law.

\subsection{Low Temperature Limit}
Again The Einstein Law is:
\begin{align*}
  C_V=3Nk_B\qty(\frac{T_E}{T})^2\frac{e^{-T_E/T}}{(1-e^{-T_E/T})^2}
\end{align*}
The only 'problematic' term is the denominator, which if we simply ignore the $1$, we would get an exponentially diverging quantity. Hence we should examine this quantity a bit more, possibly by graphing it:
\begin{figure}[H]
  \centering
  \includegraphics[width=8.0cm]{exponential.png}
  \caption{Behavior of Denominator near $T=0$}
\end{figure}
While the limit near $0$ is ill defined from either side, since temperature is positive definite, we can reasonably say we are approaching from above, giving the limit as $1$, hence the behavior near $0$ is that of an exponential:
\begin{align*}
  C_V\approx3Nk_B\qty(\frac{T_E}{T})^2e^{-T_E/T}
\end{align*}
Near $T=0$, the exponential will clearly dominate, hence the overall behavior is exponential:
\begin{align}
  \boxed{C_V\propto e^{-T_E/T}\text{ Near $T=0$}}
\end{align}

\section{Black Body Radiation}
\subsection{E\&M Wave Equations}
The vector calculus identity for a curl of a curl is:
\begin{align*}
  \curl{\curl{\vb{A}}}=\grad{(\div{\vb{A}})}-\laplacian{\vb{A}}
\end{align*}
For the specific fields; applying the free space conditions:
\begin{align*}
  \curl{\curl{\vb{E}}}&=\grad{(\div{\vb{E}})}-\laplacian{\vb{E}}
  =-\laplacian{\vb{E}}\\
  \curl{\curl{\vb{B}}}&=\grad{(\div{\vb{B}})}-\laplacian{\vb{B}}
  =-\laplacian{\vb{B}}
\end{align*}
The Maxwell Equations then become:
\begin{align*}
  -\laplacian{\vb{E}}&=-\pdv{t}\qty(\curl{\vb{B}})\\
  -\laplacian{\vb{B}}&=\mu_0\varepsilon_0\pdv{t}\qty(\curl{\vb{E}})
\end{align*}
We can then use the original equations to uncouple the equations:
\begin{align*}
  -\laplacian{\vb{E}}&=-\mu_0\varepsilon_0\pdv{t}\qty(\pdv{\vb{E}}{t})\\
  \implies\laplacian{\vb{E}}&=\mu_0\varepsilon_0\pdv[2]{\vb{E}}{t}\\
  -\laplacian{\vb{B}}&=-\mu_0\varepsilon_0\pdv{t}\qty(\pdv{\vb{B}}{t})\\
  \implies\laplacian{\vb{B}}&=\mu_0\varepsilon_0\pdv[2]{\vb{B}}{t}
\end{align*}
If we write $\mu_0\varepsilon_0$ in terms of $c$ we have:
\begin{equation}
  \boxed{
    \begin{aligned}
      \laplacian{\vb{E}}&=\frac1{c^2}\pdv[2]{\vb{E}}{t}\\
      \laplacian{\vb{B}}&=\frac1{c^2}\pdv[2]{\vb{B}}{t}
    \end{aligned}
  }
\end{equation}
Fourier transforming should change the differentiation to multiplication, specifically by $\vb{k}$, we should verify this by fourier transforming both sides of one equation, since they are practically identical:
\begin{align*}
  \int\dd{\vb{r}}e^{i\vb{k}\vdot\vb{r}}\laplacian{\vb{E}}
  =\int\dd{\vb{r}}e^{i\vb{k}\vdot\vb{r}}\frac1{c^2}\pdv[2]{\vb{E}}{t}
\end{align*}
So long as the electric field disappears out to $\pm\infty$, we can integrate by parts twice to move the laplacian from the $\vb{E}$ to the exponential, which we know the derivative of:
\begin{align*}
  \int\dd{\vb{r}}e^{i\vb{k}\vdot\vb{r}}\laplacian{\vb{E}}=
  \int\dd{\vb{r}}\qty[\laplacian{(e^{i\vb{k}\vdot\vb{r}})}]\vb{E}
\end{align*}
From this we will simply get a factor of $-\abs{\vb{k}}^2$ due to the complex exponential:
\begin{align*}
  \int\dd{\vb{r}}e^{i\vb{k}\vdot\vb{r}}\laplacian{\vb{E}}=
  -\abs{\vb{k}}^2\int\dd{\vb{r}}e^{i\vb{k}\vdot\vb{r}}\vb{E}
  =-\abs{\vb{k}}^2\widetilde{\vb{E}}
\end{align*}
Where we have defined the fourier transform of $\vb{E}$:
\begin{align*}
  \widetilde{\vb{E}}\equiv\int\dd{\vb{r}}e^{i\vb{k}\vdot\vb{r}}\vb{E}
\end{align*}
Thus we have the left hand side of our equation, the other side is simple as we are not integrating with respect to $t$, only $\vb{r}$:
\begin{align*}
  \int\dd{\vb{r}}e^{i\vb{k}\vdot\vb{r}}\frac1{c^2}\pdv[2]{\vb{E}}{t}=
  \frac1{c^2}\pdv[2]{t}\int\dd{\vb{r}}e^{i\vb{k}\vdot\vb{r}}\vb{E}=
  \frac1{c^2}\pdv[2]{\widetilde{\vb{E}}}{t}
\end{align*}
The equation of motion is then:
\begin{align*}
  \frac1{c^2}\pdv[2]{\widetilde{\vb{E}}}{t}+\abs{\vb{k}}^2\widetilde{\vb{E}}=0
\end{align*}
Rearranging to look like the harmonic oscillator equation in time, we have:
\begin{align*}
  \pdv[2]{\widetilde{\vb{E}}}{t}+c^2\abs{\vb{k}}^2\widetilde{\vb{E}}=0
\end{align*}
Where we can identify the dispersion relation $\omega(\vb{k})$:
\begin{equation}
  \boxed{
    \begin{aligned}
      \omega(\vb{k})=c\abs{\vb{k}}
    \end{aligned}
  }
\end{equation}

\subsection{Partition Function}
The Hamiltonian is:
\begin{align*}
  H=\frac12\sum_{\vb{k}}\qty(p_{\vb{k}}^2+c^2\abs{\vb{k}}^2q^2_{\vb{k}})
\end{align*}
However, since we are calculating the quantum partition function, it may be more useful to use the occupation number formalism:
\begin{align*}
  H=\sum_{i,\vb{k}}\hbar c\abs{\vb{k}}\qty(n_i(\vb{k})+\frac12)
\end{align*}
So the partition function becomes:
\begin{align*}
  Z_{ph}=\sum_{\{n_i\}}\exp{-\beta\sum_{i,\vb{k}}\hbar c\abs{\vb{k}}
    \qty(n_i(\vb{k})+\frac12)}
\end{align*}
Which is the same as the phonon partition function, which looks like:
\begin{align*}
  Z_{ph}=e^{-\beta\hbar c\absk/2}\prod_{\vb{k},i}\sum_{\{n_{\vb{k},i}\}}
  e^{-\beta\hbar c\absk n_{\vb{k},i}}=
  e^{-\beta\hbar c\absk/2}\prod_{\vb{k},i}
  \qty[\frac1{1-e^{-\beta\hbar c\absk}}]
\end{align*}
Which is our final answer:
\begin{align}
  \boxed{
    Z_{ph}=e^{-\beta\hbar c\absk/2}\prod_{\vb{k},i}
  \qty[\frac1{1-e^{-\beta\hbar c\absk}}]
  }
\end{align}

\subsection{Average Energy}
The average energy, much the same as phonons, is given as:
\begin{align*}
  \ev{H}=E_0+\sum_{\vb{k},i}\hbar c\absk\ev{n_{\vb{k},i}}
\end{align*}
With the average occupation number being exactly the same, only with the dispersion relation inserted:
\begin{align*}
  \ev{n_{\vb{k},i}}=\frac1{e^{\beta\hbar c\absk}-1}
\end{align*}
The calculation being:
\begin{align*}
  \ev{n}&=-\pdv{(\beta\hbar c\absk)}\ln(\frac1{1-e^{-\beta\hbar c\absk}})\\
  &=\frac{e^{-\beta\hbar c\absk}}{1-e^{-\beta\hbar c\absk}}=
  \frac1{e^{\beta\hbar c\absk}-1}
\end{align*}
Hence the average energy is:
\begin{align}
  \boxed{\ev{H}=E_0+\sum_{\vb{k}}\frac{\hbar c\absk}{e^{\beta\hbar c\absk}-1}}
\end{align}
If we are in the limit where we can integrate over $\vb{k}$:
\begin{align*}
  \ev{H}&=E_0+\frac{V}{(2\pi)^3}
  \int\dd[3]{\vb{k}}\frac{\hbar c\absk}{e^{\beta\hbar c\absk}-1}
  =\frac{V}{(2\pi)^3}\int\dd{k}\frac{\hbar ck^3}{e^{\beta\hbar ck}-1}
  \int\dd{\Omega}\\
  &=E_0+\frac{4\pi V}{(2\pi)^3}\frac{\pi^4}{15\hbar^3c^3\beta^4}\\
  &=E_0+\frac{\pi^2V}{30\hbar^3c^3\beta^4}
\end{align*}
\subsection{Average Pressure}
The Helmholtz free energy is related to the pressure by the following relationship:
\begin{align*}
  -\pdv{F}{V}=P
\end{align*}
And the Free energy is related to the partition function:
\begin{align*}
  F=-k_BT\log{Z}
\end{align*}
The log of the partition function is:
\begin{align*}
  -\log Z_{ph}=\beta\frac{\hbar c\absk}{2}
  +\sum_{\absk}\log(1-e^{-\beta\hbar c\absk})
\end{align*}
We can ignore the zero point term, and we can take the continuum limit, where $k$ sum turns into an integral:
\begin{align*}
  F&=-k_BT\log Z_{ph}=\frac{V}{(2\pi)^3}
  -k_BT\int\dd[3]{\vb{k}}\log(1-e^{-\hbar c\absk/k_BT})\\
  &=-\frac{Vk_BT}{2\pi^2}\int\dd{k}k^2\log(1-e^{-\hbar ck/k_BT})\\
  &=\frac{V(k_BT)^4}{2\pi^2}\frac{\pi^4}{45c^3\hbar^3}\\
  &=\frac{V\pi^2k_B^4}{90\hbar^3c^3}T^4
\end{align*}
So the pressure is:
\begin{align}
  \boxed{P=-\frac{\pi^2k_B^4}{90\hbar^3c^3}T^4}
\end{align}

\subsection{Energy Density}
From earlier, we found the energy density per $k$ to be:
\begin{align*}
  \ev{H}_{\vb{k}}=\frac{\hbar c\absk}{e^{\beta\hbar c\absk}-1}
\end{align*}
Which looks like:
\begin{figure}[H]
  \centering
  \includegraphics[width=8.0cm]{energydensity}
  \caption{Energy Density vs Wavenumber}
\end{figure}
At Low $k$, which would be short wavelengths, we do not see a divergence, but rather a constant, so we do not have a UV catastrophe.

At high $k$, which is long wavelengths, the distribution tapers off, which is expected.

\section{Electron Spin}
Note a couple properties of the Pauli matrices $\sigma_i$:
\begin{align*}
  \sigma_i^2=I_2\qquad\Tr{\sigma_i}=0
\end{align*}

\subsection{$\rho$ for Spin along $z$}
In the canonical ensemble, the density matrix is given by:
\begin{align*}
  \rho=\frac{\exp{-\beta H}}{\Tr[\exp{-\beta H}]}
\end{align*}
In this case, the Hamiltonian is:
\begin{align*}
  H=-\mu B_z\sigma_z
\end{align*}
To calculate $\exp{\beta\mu B_z\sigma_z}$ we need to use the Taylor series:
\begin{align*}
  \exp{\beta\mu B_z\sigma_z}=\sum_{n=1}^\infty
  \frac{(\beta\mu B_z)^n}{n!}\sigma_z^n
\end{align*}
We can split this up by odd and even sums, because of the property that $\sigma_z^2=I$, we get that further powers of $\sigma_z$ are:
\begin{align*}
  \sigma_i^{2n}=I_2\qquad \sigma_i^{2n+1}=\sigma_i
\end{align*}
Split into odd and even sums then:
\begin{align*}
  \exp{-\mu B_z\sigma_z}=\sum_{m=1}^\infty
  \qty(\frac{(\beta\mu B_z)^{2m}}{(2m)!}I_2
  +\frac{(\beta\mu B_z)^{2m+1}}{(2m+1)!}\sigma_z)
\end{align*}
Ignoring the matrices, the series simply converge to $\cosh$ and $\sinh$:
\begin{align*}
  \exp{\mu\beta B_z\sigma_z}=\cosh(\mu\beta B_z)I_2+\sinh(\mu\beta B_z)\sigma_z
\end{align*}
Since the pauli matrices are traceless, the trace of this is simply:
\begin{align*}
  \Tr[\exp{\mu\beta B_z\sigma_z}]=
  \Tr[\cosh(\mu\beta B_z)I_2+\sinh(\mu\beta B_z)\sigma_z]=2\cosh(\mu\beta B_z)
\end{align*}
So the density matrix is:
\begin{align*}
  \rho=\frac{\cosh(\mu\beta B_z)I_2+\cosh(\mu\beta B_z)\sigma_z}
  {2\cosh(\mu\beta B_z)}
\end{align*}
The hyperbolic trig functions have the following identities:
\begin{align*}
  e^x=\cosh(x)+\sinh(x)\\
  e^{-x}=\cosh(x)-\sinh(x)
\end{align*}
Hence in matrix form, we have:
\begin{align*}
  \cosh(x)I_2+\cosh(x)\sigma_z&=
  \bmqty{\cosh(x)+\sinh(x)&0\\0&\cosh(x)-\sinh(x)}\\
  &=\bmqty{e^x&0\\0&e^{-x}}
\end{align*}
So the density matrix in matrix form is:
\begin{align}
  \boxed{\rho=\frac1{2\cosh(\mu\beta B_z)}
  \bmqty{e^{\mu\beta B_z}&0\\0&e^{-\mu\beta B_z}}}
\end{align}

\subsection{Spin Along $x$}
The only thing that changes is that we now have $B_x$ and $\sigma_x$:
\begin{align*}
  \rho=\frac{\cosh(\mu\beta B_x)I_2+\cosh(\mu\beta B_x)\sigma_x}
  {2\cosh(\mu\beta B_x)}=
  \frac12\qty(I_2+\tanh(\mu\beta B_x)\sigma_x)
\end{align*}
In matrix form:
\begin{align}
  \boxed{\rho=\frac12\bmqty{1&\tanh\mu\beta B_x\\\tanh\mu\beta B_x&1}}
\end{align}

\subsection{Average Energy}
The Expectation value of an operator in Quantum Stat mech is:
\begin{align*}
  \ev{H}=\Tr[H\rho]
\end{align*}
For each case we have:
\begin{align*}
  \ev{H_z}&=-\frac{\mu B_z}{2\cosh(\mu\beta B_z)}
  \Tr[\bmqty{1&0\\0&-1}\bmqty{e^{\mu\beta B_z}&0\\0&e^{-\mu\beta B_z}}]\\
  &=-\frac{\mu B_z}{2\cosh(\mu\beta B_z)}
  \Tr\bmqty{e^{\mu\beta B_z}&0\\0&-e^{-\mu\beta B_z}}\\
  &=-\frac{\mu B_z}{2\cosh(\mu\beta B_z)}
  \qty(e^{\mu\beta B_z}-e^{-\mu\beta B_z})\\
  &=-\mu B_z\frac{2\sinh(\mu\beta B_z)}{2\cosh(\mu\beta B_z)}\\
  &=-\mu B_z\tanh(\mu\beta B_z)\\
  \ev{H_x}&=-\frac{\mu B_x}{2}
  \Tr[\bmqty{0&1\\1&0}\bmqty{1&\tanh(\mu\beta B_x)\\\tanh(\mu\beta B_x)&1}]\\
  &=-\frac{\mu B_x}{2}
  \Tr\bmqty{\tanh(\mu\beta B_x)&1\\1&\tanh(\mu\beta B_x)}\\
  &=-\mu B_x\tanh(\mu\beta B_x)\\
\end{align*}
Hence:
\begin{equation}
  \boxed{
    \begin{aligned}
      \ev{H_z}&=-\mu B_z\tanh(\mu\beta B_z)\\
      \ev{H_x}&=-\mu B_x\tanh(\mu\beta B_x)
    \end{aligned}
  }
\end{equation}

\section{Quantum Mechanical Entropy}
\subsection{Time Evolution of the Density Matrix}
The time evolution equation for quantum mechanics is:
\begin{align*}
  i\hbar\pdv{\rho}{t}=\comm{H}{\rho}\implies
  \pdv{\rho}{t}=\frac{i}{\hbar}\comm{\rho}{H}
\end{align*}
The time derivative of $S$ is then:
\begin{align*}
  \dv{S}{t}&=-\dv{t}\qty(\Tr[\rho\log\rho])=
  -\Tr[(1+\log\rho)\pdv{\rho}{t}]=-\frac{i}{\hbar}
  \Tr[(1+\log\rho)\comm{\rho}{H}]\\
  &=\frac{i}{\hbar}\qty(\Tr[H\rho]-\Tr[\rho H]+\Tr[(\log\rho)H\rho]
  -\Tr[(\log\rho)\rho H])
\end{align*}
However since the trace is cyclic, All of these terms cancel out, giving:
\begin{align}
  \boxed{\dv{S}{t}=0}
\end{align}

\subsection{Maximizing $S$}
We need two total constants in our equation, seeing as we need normalization of the density matrix, and average energy $E$, our $S(t)$ is:
\begin{align*}
  S&=\Tr[-\rho\ln\rho]-\alpha\qty(\Tr[\rho]-1)-\beta\qty(\Tr[\rho H]-E)\\
  &=\Tr{\rho\qty[-\ln\rho-\alpha-\beta H]}+\alpha+\beta E
\end{align*}
The derivative is a functional one due to the trace, but we will simply get
\begin{align*}
  \fdv{S(\rho)}{\rho_{max}}=-\ln\rho_{max}-1-\alpha-\beta H
\end{align*}
This is because the functional derivative produces a $\delta(\rho-\rho_{max})$ and the trace is trivial then. If this is to be maximum we need:
\begin{align*}
  \fdv{S}{\rho_m}=0\implies\ln\rho_{max}=-\qty(\alpha+1)-\beta H
\end{align*}
Solving for $\rho_{max}$:
\begin{align}
  \boxed{\rho_{max}=e^{-(\alpha+1)}e^{-\beta H}}
\end{align}
Which looks an awful lot like the canonical ensemble

\subsection{Stationary Solution}
Since the prefactor with $\alpha$ is simply a number, we have only a function of the Hamiltonian, which will commute with the Hamiltonian, a proof of which will look something like:
\begin{align*}
  H^nH=H H^{n-1}H=H H^n\implies H^n H=HH^n\implies \comm{H^n}{H}=0
\end{align*}
And Since the exponential is a convergent sum over many powers of $H$, we get:
\begin{align*}
  \pdv{\rho_{max}}{t}=i\hbar e^{-(\alpha+1)}\comm{e^{-\beta H}}{H}=0
\end{align*}
Hence the solution is stationary:
\begin{align}
  \boxed{\pdv{\rho_{max}}{t}=0}
\end{align}

\section{Ortho/para-hydrogen}
\subsection{Para Rotational Partition Function}
The partition function should be a simple sum, since angular momentum takes integer values:
\begin{align*}
  Z_p=\sum_{\text{even }\ell}\exp{-\beta\frac{\hbar^2}{2I}\ell(\ell+1)}
\end{align*}
Since for each $\ell$, there are $2\ell+1$ microstates with different $m$ values with the same $H$, we have:
\begin{align*}
  Z_p=\sum_{\text{even }\ell}(2\ell+1)\exp{-\beta\frac{\hbar^2}{2I}\ell(\ell+1)}
\end{align*}
Exchaning $\ell$ for an even integer $2n$:
\begin{align*}
  Z_p=\sum_{n=0}^\infty(4n+1)\exp{-\frac{\beta\hbar^2}{I}n(2n+1)}
\end{align*}
At low $T$, the largest contribution will be the terms near $n=0$, take the $n=0,1$ terms:
\begin{align}
  \boxed{Z_p\approx1+5\exp{-\frac{3\beta\hbar^2}{I}}}
\end{align}
And at high temperatures the sum turns into an integral since $\beta$ is small, writing it more suggestively:
\begin{align*}
  Z_p\approx\frac{I}{\beta\hbar^2}\sum_{n=0}^\infty
  \exp{-2\qty(\sqrt{\frac{\beta}{I}}\hbar n)^2}
  4\qty(\sqrt{\frac{\beta}{I}}\hbar n)
  \qty(\sqrt{\frac{\beta}{I}}\hbar)
\end{align*}
The first two terms in this product would be our integrand, and then the last one will act as our differential:
\begin{align*}
  \sum_{n=0}^\infty\exp{-2\qty(\sqrt{\frac{\beta}{I}}\hbar n)^2}
  4\qty(\sqrt{\frac{\beta}{I}}\hbar n)
  \qty(\sqrt{\frac{\beta}{I}}\hbar)
  =\int_0^\infty 4e^{-2x^2}x\dd{x}
\end{align*}
This integral is trivially evaluated:
\begin{align*}
  \int_0^\infty 4xe^{-2x^2}\dd{x}=1
\end{align*}
Giving:
\begin{align}
  \boxed{Z_p\approx\frac{I}{\beta\hbar^2}}
\end{align}

\subsection{Ortho Rotational Partition Function}
Once again we should get:
\begin{align*}
  Z_o=\sum_{\text{odd }\ell}(2\ell+1)\exp{-\beta\frac{\hbar^2}{2I}\ell(\ell+1)}
\end{align*}
However we are told it is triply degenerate, so we need to multiply by 3:
\begin{align*}
  Z_o=3\sum_{\text{odd }\ell}(2\ell+1)\exp{-\beta\frac{\hbar^2}{2I}\ell(\ell+1)}
\end{align*}
Replacing $\ell$ by an odd integer $2n+1$:
\begin{align*}
  Z_o=3\sum_{n=0}^\infty(4n+3)\exp{-\beta\frac{\hbar^2}{I}(2n+1)(n+1)}
\end{align*}
Taking the $n=0$ term for the low temperature limit since it is non-trivial:
\begin{align}
  \boxed{Z_o\approx9e^{-\beta\hbar^2/I}}
\end{align}
The high temperature limit will have the same argument, only multiplied by three:
\begin{align*}
  \boxed{Z_o\approx\frac{3I}{\beta\hbar^2}}
\end{align*}

\subsection{Equilibrium Gas Partition Function}
We should sum over various possible mixtures of each hydrogen, we should account for all the various combinations of each $N_p$ para hydrogen and $N-N_p$ ortho hydrogen:
\begin{align*}
  \sum_{N_p}\frac1{(N-N_P)!N_P!}
\end{align*}
We will also get 1 $Z_p$ for each para hydrogen, and 1 $Z_o$ for each ortho, giving the total:
\begin{align*}
  Z=\sum_{N_p}\frac1{(N-N_P)!N_P!}Z_p^{N_p}Z_o^{N-N_p}
\end{align*}
Note the similarity to the binomial expansion:
\begin{align*}
  (x+y)^n=\sum_k\frac{n!}{n!(n-k)!}x^{n-k}y^k
\end{align*}
If we simply divide out $N!$ we will have:
\begin{align*}
  (Z_o+Z_p)^N=\sum_{N_p}\frac{N!}{(N-N_P)!N_P!}Z_p^{N_p}Z_o^{N-N_p}
\end{align*}
Hence the partition function is:
\begin{align}
  \boxed{Z=\frac1{N!}(Z_o+Z_p)^N}
\end{align}

\subsection{Rotational Contribution to Internal Energy}
The average energy is simply:
\begin{align*}
  \ev{E_{rot}}=-\pdv{\log Z}{\beta}=-N\pdv{\beta}\log(Z_o+Z_p)
\end{align*}
The low energy limit:
\begin{align*}
  Z_o+Z_p\approx1+9e^{-\beta\hbar^2/I}
\end{align*}
Giving:
\begin{align}
  \boxed{\ev{E_{rot}}\approx9N\frac{\hbar^2}Ie^{-\beta\hbar^2/I}}
\end{align}
The high energy limits:
\begin{align*}
  Z_o+Z_p\approx\frac{4I}{\beta\hbar^2}
\end{align*}
Giving:
\begin{align}
  \boxed{\ev{E_{rot}}\approx Nk_BT}
\end{align}
\end{document}

