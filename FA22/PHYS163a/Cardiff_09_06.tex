\documentclass{beamer}

\usepackage{physics}
\usepackage{hyperref}
\usepackage{url}

\newcommand{\dbar}{d\hspace*{-0.08em}\bar{}\hspace*{0.1em}}

\newenvironment{itemframe}[1]{\begin{frame}{#1}\begin{itemize}}   {\end{itemize}\end{frame}}

% Changes style of actual slides
\usetheme{Dresden}
% Changes color of slides
\usecolortheme{spruce}
% removes controls at bottom right side
\usenavigationsymbolstemplate{}

% for figures
\graphicspath{ {./figs/} }

\title{Thermodynamic Potentials}
\author{Michael Cardiff}
% \logo{\large \LaTeX{}}
\subtitle{PHYS 163a \\ 09/06 Prep Work}

\begin{document}

\begin{frame}
  \titlepage
\end{frame}

\section{The Appropriate Potential}
\begin{frame}{What is our System?}
  \begin{itemize}
  \item Our system is in equilibrium with a heat bath with a temperature of $T$. 
  \item We know that $\dbar Q\neq0$ however, according to Claussius Theorem:
    \begin{align*}
      \dbar Q\leq T\dd{S}
    \end{align*}
  \item From the first law, and the fact that no work is done:
    \begin{align*}
      \dd{E}=\dbar Q+\dbar W\leq T\dd{S}
    \end{align*}
  \item Giving us the form for $F$:
    \begin{align*}
      \dd{F}\leq 0\implies \dd{F}=\eval{\dd{(E-TS)}}_{T}
    \end{align*}
  \end{itemize}
\end{frame}

\begin{frame}{Which one?}
  \begin{itemize}
  \item The Helmholtz Free Energy is the appropriate potential here
  \item The Gibbs Free Energy is also tempting, but since the system is isolated, I am tempted to believe that $\dbar W=0$
  \end{itemize}
\end{frame}

\begin{frame}{Equilibrium Equations}
  The Helmholtz Free energy is given as:
  \begin{align*}
    F=E-TS
  \end{align*}
  Its variations are given by:
  \begin{align*}
    \dd{F}=\dd{E}-\dd{(TS)}&=T\dd{S}+\vb{J}\vdot\dd{\vb{x}}-\dd{(TS)}\\
    &=T\dd{S}+\vb{J}\vdot\dd{\vb{x}}-T\dd{S}-S\dd{T}\\
    \dd{F}&=\vb{J}\vdot\dd{\vb{x}}-S\dd{T}
  \end{align*}
\end{frame}
\begin{frame}{Equilibium Equations}
  We can then find the Equilibrium forces and Entropy by taking partials of $F$ with respect to $(T,\vb{x})$:
  \begin{align*}
    J_i=\eval{\pdv{F}{x_i}}_{T,x\neq x_i}\qquad S=-\eval{\pdv{F}{T}}_{\vb{x}}
  \end{align*}
\end{frame}
\section{What are These Potentials?}
\begin{frame}{Enthalpy}
  Mathematically:
  \begin{align*}
    H=E-\vb{J\vdot x}
  \end{align*}
  \begin{itemize}
  \item Changes in enthalpy $\implies$ Changes in $S,\vb{J}$
  \item Describes the energy required to establish the system
  \end{itemize}
\end{frame}

\begin{frame}{Gibbs Free Energy}
  Mathematically:
  \begin{align*}
    G = E-TS - \vb{J\vdot x}
  \end{align*}
  \begin{itemize}
  \item Changes in Gibbs free energy $\implies$ Change in $T$, $\vb{J}$
  \item Quantifies maxiumum work that can be done by the system without change in $\vb{x}$ or $S$
  \end{itemize}
\end{frame}

\begin{frame}{Chemical Potential}
  \begin{itemize}
  \item A Chemical Potential would be associated with changes in number of particles in a system.
  \item Contrasts with $S,H,G,F$ as those all imply $\dd{N}=0$
  \end{itemize}
\end{frame}

\section{Equations}
\subsection{Gibbs-Duhem Equation}
\begin{frame}{Gibbs-Duhem Equation}
  The Extensive coordinates of a system tell us:
  \begin{equation}\label{eq:1}
    \dd{E}=T\dd{S}+\vb{J\vdot\dd{x}}+\mu\vdot\dd{\vb{N}}
  \end{equation}
  We fix the intensive coordinates, and find the extensive quantities are proportional to size/number of particles:
  \begin{align*}
    E(\lambda S,\lambda\vb{x},\lambda\vb{N})=\lambda E(S,\vb{x},\vb{N})
  \end{align*}
\end{frame}

\begin{frame}{Gibbs-Duhem Equation}
  Differentiating this equation with respect to $\lambda$ and setting it to 1:
  \begin{align*}
    \pdv{E}{S}S+\sum_ix_i\pdv{E}{x_i}+\sum_{j}N_j\pdv{E}{N_j}=E(S,\vb{x},\vb{N})
  \end{align*}
  Where the $S$ partial is at constant $\vb{x,N}$, the $x_i$ partial is at constant $S,\vb{N}$ and $x_{k\neq i}$, and the $N_j$ partial has constant $S,\vb{x}$ and $\vb{N}_{\ell\neq j}$

  These partials can be found by using the relationship of the extensive coordinates in \eqref{eq:1} to get:
  \begin{align*}
    E=TS+\vb{J\vdot x}+\vb{\mu\vdot N}
  \end{align*}
\end{frame}

\begin{frame}{Gibbs-Duhem Equation}
  Taking the total differential of this equation and keeping in mind equation \eqref{eq:1}, get:
  \begin{align*}
    \dd{E}&=T\dd{S}+S\dd{T}+\vb{J\vdot\dd{x}+x\vdot\dd{J}+\mu\dd{N}+N\dd{\mu}}\\
    &=T\dd{S}+\vb{J\vdot\dd{x}+\mu\vdot\dd{N}}
    +S\dd{T}+\vb{x\vdot\dd{J}+N\dd{\mu}}
  \end{align*}
  Notice the first three terms in the second line are what we said was $\dd{E}$, meaning the next three terms must sum to $0$:
  \begin{align*}
    \boxed{0=S\dd{T}+\vb{x\vdot\dd{J}+N\dd{\mu}}}
  \end{align*}
\end{frame}
\begin{frame}{Maxwell Relations}
  All Maxwell Relations are simply expressions of the following rule in Multivariable Calculus:
  \begin{align*}
    \pdv{f}{x}{y}=\pdv{f}{y}{x}
  \end{align*}
  Using $f$ as one of the thermodynamic potentials and $x,y$ as an appropriate variable included in the potential, we can find various relationships between the variables of the system.
\end{frame}
\begin{frame}{Example Maxwell Relation}
  For example take the internal energy of the system:
  \begin{align*}
    \dd{E}=T\dd{S}+\vb{J\vdot\dd{x}}
  \end{align*}
  The usefulness of this notation is that we can get partials very easily:
  \begin{align*}
    \eval{\pdv{E}{S}}_{\vb{x,N}}=T\qquad
    \eval{\pdv{E}{x_i}}_{S,\vb{N}x_{j\neq i}}=J_i
  \end{align*}
\end{frame}
\begin{frame}{Example Continued}
  Taking the second partial of the opposite variable ($x_i$ for $S$ and vice versa):
  \begin{align*}
    \pdv{E}{S}{x_i}=\eval{\pdv{T}{x_i}}_{S}\qquad
    \pdv{E}{x_i}{S}=\eval{\pdv{J_i}{S}}_{x_i}
  \end{align*}
  Then by the miracle of calculus we arrive at:
  \begin{align*}
    \boxed{\eval{\pdv{T}{x_i}}_{S}=\eval{\pdv{J_i}{S}}_{x_i}}
  \end{align*}
  Similar results can be obtained using the other potentials. 
\end{frame}

\begin{frame}{Gibbs Phase Rule}
  \begin{itemize}
  \item The Phase rule describes the number of degrees of freedom $f$ in a system with $p$ 'phases' coexisting, $n$ ways of doing work and $c$ chemical constituents. 
  \item The $1$ comes from thermal changes
  \item Described by the equation:
    \begin{align*}
      f=n+c+1-p
    \end{align*}
  \end{itemize}
\end{frame}

\end{document}