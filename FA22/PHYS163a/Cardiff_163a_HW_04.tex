\documentclass[12pt]{article}

\title{\vspace{-3em}PHYS 163a HW 4}
\author{Michael Cardiff}
\date{\today}

%% science symbols 
\usepackage{amssymb,amsthm,bm,physics,slashed}

%% general pretty stuff
\usepackage{caption,enumitem,float,geometry,graphicx,tikz}

% setups
\graphicspath{ {./figs/} }
\captionsetup{labelfont=bf}
\geometry{margin=1in}

% macros
\renewcommand{\L}{\mathcal{L}}
\newcommand{\D}{\partial}
\newcommand{\circled}[1]{\tikz[baseline=(char.base)]{
    \node[shape=circle,draw,inner sep=2pt](char){#1};}}

\begin{document}
\maketitle

\section{Hard Sphere Gas}
\subsection{Entropy}
In order to compute the entropy we should calculate the number of microstates, then use:
\begin{align*}
  S=k_B\log\Omega
\end{align*}
To find the corresponding entropy.

We find the number of microstates by integrating over the phase space:
\begin{align*}
  \Omega=\frac1{N!}\frac1{h^{3N}}\int\prod_{i=1}^N\dd{\vb{p}}_i\dd{\vb{q}}_i
  \delta\qty(\sum_{i=1}^N\frac{\vb{p}_i^2}{2m}-E)\Delta_E
\end{align*}
Where $\Delta_E$ is some constant with dimensions of energy.

The momentum integrals are the exact same as in the ideal gas, the surface area of a $3N$-sphere, so we get:
\begin{align*}
  \Omega=\frac{\Delta_E}{N!h^{3N}}
  \frac{(2\pi)^{3N}(2mE)^{\frac{3N}2-1}}{(\frac{3N}{2}-1)!}
  \int\prod\dd{\vb{q}}_i
\end{align*}
In the case of the ideal gas, the $N$ molecules were just integrated over the volume $V$, since they did not interact with one another, so they can occupy the entire volume. Only the first molcule inserted can explore the entire volume, so $\int\dd{\vb{q}}_1=V$. The next nested integral has similar bounds, however the only difference is that it can occupy $\omega$ less volume. The next will have $2\omega$ less volume. This continues until we add the last particle and it has $V-(N-1)\omega$ volume, so the position integrals are:
\begin{align*}
  \int\prod\dd{\vb{q}}_i=V(V-\omega)(V-2\omega)\cdots(V-(N-1)\omega)
\end{align*}
Applying the hint given in the problem, we can rewrite the $N$ terms as:
\begin{align*}
  V(V-\omega)(V-2\omega)\cdots(V-(N-1)\omega)\approx\qty(V-\frac{N\omega}{2})^N
\end{align*}
Thus the number of microstates are:
\begin{align*}
  \Omega=\frac{\Delta_E}{N!h^{3N}}
  \frac{(2\pi)^{3N}(2mE)^{\frac{3N}2-1}}{(\frac{3N}{2}-1)!}
  \qty(V-\frac{N\omega}{2})^N
\end{align*}
Simplifying:
\begin{align*}
  \Omega&=\frac{\Delta_E}{N!h^{3N}}
  \frac{(2\pi)^{3N}(2mE)^{\frac{3N}2-1}}{(\frac{3N}{2}-1)!}
  \qty(V-\frac{N\omega}{2})^N\\
  &=\frac{\Delta_E(2mE)^{-1}}{N!(\frac{3N}{2}-1)!}
  \qty(\frac{(2\pi)^3(2mE)^{3/2}}{h^3}\qty(V-\frac{N\omega}{2}))^N
\end{align*}
We can ignore the constants in front, then we have:
\begin{align*}
  S&=k_B\log\Omega\approx k_B\log\frac{\Delta_E(2mE)^{-1}}{N!(\frac{3N}{2}-1)!}
  \qty(\frac{(2\pi)^3(2mE)^{3/2}}{h^3}\qty(V-\frac{N\omega}{2}))^N\\
  &=k_B\log\frac{\Delta_E}{2mE}
  -\log(N!)-\log(\qty(\frac{3N}{2}-1)!)
  +Nk_B\log(\qty(\frac{2\pi\sqrt{2mE}}h)^3\qty(V-\frac{N\omega}{2}))
\end{align*}
Using the Sterling Approximation gives:
\begin{align*}
  \log(N!)+\log(\qty(\frac{3N}{2}-1)!)=N\log N-N+
  \qty(\frac{3N}{2}-1)\log(\frac{3N}{2}-1)-\frac{3N}{2}+1
\end{align*}
We can then write the entropy as:
\begin{align}
  \boxed{S=k_B\log(\text{const})
    +Nk_B\log(\frac1N\qty(\frac{2\pi\sqrt{2mE}}h)^3\qty(V-\frac{N\omega}{2}))}
\end{align}
\subsection{Equation of State}
The equation of state can be found using:
\begin{align*}
  \frac{P}{T}=\eval{\pdv{S}{V}}_{E,N}
\end{align*}
The derivative with respect to $V$ is fairly simply, as the only term that survives is the constant in front then the actual volume term:
\begin{align*}
  \eval{\pdv{S}{V}}_{E,N}=\frac{Nk_B}{V-\frac{N\omega}{2}}
\end{align*}
Hence the equation of state is:
\begin{align}
  \boxed{P\qty(V-\frac{N\omega}{2})=Nk_BT}
\end{align}
\subsection{Isothermal Compressibility}
This quantity is given by the following:
\begin{align*}
  \kappa_T=-\frac1V\eval{\pdv{V}{P}}_{T,N}
\end{align*}
The derivative with respect to $P$ of $V-N\omega/2$ is the same as just the derivative of just $V$, so we can ignore the rest. Hence the derivative is:
\begin{align*}
  \pdv{V}{P}=-\frac{Nk_BT}{P^2}
\end{align*}
And the compressibility is:
\begin{align}
  \boxed{\kappa_T=\frac{Nk_BT}{P^2V}}
\end{align}
\section{Molecular Oxygen}
The Hamiltonian we want to consider:
\begin{align*}
  H=\sum_{i=1}^N\qty[\frac{\vb{p}_i^2}{2m}-\mu BS_i^z]
\end{align*}
\subsection{Partition Function}
The partition function is defined as:
\begin{align*}
  Z=\sum_{\mu}e^{-\beta H}
\end{align*}
However for non-quantized degrees of freedom we have integrals:
\begin{align*}
  Z_{\vb{p,q}}=\int\prod_{i=1}^N\dd{\vb{q}}_i\dd{\vb{p}}_ie^{-\beta H_{\vb{p,q}}}
\end{align*}
Substituting the Hamiltonian in:
\begin{align*}
  Z_{\vb{p,q}}=\int\prod_{i=1}^N\dd{\vb{q}}_i\dd{\vb{p}}_i
  e^{-\beta\sum\vb{p}_i^2/2m}
\end{align*}
The $\vb{q}_i$ integrals are identical to the ones we did for the previous problem/ideal gas since there is no $q$ dependence in the Hamiltonian, so we can simply take out the $V^N$:
\begin{align*}
  Z_{\vb{p,q}}=V^N\int\prod_{i=1}^N\dd{\vb{p}}_i
  e^{-\beta\sum\vb{p}_i^2/2m}
\end{align*}
Realize that one individual momentum integral is simply a Gaussian:
\begin{align*}
  Z_{\vb{p,q}}\propto\int\dd{\vb{p}}_1e^{-\beta \vb{p}_1^2/2m}\equiv I
\end{align*}
This one integral gives:
\begin{align*}
  I=\qty(\frac{2m\pi}{\beta})^{3/2}
\end{align*}
And since we have $N$ of them, we can say:
\begin{align*}
  Z_{\vb{p,q}}=V^N\qty(\frac{2m\pi}{\beta})^{3N/2}
\end{align*}
However we also need to take into account the fact that these particles are identical, and that $Z$ needs to be dimensionless, so we need a factor of $h^{-3N}$ and an inverse $N!$:
\begin{align*}
  Z_{\vb{p,q}}&=\frac1{h^{3N}N!}V^N\qty(\frac{2m\pi}{\beta})^{3N/2}\\
  &=\frac1{N!}\qty(\frac{V}{\lambda^3})^N
\end{align*}
Where the second line uses the $\lambda$ factor defined in Kardar's Equation 4.80:
\begin{align*}
  \lambda=\frac{h}{\sqrt{2m\pi k_BT}}=h\sqrt{\frac{\beta}{2\pi}}
\end{align*}
The partial partition function for the spin degrees of freedom is instead a sum:
\begin{align*}
  Z_S=\sum_{S^z}e^{\beta\sum_i\mu BS_i^z}
\end{align*}
The sum is over all of the spin configurations of each particle. This means there are $N$ sums each summing $S^z=-1,0,1$, one of these sums is:
\begin{align*}
  Z_S\propto\sum_{S^z}e^{\beta\mu BS^z_1}&=
  e^{-\beta\mu B}+e^{\beta\mu B(0)}+e^{\beta\mu B}\\
  &=e^{-\beta\mu B}+1+e^{\beta\mu B}
\end{align*}
There are $N$ of these sums, and each are multiplied from the fact that:
\begin{align*}
  e^{\sum_ia_i}=\prod_ie^{a_i}
\end{align*}
Hence the spin partial partition function is:
\begin{align*}
  Z_S=\qty(e^{-\beta\mu B}+1+e^{\beta\mu B})^N
\end{align*}
We do not need to do the fixing again since we already did it for the position and momentum partial partition functions.

The total partition function is:
\begin{align}
  \boxed{
    Z=\frac{1}{N!}\qty(\frac{V}{\lambda^3}(e^{-\beta\mu B}+1+e^{\beta\mu B}))^N
  }
\end{align}
\subsection{Spin Probabilities}
The probability for a specific spin can be calculated analagously to the energy probability distribution for the ideal gas, however instead of an energy $E$ corresponding to the Hamiltonian, it is a particular spin $S$ corresponding to the spin part of the Hamiltonian specifically:
\begin{align*}
  p(S)=\frac{e^{-\beta\mu B S}}{Z}\sum_\mu\delta(S^z-S)
\end{align*}
Since it is a specific molecule, we need a 1-particle partition function:
\begin{align*}
  Z=e^{-\beta\mu B}+1+e^{\beta\mu B}
\end{align*}
The sum just evaluates to 1 each time, and thus the probabilities are:
\begin{equation}
  \boxed{
  \begin{aligned}
    p(S=-1)&=\frac{e^{-\beta\mu B}}{e^{-\beta\mu B}+1+e^{\beta\mu B}}\\
    p(S= 0)&=\frac1{e^{-\beta\mu B}+1+e^{\beta\mu B}}\\
    p(S= 1)&=\frac{e^{\beta\mu B}}{e^{-\beta\mu B}+1+e^{\beta\mu B}}
  \end{aligned}
  }
\end{equation}
\subsection{Average Dipole Moment}
The dipole moment is given by:
\begin{align*}
  M=\mu\sum_{i=1}^NS_i^z
\end{align*}
Using the partition function, we can get $\ev{M}$ by differentiating $\log Z$ with respect to the magnetic field $B$, and dividing out the extra $\beta$.
\begin{align*}
  \ev{M}=\frac1\beta\pdv{\log Z}B
\end{align*}
The division by volume in the problem means we are ignoring the other degrees of freedom.

The derivative is:
\begin{align*}
  \pdv{\log Z}B&=\pdv{B}\qty(N\ln(e^{\beta\mu B}+1+e^{-\beta\mu B}))\\
  &=N\beta\mu\frac{e^{\beta\mu B}-e^{-\beta\mu B}}
  {e^{\beta\mu B}+1+e^{-\beta\mu B}}
\end{align*}
Thus the average magnetic moment is:
\begin{align}
  \boxed{\ev{M}=
    N\mu\frac{e^{\beta\mu B}-e^{-\beta\mu B}}
    {e^{\beta\mu B}+1+e^{-\beta\mu B}}}
\end{align}
\subsection{Zero Field Susceptibility}
The derivative of the average moment is:
\begin{align*}
  \pdv{\ev{M}}{B}=N\beta\mu^2e^{\beta\mu B}
  \frac{1+4e^{\beta\mu B}+e^{2\beta\mu B}}
  {\qty(1+e^{\beta\mu B}+e^{2\beta\mu B})^2}
\end{align*}
Evaluating at $B=0$:
\begin{align}
  \boxed{\chi=\eval{\pdv{\ev{M}}{B}}_{B=0}=\frac23N\beta\mu^2}
\end{align}
\section{Polar Rods}
Hamiltonian:
\begin{align*}
  H_{rot}=\frac1{2I}\qty(p_\theta^2+\frac{p_\phi^2}{\sin^2\theta})
\end{align*}
\subsection{Rotational Degrees of Freedom}
Since we are working with the two angles of spherical coordinates, we have:
\begin{align*}
  \int\dd{\vb{q}}=\int_0^\pi\dd{\theta}\int_0^{2\pi}\dd{\phi}
\end{align*}
While the momentum integrals are their normal $(-\infty,\infty)$ bounds for each of $p_\theta,p_\phi$, so the rotational classical partition function is:
\begin{align*}
  Z_{rot}=\frac1{h^2}\int_0^\pi\dd{\theta}\int_0^{2\pi}\dd{\phi}
  \int_{-\infty}^\infty\dd{p}_\theta\int_{-\infty}^\infty\dd{p}_\phi
  \exp{-\frac{\beta}{2I}\qty(p_\theta^2+\frac{p_\phi^2}{\sin^2\theta})+\beta\mu E\cos\theta}
\end{align*}
Since the $\cos\theta$ term in the exponential is independent of momentum, so we can take it out:
\begin{align*}
  Z_{rot}=\frac1{h^2}\int_0^\pi\dd{\theta}\int_0^{2\pi}\dd{\phi}
  e^{\beta\mu E\cos\theta}
  \int_{-\infty}^\infty\int_{-\infty}^\infty\dd{p}_\theta\dd{p}_\phi
  \exp{-\frac{\beta}{2I}\qty(p_\theta^2+\frac{p_\phi^2}{\sin^2\theta})}
\end{align*}
We should integrate the momentum integrals first, as they are just Gaussians:
\begin{align*}
  \int_{-\infty}^\infty\int_{-\infty}^\infty\dd{p}_\theta\dd{p}_\phi
  \exp{-\frac{\beta}{2I}\qty(p_\theta^2+\frac{p_\phi^2}{\sin^2\theta})}
  =\frac{2\pi I}{\beta}\sin\theta
\end{align*}
hence the position integrals are:
\begin{align*}
  \frac{2\pi I}{\beta}\int_0^\pi\dd{\theta}\int_0^{2\pi}\dd{\phi}
  e^{\beta\mu E\cos\theta}\sin\theta
\end{align*}
Notice that there is no $\phi$ dependence, so it can be integrated out:
\begin{align*}
  \frac{2\pi I}{\beta}\int_0^\pi\dd{\theta}\int_0^{2\pi}\dd{\phi}
  e^{\beta\mu E\cos\theta}\sin\theta=
  \frac{(2\pi)^2 I}{\beta}\int_0^\pi\dd{\theta}
  e^{\beta\mu E\cos\theta}\sin\theta
\end{align*}
This integral is easy if we use $x=\cos\theta$, giving the integral:
\begin{align*}
  \frac{(2\pi)^2 I}{\beta}\int_0^\pi\dd{\theta}
  e^{\beta\mu E\cos\theta}\sin\theta&=
  \frac{(2\pi)^2 I}{\beta}\int_{-1}^1\dd{x}
  e^{\beta\mu Ex}\\
  &=\frac2{\beta\mu E}\sinh(\beta\mu E)
\end{align*}
Hence the partition function is simply:
\begin{align}
  \boxed{Z_{rot}=\frac{8\pi^2I}{h^2\beta^2\mu E}\sinh(\beta\mu E)}
\end{align}
\subsection{Mean Polarization}
The mean polarization is the average of $\mu\cos\theta$, which we can get from the log of the partition function if we differentiate with respect to $\beta E$ (if we use the original integral formulation). Hence:
\begin{align*}
  \ev{\mu\cos\theta}=\pdv{\log Z}{\beta E}=
  \mu\qty[\coth\beta\mu E-\frac1{\beta\mu E}]
\end{align*}
Which is our answer:
\begin{align}
  \boxed{P=\mu\qty[\coth\beta\mu E-\frac1{\beta\mu E}]}
\end{align}
\subsection{Zero Field Polarizability}
This is the same deal as the susceptibility from before:
\begin{align*}
  \pdv{P}{E}=\mu\qty(-\beta\mu\mathrm{csch}^2(\beta\mu E)+\frac1{\beta\mu E^2})
\end{align*}
Where $\mathrm{csch}\equiv(\sinh)^{-1}$

Subbing in $E=0$ is not enough, but it is possible with taking a limit (done with Mathematica):
\begin{align}
  \boxed{\chi_T=\frac13\beta\mu^2}
\end{align}
\subsection{Rotational Energy per Particle}
We can differentiate $\log Z_{rot}$ with respect to $\beta$ to find the energy associated with these degrees of freedom:
\begin{align*}
  \ev{E}=-\pdv{\log Z_{rot}}{\beta}=\frac2\beta-\mu E\coth(\beta\mu E)
\end{align*}
Writing out $\beta$ here is helpful:
\begin{align}
  \boxed{\ev{E}=2k_BT-\mu E\coth(\frac{\mu E}{k_BT})}
\end{align}
At high temperatures we should expect $\ev{E}=k_BT$, only kinetic energy, and at low temperatures, the $\coth$ goes to 1, so $\ev{E}=2k_BT-\mu E$
\subsection{Heat Capacity per Dipole}
Classical heat capacity at low temperatures is $2k_B$, and $k_B$ at high temperatures from simply taking a temperature derivative of the above relations. There should be some shifting behavior, whether it is a step function or other similar kink, it will be at $k_BT\approx\mu E$.
\end{document}