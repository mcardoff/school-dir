\documentclass{beamer}

\usepackage[italicdiff]{physics}
\usepackage{siunitx}
\usepackage{hyperref}
\usepackage{url}

\newcommand{\dbar}{d\hspace*{-0.08em}\bar{}\hspace*{0.1em}}

\newenvironment{itemframe}[1]{\begin{frame}{#1}\begin{itemize}}   {\end{itemize}\end{frame}}

% Changes style of actual slides
\usetheme{Dresden}
% Changes color of slides
\usecolortheme{spruce}
% removes controls at bottom right side
\usenavigationsymbolstemplate{}

% for figures
\graphicspath{ {./figs/} }

\title{Phase Transitions!}
\author{Michael Cardiff}
\subtitle{PHYS 163a \ 12/02 Prep Work}

\begin{document}

\begin{frame}
  \titlepage
\end{frame}

\begin{frame}{What is a Phase transition?}
  \begin{itemize}
  \item Example: Water freezing!
  \item At \SI{0.01}{\degreeCelsius}, we have water, but all of a sudden, once we go to \SI{-0.01}{\degreeCelsius}, it has turned into ice!
  \item For infinitesimal changes in temperature, there is a more grand change in physical properties. 
  \item Example: Water boiling
  \item All of a sudden, the volume increases with no a lot of change in the overall temperature. 
  \end{itemize}
\end{frame}

\begin{frame}{Simplifying Ising Hamiltonian}
  \begin{itemize}
  \item Immediately We can change the Hamiltonian given our assumptions about $J_{ij}$:
    \begin{align*}
      H=-\frac{J}{2N}\sum_{i,j=1}^N\sigma_i\sigma_j-h\sum_i\sigma_i
    \end{align*}
  \item We can then separate the double sum since the two spins are multiplied:
    \begin{align*}
      H=-\frac{J}{2N}\sum_{i=1}^N\sigma_i\sum_{j}\sigma_j-h\sum_i\sigma_i
    \end{align*}
  \end{itemize}
\end{frame}

\begin{frame}{Continued}
  \begin{itemize}
  \item Then Define the quantity $M$ as the sum of the spins:
    \begin{align*}
      H=-\frac{J}{2N}M^2-hM
    \end{align*}
  \item Now define $m=N/N$:
    \begin{align*}
      H=-\frac{JN}{2}m^2-hmN=\boxed{-N\qty(\frac{Jm^2}{2}+hm)}
    \end{align*}
  \end{itemize}
\end{frame}

\begin{frame}{Uniformity and Isotropy}
  \begin{itemize}
  \item Kardar Specifically Says that Uniformity and Isotropy = "All locations and directions in $\vb{x}$ are equivalent"
  \item This means if we were to rotate and shift around our system, we would not be able to tell the difference.
  \item The idea is to show that under a transformation $T$, whether that be rotations or translations, the ending computation is the same, i.e.:
    \begin{align*}
      \partial_\alpha (T\vb{m})\partial_\alpha(T \vb{m})=
      \partial_\alpha m_i\partial_\alpha m_i
    \end{align*}
  \end{itemize}
\end{frame}
\begin{frame}{Rotations}
  \begin{itemize}
  \item For rotations, the transformation is just a matrix multiplication.
  \item Since we are rotating by an arbitrary constant, the derivative commutes with the rotation operation
  \item Since we are taking the dot product of these, one will have a transpose on it
  \item Effectively, these will out, since for a rotation Matrix $R$: $R^TR=I$, the transpose is the inverse. 
    
  \end{itemize}
\end{frame}

\begin{frame}{Translations}
  \begin{itemize}
  \item This is much simpler
  \item The action of translation is adding a constant to everything
  \item Since it is just a constant vector, it has no derivative.
  \end{itemize}
\end{frame}

\begin{frame}{Ginzburg Landau Hamiltonian}
  \begin{itemize}
  \item We are integrating:
    \begin{align*}
      \beta H=\int\dd[d]{\vb{x}}\qty[\frac{t}{2}m^2(\vb{x})+um^4(x)]
    \end{align*}
  \end{itemize}
\end{frame}
\end{document}