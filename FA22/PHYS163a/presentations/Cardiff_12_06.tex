\documentclass{beamer}

\usepackage[italicdiff]{physics}
\usepackage{hyperref}
\usepackage{url}
\usepackage{siunitx}

\newcommand{\dbar}{d\hspace*{-0.08em}\bar{}\hspace*{0.1em}}

\newenvironment{itemframe}[1]{\begin{frame}{#1}\begin{itemize}}   {\end{itemize}\end{frame}}

% Changes style of actual slides
\usetheme{Dresden}
% Changes color of slides
\usecolortheme{spruce}
% removes controls at bottom right side
\usenavigationsymbolstemplate{}

% for figures
\graphicspath{ {./figs/} }

\title{Spontaneous Symmetry Breaking}
\author{Michael Cardiff}
\subtitle{PHYS 163a \ 12/06 Prep Work}

\begin{document}

\begin{frame}
  \titlepage
\end{frame}

\begin{frame}{What is Spontaneous Symmetry Breaking?}
  \begin{columns}
    \begin{column}{0.5\textwidth}
      \begin{itemize}
      \item Magnet Example, $(\vb{h}\to0)$, Hamiltonian should have rotational symmetry
      \item However, past the phase transition, it does not.
      \item That is, the symmetry is broken spontaneously
      \item Minima at $m=0$ is local and unstable
      \item There is a lower energy state in a specific direction
      \end{itemize}
    \end{column}
    \begin{column}{0.5\textwidth}
      \begin{figure}[H]
        \centering
        \includegraphics[width=5.0cm]{potential.png}
        \caption{Example for Spontaneous Symmetry Breaking}
      \end{figure}
    \end{column}
  \end{columns}
\end{frame}
\begin{frame}{Magnets}
  \begin{itemize}
  \item Magnets above $T_C$ obey rotational symmetry
  \item Below, it is energetically favorable to break this symmetry
  \end{itemize}
\end{frame}

\begin{frame}{Freezing Liquids}
  \begin{columns}
    \begin{column}{0.5\textwidth}
      \begin{itemize}
      \item Microdynamics of a liquid are rotationally invariant
      \item A solid only has a discrete symmetry
      \item The same effect can only be achieved if you rotate by a fixed amount
      \item Figure A: Invariant under Rotations of $\ang{120}$
      \item Figure B: Invariant under any Rotation
      \end{itemize}
    \end{column}
    \begin{column}{0.5\textwidth}
      \begin{figure}[H]
        \centering
        \includegraphics[width=2.4cm]{ice}
        \includegraphics[width=2.0cm]{water}
        \caption{Ice Vs Water}
      \end{figure}
    \end{column}
  \end{columns}
\end{frame}

\begin{frame}{Liquid Crystal}
  \begin{columns}
    \begin{column}{0.5\textwidth}
      \begin{itemize}
      \item Same as normal liquids and Magnets
      \item Rotational Invriance in isotropic liquid phase
      \item Alignment along some axis $\vu{n}$ in the nematic phase
      \item Translational symmetry maintained
      \end{itemize}
    \end{column}
    \begin{column}{0.5\textwidth}
      \begin{figure}[H]
        \centering
        \includegraphics[width=5.0cm]{lc}
        \caption{Phases of a Liquid Crystal}
      \end{figure}
    \end{column}
  \end{columns}
\end{frame}

\end{document}