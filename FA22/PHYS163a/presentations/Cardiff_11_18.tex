\documentclass{beamer}

\usepackage[italicdiff]{physics}
\usepackage{hyperref}
\usepackage{url}

\newcommand{\dbar}{d\hspace*{-0.08em}\bar{}\hspace*{0.1em}}

\newenvironment{itemframe}[1]{\begin{frame}{#1}\begin{itemize}}   {\end{itemize}\end{frame}}

% Changes style of actual slides
\usetheme{Dresden}
% Changes color of slides
\usecolortheme{spruce}
% removes controls at bottom right side
\usenavigationsymbolstemplate{}

% for figures
\graphicspath{ {./figs/} }

\title{Grand Canonical Ensemble}
\author{Michael Cardiff}
\subtitle{PHYS 163a \ 11/18 Prep Work}

\begin{document}

\begin{frame}
  \titlepage
\end{frame}

\section{Deriving the Ensemble}
\begin{frame}{Grand Canonical Ensemble}
  \begin{itemize}
  \item Describes macrostates of the form $(T,\mu,\vb{x})$
  \item Microstates have an unknown number of particles
  \item Like canonical ensemble, system is in equilibrium with a Reservoir at temperature $T$ and Chemical Potential $\mu$
  \item Integrating out the reservoir microstates leads to:
    \begin{align*}
      p(\mu_S)=\frac{\Omega_{R}(E_{Tot}-H_S(\mu_S),N_{Tot}-N_S)}
      {\Omega_{S+R}(E_{Tot},N_{Tot})}
    \end{align*}
  \item Using the same expansion of $\Omega=\frac1{k_B}\exp{S}$ we did in the canonical enseble, we find:
    \begin{align*}
      S(E_{Tot}-H_S,N_{Tot}-N_S)\approx S(E_{Tot},N_{Tot})
      -H_S\pdv{S}{E}-N_S\pdv{S}{N}
    \end{align*}
  \end{itemize}
\end{frame}
\begin{frame}{Continued}
  \begin{itemize}
  \item First law with $E,N$ gives:
    \begin{align*}
      \dd{S}=\frac1T\dd{E}-\frac\mu{T}\dd{N}
    \end{align*}
  \item So Our entropy in the exponential is:
    \begin{align*}
      S(E_{Tot}-H_S,N_{Tot}-N_S)\approx
      S(E_{Tot},N_{Tot})-\frac{H_S}{T}+\frac{\mu N_S}{T}
    \end{align*}
  \item So the probabilities look like:
    \begin{align*}
      p(\mu_S)=\frac1{\Omega_{S+R}}\exp{\frac1{k_BT}
        \qty(S(E_{Tot},N_{Tot})-H_S+\mu N_S)}
    \end{align*}
  \end{itemize}
\end{frame}
\begin{frame}{Grand Partition Function}
  \begin{itemize}
  \item Hence the normalized probabilities are given by:
    \begin{align*}
      p_{T,\mu,\vb{x}}(\mu)=\frac{e^{\beta\mu N-\beta H}}{Q}
    \end{align*}
  \item With $Q$ as the grand Partition function:
    \begin{align*}
      Q=\sum_{\mu_S}\exp{\beta\mu N(\mu_S)-\beta H(\mu_S)}
    \end{align*}
  \end{itemize}
\end{frame}

\section{How to get to Thermodynamics}
\begin{frame}{Thermodyamics}
  \begin{itemize}
  \item Variance of the number of particles is proportional to $N$:
    \begin{align*}
      \ev{N^2}_C=\pdv{\ev{N}}{(\beta\mu)}=\pdv[2]{(\beta\mu)}\log Q
    \end{align*}
  \item Since $N\sim10^{23}$, we know the distribution is sharply peaked, and the grand partition function can be approximated by its largest term, $\ev{N}$:
    \begin{align*}
      Q=\sum_Ne^{\beta\mu N}Z=e^{\beta\mu \ev{N}}Z(T,\ev{N},\vb{x})
    \end{align*}
  \item Where $Z$ is the canonical partition function of fixed $N,T,\vb{x}$ 
  \end{itemize}
\end{frame}
\begin{frame}{Continued}
  \begin{itemize}
  \item The Canonical Ensemble gives us the Helmholtz free energy $F$ as:
    \begin{align*}
      F=-k_BT\log Z\implies Z=e^{-\beta F}
    \end{align*}
  \item Hence we can write the grand partition function as:
    \begin{align*}
      Q=e^{\beta\mu \ev{N}}e^{-\beta F}=\exp{\beta\qty(\mu \ev{N}-F)}
    \end{align*}
  \item The potential in the argument is the Grand Potential $\mathcal{G}$:
    \begin{align*}
      Q=\exp{-\beta\mathcal{G}}x\implies\mathcal{G}=-k_BT\log Q
    \end{align*}
  \item They describe the same thermodynamics, they are equivalent!
  \end{itemize}
\end{frame}
\end{document}