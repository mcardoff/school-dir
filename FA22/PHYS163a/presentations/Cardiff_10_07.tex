\documentclass{beamer}

\usepackage[italicdiff]{physics}
\usepackage{siunitx}
\usepackage{hyperref}
\usepackage{url}

\newcommand{\dbar}{d\hspace*{-0.08em}\bar{}\hspace*{0.1em}}

\newenvironment{itemframe}[1]{\begin{frame}{#1}\begin{itemize}}   {\end{itemize}\end{frame}}

% Changes style of actual slides
\usetheme{Dresden}
% Changes color of slides
\usecolortheme{spruce}
% removes controls at bottom right side
\usenavigationsymbolstemplate{}

% for figures
\graphicspath{ {./figs/} }

\title{Kinetic Theory II}
\author{Michael Cardiff}
\subtitle{PHYS 163a \ 10/07 Prep Work}

\begin{document}

\begin{frame}
  \titlepage
\end{frame}

\section{Estimation of Parameters}
\begin{frame}{Estimating Size Parameters}
  \begin{itemize}
  \item In order to find what we are dealing with, we should use dimensional analysis to find units for the following:
    \begin{align*}
      \pdv{U}{\vb{q}}\pdv{f_s}{\vb{p}}
    \end{align*}
  \item The density $f_s$ is a number density, so it is dimensionless.
  \item We have the other dimensions:
    \begin{align*}
      [\vb{q}]=L\quad[\vb{p}]=MLT^{-1}\quad[U]=[E]=ML^2T^{-2}
    \end{align*}
  \item The dimensions for a derivative are:
    \begin{align*}
      \qty[\pdv{f}{x}]=[x^{-1}][f(x)]
    \end{align*}
  \end{itemize}
\end{frame}
\begin{frame}{The First Quantity}
  \begin{itemize}
  \item Thus, we are looking for:
    \begin{align*}
      \pdv{U}{\vb{q}}\pdv{f_s}{\vb{p}}=T^{-1}
    \end{align*}
  \item Since you can only add quantities with the same units, every other quantity is also an inverse time, that is the only way the equality can work out.
  \end{itemize}
\end{frame}
\section{Finding Time Scales}
\begin{frame}{How do we find a time scale?}
  \begin{itemize}
  \item Since each term has a potential energy $U$ or $V$, which is dependent on the distance between particles, so we have a length scale $L$
  \item We are also considering a particle of momentum $p$, or more importantly, a velocity $v$
  \item To find a time from a length and velocity, we can take
    \begin{align*}
      t=\frac{L}{v}
    \end{align*}
  \item The typical velocity of a gas particle at room temperature is on the order of $\SI{100}{\m\per\s}$
  \end{itemize}
\end{frame}

\section{External Potential}
\begin{frame}{Time scale for the External Potential}
  \begin{itemize}
  \item We can think of the derivative $\pdv{U}{\vb{q}}$ as large scale variations of the particle, so for the typical length of $\SI{1}{\mm}$, we get a time scale of:
    \begin{align*}
      t_U\approx\frac{10^{-3}}{10^2}= 10^{-5}\unit{\s}
    \end{align*}
  \end{itemize}
\end{frame}

\section{Interaction Potential}
\begin{frame}{Time scale for the Interaction Potential}
  \begin{itemize}
  \item We have the exact same term as before except with the interaction potential $V$
  \item These are typically some form of collision, so they occur at the scale of atomic size $\sim10^{-10}\unit{\m}$
  \item Time scale is then:
    \begin{align*}
      t_V\approx\frac{10^{-10}}{10^2}= 10^{-12}\unit{\s}
    \end{align*}
  \end{itemize}
\end{frame}
\begin{frame}{Other time scale for the Interaction Potential}
  \begin{itemize}
  \item The last term is a bit different, it is a volume integral
  \item How will it be non-zero?
  \item This scale should be related to the previous one we found
    \begin{align*}
      t_f\approx\alpha t_V
    \end{align*}
  \item $f_{s+1}/f_s$ should be a quantity per unit volume
  \item Number of particles per unit volume is $n=10^{26}\unit{\m}$
    \begin{align*}
      t_f\approx\frac{t_V}{nV}=\frac{1}{nvd^2}= 10^{-8}\unit{\s}
    \end{align*}
  \end{itemize}
\end{frame}

\section{Reduced Distributions}
\begin{frame}{Reduced Distribution Function}
  \begin{itemize}
  \item For the $m$ particle reduced distribution, find $m$ particles with specific position and momenta $\vb{q}$, $\vb{p}$
  \item Do this with an average:
    \begin{align*}
      f_1(\vb{p},\vb{q},t)=
      \ev{\sum_{i=1}^N\delta^3(\vb{p-p}_i)\delta^3(\vb{q-q}_i)}
    \end{align*}
  \item For $m=1$, you get $N$ equations, since you dont care which specific particle has momentum $p$ or position $q$, just that one does. 
  \item For $m$ particles, you will have $m$ sums with $2m$ delta functions, however you must sum over less and less terms each time. 
  \end{itemize}
\end{frame}
\begin{frame}{Liouville Equation}
  \begin{itemize}
  \item We use the typical Hamiltonian $H$
  \item The Liouville Equation states:
    \begin{align*}
      \pdv{\rho}{t}=-\{\rho,H\}_{PB}
    \end{align*}
  \item We want to consider the density $\rho_s$, but Liouville only talks about $\rho$, so we find:
    \begin{align*}
      \pdv{\rho_s}{t}=\int\prod_{i=s+1}^N\dd{V}_i\pdv{\rho}{t}=
      -\int\prod_{i=s+1}^N\dd{V}_i\{\rho,H\}_{PB}
    \end{align*}
  \item The Poisson Bracket has the following form:
    \begin{align*}
      \{H,\rho\}_{PB}=\sum_i\qty(
      \pdv{H}{q_i}\pdv{\rho}{p_i}-\pdv{H}{p_i}\pdv{\rho}{q_i}
      )
    \end{align*}
  \end{itemize}
\end{frame}
\begin{frame}{Liouville Equation}
  \begin{itemize}
  \item If we split the Hamiltonian into the form of:
    \begin{align*}
      H=H_s+H_{N-s}+H^{int}
    \end{align*}
  \item The sums in $H_s$ go from $i=1,s$ and $H_{N-s}$ covers the rest. 
  \item If we only consider this, we can rewrite the Poisson Bracket as:
    \begin{align*}
      \int\prod_{i=s+1}^N\dd{V}_i\{\rho,H_s\}_{PB}=\{\rho_s,H_s\}_{PB}
    \end{align*}
    Since the Hamiltonian has the same number of sum variables as the PB now
  \end{itemize}
\end{frame}
\begin{frame}{Continued}
  \begin{itemize}
  \item We must also now consider the PB with $H_{N-s}$, which has the following partials:
    \begin{gather*}
      \pdv{H_{N-s}}{\vb{q}_j}=\pdv{U}{\vb{q}_j}+
      \frac12\sum_{k=s+1}^N\pdv{V(\vb{q}_j-\vb{q}_k)}{\vb{q}_j}\\
      \pdv{H_{N-s}}{\vb{p}_j}=\frac{\vb{p}_j}{m}
    \end{gather*}
  \item From here you will find that the interaction hamiltonian term is proportional to $\rho_{s+1}$, such that:
    \begin{align*}
      \pdv{\rho_s}{t}-\{H_s,\rho_s\}_{PB}=(N-s)\sum_{n=1}^s\int\dd{V_{s+1}}
      \frac{V(\vb{q}_n-\vb{q}_{s+1})}{\vb{q}_n}\pdv{\rho_{s+1}}{\vb{p}_n}
    \end{align*}
  \end{itemize}
\end{frame}
\end{document}