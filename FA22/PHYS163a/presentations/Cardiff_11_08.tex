\documentclass{beamer}

\usepackage[italicdiff]{physics}
\usepackage{multicol}
\usepackage{hyperref}
\usepackage{url}

\newcommand{\dbar}{d\hspace*{-0.08em}\bar{}\hspace*{0.1em}}

\newenvironment{itemframe}[1]{\begin{frame}{#1}\begin{itemize}}   {\end{itemize}\end{frame}}

% Changes style of actual slides
\usetheme{Dresden}
% Changes color of slides
\usecolortheme{spruce}
% removes controls at bottom right side
\usenavigationsymbolstemplate{}

% for figures
\graphicspath{ {./figs/} }

\title{Introductory Quantum Stat Mech}
\author{Michael Cardiff}
\subtitle{PHYS 163a \\ 11/08 Prep Work}

\begin{document}
\begin{frame}
  \titlepage
\end{frame}

\section{Blackbody Radiation}
\begin{frame}{Blackbody Radiation? Huh?}
  \begin{columns}
    \begin{column}{0.5\textwidth}
      \begin{itemize}
      \item First, what is a blackbody?
        \begin{itemize}
        \item Absorbs all incident electromagnetic radiation
        \item Idealized
        \end{itemize}
      \item Radiation spectrum at equilibrium described only by temperature
      \item Classical mechanics describes it poorly, distribution diverges when $\lambda\to0$
      \item We can estimate the temp of an object based on its color. 
      \end{itemize}
    \end{column}
    \begin{column}{0.5\textwidth}
      \begin{figure}[H]
        \centering
        \includegraphics[width=5.0cm]{blackbody}
        \caption{Actual Blackbody Spectrum}
      \end{figure}
    \end{column}
  \end{columns}
\end{frame}

\begin{frame}{Why is it special}
  \begin{columns}
    \begin{column}{0.5\textwidth}
      \begin{itemize}
      \item All matter emits EM radiation
      \item However all matter does not absorb it in the same way
        \begin{itemize}
        \item Glass is see-through in visible spectrum, but acts like a mirror in the UV
        \end{itemize}
      \item Blackbodies absorb at all wavelengths
      \end{itemize}
    \end{column}
    \begin{column}{0.5\textwidth}
      \begin{figure}[H]
        \centering
        \includegraphics[width=5.0cm]{transmittance}
        \caption{Transmittance of Glass, NOT a Black Body}
      \end{figure}
    \end{column}
  \end{columns}
\end{frame}
\section{Rayleigh-Jean's Law}
\begin{frame}{Rayleigh-Jean's Solution}
  \begin{itemize}
  \item One attempt to describe the spectrum of blackbody radiation was done by Rayleigh and Jeans, which gives the following form:
    \begin{align*}
      B_\lambda(T)=\frac{2ck_BT}{\lambda^4}
    \end{align*}
  \item The actual answer is a limit of this, so it agrees at high wavelengths
  \item This law is the origin of the UV-catastrophe
  \end{itemize}
\end{frame}
\begin{frame}{Outline of Derivation}
  \begin{columns}
    \begin{column}{0.6\textwidth}
      \begin{itemize}
      \item Solve wave equation for standing EM waves in a cubical cavity
      \item Solution is an infinite number of normal modes:\small
        \begin{align*}
          E=\sin(n_xx)\sin(n_yy)\sin(n_zz)\sin(2\pi ct/\lambda)
        \end{align*}\normalsize
      \item Find the number of modes possible in the cavity
      \item Find modes per wavelength
      \item Relate modes per wavelength to energy per wavelength using equipartition theorem
      \end{itemize}
    \end{column}
    \begin{column}{0.4\textwidth}
      \begin{figure}[H]
        \centering
        \includegraphics[width=4.0cm]{modes}
        \caption{Counting Modes}
      \end{figure}
    \end{column}
  \end{columns}
\end{frame}
\section{UV Catastrophe}
\begin{frame}{UV Disaster!}
  \begin{columns}
    \begin{column}{0.5\textwidth}
      \begin{itemize}
      \item In the Rayleigh-Jeans Law, take $\lambda\to0$, the radiation blows up!
      \item Indication that this is not the correct law
      \item This is clearly not the behavior of something like the sun, which emits UV radiation similarly to a blackbody
      \end{itemize}
    \end{column}
    \begin{column}{0.5\textwidth}
      \begin{figure}[H]
        \centering
        \includegraphics[width=5.0cm]{catastrophe}
        \caption{Demonstration of UV Catastrophe from Rayleigh-Jeans}
      \end{figure}
    \end{column}
  \end{columns}
\end{frame}

\section{Stefan Boltzmann Law}
\begin{frame}{Stefan Boltzmann}
  \begin{itemize}
  \item Rayleigh-Jean's Law talks about a Power per unit area, per unit wavelength, aka Energy Flux per wavelength
  \item Stefan-Boltzmann Talks about Energy flux across all wavelengths, so the Law simply integrates out the wavelength 
  \item Expressed as:
    \begin{align*}
      \frac{P}{A}=\sigma T^4
    \end{align*}
  \item $\sigma$ is a fixed constant on the order of $5.0\times10^{-8}$ W/(m$^2$K$^4$)
  \end{itemize}
\end{frame}

\section{Quantum Harmonic Oscillator?}
\begin{frame}{What does this have to do with Harmonic Oscillators?}
  \begin{itemize}
  \item Radiation and Photons: Electromagnetic Waves
  \item Classical Potential for Waves: Harmonic Oscillator
  \item Photons and Blackbody radiation are quantum effects
    \begin{itemize}
    \item Photons: Quantized Excitations of EM Field: waves!
    \item Radiation: Waves resulting from accelerating charges
    \end{itemize}
  \item Waves, especially EM waves and radiation obey Harmonic Oscillator Equations
  \item This makes the Harmonic Oscillator Potential one of the most important ones in Physics
  \end{itemize}
\end{frame}
\end{document}