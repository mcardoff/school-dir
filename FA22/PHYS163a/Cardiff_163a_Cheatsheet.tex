\documentclass[8pt]{article}

% -------------------------------------------------------------------
% packages 
% -------------------------------------------------------------------

\usepackage{geometry}
\usepackage{multicol}
\usepackage{amsmath,amsthm,amsfonts,amssymb,physics,bm}
\usepackage{graphicx,float}
% margins & spacing 
\geometry{margin = 0.5in}
\pagestyle{empty}
\setlength{\parindent}{0pt} 
\setlength{\parskip}{0pt}

% multicol parameters
\setlength{\premulticols}{2pt}
\setlength{\postmulticols}{2pt}
\setlength{\multicolsep}{2pt}
\setlength{\columnsep}{20pt}

\begin{document}
\small
\begin{multicols*}{2}
  \begin{center}
    \normalsize{Axiomatic Thermodynamics}
  \end{center}
    \hrule~\\
  The laws of thermodynamics are:
  \begin{enumerate}
  \setcounter{enumi}{-1}
  \item Thermal equilibrium is transitive
  \item Two ways to put $E$ in a sys, $\dd{Q},\dd{W}$
  \item Cant 100\% turn $Q\to W$, $Q$ flows $T_H$ to $T_C$
  \item $S$ goes to 0 as $T$ goes to 0
  \end{enumerate}
  Extensive=External size, Intensive=Intrinsic.

  Thermodynamic Potentials:
  \begin{gather*}
    \footnotesize
    F=E-TS \qquad \dd{F}=-S\dd{T}+J\dd{x}\\
    G=E-TS-Jx \qquad \dd{G}=-S\dd{T}-x\dd{J}\\
    H=E-Jx \qquad \dd{H}=-S\dd{T}+J\dd{x}
  \end{gather*}
  Table 1.2:
  \begin{table}[H]
    \footnotesize
    \centering
    \begin{tabular}{|c|cc|}
      \hline&$\dd{Q}=0$ & const $T$ \\\hline
      $\dd{W}=0$ & $\dd{S}\geq0$ & $\dd{F}\leq0$\\
      const $J$ & $\dd{H}\leq0$ & $\dd{G}\leq0$ \\\hline
    \end{tabular}
  \end{table}

   Gibbs-Duhem, take $\partial$: $E(\lambda\text{Ex})=\lambda E(\text{Ex})$ and T1\\
   Maxwell relations: Take mixed $\partial$ of potentials\\
   Stability: $\dd{T}\dd{S}+\dd{J}\dd{x}+\dd{\mu}\dd{N}\geq 0$ or $C_V\geq0$
   \\\textbf{or}\\ second partial of potential is positive
  \begin{center}
    \normalsize{Probability}
  \end{center}
    \hrule~\\
  Characteristic function: $\tilde{p}(k)=\ev{e^{-ikx}}$ \\
  Cumulant generating function: $\ln\tilde{p}(k)$
  \begin{gather*}
    \tilde{p}(k)=\int\dd{x}p(x)e^{ikx}=\sum_{n}\frac{(-ik)^n}{n!}\ev{x^n}\\
    \ln\tilde{p}(k)=\sum_n\frac{(-ik)^n}{n!}\ev{x^n}_c
  \end{gather*}
  Central Limit Thm: Sum of many random vars $\sim$ normal distro
  Saddle point: Integral of exponential times $N\phi$ can be done by expanding about $x_{max}$:
  \begin{align*}
    \int\dd{x}\exp{N\phi(x)}\approx e^{N\phi(x_{m})}
    \sqrt{\frac{2\pi}{N\abs{\phi''(x_{m})}}}
  \end{align*}
  Information entropy:
  \begin{align*}
    S=-\sum_ip_i\ln p_i=-\int\dd{p}p\ln p
  \end{align*}
  To take a functional derivative, use
  \begin{gather*}
    \fdv{f(x)}{f(y)}=\delta(x-y)
  \end{gather*}
  To maximize entropy, Add $\mathcal{L}$ multipliers, take functional derivative with respect to $p$ and apply constraints
  \begin{center}
    \normalsize{Kinetic Theory}
  \end{center}
    \hrule~\\
  Liouville Thm: At diff times, $\dd{\vb{p}'}\dd{\vb{q}'}=\dd{\vb{p}}\dd{\vb{q}}$
  Liouville Eqn: bc of Liouville thm, $\dv{\rho}{t}=0$, get
  \begin{gather*}
    \pdv{\rho}{t}+\{\rho,H\}=0\\
    \{\rho_{eq},H\}=0
  \end{gather*}
  First Equation of BBGKY has longer relaxation time than all of the other ones, so when first eqn relaxes, all others have as well\\
  Coarse Graining in Time: limited range of $V_{int}$, Reduction to separation and center of mass, Center of mass deriv is small \\
  Coarse Graining in space: Change of vars to scattering notation, assumption of molecular chaos:
  \begin{gather*}
    f_2(\Gamma_1,\Gamma_2,t)=f_1(\Gamma_1,t)f_1(\Gamma_2,t)
  \end{gather*}
  only really unjustified assumption\\
  H-Theorem: $\dv{H(t)}{t}\leq0$:
    \begin{gather*}
      H(t)=\int\dd{\vb{p}}\dd{\vb{q}}f_1\ln f_1
    \end{gather*}
    Irreversible as we are treating $f_2(+)$ and $f_2(-)$ (b/f and after the col) differently, one IS the product of 2 $f_1$s, the other is not\\
  Necessary condition for $\dv{H}{t}=0$:
  \begin{gather*}
    f_1(\vb{p}_1,\vb{q})f_1(\vb{p}_2,\vb{q})=
    f_1(\vb{p}_1',\vb{q})f_1(\vb{p}_2',\vb{q})
  \end{gather*}
  Described by additive quantities conserved in collision, giving:
  \begin{align*}
    f_1(\vb{p},\vb{q})&=N(\vb{q})\exp{-\alpha(\vb{q})\vdot\vb{p}
      -\beta(\vb{q})\qty(p^2/2m+U(\vb{q}))}\\
    &=n\qty(\frac{\beta}{2\pi m})^{3/2}\exp[-\beta(\vb{p-p}_0)^2/2m]
  \end{align*}
  With $\vb{p}_0=m\bm{\alpha}/\beta$
  \begin{center}
    \normalsize{Classical Statistical Mechanics}
  \end{center}
    \hrule~\\
  Fundamental Postulate: equal a priori probabilities for $\mu$states for isolated systems
  \begin{gather*}
    p(\mu)=\frac1{\Omega(E)}
  \end{gather*}
  $\Omega(E)=$ \# of states with energy $E$\\
  Information Entropy definition gives $S=k_B\log\Omega$\\
  Canonical Ensemble: Heat sink in EQL with smaller system, get:
  \begin{align*}
    p(\mu)=Z^{-1}e^{-\beta H(\mu)}\quad
    Z=\sum_\mu e^{-\beta H(\mu)}\quad
    F=-k_BT\log Z
  \end{align*}
  To derive T0:
  \begin{itemize}
  \item Get expression for total $S$, saddle pt, etc
  \item Find $\pdv{S_i}{E_i}$ are equal, conclude it is temp
  \end{itemize}
  To derive T1:
  \begin{itemize}
  \item Expand differential of $S$
  \item Find form for $\pdv{S}{x}$, use T0 and rewrite 
  \end{itemize}
  To derive T2:
  \begin{itemize}
  \item The EQL point has greater $\Omega$
  \item From lower energies to higher, there is a loss of info $\implies\dd{S}\geq0$
  \end{itemize}
  \begin{center}
    \normalsize{Misc}
  \end{center}
  \hrule~\\
  \begin{itemize}
  \item Sterling Approx:
    \begin{gather*}
      \ln N!\approx N\ln N-N
    \end{gather*}
  \item Gaussian normalization:
  \begin{gather*}
    \int\dd{x}\exp{-\alpha(x-\beta)^2}=\sqrt{\frac{\pi}{\alpha}}
  \end{gather*}
  
  \end{itemize}
\end{multicols*}
\end{document}
