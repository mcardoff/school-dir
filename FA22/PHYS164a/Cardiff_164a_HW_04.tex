\documentclass[12pt]{article}

\title{\vspace{-3em}PHYS 164a HW 4}
\author{Michael Cardiff}
\date{\today}

%% science symbols 
\usepackage{amsmath}
\usepackage{amssymb}
\usepackage{physics}
\usepackage{slashed}

%% general pretty stuff
\usepackage{bm}
\usepackage{enumitem}
\usepackage{float}
\usepackage{graphicx}
\usepackage[margin=1in]{geometry}
\usepackage[labelfont=bf]{caption}

% figures
\graphicspath{ {./figs/} }

\newcommand{\fig}[3]
{
  \begin{figure}[H]
    \centering
    \includegraphics[width=#1cm]{#2}
    \caption{#3}
  \end{figure}
}

\newcommand{\figref}[4]
{
  \begin{figure}[H]
    \centering
    \includegraphics[width=#1cm]{#2}
    \caption{#3}
    \label{#4}
  \end{figure}
}

\renewcommand{\L}{\mathcal{L}}
\newcommand{\D}{\partial}
\newcommand{\munu}{{\mu\nu}}
\newcommand{\sla}[1]{\slashed{#1}}

\begin{document}
\maketitle
\section{Power Law with $G,c$}
We can parameterize the constant $k$ in terms of powers of $G,c$:
\begin{align*}
  k=G^\alpha c^\beta
\end{align*}
The dimensions of $k$ will be:
\begin{align*}
  [k]=L^{3\alpha+\beta}M^{-\alpha}T^{-2\alpha-\beta}
\end{align*}
If we use $\alpha=1$ and $\beta=-2$, we get:
\begin{align*}
  \eval{[k]}_{\alpha=1,\beta=-2}=L^{3-2}M^{-1}T^{2-2}=M^{-1}L
\end{align*}
Which looks pretty nice! To match our desired power law:
\begin{align*}
  \ell=k\mu^m
\end{align*}
We need dimensions of $L$ at the end, and since our $k$ has dimensions of length per mass we only need a single power of mass, hence our law is:
\begin{align}
  \boxed{\ell=Gc^{-2}\mu}
\end{align}
The magnitude of this quantity:
\begin{align*}
  \frac{G}{c^2}\approx 10^{-27}
\end{align*}\newpage
\section{Power Law with $\hbar,c$}
We can now parameterize the constant $k$ in terms of powers of $\hbar,c$:
\begin{align*}
  k=\hbar^\alpha c^\beta
\end{align*}
The dimensions of $k$ will be:
\begin{align*}
  [k]=M^{\alpha} L^{2\alpha+\beta} T^{-\alpha-\beta}
\end{align*}
If we use $\alpha=1$ and $\beta=-1$, we get:
\begin{align*}
  \eval{[k]}_{\alpha=1,\beta=-1}=M^{1} L^{2-1} T^{-1+1}=ML
\end{align*}
Which is eerily similar to what we had before, to match our power law again:
\begin{align*}
  \ell=k\mu^m
\end{align*}
We need dimensions of $L$ at the end, and since our $k$ has dimensions of length times mass, so we need $-1$ for our power of mass
\begin{align}
  \boxed{\ell=\hbar c^{-1}\mu^{-1}}
\end{align}
The magnitude of this value is:
\begin{align*}
  \frac{\hbar}{c}\approx 10^{-42}
\end{align*}
\end{document}