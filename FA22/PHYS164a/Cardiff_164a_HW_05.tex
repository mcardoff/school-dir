\documentclass[12pt]{article}

\title{\vspace{-3em}PHYS 164a HW 5}
\author{Michael Cardiff}
\date{\today}

%% science symbols 
\usepackage{amsmath}
\usepackage{amssymb}
\usepackage{physics}
\usepackage{slashed}
\usepackage{siunitx}

%% general pretty stuff
\usepackage{bm}
\usepackage{enumitem}
\usepackage{float}
\usepackage{graphicx}
\usepackage[margin=1in]{geometry}
\usepackage[labelfont=bf]{caption}

% figures
\graphicspath{ {./figs/} }

\newcommand{\fig}[3]
{
  \begin{figure}[H]
    \centering
    \includegraphics[width=#1cm]{#2}
    \caption{#3}
  \end{figure}
}

\newcommand{\figref}[4]
{
  \begin{figure}[H]
    \centering
    \includegraphics[width=#1cm]{#2}
    \caption{#3}
    \label{#4}
  \end{figure}
}

\renewcommand{\L}{\mathcal{L}}
\newcommand{\D}{\partial}
\newcommand{\munu}{{\mu\nu}}
\newcommand{\sla}[1]{\slashed{#1}}

\begin{document}
\maketitle

\section{Significance}
The formula for the compton length is:
\begin{equation}
  \lambda= \frac{\hbar}{mc}
\end{equation}
It is the wavelength of a photon whose energy is the same as the rest energy of that particle, so it is inherently a relativistic quantity.

It is not the wavelength of some wave, more of a hypothetical wavelength.
\section{Comparison to de Broglie}
The formula for the de Broglie wavelength is:
\begin{equation}
  \lambda=\frac{\hbar}{p}
\end{equation}
Which is not necessarily relativistic.

The de Broglie wavelength is an actual wavelength of a wave. It describes the wave nature of particles, and was shown experimental using electrons, specifically diffracting a large number of them, and there was an explicit wavelike diffraction pattern shown, demonstrating a wave nature.

Since the Compton length explicitly uses $c$ as a velocity, the difference in the Compton length and de Broglie wavelength actually depends on the system.
\section{Who was Compton?}
The Compton length is named after Arthur Compton, who showed the sort of converse statement that the de Broglie wavelength demonstrates, that electromagnetic radiation is inherently particle-like. He won the Nobel Prize in Physics for this effect, called the Compton effect. 
\section{Comparison}
We already have a table of 'physical' sizes and masses of objects, we should calculate the compton length (at least to order of magnitude:
\begin{table}[H]
  \centering
  \begin{tabular}{c|c|c|c}
    Object & Mass (kg) & Size (m) & Compton Length \\ \hline
    Photon           & \num{0}       & Point         & Undefined?    \\
    Electron         & \num{9.0e-31} & Point         & \num{2.4e-11} \\
    Proton           & \num{1.6e-27} & \num{8.4e-16} & \num{6.2e-16} \\
    Hydrogen Atom    & \num{1.7e-27} & \num{1.2e-10} & \num{5.9e-16} \\
    Carbon Atom      & \num{2.0e-26} & \num{1.7e-10} & \num{5.0e-17} \\
    Bacterium        & \num{1.0e-15} & \num{1.1e-6 } & \num{1.0e-27} \\
    Blood Cell       & \num{2.7e-14} & \num{3.8e-6 } & \num{3.7e-29} \\
    Baseball         & \num{1.4e-1}  & \num{2.3e-1 } & \num{7.1e-42} \\
    People           & \num{6.2e1  } & \num{1.6e1  } & \num{1.6e-44} \\
    Antartic Whale   & \num{1.8e5  } & \num{2.0e1  } & \num{5.6e-48} \\
    Moon             & \num{7.3e22 } & \num{1.7e6  } & \num{1.4e-65} \\
    Earth            & \num{5.9e24 } & \num{6.4e6  } & \num{1.7e-67} \\
    Sun              & \num{1.9e30 } & \num{7.0e8  } & \num{5.3e-73} \\
    Solar System     & \num{2.0e30 } & \num{1.4e14 } & \num{5.0e-73} \\
    Milky Way Galaxy & \num{2.9e42 } & \num{5.0e20 } & \num{2.9e-85}
  \end{tabular}
  \caption{Mass, Size, Compton Length}
\end{table}

\end{document}