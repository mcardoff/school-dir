\documentclass[12pt]{article}

\title{\vspace{-3em}PHYS 518 HW 3}
\author{Michael Cardiff}
\date{\today}

%% science symbols 
\usepackage{amsmath}
\usepackage{amssymb}
\usepackage{physics}

%% general pretty stuff
\usepackage{bm}
\usepackage{enumitem}
\usepackage{float}
\usepackage{graphicx}
\usepackage[margin=1in]{geometry}

% figures
\graphicspath{ {./figs/} }

\newcommand{\fig}[3]
{
  \begin{figure}[H]
    \centering
    \includegraphics[width=#1cm]{#2}
    \caption{#3}
  \end{figure}
}

\newcommand{\figref}[4]
{
  \begin{figure}[H]
    \centering
    \includegraphics[width=#1cm]{#2}
    \caption{#3}
    \label{#4}
  \end{figure}
}

\renewcommand{\L}{\mathcal{L}}
\newcommand{\D}{\partial}
\newcommand{\h}{\phi}
\newcommand{\s}{\psi}

\begin{document}
\maketitle

\section{Antisymmetric Tensor}
Expanding out everything:
\begin{align*}
  &\D_\lambda F_{\mu\nu}+\D_\nu F_{\lambda_\mu}+\D_\mu F_{\nu\lambda}\\
  &=\D_\lambda F_{\mu\nu}-\Gamma^{\alpha}_{\lambda\mu}F_{\alpha\nu}
  -\Gamma^{\alpha}_{\lambda\nu}F_{\mu\alpha}+\D_\nu F_{\lambda\mu}
  -\Gamma^{\alpha}_{\nu\alpha}F_{\alpha\mu}-\Gamma^\alpha_{\nu\mu}F_{\lambda\alpha}
  +\D_\mu F_{\nu\lambda}-\Gamma^\alpha_{\mu\nu}F_{\alpha\lambda}
  -\Gamma^{\alpha}_{\mu\lambda}F_{\nu\alpha}
\end{align*}
Consider the first and last terms with Christoffel symbols:
\begin{align*}
  \Gamma^\alpha_{\lambda\mu}F_{\alpha\nu}+\Gamma^{\alpha}_{\mu\lambda}F_{\nu\alpha}
  &=\Gamma^\alpha_{\lambda\mu}\qty(F_{\alpha\nu}+F_{\nu\alpha})\\
  &=\Gamma^\alpha_{\lambda\mu}\qty(F_{\alpha\nu}-F_{\alpha\nu})\\
  &=0
\end{align*}
This is due to the symmetry of the Christoffel symbol in the lower indices and the defined antisymmetry of the tensor $F$. Most of the terms cancel out this way, so we are left with:
\begin{align*}
  \D_\lambda F_{\mu\nu}-\Gamma^{\alpha}_{\lambda\mu}F_{\alpha\nu}
  -\Gamma^{\alpha}_{\lambda\nu}F_{\mu\alpha}&=\grad_\lambda{F_{\mu\nu}}\\
  \D_\nu F_{\lambda\mu}-\Gamma^{\alpha}_{\nu\lambda}F_{\alpha\mu}
  -\Gamma^{\alpha}_{\mu\nu}F_{\lambda\alpha}&=\grad_\nu{F_{\lambda\nu}}\\
  \D_\mu F_{\nu\lambda}-\Gamma^{\alpha}_{\mu\nu}F_{\alpha\lambda}
  -\Gamma^{\alpha}_{\mu\lambda}F_{\nu\alpha}&=\grad_\mu{F_{\nu\lambda}}
\end{align*}
Hence we end up with only covariant derivatives, so the sum is in fact a tensor:
\begin{align*}
  \D_\lambda F_{\mu\nu}+\D_\nu F_{\lambda_\mu}+\D_\mu F_{\nu\lambda}
  =\boxed{
    \grad_\lambda F_{\mu\nu}+
    \grad_\nu F_{\lambda_\mu}+
    \grad_\mu F_{\nu\lambda}}
\end{align*}
\section{Hyperbolic Surface}
The metric is:
\begin{align*}
  \dd{s}^2=a^2\qty(\dd{\psi}^2+\sinh^2\psi\dd{\phi}^2)
\end{align*}
Such that the metric tensor has the following components:
\begin{gather*}
  g_{\phi\phi}=a^2\sinh^2\psi\quad g_{\s\s}=a^2\\
  g^{\phi\phi}=\frac{1}{a^2\sinh^2\psi}\quad g^{\s\s}=\frac{1}{a^2}\\
  g_{\h\s}=g_{\s\h}=0=g^{\h\s}=g^{\s\h}
\end{gather*}
\subsection{Christoffel Symbol}
The only non-zero indices will be $\Gamma^\s_{\h\h}$ and $\Gamma^\h_{\h\s}$:
\begin{align*}
  \Gamma^\s_{\h\h}&=\frac{1}{2}g^{\s\sigma}=-\frac{1}{2}g^{\s\s}\D_\s g_{\h\h}\\
  &=\boxed{-a^2\sinh\s\cosh\s}\\
  \Gamma^\h_{\h\s}&=\frac{1}{2}g^{\h\sigma}\qty(\D_\h g_{\sigma\s}+
  \D_\s g_{\sigma\h}-\D_\sigma g_{\s\h})\\
  &=\frac{1}{2}g^{\h\h}\qty(\D_\h g_{\h\s}+\D_\s g_{\h\h})=\coth\s\\
  \Gamma^\h_{\h\s}=\Gamma^\h_{\s\h}&=\boxed{\coth\s}
\end{align*}
\subsection{Laplacian}
The Laplacian is given via the scale factors as:
\begin{align*}
  \laplacian = \frac{1}{h_\s h_\h}\D_\sigma\qty(\frac{\h_\s h+\h}{h^2}\D_\sigma)
\end{align*}
Going from here:
\begin{align*}
  \laplacian&=\frac{1}{a^4\sinh^2\s}\qty(\D_\h\qty(\frac{1}{\sinh^2\psi}\D_h)
  +\D_\s\qty(\sinh^2\s)\D_\s)\\
  &=\boxed{\frac{1}{a^4\sinh^2\psi}\qty(\frac{1}{\sinh^2\psi}\D^2_\phi
    +\D_\psi\qty(\sinh^2\psi\D_\psi))}
\end{align*}
\subsection{Riemann Tensor}
The indices of the Riemann Tensor are:
\begin{align*}
  R_{abcd}=\frac{1}{2}(\D_d\D_ag_{bc}-\D_d\D_bg_{ac}+\D_c\D_bg_{ad}-\D_c\D_ag_{bd})
  -g_{ef}\qty(\Gamma^e_{ac}\Gamma^f_{bd}-\Gamma^e_{ad}\Gamma^f_{bc})
\end{align*}
The number of independent components for $N=2$:
\begin{align*}
  \frac{N^2(N^2-1)}{12}=\frac{4(4-1)}{12}=1
\end{align*}
Choose $R_{\s\h\s\h}$ to be the unique component:
\begin{align*}
  R_{\s\h\s\h}&=\frac{1}{2}\qty(\D_\h\D_\s g_{\s\h}-\D_\h\D_\h g_{\h\s}
  +\D_\h\D_\h g_{\h\s}-\D_\s\D_\h g_{\h\h})-g_{\mu\sigma}
  \qty(\Gamma^\lambda_{\s\s}\Gamma^\sigma_{\h\h}-
  \Gamma^\lambda_{\s\h}\Gamma^\sigma_{\s\h})\\
  &=-\frac{a^2}{2}
  \D_\s\qty(\sinh\s\cosh\s)+a^2\sinh^2\s\frac{\cosh^2\s}{\sinh^2\s}\\
  &=\boxed{-\frac{a^2}{2}\sinh^2\s}
\end{align*}
\subsection{Ricci Tensor}
We need to use:
\begin{align*}
  R_{\nu\nu}=R^\lambda_{\mu\lambda\nu}
\end{align*}
This means:
\begin{align*}
  R^\psi_{\h\s\h}=R_{\h\h}=\frac{1}{a^2}R_{\s\h\s\h}=\boxed{-\sinh^2\s}
\end{align*}
\subsection{Ricci Scalar}
Once again we need:
\begin{align*}
  R^\mu_\mu=g^{\mu\nu}R_{\mu\nu}
\end{align*}
Hence:
\begin{align*}
  R_{\h\h}=g_{\h\h}R^{\h}_\h=\implies \boxed{R=-\frac{1}{a^2}}
\end{align*}
The formula we need to confirm is:
\begin{align*}
  R_{\s\h\s\h}&=-\frac{1}{2}(g_{\s\s}g_{\h\h}-g_{\s\h}g_{\s\h})R\\
  &=-\frac{a^2}{2}\sinh^2\s
\end{align*}
\section{Conformally Flat Metric}
The metric is given by the following:
\begin{align*}
  \dd{s}^2=\Omega^2(x,y)\qty(\dd{x}^2+\dd{y}^2)
\end{align*}
So we have the following for our metric tensor:
\begin{gather*}
  g_{xx}=g_{yy}=\Omega^2\qquad g^{xx}=g^{yy}=\Omega^{-2}\\
  g_{xy}=g_{yx}=g^{xy}=g^{yx}=0
\end{gather*}
\subsection{Christoffel Symbol}
The formula for the Christoffel symbol is:
\begin{align*}
  \Gamma^\lambda_{\mu\nu}=\frac{1}{2}g^{\lambda\sigma}
  \qty(\D_\mu g_{\sigma\nu}+\D_\nu g_{\mu\sigma}-\D_\sigma g_{\mu\nu})
\end{align*}
And we can plug and chug to continue:
\begin{align*}
  \Gamma^x_{xx}&=\frac{1}{2}g^{x\sigma}
  \qty(\D_xg_{\sigma x}+\D_x g_{\sigma x}-\D_\sigma g_{xx})\\
  &=\frac{g^{xx}}{2}\qty(\D_x g_{xx}+\D_x g_{xx}-\D_xg_{xx})\\
  &=\boxed{\frac{1}{2}\frac{\D_x\Omega^2}{\Omega^2}}
\end{align*}
\begin{align*}
  \Gamma^y_{xx}&=\frac{1}{2}g^{y\sigma}
  \qty(\D_xg_{\sigma x}+\D_x g_{\sigma x}-\D_\sigma g_{xx})\\
  &=\frac{g^{yy}}{2}\qty(\D_x g_{yx}+\D_x g_{yx}-\D_yg_{xx})\\
  &=\boxed{-\frac{1}{2}\frac{\D_y\Omega^2}{\Omega^2}}
\end{align*}
\begin{align*}
  \Gamma^y_{yy}&=\frac{1}{2}g^{y\sigma}
  \qty(\D_yg_{\sigma y}+\D_y g_{\sigma y}-\D_\sigma g_{yy})\\
  &=\frac{g^{yy}}{2}\qty(\D_y g_{yy}+\D_y g_{yy}-\D_yg_{yy})\\
  &=\boxed{\frac{1}{2}\frac{\D_y\Omega^2}{\Omega^2}}
\end{align*}
\begin{align*}
  \Gamma^x_{yy}&=\frac{1}{2}g^{x\sigma}
  \qty(\D_yg_{\sigma y}+\D_y g_{\sigma y}-\D_\sigma g_{yy})\\
  &=\frac{g^{xy}}{2}\qty(\D_y g_{yy}+\D_y g_{yy}-\D_yg_{yy})\\
  &=\boxed{-\frac{1}{2}\frac{\D_x\Omega^2}{\Omega^2}}
\end{align*}
\begin{align*}
  \Gamma^x_{xy}&=\frac{1}{2}g^{x\sigma}
  \qty(\D_xg_{\sigma y}+\D_y g_{\sigma x}-\D_\sigma g_{yx})\\
  &=\boxed{-\frac{1}{2}\frac{\D_y\Omega^2}{\Omega^2}}
\end{align*}
\begin{align*}
  \Gamma^y_{xy}&=\frac{1}{2}g^{y\sigma}
  \qty(\D_xg_{\sigma y}+\D_y g_{\sigma x}-\D_\sigma g_{yx})\\
  &=\boxed{-\frac{1}{2}\frac{\D_x\Omega^2}{\Omega^2}}
\end{align*}
We can simplify some of these derivatives since it is $\Omega^2$ not just $\Omega$ so the chain rule can be invoked:
\begin{equation*}
  \boxed{
    \begin{aligned}
      \Gamma^{x}_{xx}=\frac{\D_x\Omega}{\Omega}\quad
      \Gamma^{y}_{xx}=-\frac{\D_y\Omega}{\Omega}\quad
      \Gamma^{y}_{yy}=\frac{\D_y\Omega}{\Omega}\\
      \Gamma^{x}_{yy}=-\frac{\D_x\Omega}{\Omega}\quad
      \Gamma^{x}_{xy}=\frac{\D_y\Omega}{\Omega}\quad
      \Gamma^{y}_{xy}=\frac{\D_y\Omega}{\Omega}
    \end{aligned}
  }
\end{equation*}
\subsection{Riemann Tensor}
The indices of the Riemann Tensor are:
\begin{align*}
  R_{abcd}=\frac{1}{2}(\D_d\D_ag_{bc}-\D_d\D_bg_{ac}+\D_c\D_bg_{ad}-\D_c\D_ag_{bd})
  -g_{ef}\qty(\Gamma^e_{ac}\Gamma^f_{bd}-\Gamma^e_{ad}\Gamma^f_{bc})
\end{align*}
Due to the dimension $N=2$, there is one independent component of the Riemann tensor:
\begin{align*}
  R_{xyxy}&=\frac{1}{2}
  \qty(\D_y\D_xg_{yx}-\D_y\D_yg_{xx}+\D_x\D_yg_{xy}-\D_x\D_xg_{yy})
  -g_{ef}\qty(\Gamma^e_{xx}\Gamma^f_{yy}-\Gamma^e_{xy}\Gamma^f_{yx})\\
  &=-\frac{1}{2}\qty(\D^2_y\Omega^2+\D^2_x\Omega^2)
  -g_{xx}\qty(\Gamma^x_{xx}\Gamma^x_{yy}-\Gamma^x_{xy}\Gamma^x_{xy})
  -g_{yy}\qty(\Gamma^y_{xx}\Gamma^y_{yy}-\Gamma^y_{xy}\Gamma^y_{xy})
\end{align*}
We can reduce this going term by term:
\begin{align*}
  -\frac{1}{2}\qty(\D^2_y\Omega^2+\D^2_x\Omega^2)&=
  -\qty(\D_y\qty(\Omega\D_y\Omega)+\D_x(\Omega\D_x\Omega))\\
  &=-\Omega\qty(\D_x^2\Omega+\D_y^2\Omega)
\end{align*}
\begin{align*}
  -g_{xx}\qty(\Gamma^x_{xx}\Gamma^x_{yy}-\Gamma^x_{xy}\Gamma^x_{xy})&=
  -\Omega^2\qty(-\frac{\D_x\Omega}{\Omega}\frac{\D_x\Omega}{\Omega}
  -\frac{\D_y\Omega}{\Omega}\frac{\D_y\Omega}{\Omega})\\
  &=(\D_x\Omega)^2+(\D_y\Omega)^2
\end{align*}
\begin{align*}
  -g_{yy}\qty(\Gamma^y_{xx}\Gamma^y_{yy}-\Gamma^y_{xy}\Gamma^y_{xy})&=
  \Omega^2\qty(\frac{\D_y\Omega}{\Omega}\frac{\D_y\Omega}{\Omega}
  +\frac{\D_x\Omega}{\Omega}\frac{\D_x\Omega}{\Omega})\\
  &=(\D_y\Omega)^2+(\D_x\Omega)^2
\end{align*}
Combining these gives:
\begin{align*}
  \boxed{R_{xyxy}=(\D_x\Omega)^2+(\D_y\Omega)^2
    -\Omega(\D_x^2\Omega+\D_y^2\Omega)}
\end{align*}
\subsection{Ricci Tensor}
The Ricci tensor will be given by:
\begin{align*}
  R_{\mu\nu}=R^\lambda_{\mu\lambda\nu}
\end{align*}
In our case:
\begin{align*}
  R_{xyxy}=g_{xx}R^x_{yxy}
\end{align*}
The one unique component will be $R_{yy}$:
\begin{align*}
  R_{yy}=\boxed{\frac{(\D_x\Omega)^2+(\D_y\Omega)^2}{\Omega^2}
    -\frac{\D_x^2\Omega+\D_y^2\Omega}{\Omega}}
\end{align*}
\subsubsection{Ricci Scalar}
Finally the Ricci scalar:
\begin{align*}
  R=R^\mu_\mu
\end{align*}
So in our case we just need to multiply by $g_{yy}$
\begin{align*}
  R=\boxed{\frac{(\D_x\Omega)^2+(\D_y\Omega)^2}{\Omega^4}
    -\frac{\D_x^2\Omega+\D_y^2\Omega}{\Omega^3}}
\end{align*}
\section{Variation of the Christoffel Symbol}
We start with the definition of the symbol as derivatives of the metric:
\begin{align*}
  \Gamma^\lambda_{\mu\nu}=\frac{1}{2}g^{\lambda\rho}
  \qty(\D_\mu g_{\rho\nu}+\D_\nu g_{\rho\mu}-\D_\rho g_{\mu\nu})
\end{align*}
Variations of $\Gamma$ yield:
\begin{align*}
  \delta\Gamma^\lambda_{\mu\nu}&=\frac{1}{2}\delta g^{\lambda\rho}
  \qty(\D_\mu g_{\rho\nu}+\D_\nu g_{\rho\mu}-\D_\rho g_{\mu\nu})+
  \frac{1}{2}g^{\lambda\rho} \qty(\D_\mu\delta g_{\rho\nu}+
  \D_\nu\delta g_{\rho\mu}-\D_\rho\delta g_{\mu\nu})\\
  &=-\frac{1}{2}g^{\lambda\rho}\delta g^{\rho\sigma}
  \qty(\D_\mu g_{\rho\nu}+\D_\nu g_{\rho\mu}-\D_\rho g_{\mu\nu})+
  \frac{1}{2}g^{\lambda\rho} \qty(\D_\mu\delta g_{\rho\nu}+
  \D_\nu\delta g_{\rho\mu}-\D_\rho\delta g_{\mu\nu})\\
  &=-g^{\lambda\rho}(\delta g_{\rho\sigma})\Gamma^\sigma_{\mu\nu}
  +\frac{1}{2}g^{\lambda\rho} \qty(\D_\mu\delta g_{\rho\nu}+
  \D_\nu\delta g_{\rho\mu}-\D_\rho\delta g_{\mu\nu})\\
  &=\frac{1}{2}g^{\lambda\rho}\qty(\D_\mu\delta g_{\rho\nu}+
  \D_\nu\delta g_{\rho\mu}-\D_\rho\delta g_{\mu\nu}
  -2\delta_{\rho\sigma}\Gamma^{\sigma}_{\mu\nu})
\end{align*}
By adding and subtracting pairs of Christoffel symbols, we can make all of these into the covariant derivatives $\grad_\mu$:
\begin{align*}
  \boxed{\delta\Gamma^\lambda_{\mu\nu}=\frac{1}{2}g^{\lambda\rho}\qty(
   \grad_\mu\delta g_{\rho\mu}
  +\grad_\nu\delta g_{\rho\mu}
  -\grad_\rho\delta g_{\mu\nu})}
\end{align*}
\end{document}