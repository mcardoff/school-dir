\documentclass[12pt]{article}

\title{\vspace{-3em}PHYS 437 HW 7}
\author{Michael Cardiff}
\date{\today}

%% science symbols 
\usepackage{amsmath}
\usepackage{amssymb}
\usepackage{physics}

%% general pretty stuff
\usepackage{bm}
\usepackage{enumitem}
\usepackage{float}
\usepackage{graphicx}
\usepackage[margin=1in]{geometry}

% figures
\graphicspath{ {./figs/} }

\newcommand{\fig}[3]
{
  \begin{figure}[H]
    \centering
    \includegraphics[width=#1cm]{#2}
    \caption{#3}
  \end{figure}
}

\newcommand{\figref}[4]
{
  \begin{figure}[H]
    \centering
    \includegraphics[width=#1cm]{#2}
    \caption{#3}
    \label{#4}
  \end{figure}
}

\renewcommand{\L}{\mathcal{L}}

\begin{document}
\maketitle
\section*{Question 1}
The effective mass is given in terms of the dispersion relation:
\begin{align*}
  m^*=\frac{\hbar^2}{\dv[2]{E}{k}}
\end{align*}
The free particle dispersion relation gives:
\begin{align*}
  E&=\frac{\hbar^2k^2}{2m}\\
  \dv[2]{E}{k}&=\frac{\hbar^2}{m}
\end{align*}
So for a free particle:
\begin{align*}
  \boxed{m^*=m}
\end{align*}
At the $L$-point, we get:
\begin{align*}
  m_L=\frac{\hbar^2k_L^2}{2E_L}=\frac{\hbar^2}{8}\qty(\frac{3\pi^2}{a^2})
\end{align*}
Where we had $k_L\propto\qty(\frac{1}{2},\frac{1}{2},\frac{1}{2})$ and $E_L=4$.

Now we need to find the heavy and light hole masses at $\Gamma$, where we solve for the mass isolating it at a point $\Gamma$:
\begin{align*}
  E_{hh}&\approx\frac{\hbar^2k^2}{2m_{hh}}\\
  \implies m_{hh}&=\frac{\hbar^2\Gamma^2}{0.18}\\
  E_{lh}&\approx\frac{\hbar^2k^2}{2m_{lh}}\\
  \implies m_{lh}&=\frac{\hbar^2\Gamma^2}{0.58}
\end{align*}

\section{Question 2}
The fermi energy is half of the gap energy $E_g$, which can be described by:
\begin{align*}
  E_i=\frac{E_g}{2}=\frac{E_v-E_c}{2}
\end{align*}
For electron densities $n$ and hole densities $p$:
\begin{align*}
  n&=2\qty(\frac{m_ek_bT}{2\pi\hbar^2})^{3/2}e^{(\mu-E_c)/k_bT}\\
  p&=2\qty(\frac{m_ek_bT}{2\pi\hbar^2})^{3/2}e^{(E_v-\mu)/k_bT}
\end{align*}
In this case, we have $m_e=m=m_h$, and our $\mu=E_F$:
\begin{align*}
  n&=2\qty(\frac{mk_bT}{2\pi\hbar^2})^{3/2}e^{(E_F-E_c)/k_bT}\\
  p&=2\qty(\frac{mk_bT}{2\pi\hbar^2})^{3/2}e^{(E_v-E_F)/k_bT}  
\end{align*}
Solving for the Energy differences:
\begin{align*}
  E_F-E_c&=k_bT\ln(\frac{n}{2}\qty(\frac{2\pi\hbar^2}{mk_bT})^{3/2})\\
  E_v-E_F&=k_bT\ln(\frac{p}{2}\qty(\frac{2\pi\hbar^2}{mk_bT})^{3/2})
\end{align*}
Let the constant in the argument of the logs be $C$:
\begin{align*}
  E_F-E_c&=k_bT\ln(nC)\\
  E_v-E_F&=k_bT\ln(pC)  
\end{align*}
Finding the differences gives:
\begin{align*}
  \boxed{E_F-E_i=k_bT\ln(\frac{n}{p})}
\end{align*}
I am not sure how to do this problem and I am already way past the deadline

\section{Question 3}
Using Ampere's law, we get the following expression:
\begin{align*}
  \oint\vb{B}\vdot\dd{\bm{\ell}}=\mu_0NI
\end{align*}
We are only considering a single loop, so $N=1$, choosing the correct loop at a distance $r$ out makes this integral fairly trivial:
\begin{align*}
  \oint\vb{B}\vdot\dd{\bm{\ell}}=B\oint\dd{\ell}=B(2\pi r)
\end{align*}
Solving for $B$:
\begin{align*}
  B=\frac{\mu_0}{2\pi}\frac{I}{r}\implies I=B\frac{2\pi r}{\mu_0}
\end{align*}
Plugging in the wire radius as well as the critical field, we can solve for $I_{max}$:
\begin{align*}
  I_{max}=\frac{2\pi r_w B_C}{\mu_0}=\boxed{125 A}
\end{align*}
\end{document}