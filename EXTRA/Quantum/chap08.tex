% -*- TeX-master: "master.tex" -*-
\section{The Variational Principle}
The variational principle states that:
\begin{align*}
  E_{gs}\leq\ev{H}=\mel{\psi}{H}{\psi}
\end{align*}
Where $\psi$ is any normalized function.
The process for the variational method is very simple, and follows these steps:
\begin{enumerate}
\item Find a wavefunction that matches the potential with a parameter that can be varied. For example it could be the Gaussian:
  \begin{align*}
    \psi = A\exp{-bx^2}
  \end{align*}
Which matches very well with potentials that do not have barriers. 
\item Normalize the function, it will usually be in terms of the parameter that you built in.
\item Find the expectation value of the Hamiltonian, which can be written in terms of the kinetic and potential energy:
  \begin{align*}
    \ev{H}&=\ev{T}+\ev{V}\\
    \ev{T}&=\ev{-\frac{\hbar^2}{2m}\laplacian}\\
    \ev{V}&=\ev{V}
  \end{align*}
  Where $V$ is the potential for the specific problem.
\item Minimize the Hamiltonian with respect to the parameter:
  \begin{align*}
    \ev{H}_{min}=E_{gs}\implies \pdv{\ev{H}}{b}=0
  \end{align*}
  If you have multiple parameters you can minimize with respect to all of them
\item The previous result should give a value for the parameter, plug that into the Hamiltonian, and you should get a result without any of the parameters. This results in the estimation for the ground state. 
\end{enumerate}