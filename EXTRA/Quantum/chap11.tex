% -*- TeX-master: "master.tex" -*-
\section{Quantum Dynamics}
This is when we talk about time dependent Hamiltonians

We begin with systems that have two eigenstates of an unperturbed Hamiltonian. They obey the time independent Schrodinger equation:
\begin{align*}
  H^0\psi_a=E_a\psi_a\qquad H^0\psi_b=E_b\psi_b
\end{align*}
They are orthonormal:
\begin{align*}
  \ip{\psi_i}{\psi_j}=\delta_{ij}
\end{align*}
The initial state can be written:
\begin{align*}
  \Psi(0)=c_a\psi_a+c_b\psi_b
\end{align*}
In the absence of perturbation, the full time dependent solution is:
\begin{align*}
  \Psi(t)=c_a\psi_a\exp{-iE_at/\hbar}+c_b\psi_b\exp{-iE_bt/\hbar}
\end{align*}
Normalization requires that:
\begin{align*}
  \abs{c_a}^2+\abs{c_b}^2=1
\end{align*}
Now if we add a time dependent perturbation $H'(t)$, we can still write the solution in terms of the eigenstates since they form a complete set, but the constants need to be functions of time.
\begin{align*}
  \Psi(t)=c_a(t)\psi_a\exp{-iE_at/\hbar}+c_b(t)\psi_b\exp{-iE_bt/\hbar}
\end{align*}
We can solve for the coefficients by plugging this solution into the time dependent Schrodinger equation. We get a system of first order ODEs:
\begin{align*}
  \dot{c}_a&=-\frac{i}{\hbar}\qty[c_aH'_{aa}+c_bH'_{ab}\exp{-i(E_b-E_a)t/\hbar}]\\
  \dot{c}_b&=-\frac{i}{\hbar}\qty[c_bH'_{bb}+c_aH'_{ba}\exp{-i(E_a-E_b)t/\hbar}]\\
\end{align*}
It is fair to say that the $H'$ diagonal matrix elements will be $0$, so the equations will simplify to:
\begin{align*}
  \dot{c}_a=-\frac{i}{\hbar}H'_{ab}\exp{-i\omega_0t}c_b\qquad
  \dot{c}_b=-\frac{i}{\hbar}H'_{ba}\exp{i\omega_0t}c_a
\end{align*}
Where the characteristic frequency $\omega_0$ is given by:
\begin{align*}
  \omega_0=\frac{E_b-E_a}{\hbar}
\end{align*}
\subsection{Time-Dependent Perturbation Theory}
Consider a two level system which starts in the lower state (here we make it $a$):
\begin{align*}
  c_a(0)=1\qquad c_b(0)=0
\end{align*}
To zeroth order, (where there is no perturbation) these solutions do not change at all hence we end up with:
\begin{align*}
  c_a^{(0)}(t)=1\qquad c_b^{(0)}(t)=0
\end{align*}
To get the first order approximation, we sub the 0 order values on the right side of the equations:
\begin{align*}
  \dot{c}_a^{(1)}=0&\implies c_a^{(1)}(t)=1\\
  \dot{c}_b^{(1)}=-\frac{i}{\hbar}H'_{ba}\exp{-\omega_0t}&\implies
  c_b^{(1)}(t)=-\frac{i}{\hbar}\int_0^tH'_{ba}(t')\exp{i\omega_0t'}\dd{t'}
\end{align*}
Inserting these into the systems again gives the second order term:
\begin{align*}
  c_a^{(2)}(t)&=1-\frac{1}{\hbar^2}\int_0^tH'_{ab}(t')\exp{-i\omega_0t'}
  \qty[\int_0^{t'}H'_{ba}(t'')\exp{i\omega_0t''}\dd{t''}]\dd{t'}\\
  c_b^{(2)}(t)=c_b^{(1)}(t)
\end{align*}
This continues forever. The first order term is ovbiously not properly normalized.
\subsection{Sinusoidal Perturbations}
Consider perturbations of the following form
\begin{align*}
  H'(\vb{r},t)=V\vb{r}\cos\omega t
\end{align*}
The matrix elements are:
\begin{align*}
  H'_{ab}=V_{ab}\cos\omega t
\end{align*}
Assuming that the diagonal elements vanish, we get:
\begin{align*}
  c_b(t)\approx-\frac{V_{ba}}{2\hbar}\qty
  [\frac{\exp{i(\omega_0+\omega)t}-1}{\omega_0+\omega}+
    \frac{\exp{i(\omega_0-\omega)t}}{\omega_0-\omega}]
\end{align*}
Now we want to assume that:
\begin{align*}
  \omega_0+\omega>>\abs{\omega_0-\omega}
\end{align*}
Such that we get:
\begin{align*}
  c_b(t)\approx-i\frac{V_{ba}}{\hbar}\frac{\sin(\omega_0-\omega)t/2}{\omega_0-\omega}
  \exp{i(\omega_0-\omega)t/2}
\end{align*}
The probability of a transistion is the amplitude of this coefficient:
\begin{align*}
  P_{a\to b}(t)\approx\frac{\abs{V}_{ab}^2}{\hbar^2}\frac{\sin^2(\omega_0-\omega)t/2}
  {(\omega_0-\omega)^2}
\end{align*}
So the probability flops around
\subsection{Emission and Absorption}
Consider a particle exposed to a sinuisoidally varying electric field $\vb{E}$:
\begin{align*}
  \vb{E}=E_0\cos\omega t\vu{z}
\end{align*}
The corresponding perturbation is:
\begin{align*}
  H'=-qE_0z\cos\omega t
\end{align*}
The matrix element is:
\begin{align*}
  H'_{ba}&=-\mathcal{P}E_0\cos\omega t\\
  \mathcal{P}&=q\mel{\psi_b}{z}{\psi_a}
\end{align*}
Matching this with sinuisoidal perturbation from last section:
\begin{align*}
  V_{ba}=-\mathcal{P}E_0
\end{align*}
The transistion probability is:
\begin{align*}
  P_{a\to b}(t)=\qty(\frac{\abs{\mathcal{P}}E_0}{\hbar})^2
  \frac{\sin^2(\omega_0-\omega)t/2}{(\omega_0-\omega)^2}
\end{align*}
The probability of a transition down, called stimulated emission is:
\begin{align*}
  P_{b\to a}(t)=\qty(\frac{\abs{\mathcal{P}}E_0}{\hbar})^2
  \frac{\sin^2(\omega_0-\omega)t/2}{(\omega_0-\omega)^2}
\end{align*}
\subsection{Incoherent Perturbations}
The energy density in an electromagnetic wave is:
\begin{align*}
  u=\frac{veps_0}{2}E_0^2
\end{align*}
We can rewrite the stimulated emission probability in terms of this as:
\begin{align*}
  P_{b\to a}(t)=\frac{2u}{\veps_0\hbar^2}\abs{\mathcal{P}}^2
  \frac{\sin^2(\omega_0-\omega)t/2}{(\omega_0-\omega)^2}
\end{align*}
For a wave with a range of frequencies we use a density:
\begin{align*}
  P_{b\to a}(t)=\frac{2}{\veps_0\hbar^2}\abs{\mathcal{P}}^2\int_0^\infty\rho(\omega)
  \qty[\frac{\sin^2(\omega_0-\omega)t/2}{(\omega_0-\omega)^2}]\dd{\omega}
\end{align*}
Most of the time the density is peaked around $\omega_0$, hence we can write it as:
\begin{align*}
  P_{b\to a}(t)\approx\frac{2}{\veps_0\hbar^2}\abs{\mathcal{P}}^2\rho(\omega_0)\int_0^\infty
  \frac{\sin^2(\omega_0-\omega)t/2}{(\omega_0-\omega)^2}\dd{\omega}
\end{align*}
The integral is fairly simple to evaluate:
\begin{align*}
  P_{b\to a}(t)\approx\frac{\pi}{\veps_0\hbar^2}\abs{\mathcal{P}}^2\rho(\omega_0)t
\end{align*}
The tranisition rate, which is the time derivative of the probability, is:
\begin{align*}
  R_{b\to a}(t)\approx\frac{\pi}{\veps_0\hbar^2}\abs{\mathcal{P}}^2\rho(\omega_0)
\end{align*}
If we have a generalized polarization we just replace the square by one third of the average:
\begin{align*}
  R_{b\to a}(t)\approx\frac{\pi}{3\veps_0\hbar^2}\abs{\vb{\mathcal{P}}}^2\rho(\omega_0)
\end{align*}
The vector polarization is:
\begin{align*}
  \vb{\mathcal{P}}=q\mel{\psi_b}{\vb{r}}{\psi_a}
\end{align*}
\subsection{Einstein's A and B Coefficients}
Imagine a system which houses $N$ particles, $N_a$ of them in the lower state $\psi_a$, and $N_b$ of them in the higher $\psi_b$ state. Let $A$ be the spontaneous emission rate, then the number of particles leaving the upper state per unit time is $N_bA$. The transition rate is proportional to the energy density of the electromagnetic field some constant $B_{ba}\rho(\omega_0)$, we can presume that the absorption rate is similarly proportional to the energy density. The total rate is:
\begin{align*}
  \dv{N_b}{t}=-N_bA-N_bB_{ba}\rho(\omega_0)+N_aB_{ab}\rho(\omega_0)
\end{align*}
If the system is in thermal equilibrium, then the energy density is given by:
\begin{align*}
  \rho(\omega_0)=\frac{A}{(N_a/N_b)B_{ab}-B_{ba}}
\end{align*}
From statistical mechanics, the factors of $N$ are proportional to the boltzmann factor:
\begin{align*}
  \rho(\omega_0)=\frac{A}{\exp{\hbar\omega_0/k_BT}B_{ab}-B_{ba}}
\end{align*}
Comparing to Planck's blackbody formula:
\begin{align*}
  \rho(\omega)=\frac{\hbar}{\pi^2c^3}\frac{\omega^3}{\exp{\hbar\omega/k_BT}-1}
\end{align*}
We get the following conditions for the coefficients:
\begin{align*}
  B_{ab}&=B_{ba}\\
  A&=\frac{\omega_0^3\hbar}{\pi^2c^3}B_{ba}
\end{align*}
Subbing in what we know about $B_{ba}$:
\begin{align*}
  B_{ba}+\frac{\pi}{3\veps_0\hbar^2}\abs{\vb{\mathcal{P}}}^2
\end{align*}
Giving $A$ as:
\begin{align*}
  A=\frac{\omega_0^3\abs{\vb{\mathcal{P}}}^2}{3\pi\veps_0\hbar c^3}
\end{align*}
\subsection{Lifetime of an Excited State}
If we have a large amount of particles in the excited state, we have the following relation:
\begin{align*}
  \dd{N_b}=-AN_b\dd{t}
\end{align*}
Which has solution:
\begin{align*}
  N_b(t)+N_b(0)\exp{-At}
\end{align*}
The time constant of this is:
\begin{align*}
  \tau=\frac{1}{A}
\end{align*}
Since it is decaying exponentially.
\subsection{Selection Rules}
When we calculate the spontaneous emission rates, we need to find the following matrix elements:
\begin{align*}
  \mel{\psi_a}{\vb{r}}{\psi_a}
\end{align*}
Consider a hydrogen-like system, with quantum numbers $n,\ell,m$, the matrix elements to be computed would be:
\begin{align*}
  \mel{n'\,\ell'\,m'}{\vb{r}}{n\,\ell\,m}
\end{align*}
From chapter 6, we obtained the following selection rules:
\begin{align*}
  \Delta{\ell}\equiv\ell'-\ell=\pm1\qquad
  \Delta{m}\equiv m'-m=0,\pm1
\end{align*}
We can make the following simplifications as well:
\begin{align*}
  \mqty{
    m'=m & \implies &
    \mel{n'\,\ell'\,m'}{x}{n\,\ell\,m}=\mel{n'\,\ell'\,m'}{0}{n\,\ell\,m}=0\\
    m'=m\pm1 & \implies &
    \mel{n'\,\ell'\,m'}{x}{n\,\ell\,m}=\pm i\mel{n'\,\ell'\,m'}{0}{n\,\ell\,m}\\
    & \implies &
    \mel{n'\,\ell'\,m'}{z}{n\,\ell\,m}=0
  }
\end{align*}
\subsection{Fermi's Golden Rule}
We are going to consider transitions to a continuum of states, such as in the photoelectric effect now. Transition to a precise state is undefined, so we instead integrate over a range of energies:
\begin{align*}
  P=\int_{E_f-\Delta{E}/2}^{E_f+\Delta{E}/2}\frac{\abs{V_{in}}^2}{\hbar^2}
  \qty(\frac{\sin^2(\omega_0-\omega)t/2}{(\omega_0-\omega)^2})\rho(E_n)\dd{E_n}
\end{align*}
For large $t$, it can be approximated as:
\begin{align*}
  P=\frac{\abs{V_{if}}^2}{\hbar^2}\rho(E_f)
  \int_{-\infty}^\infty\frac{\sin^2(\omega_0-\omega)t/2}{(\omega_0-\omega)^2}\dd{E_n}
\end{align*}
This intergal was evaluated already:
\begin{align*}
  P=\frac{2\pi}{\hbar}\abs{\frac{V_{if}}{2}}^2\rho(E_f)t
  R=\frac{2\pi}{\hbar}\abs{\frac{V_{if}}{2}}^2\rho(E_f)
\end{align*}
\subsection{Adiabatic Approximation}
The adiabatic theorem states that if a particle was initially in the $n^{th}$ eigenstate of the initial perturbed Hamiltonian, it will be carried into the $n^{th}$ eigenstate of the Hamiltonian at some time $T$. This leads to the introduction of a dynamic phase factor $\theta_n(t)$:
\begin{align*}
  \exp{-iE_nt/\hbar}\to\exp{i\theta_n(t)}\qquad\theta_n(t)\equiv-\frac{1}{\hbar}
  \int_0^tE_n(t')\dd{t'}
\end{align*}
There may be an additional phase factor $\gamma_n$, so that the wavefunction is:
\begin{align*}
  \Psi_n(t)=\exp{i\theta_n(t)}\exp{i\gamma_n(t)}\psi_n
\end{align*}
If you want to calculate the geometric phase $\gamma$, it is given by:
\begin{align*}
  \gamma_n(t)\equiv i\int_0^t\ip{\psi_n(t')}{\pdv{\psi_m(t')}{t'}}\dd{t'}
\end{align*}
We know that the stationary eigenstaets depend on $t$ only because of some parameter(s) $R$, so the partial can be expanded:
\begin{align*}
  \pdv{\psi_n}{t}=\pdv{\psi_n}{R}\dv{R}{t}
\end{align*}
If there are some number $N$ of these, we can represent this as a gradient of some vector of parameters $\vb{R}$ such that the geometric phase is given as:
\begin{align*}
  \gamma_n(t)=i\int_{\vb{R}_i}^{\vb{R}_i}\ip{\psi_n}{\grad_R\psi_n}\vdot\dd{\vb{R}}
\end{align*}
For Berry's phase, this is a closed loop integral:
\begin{align*}
  \gamma_n(T)=i\oint\ip{\psi_n}{\grad_R\psi_n}\vdot\dd{\vb{R}}
\end{align*}
This means that the geometric phase depends only on the path taken (a fitting name) and is called Berry's phase.