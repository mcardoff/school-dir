% -*- TeX-master: "master.tex" -*-
\section{Quantum Electrodynamics}
In deriving the Dirac equation, we formed Lorentz invariants out of Dirac spinors such as $\bar{\psi}\psi$.

Under the Lorentz group, $\bar{\psi}\psi$ is a scalar, $\bar{\psi}\gamma^\mu\psi$ is a 4-vector, so $\bar{\psi}\sla{\D}\psi$ is also a scalar.

We want a Lagrangian $\L$ invariant under the Lorentz group:
\begin{align*}
  \L=i\bar{\psi}\sla{\D}\psi-m\bar{\psi}\psi
\end{align*}
Equation of motion comes from the Principle of Least Action:
\begin{align*}
  \fdv{\L}{\bar{\psi}}=0\implies i\sla{D}\psi-m\psi=0
\end{align*}
Recall that the other term from the Euler-Lagrange is just $0$ since theres no derivatives of $\bar{\psi}$.

In addition to the Lorentz group, $\L$ is also invariant under the following symmetry:
\begin{align*}
  \psi&\to e^{iQ\theta}\psi\\
  \bar{\psi}&\to \bar{\psi}e^{-iQ\theta}
\end{align*}
Where $Q\in\mathbb{R}$, a $U(1)$ symmetry.

This is a \underline{global} symmetry, $\theta$ is the same over all spacetime. The symmetry group is $U(1)$. QED is introduced by demanding the $U(1)$ symmetry be \underline{local}, where $\theta=\theta(x)$ can change smoothly.

Clearly $m\bar{\psi}\psi$ is invariant:
\begin{align*}
  m\bar{\psi}\psi\to m\bar{\psi}e^{-iQ\theta(x)}e^{iQ\theta(x)}\psi
  =m\bar{\psi}\psi
\end{align*}
The other term:
\begin{align*}
  i\bar{\psi}\sla{\D}\psi\to
  \bar{\psi}e^{-iQ\theta(x)}\sla{\D}(e^{iQ\theta(x)}\psi)
  &=\bar{\psi}e^{-iQ\theta(x)}e^{iQ\theta(x)}\sla{\D}\psi
  +iQ\bar{\psi}\gamma^\mu\psi(\D_\mu\theta)\\
  &=i\bar{\psi}\sla{\D}\psi+i\bar{\psi}(iQ\sla{\D}\theta)\psi
\end{align*}
Is not invariant.

This is remedied by introducing a new field $A^\mu$, into the Lagrangian, and demanding $A^\mu$ transform in just the right way to cancel the offending term. Try the following:
\begin{align*}
  \L=i\bar{\psi}(+ieQ\sla{A})\psi
\end{align*}
Where $\sla{A}=A_\mu\gamma^\mu$, the photon field

The second term in the transformation of $i\bar{\psi}\sla{\D}\psi$ is cancelled if:
\begin{align*}
  A^\mu\to A^\mu-\frac1e\D^\mu\theta(x)
\end{align*}
This local transformation is called a \underline{gauge} transformation, and we say $\L$ is gauge invariant.

The field $A^\mu$ corresponds to the vector potential of the electromagnetic field, and $eQ$ is the electric charge of the fermion.
\begin{itemize}
\item $e$ is called the coupling constant $\alpha=e^2/4\pi\approx 1/137$ is the fine structure constant
\item $Q$ is the charge of particle in units of $e$, $Q=-1$ for electrons, and $Q=+\frac23$ for up quark, etc.
\end{itemize}
The quantum of the EM field is the photon. A mass term in $\L$ would be:
\begin{align*}
  -\frac12M^2A^\mu A_\mu
\end{align*}
But this is \underline{not} gauge invariant. We regard the masslessness of the photon as being a consequence of gauge invariance (although the inverse transformation is also reasonable).

The photon is a massless spin-1 particle and transforms as the $(1,0)\oplus(0,1)$ representation of the Lorentz group. It therefore as helicity $h=\pm1$.

Due to gauge invariance, $A_\mu\to A_\mu-\frac1e\D_\mu\theta$, we can always find a gauge (some $\theta(x)$) where:
\begin{align*}
  \D_\mu A^\mu=0
\end{align*}
This is called the Lorentz gauge, so called because of its manifest Lorentz invariance.

The photon field is:
\begin{align*}
  A^\mu(x)=\int\frac{\dd[3]{p}}{2E}\qty[
  e^{-ip\vdot x}a_\lambda\veps^\mu_\lambda(p)+
  e^{+ip\vdot x}a^\dag_\lambda\veps^{\mu*}_\lambda(p)
  ]
\end{align*}
Where $a,a^\dag$ are the destruction and creation operators for the photon, and $\veps^\mu_\lambda(p)$ is a four-vector which transforms as either the $(1,0)$ rep ($\lambda=+1$), or the $(0,1)$ rep ($\lambda=-1$) of the Lorentz group.

Recall $\psi(x)$:
\begin{align*}
  \psi(x)=\int\frac{\dd[3]{p}}{2E}\qty[
  e^{-ip\vdot x}a_\lambda u_\lambda(p)+
  e^{+ip\vdot x}b^\dag_\lambda v_\lambda(p)
  ]
\end{align*}
The Lorentz gauge condition implies:
\begin{align*}
  p_\mu\veps^\mu_\lambda(p)=0
\end{align*}
This means $\veps^\mu_\lambda$ is the Fourier transform of the photon field, and implies the photon is its own antiparticle.

The ``polarization vectors'' $\veps^\mu_\lambda$ for a photon with momentum $p^\mu=(E,0,0,E)$ are:
\begin{align*}
  \veps^\mu_+&=\frac1{\sqrt{2}}(0,1,i,0)\\
  \veps^\mu_-&=\frac1{\sqrt{2}}(0,1,-i,0)
\end{align*}

\begin{aside}
  Lorentz group (photon not massless) $\lambda=0,\pm1$, for a real photon we would have to remove $\lambda=0$. The radiation gauge, $A^0=0, \grad_i\vdot A^i$, $\lambda=\pm1$ manifestly.
\end{aside}

\subsection{Perturbation Theory}
QED is a weakly-coupled theory --- photons and electrons (or quarks) do not interact very strongly. To zeroth order they do not interact at all. Hence, one can regard interactions as perturbations of this simple picture. The outcome of treating QED perturbatively can be represented pictorially via Feynman diagrams. These ``Feynman rules'' are derived from quantum field theory. In practice, all one needs for simple calculations are the Feynman rules.

Lets draw the Feynman diagram for $e^+e^-\to\mu^+\mu^-$ then work out the details:
\begin{figure}[H]
  \centering
  \begin{tikzpicture}
    \begin{feynhand}
      \vertex (a) at (0,0);
      \vertex (e1) at (-1, 1) {$e^-$};
      \vertex (e2) at (-1,-1) {$e^+$};
      \vertex (m1) at (2.5, 1) {$\mu^-$};
      \vertex (m2) at (2.5,-1) {$\mu^+$};
      \vertex (b) at (1.5,0);
      \propag[mom=\(p_1\)] (e1) to (a);
      \propag[mom'=\(p_2\)] (e2) to (a);
      \propag[mom=\(p_3\)] (b) to (m1);
      \propag[mom'=\(p_4\)] (b) to (m2);
      \propag[bos, mom=\(q\)] (a) to (b);
    \end{feynhand}
    \draw[->] (3.0, 0.0) -- (4.0, 0.0);
    \node at (3.45, 0.2) {Time};
  \end{tikzpicture}
  \caption{Feynman Diagram for $e^+e^-\to\mu^+\mu^-$}
\end{figure}

\begin{aside}
  Many older books draw time going up. More modern notation has time flowing to the right. 
\end{aside}

The interaction ``vertex'' (seen below) comes from the Lagrangian:
\begin{align*}
  \L=i\bar{\psi}(\sla{\D}+ieQ\sla{A})\psi-m\bar{\psi}\psi
\end{align*}
\begin{figure}[H]
  \centering
  \begin{tikzpicture}
    \begin{feynhand}
      \vertex (a) at (0,0);
      \vertex (e1) at (-1, 1) {$e^-$};
      \vertex (e2) at (-1,-1) {$e^+$};
      \vertex (b) at (1.5,0) {$\gamma$}; 
      \propag (e1) to (a);
      \propag (e2) to (a);
      \propag[bos] (a) to (b);
    \end{feynhand}
  \end{tikzpicture}
  \caption{The interaction vertex}
\end{figure}
What does this mean?
\begin{itemize}
\item $\psi$ Destorys an $e^-$
\item $\bar{\psi}$ Destorys an $e^+$
\item $A^\mu$ creates a photon ($\gamma$)
\end{itemize}

We can ``read-off'' the Feynman rule for the interaction vertex: (Note that overall sign is convention, not all vertices are simple):
\begin{align*}
  \begin{tikzpicture}[baseline=-0.15cm]
    \begin{feynhand}
      \vertex (a) at (0,0);
      \vertex (e1) at (-1, 1) {$e^-$};
      \vertex (e2) at (-1,-1) {$e^+$};
      \vertex (b) at (1.5,0) {$\gamma$}; 
      \propag[fer] (e1) to (a);
      \propag[antfer] (e2) to (a);
      \propag[bos] (a) to (b);
    \end{feynhand}
  \end{tikzpicture}
  =-ieQ\gamma^\mu
\end{align*}
Where $Q=$ particle charge ($-1$ for $e^-$ and $e^+$).

The second ingredient we need is the ``propagator'' of the photon which mediates the interaction. This is not a real photon. A real photon is not kinematically allowed.
\begin{proof}
  Let the electron and positron 4-momenta respectively be:
  \begin{align*}
    p_1=(E,0,0,p)\qquad p_2=(E,0,0,-p)
  \end{align*}
  And that they are colliding head-on (in c.o.m.\ frame) in an experiment (BEPC, Tristan, CESR, SLAC, LEP, etc.)
  The photon momentum is the sum of these two:
  \begin{align*}
    q_{\gamma}&=p_1+p_2=(2E,0,0,0)\\
    \implies q_{\gamma}^2=4E^2\neq 0
  \end{align*}
  A real photon would have $q^2=0$ always. 
\end{proof}

The interaction is mediated by a ``virtual'' photon (a particle \underline{not} ``on the mass shell'', i.e.\ $q^2\neq m^2$). These ``particles'' do not actually exist in the same sense that real photons (w/ $q^2=0$) do. Also $q^2$ is \underline{not} constrained to be greater than $0$ in general (though it is here).

What is the time scale of this interaction? Dimensional analysis says:
\begin{align*}
  t\sim\frac1{\sqrt{q^2}}\sim\frac1{2E}
\end{align*}
The minimum (threshold) energy for this interaction is that $2E\sim\SI{200}{\MeV}$ Hence $t$ is about:
\begin{align*}
  t\sim\frac{\hbar c}{2Ec}\sim 10^{-23}\text{s}
\end{align*}

When a virtual particle appears in a Feynman diagram, this corresponds to a propagator:
\begin{align*}
  \text{spin 0}\quad
  \begin{tikzpicture}[baseline=-0.10cm]
    \begin{feynhand}
      \vertex (a) at (0,0);
      \vertex (b) at (1.5,0); 
      \propag[sca, mom=\(p\)] (a) to (b);
    \end{feynhand}
  \end{tikzpicture}
  &=\frac{i}{p^2-m^2}\\
  \text{spin }\frac12\quad
  \begin{tikzpicture}[baseline=-0.10cm]
    \begin{feynhand}
      \vertex (a) at (0,0);
      \vertex (b) at (1.5,0); 
      \propag[fer, mom=\(p\)] (a) to (b);
    \end{feynhand}
  \end{tikzpicture}
  &=\frac{i}{\sla{p}-m}=\frac{i(\sla{p}+m)}{p^2-m^2}\\
  \text{(massless) spin 1}\quad
  \begin{tikzpicture}[baseline=-0.10cm]
    \begin{feynhand}
      \vertex (a) at (0,0);
      \vertex (b) at (1.5,0); 
      \propag[bos, mom=\(p\)] (a) to (b);
    \end{feynhand}
  \end{tikzpicture}
  &=\frac{i(-g^{\mu\nu}+p^\mu p^\nu/p^2)}{p^2}
\end{align*}
\begin{remark}
  The charge arrow on the fermion above is defined such that the direction of motion of the \emph{particle} is the direction of the charge arrow.\@\emph{Antiparticles} go backward in time.
\end{remark}

The numerators of the propagators come from summing over helicities:
\begin{align*}
  \sum_{\lambda=\pm}u_\lambda\bar{u}_\lambda&=\sla{p}+m\\
  \sum_{\lambda=\pm,0}\veps^\mu_\lambda\bar{\veps}^\nu_\lambda&=
  -g^{\mu\nu}+\frac{p^\mu p^\nu}{p^2}
\end{align*}

The photon propagator looks complicated, but we will see that the second term will not contribute in practice in either of QED or QCD.\@

\subsection{Cross section calculation: $e^+e^-\to\mu^+\mu^-$}
Lets now turn the pictogram into an actual calculation of the scattering amplitude:
\begin{figure}[H]
  \centering
  \begin{tikzpicture}[scale=2.0]
    \begin{feynhand}
      % vertices
    \vertex (p11) at (-1,1) {$e^-$};
    \vertex (p12) at (-1,-1) {$e^+$};
    \vertex (p21) at (3,1) {$\mu^-$};
    \vertex (p22) at (3,-1) {$\mu^+$};
    \vertex (a) at (0,0);
    \vertex (b) at (2,0);
    % particles
    \propag [fer, mom=\(p_1\)] (p11) to (a);
    \propag [antfer, mom'=\(p_2\)] (p12) to (a);
    \propag [fer, mom=\(p_3\)] (b) to (p21);
    \propag [antfer, mom'=\(p_4\)] (b) to (p22);
    % propagator
    \propag [bos, mom=\(q\)] (a) to [edge label'=\(\gamma\)] (b);
    \end{feynhand}
  \end{tikzpicture}
  \caption{$e^+e^-\to\mu^+\mu^-$}\label{fig:feynman}
\end{figure}
The arrows on the actual edges indicate charge or ``fermion'' flow, read the spinor chain opposite in the direction of these arrows, for example, the left chain would be read as:
\begin{align*}
  \bar{v}(p_2)(-ieQ_e\gamma^\mu)u(p_1)
\end{align*}
And the right chain would read as:
\begin{align*}
  \bar{u}(p_3)(-ieQ_\mu\gamma^\nu)v(p_4)
\end{align*}
The last thing to get is the propagator, then we can write out $\M$, the matrix element:
\begin{align*}
  i\M=&(-ieQ_e)\bar{v}(p_2)\gamma^\mu u(p_1)\\
  \times&\frac{i}{q^2}\qty(-g_{\mu\nu}+\frac{q_\mu q_\nu}{q^2})\\
  \times&(-ieQ_\mu)\bar{u}(p_3)\gamma^\nu v(p_4)
\end{align*}
$\M$ is the amplitude, we will square $\M$ to get the probability that feeds into the cross section or particle decay width.

Lets first show that the second term in the photon propagator does not contribute:
\begin{proof}
  Consider what is actually being calculated:
  \begin{align*}
    \bar{v}(p_2)\gamma^\mu u(p_1)q_\mu=\bar{v}(p_2)\sla{q}u(p_1)
  \end{align*}
  By conservation of momentum, we know that $q=p_1+p_2$:
  \begin{align*}
    \bar{v}(p_2)\sla{q}u(p_1)=\bar{v}(p_2)(\sla{p}_1+\sla{p}_2)u(p_1)
  \end{align*}
  Note that on each side we can use the momentum-space Dirac equation, namely that:
  \begin{align*}
    \sla{p}_1u(p_1)&=m u(p_1)\\
    \bar{v}(p_2)\sla{p}_2&=-m\bar{v}(p_2)
  \end{align*}
  This gives a factor of $m$ and $-m$ in the parentheses:
  \begin{align*}
    \bar{v}(p_2)(\sla{p}_1+\sla{p}_2)u(p_1)
    =\bar{v}(p_2)(m-m)u(p_1)=0
  \end{align*}
  The factors $\bar{v}(p_2)\gamma^\mu u(p_1)$ and $\bar{u}(p_3)\gamma^\nu v(p_4)$ are called the fermion current, $j^\mu$, and the fact that $q_\mu j^\mu=0$ is called current conservation. This is directly from Maxwell's equations in covariant form.
\end{proof}
Thus our matrix element can be reduced drastically:
\begin{align*}
  i\M=&(-ieQ_e)\bar{v}(p_2)\gamma^\mu u(p_1)
  \times-\frac{i}{q^2}g_{\mu\nu}
  \times(-ieQ_\mu)\bar{u}(p_3)\gamma^\nu v(p_4)\\
  =&i\frac{e^2Q_e Q_\mu}{q^2}\bar{v}(p_2)\gamma^\mu u(p_1)
  g_{\mu\nu}\bar{u}(p_3)\gamma^\nu v(p_4)\\
  =&i\frac{e^2Q_e Q_\mu}{q^2}\bar{v}(p_2)\gamma^\mu u(p_1)
  \bar{u}(p_3)\gamma_\mu v(p_4)
\end{align*}
Now we can square the amplitude: $\abs{\M}^2=\M\M^*$:
\begin{align*}
  \abs{\M}^2=&e^4Q_e^2Q_\mu^2q^{-4}\bar{v}p_2\gamma^\mu u(p_1)
  {(\bar{v}(p_2)\gamma^\nu u(p_1))}^*\\
  \times&\bar{u}(p_3)\gamma_\mu v(p_4)
  {(\bar{u}(p_3)\gamma_\nu v(p_4))}^*
\end{align*}
Notice that the factor of $\M^*$ has a different index on it. We can also trivially turn the complex conjugate into a conjugate transpose as we are dealing with scalars in Dirac-space:
\begin{align*}
  {(\bar{v}(p_2)\gamma^\nu u(p_1))}^*&={(\bar{v}(p_2)\gamma^\nu u(p_1))}^\dag\\
  =u^\dag(p_1){(\gamma^\nu)}^\dag{(\gamma^0)}^\dag v(p_2)
  &=u^\dag(p_1)\gamma^0\gamma^\nu\gamma^0\gamma^0 v(p_2)\\
  &=\bar{u}(p_1)\gamma^\nu v(p_2)
\end{align*}

Often the colliding beams at these experiments are unpolarized, so it is useful to average over helicities:
\begin{align*}
  \frac12\frac12\sum_{\lambda=\pm}\sum_{\lambda'=\pm}
  \bar{v}_{\lambda'}(p_2)\gamma^\mu u_\lambda(p_1)
  \bar{u}_{\lambda}(p_1)\gamma^\nu v_{\lambda'}(p_2)
\end{align*}
We now need to make use of the matrix-nature of this expression. Since the expression is a complex number, we have:
\begin{align*}
  \sum_i M_{ii}=\Tr[M_{ij}]
\end{align*}
Because of this, we can move $\bar{v}_{\lambda'}$ to the end of the chain:
\begin{align*}
  \frac12\frac12\sum_{\lambda=\pm}\sum_{\lambda'=\pm}\Tr[
  \gamma^\mu u_\lambda(p_1)\bar{u}_{\lambda}(p_1)
  \gamma^\nu v_{\lambda'}(p_2)\bar{v}_{\lambda'}(p_2)]
\end{align*}
Now we can use:
\begin{align*}
  \sum_{\lambda=\pm}u_\lambda(p_1)\bar{u}_\lambda(p_1)&=\sla{p}_1+m\\
  \sum_{\lambda'=\pm}v_\lambda(p_2)\bar{v}_\lambda(p_2)&=\sla{p}_2-m
\end{align*}
This is called ``Casimir's Trick'', and we can now use the trace ``theorems'':
\begin{align*}
  \acomm{\gamma^\mu}{\gamma^\nu}&=2g^{\mu\nu}\\
  \Tr[\gamma^\mu]&=0\\
  \Tr[\gamma^\mu\gamma^\nu]&=4g^{\mu\nu}\\
  \Tr[\gamma^\mu\gamma^\nu\gamma^\sigma\gamma^\rho]&=
  4[g^{\mu\nu}g^{\sigma\rho}+g^{\nu\sigma}g^{\mu\rho}-g^{\nu\rho}g^{\mu\sigma}]\\
  \Tr[\text{odd \# of $\gamma$'s}]&=0
\end{align*}
So our trace is fairly easy:
\begin{align*}
  \frac14\Tr[\gamma^\mu(\sla{p}_1+m_e)\gamma^\nu(\sla{p}_2-m_e)]
  &=\frac14\qty{\Tr[\gamma^\mu\sla{p}_1\gamma^\nu\sla{p_2}]
    -m_e^2\Tr[\gamma^\mu\gamma^\nu]}\\
  &=\frac144\qty{p_{1\rho}p_{2\sigma}
    (g^{\mu\rho}g^{\nu\sigma}+g^{\nu\rho}g^{\mu\sigma}-g^{\mu\nu}g^{\rho\sigma})
    -m_e^2g^{\mu\nu}}\\
  &=p_1^\mu p_2^\nu+p_1^\nu p_2^\mu-p_1\vdot p_2g^{\mu\nu}-m_e^2g^{\mu\nu}
\end{align*}
Similarly, the helicities of the final-state fermions are often unobserved, so we can sum over all helicities. (Note that if we wanted to pull out for example $u_L$, we could insert a $P_L=\frac{1-\gamma_5}2$, or for $u_R$: $P_R=\frac{1+\gamma_5}2$)
\begin{align*}
  \sum_\lambda&\sum_{\lambda'}\bar{u}(p_r)\gamma_\mu v(p_4){\qty(\bar{u}(p_3)\gamma_\nu v(p_4))}^*\\
  &=\Tr[\gamma_\mu(\slap_4-m_\mu)\gamma_\nu(\slap_3+m_\mu)]\\
  &=4\qty[p_{4\mu}p_{3\nu}+p_{4\nu}p_{3\mu}-p_3\vdot p_4g_{\mu\nu}-m_\mu^2g_{\mu\nu}]
\end{align*}
Since $m_e\ll m_\mu$, we'll ignore $m_e$ from this point forward. So:
\begin{align*}
  \overline{\abs{\M}^2}=&\frac{e^4}{q^4}\qty[p_1^\mu p_2^\nu+p_1^\nu p_2^\mu-p_1\vdot p_2g^{\mu\nu}]\\
  &\times4\qty[p_{4\mu}p_{3\nu}+p_{4\nu}p_{3\mu}-p_3\vdot p_4g_{\mu\nu}-m_\mu^2g_{\mu\nu}]\\
  &=4\frac{e^4}{q^4}\qty[
  \begin{aligned}
    2&(p_1\vdot p_4)(p_2\vdot p_3)+2(p_1\vdot p_3)(p_2\vdot p_4)\\
    -2&(p_1\vdot p_2)(p_3\vdot p_4)-2(p_3\vdot p_4)(p_1\vdot p_2)+4(p_1\vdot p_2)(p_3\vdot p_4)\\
    -&m_\mu^2(2(p_1\vdot p_2)-4(p_1\vdot p_2))
  \end{aligned}
  ]\\
  &=4\frac{e^4}{q^4}\qty[2(p_1\vdot p_4)(p_2\vdot p_3)+2(p_1\vdot p_3)(p_2\vdot p_4)+2m_\mu^2(p_1\vdot p_2)]
\end{align*}
Use momentum conservation to simplify:
\begin{align*}
  p_1+p_2&=p_3+p_4\\
  {(p_1-p_3)}^2&={(p_4-p_2)}^2\\
  \implies -2p_1\vdot p_3+m_\mu^2=-2p_2\vdot p_4+m_\mu^2
\end{align*}
Hence we have two identities:
\begin{align*}
  p_1\vdot p_3=p_2\vdot p_4\qquad p_1\vdot p_4=p_2\vdot p_3
\end{align*}
Hence the fully simplified matrix element is:
\begin{align*}
  \overline{\abs{\M}^2}=8e^4\frac1{q^4}\qty[{(p_2\vdot p_3)}^2+{(p_1\vdot p_3)}^2+m_\mu^2(p_1\vdot p_2)]
\end{align*}

\paragraph{From $\overline{\abs{\M}^2}$ to $\dd{\sigma}$ (differential cross section)}
\begin{align*}
  \dd{\sigma}=\underbrace{\frac1{2s}}_{\text{``Flux factor''}}
  \underbrace{\frac{\dd[3]{p_3}}{{(2\pi)}^3 2E_3}\frac{\dd[3]{p_4}}{{(2\pi)}^3 2E_4}{(2\pi)}^4\delta^4(p_1+p_2-p_3-p_4)}_{\text{``Phase Space''}}
\end{align*}
Where $s\equiv{(p_1+p_2)}^2=q^2$ ``Mandelstam variable''

Two-body phase space is quite siumple. Everything is determined by kinematics, except the angles of the outgoing particles:
\begin{figure}[H]
  \centering
  \begin{tikzpicture}
    \begin{feynhand}
      \vertex (em) at (-2,0) {$e^-$};
      \vertex (ep) at ( 2,0) {$e^+$};
      \vertex (i) at (0,0);
      \vertex (mp) at (-1.0, -1.73205) {$\mu^+$};
      \vertex (mm) at ( 1.0,  1.73205) {$\mu^-$};
      \propag[fer] (em) to (i);
      \propag[fer] (ep) to (i);
      \propag[fer] (i) to (mp);
      \propag[fer] (i) to (mm);
    \end{feynhand}
    \draw (0.5,0.0) arc
    [start angle=0, end angle=60, x radius=0.5,y radius=0.5];
    \node at (0.7, 0.3) {$\theta$};
  \end{tikzpicture}
  \caption{Kinematic diagram of our collision}\label{fig:kinematics}
\end{figure}


\begin{remark}
  We will solve for $\theta$ dependence. There is also an angle $\phi$ (angle of $\mu^+\mu^-$ plane with respect to some vertical plane), but in most cases $\bar{\abs{\M}}^2$ has no explicit $\phi$ dependence, so this is trivial. That is to say, there is a cylindrical symmetry in geenral.
\end{remark}

Two-body phase space, ignoring some factors of $2\pi$:
\begin{align*}
  \frac{\dd[3]{\vb{p}_3}}{2E_3}\frac{\dd[3]{\vb{p}_4}}{2E_4}
  \underbrace{\delta^4(p_1+p_2-p_3-p_4)}_{\text{Energy conservation}}
\end{align*}
Let $p=p_1+p_2$, in the $p$ rest frame:
\begin{align*}
  p&=(\sqrt{s},0,0,0)\\
  p^2&=s={(p_1+p_2)}^2
\end{align*}
Phase space doe snot look Lorentz invariant, but it is. To see this, and where this came from, notice:
\begin{align*}
  \frac{\dd[3]{\vb{p}}}{2E}=\dd[4](p)\delta(p^2-m^2)\Theta(E)
\end{align*}
For a particle of mass $m$
\begin{proof}
  \begin{align*}
    \dd[4]{p}\delta(p^2-m^2)\Theta(E)
    &=\dd[3]{\vb{p}}\dd{E}\delta(E^2-\vb{p}^2-m^2)\Theta(E)
  \end{align*}
  The $\delta$ function has the following property:
  \begin{align*}
    \delta(f(x))&=\sum_{x_0}\frac{\delta(x-x_0)}{\abs{f'(x_0)}}\\
    \text{Where }\quad f(x_0)&=0
  \end{align*}
  So we can expant the Energy delta function:
  \begin{align*}
    \delta(E^2-(\vb{p}^2+m^2))=
    \frac{\delta(E-\sqrt{\vb{p}^2+m^2})}{2E}+
    \frac{\delta(E+\sqrt{\vb{p}^2+m^2})}{2E}
  \end{align*}
  The first term requires $E>0$, where the second requires $E<0$, so the latter is ``killed'' by the $\Theta(E)$, hence we have:
  \begin{align*}
    \int\dd{E}\delta(E^2-(\vb{p}^2+m^2))\Theta(E)=\frac1{2E}
    \implies\dd[4]{p}\delta(p^2-m^2)=\frac{\dd[3]{\vb{p}}}{2E}
  \end{align*}
\end{proof}
We will exploit this relationship to aid in solving the phase-space integral.
\begin{align*}
  \frac{\dd[3]{\vb{p}_3}}{2E_3}\frac{\dd[3]{\vb{p}_4}}{2E_4}\delta^4(p-p_3-p_4)
  &=\frac{\dd[3]{\vb{p}_3}}{2E_3}\delta[4]{p_4}(p_4^2-m^2)\Theta(E_4)
  \delta^4(p-p_3-p_4)
\end{align*}
We can use the extra $\delta^4$ to solve the $p_4$ integral, setting $p_4=p-p_3$, and we can ignore the $\Theta(E_4)$:
\begin{align*}
  \frac{\dd[3]{\vb{p}_3}}{2E_3}\delta[4]{p_4}(p_4^2-m^2)\Theta(E_4)
  \delta^4(p-p_3-p_4)
  =\frac{\dd[3]{\vb{p}_3}}{2E_3}\delta({(p-p_3)}^2-m^2)
\end{align*}
In our special rest frame, ${(p-p_3)}^2-m^2=s-2\sqrt{s}E_3$. We then take advantage of spherical coordinates in $\vb{p}_3$:
\begin{align*}
  \dd[3]{\vb{p_3}}=\dd{\Omega}\abs{\vb{p_3}}^2\dd{p_3}
\end{align*}
With the energy momentum constraint, we also have:
\begin{align*}
  E^2_3=\vb{p_3}^2+m^2\implies 2E_3\dd{E_3}=2\abs{\vb{p_3}}\dd{p_3}
\end{align*}
Changing out differential to:
\begin{align*}
  \frac{\dd[3]{\vb{p}_3}}{2E_3}\delta(s-2\sqrt{s}E_3)
  &=\dd{\Omega}\underbrace{\abs{\vb{p_3}}^2\dd{p_3}}_{\abs{\vb{p_3}}E_3\dd{E_3}}
  \frac1{2E_3}\delta(s-2\sqrt{s}E_3)\\
  &=\frac12\abs{\vb{p_3}}\dd{\Omega}\dd{E_3}\delta(s-2\sqrt{s}E_3)
\end{align*}
We can then finish the integral over $E_3$:
\begin{align*}
  \int\dd{E_3}\delta(s-2\sqrt{s}E_3)=\frac{1}{2\sqrt{s}}
\end{align*}
Giving our integrated result:
\begin{align*}
  \frac{\dd[3]{\vb{p}_3}}{2E_3}\frac{\dd[3]{\vb{p}_4}}{2E_4}\delta^4(p-p_3-p_4)
  =\frac1{4\sqrt{s}}\abs{\vb{p}_3}\dd{\Omega}
\end{align*}
Kinematically, tis process is like a ``virtual'' photon of mass $\sqrt{s}$ decaying into two muons of mass $m$.

In the $CM$ frame, $\abs{\vb{p}_3}=\abs{\vb{p}_4}=\frac{\sqrt{s}}2\beta$, and in terms of the mass and $s$:
\begin{align*}
  \beta=\sqrt{1-\frac{4m_\mu^2}{s}}
\end{align*}
So, for our process:
\begin{align*}
  \frac1{{(2\pi)}^2}\dd[3]{\vb{p}_3}{2E_3}\dd[3]{\vb{p}_4}{2E_4}
  \delta^4(p_1+p_2-p_3-p_4)=\frac1{{(2\pi)}^2}\frac18\beta\dd{\Omega}
\end{align*}
So the differential for the cross section is:
\begin{align*}
  \dd{\bar{\sigma}}&=\frac1{2s}\overline{\abs{\M}^2}\times PS\\
  &=\frac1{2s}\frac{8e^4}{s^2}
  [{(p_2\vdot p_3)}^2+{(p_1\vdot p_3)}^2+m_\mu^2(p_1\vdot p_2)]
  \frac1{{(2\pi)}^2}\frac{\beta}{8}\dd{\Omega}
\end{align*}
Choose the following values for the momenta:
\begin{align*}
  p_1&=\frac{\sqrt{s}}{2}(1,0,0,1)\\
  p_2&=\frac{\sqrt{s}}{2}(1,0,0,-1)\\
  p_3&=\frac{\sqrt{s}}{2}(1,0,\beta\sin\theta,\beta\cos\theta)\\
  p_4&=\frac{\sqrt{s}}{2}(1,0,-\beta\sin\theta,-\beta\cos\theta)
\end{align*}
The dot products are:
\begin{align*}
  p_2\vdot p_3&=\frac{s}{4}(1+\beta\cos\theta)\\
  p_1\vdot p_3&=\frac{s}{4}(1-\beta\cos\theta)\\
  p_1\vdot p_2&=\frac{s}{2}
\end{align*}
Subbing these in:
\begin{align*}
  \dd{\bar{\sigma}}&=\frac1{2s}\frac{e^4}{{(2\pi)}^2}\frac{\beta}{s^2}s^2
  \qty[\frac1{16}{(1+\beta\cos\theta)}^2+{(1-\beta\cos\theta)}^2+\frac{m_\mu^2}{2s}]
  \dd{\Omega}\\
  &=\frac1{2s}\frac{e^4}{{(2\pi)}^2}\beta
  \qty[\frac18(1+\beta^2\cos^2\theta)+\frac{m_\mu^2}{2s}]\dd{\Omega}
\end{align*}
Using the fact that there is $\phi$ symmetry, we can integrate it out for an extra factor of $2\pi$, giving:
\begin{align*}
  \dv{\bar{\sigma}}{(\cos\theta)}&=\frac\pi{8s}\frac{e^4}{{(2\pi)}^2}\beta
  \qty[1+\beta^2\cos^2\theta+4\frac{m_\mu^2}{s}]
\end{align*}
Using $\alpha=e^2/4\pi$ and $z=\cos\theta$:
\begin{align*}
  \dv{\bar{\sigma}}{z}&=\frac\pi{2s}\alpha^2\beta
  \qty[1+\beta^2z^2+4\frac{m_\mu^2}{s}]
\end{align*}
The total cross section is then found using
\begin{align*}
  \bar{\sigma}=\int_{-1}^1\dv{\bar{\sigma}}{z}\dd{z}
  =\frac{\pi}{2s}\alpha^2\beta\qty[2+\frac23\beta^2+2\frac{4m_\mu^2}{s}]
\end{align*}
The parenthetical can be reduced to:
\begin{align*}
  2+\frac23\beta^2+2\frac{4m_\mu^2}{s}=\frac83+\frac{16}3\frac{m_\mu^2}{s}
\end{align*}
Hence our final result!
\begin{align*}
  \boxed{\bar{\sigma}=\frac{4\pi}{3s}\alpha^2\beta\qty(1+2\frac{m_\mu^2}{s})}
\end{align*}
We've finally reached the total cross section for $e^-e^+\to\mu^-\mu^+$. This is \emph{the} basic cross section of $e^-e^+$ physics, other cross sections are compared to this one.

Recent (and future) $ee$ colliders run at energies far above $2m_\mu$:
\begin{table}[H]
  \centering
  \begin{tabular}{ccc}
    Collider & $\sqrt{s}$        & Location\\ \midrule
    BEPC   & \SI{3}{\GeV}        & Beijing\\ \midrule
    CESR   & \SI{10}{\GeV}       & Cornell\\ \midrule
    SLC    & \SI{91}{\GeV}       & Stanford\\ \midrule
    LEP I  & \SI{91}{\GeV}       & CERN\\ \midrule
    LEP II & 91-\SI{210}{\GeV}   & CERN\\ \midrule
    ILC    & 500-\SI{1000}{\GeV} & Future\\
  \end{tabular}
  \caption{$\sqrt{s}$ Values at various colliders}\label{tab:1}
\end{table}
The value we found, $\bar{\sigma}(e^+e^-\to\mu^+\mu^-)=\frac{4\pi\alpha^2}{2s}$ is a good approximation.

\subsection{$e^+e^-\to$ hadrons}
The basic process here is $e^-e^+\to q\bar{q}$:
\begin{figure}[H]
  \centering
  \begin{tikzpicture}
    \begin{feynhand}
      \vertex (a) at (0,0);
      \vertex (e1) at (-1, 1) {$e^-$};
      \vertex (e2) at (-1,-1) {$e^+$};
      \vertex (m1) at (2.5, 1) {$q$};
      \vertex (m2) at (2.5,-1) {$\bar{q}$};
      \vertex (b) at (1.5,0);
      \propag (e1) to (a);
      \propag (e2) to (a);
      \propag (b) to (m1);
      \propag (b) to (m2);
      \propag[bos] (a) to (b);
    \end{feynhand}
  \end{tikzpicture}
  \caption{Feynman Diagram for $e^+e^-\to q\bar{q}$}
\end{figure}
This is just like $e^-e^+\to\mu^-\mu^+$. The quarks do not exist in free form, and must ``hadronize'' over a time scale $1/\Lambda_{QCD}\sim\SI{1e-23}{\s}$. This picture is therefore valid if $q\bar{q}$ creation occurs over a time small compared to $1/\Lambda_{QCD}$, i.e.\ at energies $\sqrt{s}\gg\Lambda_{QCD}\sim\SI{200}{\MeV}$.

For example, $e^+e^-\to\pi^-\pi^+$ looks like this:
\begin{figure}[H]
  \centering
  \begin{tikzpicture}[scale=2.0]
    \begin{feynhand}
      \vertex (a) at (0,0);
      \vertex (b) at (3/2,0);
      \vertex (ga) at (11/6, 1/3);
      \vertex (gb) at (13/6, 2/3);
      \vertex (gc) at (11/6,-1/3);
      \vertex (gd) at (13/6,-2/3);
      \vertex (c) at (13/6, 0);
      \vertex (d) at ( 5/2, 1/3);
      \vertex (e) at ( 5/2,-1/3);
      \vertex (f1) at (-1, 1) {$e^-$};
      \vertex (f2) at (-1,-1) {$e^+$};
      \vertex (ur) at (5/2, 1) {$u$};
      \vertex (ub) at (5/2,-1) {$\bar{u}$};
      \vertex (db) at (17/6, 2/3) {$\bar{d}$};
      \vertex (dr) at (17/6,-2/3) {$d$};
      \propag (f1) to (a);
      \propag (f2) to (a);
      \propag (b) to (ur);
      \propag (b) to (ub);
      \propag[bos] (a) to (b);
      \propag[glu] (ga) to (c);
      \propag[glu] (gb) to (d);
      \propag[glu] (gc) to (c);
      \propag[glu] (gd) to (e);
      \propag (c) to (d);
      \propag (c) to (e);
      \propag (d) to (db);
      \propag (e) to (dr);
    \end{feynhand}
    \draw[rotate around={45:(8/3, 5/6)}] (8/3, 5/6) ellipse (0.2 and 0.5);
    \node at (9/3, 7/6) {$\pi^+$};
    \draw[rotate around={45:(8/3,-5/6)}] (8/3,-5/6) ellipse (0.5 and 0.2);
    \node at (9/3, -7/6) {$\pi^-$};
  \end{tikzpicture}
  \caption{Feynman Diagram for $e^+e^-\to \pi^+\pi^-$}
\end{figure}

If we sum over all possible hadronic final states:
\begin{align*}
  \sigma(e^-e^+\to q\bar{q})=\sigma(e^-e^+\to\text{hadrons})
\end{align*}
Since we already solved $e^-e^+\to\mu^-\mu^+$, we can write down this cross section immediately:
\begin{align*}
  \sigma(e^-e^+\to q\bar{q})=\frac{4\pi\alpha^2}{2s}\beta\qty(1+2\frac{m_q^2}{s})
  Q_q^2\times 3
\end{align*}
The multiple of 3 comes from the quark colors, and $Q_q$ is the quark charge.

Taking this one step further, one of the most important tests of QCD (and a strong constraint on \emph{all} possible low-energy physics) is the measurement of $R_{\text{hadrons}}$:
\begin{align*}
  R_{\text{h}}=\frac{\sigma(e^-e^+\to\text{hadrons})}{\sigma(e^-e^+\to\mu^-\mu^+)}
\end{align*}
At $\sqrt{s}=\SI{8}{\GeV}$, for example, one is above threshold for $u\bar{u}, d\bar{d},s\bar{s}$, and $c\bar{c}$ production. Hence:
\begin{align*}
  R_{\text{h}}(\SI{8}{\GeV})&=3[Q_u^2+Q_d^2+Q_s^2+Q_c^2]\\
  &=3\qty[{\qty(\frac23)}^2+{\qty(-\frac13)}^2
  +{\qty(-\frac13)}^2+{\qty(\frac23)}^2]\\
  &=\frac{10}{3}
\end{align*}
Measurements of $R_{\text{h}}$ have been made at the $\sim1\%$ level at several energies. Since this number is a ratio, many experimental systematic errors cancel, leaving a very strong case that there really are 3 colors of quarks.

Where the measurement of $R_{\text{h}}$ is challenging is near threshold for a $q\bar{q}$ pair. The $q\bar{q}$ pair binds into a meson due to the strong force:
\begin{TODO}
  binding
\end{TODO}
These must have the quantum numbers of the photon, $J^P=1^-,I=0$, such as:
\begin{table}[H]
  \centering
  \begin{tabular}{ccc}
    Quarks & Meson      & Threshold \\ \midrule
    $u,d$  & $\rho$     & \SI{720}{\MeV} \\ \midrule
    $s$    & $\phi$     & \SI{1020}{\MeV} \\ \midrule
    $c$    & $J/\psi$   & \SI{3.1}{\GeV} \\ \midrule
    $b$    & $\Upsilon$ & \SI{9.46}{\GeV}
  \end{tabular}
  \caption{Threshold production for some $q\bar{q}$ pairs}
\end{table}
