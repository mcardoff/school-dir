% -*- TeX-master: "master.tex" -*-
\newcommand{\up}{\uparrow}
\newcommand{\dn}{\downarrow}
\section{Symmetry}
The study of symmetry has been a central theme in the study of physics. In 1917 Emmy Noether proved that (for a specific type of field-complex, 4-dimensional) a symmetry of the Hamiltonian implied the existence of a conserved quantity

You have studied the correspondence of symmetries and conserved quantities in classical and quantum mechanics:
\begin{itemize}
\item Time translation invariance $\iff$ Energy conservation
\item Space translation invariance $\iff$ Momentum conservation
\item Rotational invariance $\iff$ Angular momentum conservation
\end{itemize}

The construction of the Standard Model (and much of theoretical thought) has been largely the identification of the symmetries associated with observed conserved quantities.

\begin{aside}
  Not all conserved quantities must come from symmetries, e.g. solitons and other topological properties etc.
\end{aside}

A conserved quantity is one that is unchanged by a transformation, e.g. a Lorentz scalar under a Lorentz transformation.

The set of possible transformations of a given type have the properties of a \underline{group}. Hence, we study groups and algebras

\subsection{Group Theory}
Reminder of group definitions
\begin{enumerate}
\item Clousure: if $A\in G$, $B\in G$, then $AB\in G$
\item $\exists$ an identity: $I\in G$
\item Every element has an inverse: $A\in G$, $AA^{-1}=A^{-1}A=I$ with $A^{-1}\in G$
\item Associativity: $(AB)C=A(BC)$
\end{enumerate}

\subsubsection{Rotation Group}
Consider rotation by angle $\theta$ about the $z$-axis:
\begin{align*}
  \pmqty{x'\\y'\\z'}=
  \underbrace{\pmqty{\cos\theta & \sin\theta&\\
      -\sin\theta&\cos\theta&\\&&1}}_{R_z(\theta)}
  \pmqty{x\\y\\z}
\end{align*}
Similarly:
\begin{align*}
  R_x(\theta)=
  \pmqty{1\\&\cos\theta&\sin\theta\\&-\sin\theta&\cos\theta}
  \qquad
  R_y(\theta)=
  \pmqty{\cos\theta&&\sin\theta\\&1&\\-\sin\theta&&\cos\theta}
\end{align*}
Two successive rotations is also a rotation (closure). Can you quickly show the other group requirements?

Rotations leave the length of a vector invariant:
\begin{TODO}
  Show above
\end{TODO}

So rotations form a three-dimensional orthogonal group. In addition:
\begin{align*}
  \det{R}=\cos^2\theta+\sin^2\theta=1
\end{align*}
So the rotation group is $SO(3)$, S for special, O for orthogonal.

Recal the Lorentz group is $SO(3,1)$. Rotations are a \underline{subgroup} of the Lorentz Group:
\begin{align*}
  \Lambda^\mu_\nu=\pmqty{\text{Boosts}&\\&3\times3\text{ Rotations}}
\end{align*}
So the Lorentz group $=$ Boosts + Rotations

\subsubsection{Spin}
Consider a particle of mass $m$ at rest. Its four-momentum is:
\begin{align*}
  p^\mu=\pmqty{m\\0\\0\\0}
\end{align*}
Certainly the particle's four-momentum is invariant under rotations. The spin of the particle tells you how the wavefunction of the particle transforms  under the rotation group:
\begin{itemize}
\item Spin 0: $R=\bm{1}$ Trivial
\item Spin $\frac12$: $R_i=\exp{i\frac12\theta_i\sigma_i}$ where $\sigma_i$ are the Pauli matrices $(2\times2)$
\item Spin 1: $R=3\times3$ Rotation matrices (see above)
\item Spin $\frac32$: $R=4\times4$ matrices
\item Spin $j$: $R=(2j+1)\times(2j+1)$ matrices
\end{itemize}
So the spin of a particle is related to its \underline{representation} of the rotation group

Formally, a representation is a homomorphic mapping of a group onto the matrices $(GL_n)$ of rank $n$.

The symmetry associated with rotations is continuous. Hence, rather than work directly with the rotation matrices, we may work with the significantly simpler infinitesimal roations:

Let $R(\theta)=\exp{i\theta_iT_i}\equiv e^{i\theta_iT_i}$, where $T_i$ are matrices, $i=x,y,z$, and $\exp$ of a matrix is just the shorthand for:
\begin{align*}
  \exp{M}=e^M\equiv\sum_{i=0}^\infty\frac{M^n}{n!}
\end{align*}
We know that $R^TR\equiv\bm{1}$, so that $\exp{i\theta_iT_i}^T\exp{i\theta_iT_i}=\bm{1}$. Consider $\theta_i$ to be infinitesimal and expand:
\begin{align*}
  (\bm{1}+i\theta_iT_i^T+\cdots)
  (\bm{1}+i\theta_iT_i^T+\cdots)&=\bm{1}\\
  \implies \theta_i\qty[T_i^T+T_i]&=0
\end{align*}
Thus we arrive at $T_i^T=-T_i$, that is, $T_i$ are \underline{antisymmetric matrices}.

Consider: $R_x(\theta)$ with $\theta\ll 1$:
\begin{align*}
  R_x(\theta\ll1)&=\pmqty{1\\&1&-\theta\\&\theta&1}\\
  &=\bm{1}+\theta\pmqty{0&0&0\\0&0&-1\\0&1&0}
\end{align*}
From above, we can then identify $iT_x$ as this second matrix. Repeating the above process for $R_y,R_z$, we obtain a full set of generators of $SO(3)$:
\begin{align*}
  iT_x=\pmqty{0&0&0\\0&0&-1\\0&1&0}\quad
  iT_y=\pmqty{0&0&-1\\0&0&0\\1&0&0}\quad
  iT_z=\pmqty{0&-1&0\\1&0&0\\0&0&0}
\end{align*}

Consider two successive rotations:
\begin{gather*}
  R(\theta)R(\theta')=R(\theta'')\\
  R(\theta)=\bm{1}+i\theta\vdot T
\end{gather*}
Since they form a group, the right hand side should also be a rotation, and we have our definition of the generator below, with $\theta\vdot T=\theta_iT_i$. Plugging this form into the above:
\begin{align*}
  (\bm{1}+i\theta\vdot T+\frac12(i\theta\vdot T)^2+\cdots)
  (\bm{1}+i\theta'\vdot T+\frac12(i\theta'\vdot T)^2+\cdots)=
  (\bm{1}+i\theta''\vdot T+\frac12(i\theta''\vdot T)^2+\cdots)
\end{align*}
This gives you that $\theta+\theta'=\theta''$ to first order, $\order{\theta}$. If we go to second order, we would find that $\theta+\theta'=\theta''+\alpha$ where $\alpha\sim\order{\theta^2}$. Here we find that:
\begin{align*}
  -\frac12\qty[(\theta\vdot T)^2+(\theta'\vdot T)^2+2(\theta\vdot T)(\theta'\vdot T)]=-\frac12\qty[(\theta''\vdot T)^2-2i\alpha\vdot T]
\end{align*}
Now the second order term will tell us:
\begin{align*}
  (\theta''\vdot T)^2=\qty[\theta\vdot T+\theta'\vdot T-\alpha\vdot T]^2
\end{align*}
However we can essentially ignore the $\alpha\vdot T$ cross terms since all the other terms are at least order $\theta$, and the resulting multiplied term would be $\order{\theta^3}$ or greater, which we do not need to consider. Hence we find:
\begin{align*}
  (\theta''\vdot T)^2=(\theta\vdot T)^2+(\theta'\vdot T)^2+(\theta'\vdot T)(\theta\vdot T)+(\theta\vdot T)(\theta'\vdot T)
\end{align*}
This gives a relation between $\alpha$ and the other second order terms:
\begin{align*}
  -2i\alpha\vdot T=
  (\theta\vdot T)(\theta'\vdot T)-(\theta'\vdot T)(\theta\vdot T)
\end{align*}
Since $\alpha\sim\order{\theta^2}$, and for the above equation to hold it must be proportional to $\theta\theta'$. Let $\alpha_i\equiv-\frac12f_{jki}\theta_jtheta'_k$. We then find:
\begin{align*}
  if_{jki}\theta_j\theta'_kT_i&=\theta_j\theta'_k\qty[T_jT_k-T_kT_j]\\
  \implies if_{jki}T_i=T_jT_k-T_kT_j
\end{align*}
Define the commutator: $\comm{A}{B}\equiv AB-BA$, so that we can relabel:
\begin{align*}
  \boxed{\comm{T_i}{T_j}=if_{ijk}T_k}
\end{align*}
From this we find $f_{ijk}=-f_{jik}$. These are called structure constants, and are real valued.

This commutator above tells us that $SO(3)$ is a \underline{Lie Group}, i.e. a continuous group (on a differentiable manifold).

The generators $T_i$ of $SO(3)$ form a \underline{graded Lie algebra} with structure constants $f_{ijk}$

The structure constants for $SO(3)$ are $f_{ijk}=\veps_{ijk}$, with $\veps_{123}=+1$, the totally antisymmetric tensor. In other words the commutators are given as:
\begin{align*}
  \comm{T_x}{T_y}&=iT_z\\
  \comm{T_y}{T_z}&=iT_x\\
  \comm{T_z}{T_x}&=iT_y
\end{align*}

A \underline{representation} of the $SO(3)$ Lie algebra is any set of matrices which satisfy
\begin{align*}
  \comm{T_i}{T_j}=i\veps_{ijk}T_k
\end{align*}
The $3\times 3$ matrices we have written are called the \underline{fundamental representation} (also known as the \underline{defining representation}).

Recal this was associated with spin-1 particles.

\subsection{Spin $\frac12$ Representation}
The generators associated with spin-$\frac12$ are:
\begin{align*}
  T_x=\frac12\pmqty{&1\\1&}\quad
  T_y=\frac12\pmqty{&i\\-i&}\quad
  T_z=\frac12\pmqty{1& \\ &-1}
\end{align*}
Which are related to the Pauli Matrices, i.e. $T_i=\frac12\sigma_i$.

One can show that these $2\times 2$ matrices $T_i$ satisfy $\comm{T_i}{T_j}=i\veps_{ijk}T_k$, so they do form a representation of $SO(3)$.

The $T_i$ for this representation are \underline{not} antisymmetric, but they are Hermitian:
\begin{align*}
  T_i^\dag=T_i
\end{align*}
The rotation matrices $R=\exp{iT}$ are therefore \underline{unitary}:
\begin{align*}
  R^\dag R=(e^{iT})^\dag e^{iT}=e^{-iT^\dag}e^{iT}=e^{-iT}e^{iT}=\bm{1}
\end{align*}
To emphasize this, let's call these $2\times2$ matrices $U$ instead of $R$, so:
\begin{align*}
  U^\dag U=\bm{1}
\end{align*}

These $2\times2$ complex, unitary rotation matrices represent the group $SO(3)$, \underline{but} they also represent the group $SU(2)$, where $SU(2)$ means \textbf{s}pecial (determinant 1) \textbf{u}nitary $\mathbf{2\times2}$ matrices.

\begin{TODO}
  prove det $U$ is 1 in an aside
\end{TODO}

So we have proven the groups $SU(2)$ and $SO(3)$ have the \underline{same} Lie algebra. Thus the groups are identical as far as ``small'' rotations are concerned:
\begin{align*}
  SU(2)\underbrace{\approxeq}_{\text{Isomorphic}} SO(3)
\end{align*}

While the $3\times3$ $SO(3)$ matrices act on ordinary space $(x,y,z)$, the $2\times2$ complex $SU(2)$ matrices act on a complex, 2-dimensional ``spin'' space. We call this the \underline{fundamental} representation of $SU(2)$, but the \underline{spinor} representation of $SO(3)$.
\begin{remark}
  The so-called \underline{Classical Lie Groups} are $SO(n+1)$, $SU(n)$, and $SP(n+1)$, with the $p$ for symplectic. Some exceptions to these are $G_2, F_4, E_6, E_7, E_8$.

  It is also important to note that $SO(2)\approxeq U(1)$, so 2D rotations are equivalent to unitary numbers $e^{i\theta}$
\end{remark}

\subsection{Isospin}
Isospin (or Isotropic spin) is the symmetry of the strong interaction (QCD):
\begin{figure}[H]
  \centering
  \begin{tikzpicture}[scale=2.0]
    \begin{feynhand}
      % vertices
      \vertex (p11) at (1.5,0.5) {$u$};
      \vertex (p12) at (1.5,-0.5) {$\bar{u}$};
      \vertex (a1) at (0,0) {$g$};
      \vertex (b1) at (1,0);
      % propagators
      \propag [gluon] (a1) to (b1);
      \propag (b1) to (p11);
      \propag (b1) to (p12);
      \vertex (eq) at (1.75,0) {=};
      % vertices
      \vertex (p21) at (3.5,0.5) {$d$};
      \vertex (p22) at (3.5,-0.5) {$\bar{d}$};
      \vertex (a2) at (2,0) {$g$};
      \vertex (b2) at (3,0);
      % propagators
      \propag [gluon] (a2) to (b2);
      \propag (b2) to (p21);
      \propag (b2) to (p22);
    \end{feynhand}
  \end{tikzpicture}
  \caption{Isospin Conservation in QCD}
  \label{fig:isospin}
\end{figure}
So in QCD, $u$ and $d$ are interchangeable. This is the reason for isospin.

But, recall that $m_u\neq m_d$, how is it possible that we can interchange $u$ and $d$.
\begin{TODO}
  Add proton fig again
\end{TODO}
Recall the structure of the proton, and that $m_p\gg m_u,m_d$, this is because the proton mass is ``dynamical'' and is not due to the quark masses. So isospin is a good symmetry due to this difference in proton and up/down masses.

In group theory terms, we say that $\pmqty{u\\d}$ transform as a doublet under $SU(2)$ isospin

\begin{remark}
  \underline{\textbf{WARNING}}: $SU(2)$ ``isospin'' has nothing to do with $SU(2)$ spin! The mathematics is the same, but the physics is entirely different!
\end{remark}
One consequence of isospin is that $m_p=m_n$, this means that $\pmqty{p\\n}$ transform as an isospin doublet. Historically, this was the motivation for Heisenberg to suggest isospin differentiated ``nucleons,'' like $s_z=\pm\frac12$ differentiates the degeneracy in spin $\frac12$ systems.
\begin{note}
  The nuclear physicists use the much better term ``isobaric'' spin.

  E.g., Mirror nuclei, like $^7_3$Li/$^7_4$Be and $^{11}_5$B/$^{11}_6$C have nearly identical binding energies (after correcting for coulomb interaction)
\end{note}

\begin{definition}[Antiquarks]
  The so called antiquarks $\bar{u}$ and $\bar{d}$ are simply defined to be the complex conjugate of $u,d$.

  In group theory, this corresponds to $T_i$ representating the same algebra as $-T_i^*$:
  \begin{align*}
    \comm{T_i}{T_j}&=if_{ijk}T_k\\
    \comm{T_i^*}{T_j^*}&=-if_{ijk}T_k^*\\
    \implies\comm{(-T_i^*)}{(-T_j^*)}&=if_{ijk}(-T_k^*)
  \end{align*}
\end{definition}

For $SU(2)$, the complex conjugate representation is unitarily equivalent to the original representation:
\begin{align*}
  \underbrace{\pmqty{&-1\\1}}_{U}\qty(-T_i^*)
  \underbrace{\pmqty{&1\\-1}}_{U^\dag}=T_i
\end{align*}
For $T_i$ in the spin $\frac12$ representation, the representation is \underline{pseudoreal}.

Antiquarks transform as the c.c. rep., called the $\bar{2}$ representation:
\begin{align*}
  -T_i^*\pmqty{\bar{u}\\\bar{d}}&=\pmqty{\bar{u}'\\\bar{d}'}\\
  T_i\pmqty{-\bar{d}\\\bar{u}}&=\pmqty{-\bar{d}\\\bar{u}}
\end{align*}
Therefore, we can say that instead of using complex conjugate versions of the same algebra, we can use the usual spin-$\frac12$, instead on $\pmqty{-\bar{d}\\\bar{u}}$
\begin{definition}[Pions]
  Pions (or more generally mesons) are pairs of $q\bar{q}$, quarks and antiquarks.
\end{definition}
The strong interaction (QCD) not only binds quarks into $p,n$, it also binds quarks with antiquarks into \underline{pions}.

\begin{figure}[H]
  \centering
  \begin{TODO}
    Pion figure
  \end{TODO}
  \caption{Pion definition}
  \label{fig:pion}
\end{figure}
The charge of the pion is given by the sum of the charges of the $u$ and $\bar{d}$ quarks, that is $+\frac23+\frac13=1$.

In particular, pions are formed from the product of $\pmqty{u\\d}$ and $\pmqty{-\bar{d}\\u}$ isospin doublets. These are combined in the same way that we add spins:
\begin{example}[Adding Spins]
  Adding 2 spin-$\frac12$ particles, we have the following:
  \begin{align*}
    &\ket{J,J_z}\\
    &\ket{1, 1}=\ket{\up\up}\\
    &\ket{1, 0}
    =\frac1{\sqrt{2}}\qty(\ket{\up\dn}+\ket{\dn\up})\\
    &\ket{1,-1}=\ket{\dn\dn}\\
    &\ket{0,0}
    =\frac1{\sqrt{2}}\qty(\ket{\up\dn}-\ket{\dn\up})
  \end{align*}
  The first 3 are the spin 1 triplet, and the last 1 is the spin 0 singlet
\end{example}
\begin{example}[Adding IsoSpins]
  The math is exactly the same, except instead of $\up/\dn$ we will have $u,\bar{u},d,-\bar{d}$
  \begin{align*}
    &\ket{I,I_3}\\
    &\ket{1, 1}=\ket{u(-\bar{d})}\equiv\pi^+\\
    &\ket{1, 0}=\frac1{\sqrt{2}}\qty(\ket{u\bar{u}+d(-\bar{d})})\equiv\pi^0\\
    &\ket{1,-1}=\ket{d\bar{u}}\equiv\pi^-\\
    &\ket{0,0}=\frac1{\sqrt{2}}\qty(\ket{u\bar{u}-d(-\bar{d})})\equiv???
  \end{align*}
  The first 3 are the isospin 1 triplet, the pions, and the last is different, we will come back to it
\end{example}
So, we have isospin doublets of $\pmqty{p\\n}$ and $\pmqty{-n\\p}$, and the isospin triplet of $\pmqty{\pi^+\\\pi^0\\\pi^-}$. The pion masses are:
\begin{align*}
  m_{\pi^\pm}&=\SI{139.56}{\mega\eV}\\
  m_{\pi^0}&=\SI{134.97}{\mega\eV}
\end{align*}
\begin{note}
  Isospin is \underline{not} a symmetry of the electromagnetic interaction, that is:
  \begin{figure}[H]
  \centering
  \begin{tikzpicture}[scale=2.0]
    \begin{feynhand}
      % vertices
      \vertex (p11) at (1.5,0.5) {$u$};
      \vertex (p12) at (1.5,-0.5) {$\bar{u}$};
      \vertex (a1) at (0,0) {$\gamma$};
      \vertex (b1) at (1,0);
      % propagators
      \propag [gluon] (a1) to (b1);
      \propag (b1) to (p11);
      \propag (b1) to (p12);
      \vertex (eq) at (1.75,0) {$\neq$};
      % vertices
      \vertex (p21) at (3.5,0.5) {$d$};
      \vertex (p22) at (3.5,-0.5) {$\bar{d}$};
      \vertex (a2) at (2,0) {$\gamma$};
      \vertex (b2) at (3,0);
      % propagators
      \propag [gluon] (a2) to (b2);
      \propag (b2) to (p21);
      \propag (b2) to (p22);
    \end{feynhand}
  \end{tikzpicture}
  \caption{QED does \underline{not} care about isospin}
  \label{fig:isospin}
\end{figure}
This is because the electromagnetic charge of $u$ and $d$ are not equal, $Q_u=+\frac23$ and $Q_d=-\frac13$
\end{note}
Isospin has \underline{dynamical} implications. Consider $\pi N$ ($N=n,p$), that is $\pi N\to\pi'N'$. Since $\pi$ has $I=1$ and $N$ has $I=\frac12$, the state $(\pi N)$ can have $I=\frac12,\frac32$. Since $I$ is conserved, all $\pi N$ scattering processes can be reduced to just $2$ amplitudes.

\begin{example}[$\pi^+p\to\pi^+p$]
  Isospin: $\ket{\frac32,\frac32}\to\ket{\frac32,\frac32}$.

  The reaction is pure in isospin, so $A(\pi^+p\to\pi^+p)\equiv A_{3/2}$
\end{example}
Next consider a mixed process:
\begin{example}[$\pi^-p\to\pi^-p$]
  Isospin: We are adding $\ket{1,-1}$ to $\ket{\frac12,\frac12}$:
  \begin{align*}
    \ket{1,-1}\oplus\ket{\frac12,\frac12}
    =\frac1{\sqrt{3}}\ket{\frac32,-\frac12}
    -\sqrt{\frac23}\ket{\frac12,-\frac12}
  \end{align*}
  The reaction is no longer pure in isospin, but the amplitude would be the square of the coefficients: so $A(\pi^-p\to\pi^-p)= \frac13A_{3/2}+\frac23A_{1/2}$
\end{example}
The takeaway from this is that all ten $\pi N$ scattering amplitudes can be expressed in terms of $A_{3/2}$ and $A_{1/2}$.

Isospin conservation also imposes important constraints on string interaction processes.

\begin{example}[$d+d\to \,^4\text{He}+\pi^0$]
  Since the deuteron $d$ is a combination of $p\oplus n$, $I=0$, a similar argument shows $I=0$ for $^4\text{He}$, but $\pi^0$ has $I=1$. Hence this process would violate isospin conservation, and can only occur electromagnetically. It was only in the mid-2000's that experimenters at Indiana University claimed a detection of this rare process.
\end{example}

\begin{example}
  The branching fraction (\% of decays that occur in a given channel) for $\psi'\to J/\psi+\pi^0+\pi^0$ is $\sim20\%$, while $\psi'\to J/\psi+\pi^0$ is only $\sim0.1\%$. The $\psi'$ and $J/\psi$ are $c\bar{c}$ bound states ($I=0$), and again, the decay to one pion is inhibited, while the decay to two pions is not.
\end{example}

\begin{definition}{$\rho$ meson}
  The pions have spin $0$, the quark and antiquark have opposite spin:
  \begin{align*}
    \ket{\pi^+}=\ket{u(-\bar{d})}\otimes\underbrace{\frac1{\sqrt{2}}
      \qty(\ket{\up\dn-\dn\up})}
    _{\ket{0,0}\text{ state}}
  \end{align*}
  There is also a state with aligned spins:
  \begin{align*}
    \ket{\rho^+}=\ket{u(-\bar{d})}\otimes
    \begin{cases}
      \ket{\up\up} & =\ket{1,1} \\
      \frac1{\sqrt{2}}\ket{\up\dn+\dn\up}
      & =\ket{1,0} \\
      \ket{\dn\dn} & =\ket{1,-1}
    \end{cases}
  \end{align*}
\end{definition}
The $\rho$ (``rho'') is a spin-1 particle and an isospin triplet (like the pion):
\begin{align*}
  \pmqty{\rho^+\\\rho^0\\\rho^-}\text{ Isospin triplet}\quad
  m_{\rho^\pm}\approx m_{\rho^0}\equiv m_{\rho}\approx\SI{770}{\MeV}
\end{align*}

\begin{definition}{$\omega$ meson}
  This is the $I=0$ version of the $\rho$:
  \begin{align*}
    \ket{\omega}=\frac1{\sqrt{2}}\qty(\ket{u\bar{u}}+\ket{d(-\bar{d})})
    \otimes
    \begin{cases}
      \ket{\up\up} & =\ket{1,1} \\
      \frac1{\sqrt{2}}\ket{\up\dn+\dn\up}
      & =\ket{1,0} \\
      \ket{\dn\dn} & =\ket{1,-1}
    \end{cases}
  \end{align*}
\end{definition}
The $\omega$ is a singlet and has mass $m_\omega=\SI{783}{\MeV}\approx m_\rho$, this is NOT a consequence of isospin.

We will return to $I=0$ version of the pion later.

\subsection{Nomenclature}
\begin{definition}[Hadron]
  A hadron is a particle composed of quarks and/or antiquarks:
  \begin{enumerate}[label=(\alph*)]
  \item \underline{Mesons}: $q\bar{q}$ pairs like $\pi,\rho,\omega$
  \item \underline{Baryons}: $qqq$ states like $p,n$,

    Antibaryons are $\bar{q}\bar{q}\bar{q}$ states
  \end{enumerate}
\end{definition}

\begin{definition}[$\Delta$ baryon]
  This is the $I=\frac32$, $J=\frac32$ version of the proton:
  \begin{align*}
    \Delta^{++}&=``uuu''\\
    \Delta^{+ }&=``uud''\\
    \Delta^{0 }&=``udd''\\
    \Delta^{- }&=``ddd''
  \end{align*}
  With mass $m_\Delta=\SI{1232}{\MeV}$, we will write these more carefully in a moment.
\end{definition}
Consider the $\Delta^{++}$ with $J_z=+\frac32$, this is the state $\ket{\Delta^{++},J_z=+\frac32}=\ket{uuu}\otimes\ket{\up\up\up}$
This state is totally symmetric under the interchange of any pair of quarks. But the Pauli statistics that govern fermions insist this must be antisymmetric!

The solution? Color. We now write the state as $\ket{\Delta^{++},J_z=+\frac32}=\frac1{\sqrt{6}}\veps_{ijk}\ket{uuu}\otimes\ket{\up\up\up}$, where $i=1,2,3$ are the 3 ``colors'' red, green, blue. Thus we have totally symmetric \underline{quark} and \underline{spin} states, but an antiymmetric \underline{color} state, with a general state of
\begin{align*}
  \psi=\psi(\text{space})\otimes
  \psi(\text{flavor})\otimes
  \psi(\text{spin})\otimes
  \psi(\text{color})
\end{align*}
So the $\Delta$ states have ``flavor'' structure of:
\begin{align*}
  \Delta^{++}&=\ket{uuu}\\
  \Delta^{+ }&=\frac1{\sqrt{3}}\ket{uud+udu+duu}\\
  \Delta^{0 }&=\frac1{\sqrt{3}}\ket{udd+dud+ddu}\\
  \Delta^{- }&=\ket{ddd}
\end{align*}
With the color indices and spin states suppressed.

\textbf{Question}: The proton has $J=\frac12$. Is there are $\ket{uuu}$ state with $J=\frac12$? No, only $J=\frac32$ has a completely symmetric spin of three spin-$\frac12$ quarks

\subsection{Proton and Neutron}
$p,n,$ and $\Delta$ are all constructed from three $I=\frac12$ doublets:
\begin{align*}
  \pmqty{u\\d}\otimes\pmqty{u\\d}=
  \begin{cases}
    uu \\ \frac1{\sqrt{2}}(ud+du) \\ dd \\
    \frac1{\sqrt{2}}(ud-du)
  \end{cases}
\end{align*}
With the first 3 being isospin 1 and the last isospin 0.

In the language of group theory, this is:
\begin{align*}
  \underbrace{2}_{\text{doublet}}\otimes 2 = \underbrace{3}_{\text{triplet}}
  \oplus \underbrace{1}_{\text{singlet}}
\end{align*}
Let's add the third quark:
\begin{align*}
  \pmqty{u\\d}\otimes
  \pmqty{uu\\\frac1{\sqrt{2}}(ud+du)\\dd}
  =\pmqty{uuu\\\frac1{\sqrt{3}}duu+\sqrt{\frac23}u\frac1{\sqrt{2}}(ud+du)\\
    \frac1{\sqrt{3}}udd+\sqrt{\frac23}d\frac1{\sqrt{2}}(ud+du)\\ ddd}
  \oplus \pmqty{\sqrt{\frac23}duu-\frac1{\sqrt{6}}u(ud+du)\\
  \frac1{\sqrt{6}}d(ud+du)-\sqrt{\frac23}udd}
\end{align*}
Or $2\otimes3=4\oplus2$, The 4 is clearly the $\Delta$, but the doublet is NOT the $p,n$ doublet, since it is not symmetric in exchange of the first 2 quarks.

Of course there is also the state from $2\otimes 1$:
\begin{align*}
  \pmqty{u\\d}\otimes\frac1{\sqrt{2}}\qty(ud-du)=
  \begin{cases}
    \frac{1}{\sqrt{2}}u(ud-du)\\
    \frac{1}{\sqrt{2}}d(ud-du)
  \end{cases}
\end{align*}
Which is antisymmetric in the last two quarks.

Group theory: $2\otimes2\otimes2=2\otimes(3\oplus1)=4\oplus2_S\oplus2_A$, where $S,A$ mean symmetric/antisymmetric in the last 2 quarks.

To build $p,n$, we need to combine isospin with spin to make a totally symmetric state. Color will then make it antisymmetric as before.

We've shown spin and isospin share the same mathematics. Let's use that property. For spin, we also get $2\otimes2\otimes2=4\oplus2_S\oplus2_A$. So $\Delta=(4,4)$, where the first is isospin and the second is spin. This means that $p,n=(2_S,2_S)\oplus(2_A,2_A)$.

So to construct $p$ with $J_z=+\frac12$:
\begin{align*}
  \qty[\sqrt{\frac23}duu-\frac1{\sqrt{6}}u(ud+du)]\otimes
  \qty[\sqrt{\frac23}\dn\up\up-\frac1{\sqrt{6}}\up(\up\dn+\dn\up)]\oplus
  \qty[\frac1{\sqrt{2}}u(ud-du)]\otimes
  \qty[\frac1{\sqrt{2}}\up(\up\dn-\dn\up)]\\
  \equiv duu\qty[\frac23\dn\up\up-\frac13\up\up\dn-\frac13\up\dn\up]
  uud\qty[-\frac13\dn\up\up+\frac23\up\up\dn-\frac13\up\dn\up]
  udu\qty[-\frac13\dn\up\up-\frac13\up\up\dn+\frac23\up\dn\up]
\end{align*}
There is also an overall normalization of $\frac1{\sqrt{2}}$

State is symmetric under all interchanges.

Summary of hadrons composed of $u,d$ quarks:

\paragraph{Mesons}
\begin{align*}
  &J=0,I=1&\qquad & \pi^-,\pi^0,\pi^+   \qquad &m_\pi\approx\SI{140}{\MeV}\\
  &J=1,I=1&\qquad & \rho^-,\rho^0,\rho^+\qquad &m_\rho\approx\SI{776}{\MeV}\\
  &I=0    &\qquad & \omega              \qquad &m_\omega\approx\SI{783}{\MeV}
\end{align*}

\paragraph{Baryons}
\begin{align*}
  &J=\frac12,I=\frac12 &\qquad
  & n,p \qquad
  &m_{p/n}\approx\SI{939.565}{\MeV}\\
  &J=\frac32,I=\frac32 &\qquad
  & \Delta^-,\Delta^0,\Delta^+,\Delta^{++}\qquad
  &m_\rho\approx\SI{1232}{\MeV}
\end{align*}

\subsection{Broken Symmetry}
We have $p=uud$ and $n=udd$, with $m_N\sim\SI{938}{\MeV}$, which means the constituent quark masses should be $m_u,m_d\sim\SI{300}{\MeV}$, this is consistent across the $\rho,\omega$, not so much the $\Delta$, but approximately.

However, the $\pi$ is not consistent with this, in fact $m_\pi\approx\SI{140}{\MeV}\ll\rho(775)$