\documentclass{beamer}

\title{PHYS 19a Lab 1}
\author{Michael Cardiff}
\subtitle{How a measurement works}

\usepackage[italicdiff]{physics}
\usepackage{multicol}
\usepackage{hyperref}
\usepackage{url}

% Changes style of actual slides
\usetheme{Dresden}
% Changes color of slides
\usecolortheme{beaver}
% removes controls at bottom right side
\usenavigationsymbolstemplate{}

% for figures
\graphicspath{ {./figs/} }

\begin{document}

\begin{frame}
  \titlepage
\end{frame}

\section{Introduction}
\begin{frame}{Basic Idea}
  \begin{itemize}
  \item A crater of diameter $d$ was created by a ``meteor'' with kinetic energy $E$, according to the following equation:
    \begin{align*}
      d=CE^{1/4}
    \end{align*}
  \item Replace meteor with steel ball bearings and a planet with a bowl of sand 
  \item Turn gravitational potential energy ($V=mgy$) into kinetic energy, impacting the ``planet''
  \end{itemize}
\end{frame}

\section{Procedure}
\begin{frame}{One Height}
  \begin{columns}
    \begin{column}{0.5\textwidth}
      \begin{itemize}
      \item Many things we \emph{can} change, but we will focus on changing height
      \item Each height will have $n$ trials
      \item Average will be central value of measurement
      \item Error on average calculated from $\sigma$ (manually calc'd)
      \item Use students as our guinea pigs for test right size of ball
      \end{itemize}
    \end{column}
    \begin{column}{0.5\textwidth}
      \begin{figure}[H]
        \centering
        \includegraphics[width=5.0cm]{measurement}
        \caption{One measurement from many trials}
        \label{fig:1}
      \end{figure}
    \end{column}
  \end{columns}
\end{frame}

\begin{frame}{One Trial}
  \begin{itemize}
  \item One trial is done by
    \begin{enumerate}
    \item Ensuring the ``planet'' is sufficiently flat using rake and shaking the bowl
    \item Attaching the bearing to the solenoid
    \item (Optional) Place the shield on the bowl to prevent a splash of sand
    \item Turn off the solenoid and watch the ball drop
    \item Using the calipers, measure the diameter of the crater, lip to lip
    \item Go back to step 1
    \item Repeat $n$ times
    \end{enumerate}
  \item Once a trial is done, change the height, repeat above
  \item Maximize number of heights, suggestion: 6 heights, $n=8$
  \end{itemize}
\end{frame}

\section{Data Analysis}
\begin{frame}{Caveats}
  \begin{itemize}
  \item There are many innacuracies with the height (see the giant blob about it in the manual)
  \item Try to recognize them
  \item A few noticed in my run:
    \begin{itemize}
    \item Sand is not level with $0$ on measuring tape
    \item Bar holding solenoid droops
    \item Solenoid is not level with bar
    \item Ball is not a point, take diameter into account
    \end{itemize}
  \item All of these are mm or cm level, but significantly contribute to final result
  \end{itemize}
\end{frame}

\begin{frame}{Data Analysis}
  \begin{itemize}
  \item On one trial analysis, need to find average and error on average
  \item From height $\to$ potential energy
  \item Potential energy $\to$ $\ln(\text{potential energy})$
  \item Diameter $\to$ $\ln(\text{diameter})$
  \item Scatter plot with $\ln(E)$ on $x$ axis, $\ln(d)$ on $y$ axis
  \item Add Linear trendline, slope should be \underline{\qquad} based on initial law (tell me?)
  \end{itemize}
\end{frame}

\begin{frame}{What YOU Need to know}
  \begin{itemize}
  \item Helpful to have excel sheet open with data collection
  \item Include offsets, errors, etc. in sheet
  \item Actual data later in sheet
  \item Trendline does not include error on fit parameters, use function \texttt{LINEST}, I have a presentation on this because its a bit clunky to use
  \item Do not use a million decimal places, limit to $2-3$ significant digits
  \end{itemize}
\end{frame}

\begin{frame}{Basics of \texttt{LINEST}}
  \begin{columns}
    \begin{column}{0.5\textwidth}
      \begin{itemize}
      \item \texttt{LINEST} is a matrix function, so it will give data in multiple cells if you provide them
      \item Technically can be used for a general degree polynomial (Trendline option supports this as well)
      \end{itemize}
    \end{column}
    \begin{column}{0.5\textwidth}
      \begin{figure}[H]
        \centering
        \includegraphics[width=5.0cm]{linest}
        \caption{Example output of \texttt{LINEST}}
        \label{fig:3}
      \end{figure}
    \end{column}
  \end{columns}
\end{frame}
\begin{frame}{Basics of \texttt{LINEST} Cont.}
  \begin{itemize}
  \item Actual function call: \texttt{LINEST(y\textunderscore data,x\textunderscore data,intercept?,extra?)}
  \item First two arguments are hopefully trivial
  \item Third argument: if FALSE, set constant term to $0$:
    \begin{align*}
      \text{FALSE: }y&=mx\\
      \text{TRUE: }y&=mx+b
    \end{align*}
  \item Fourth arguemnt should always be TRUE, as it determines whether or not you get your extra data, i.e. the errors
  \item Only care about first two rows of output, the first row being the parameter value, and the second its associated error in decreasing degree
  \end{itemize}
\end{frame}

\section{Results}
\begin{frame}{Proof it Works!}
  \begin{figure}[H]
    \centering
    \includegraphics[width=7.5cm]{chart}
    \caption{Example of log-log Plot}
    \label{fig:2}
  \end{figure}
\end{frame}

\begin{frame}{Overarching Goals}
  \begin{itemize}
  \item Establish expectations for the course
  \item Familiarize students with what a measurement is
  \item Show students their data analysis tool for the semester (Excel)
  \item Introduce students to error propagation
  \item Let the students know who you are as a teacher
  \end{itemize}
\end{frame}
\end{document}