\documentclass[aspectratio=169]{beamer}

\title{PHYS 19a Lab 1}
\author{Michael Cardiff}
\subtitle{What is a Measurement?}

\usepackage[italicdiff]{physics}
\usepackage{multicol}
\usepackage{hyperref}
\usepackage{xcolor}
\usepackage{siunitx}
\usepackage{url}

% Changes style of actual slides
\usetheme{Dresden}
% Changes color of slides
\usecolortheme{spruce}
% removes controls at bottom right side
\usenavigationsymbolstemplate{}

% for figures
\graphicspath{ {./figs/} }

\begin{document}

\begin{frame}
  \titlepage
\end{frame}

\section{Intro}
\begin{frame}{What is a Measurement}
  \begin{itemize}
  \item All measurements have a {\color{red}\textbf{central value}} and an {\color{blue}\textbf{error}}
    \begin{align*}
      m={\color{red}v}\pm{\color{blue}\sigma_v}
    \end{align*}
  \item Every time you measure something in this lab, the value you find on your measuring device acts as the central value, ${\color{red}v}$.
  \item The error on your measurement, usually called ${\color{blue}\sigma_v}$, is determined by how good your measurement device is.
    \begin{itemize}
    \item The error on all measurement devices should be the smallest unit the device can measure
    \end{itemize}
  \end{itemize}
\end{frame}

\begin{frame}
  \frametitle{Example: Meterstick}
  \begin{columns}
    \begin{column}{0.5\textwidth}
      \begin{itemize}
      \item A meterstick usually gives a central value in cm, with one decimal point of accuracy.
      \item This means the error on a measurement with that meterstick will be $0.1$ cm, or 1 mm
      \item Or, we could say $\sigma_{\text{meterstick}}=1$mm
      \end{itemize}
    \end{column}
    \begin{column}{0.5\textwidth}
      \begin{figure}[H]
        \centering
        \includegraphics[width=5.0cm]{meterstick}
        \caption{A meterstick with 1mm resolution}
      \end{figure}
    \end{column}
  \end{columns}
\end{frame}

\begin{frame}
  \frametitle{Speaking of Units...}
  \begin{itemize}
  \item There are a lot of different units you will be using, so it is best to keep it to a single system you can make sense of all your numbers
  \item Ensure you are using SI units for \emph{everything}, this means:
    \begin{table}[H]
      \centering
      \begin{tabular}{c|c}
        Measurement & SI Unit \\\hline
        Length      & Meter (m) \\
        Mass        & Kilogram (kg)
      \end{tabular}
      \caption{Guide for units}
    \end{table}
  \item So it will be necessary to convert your raw measurements to SI units before doing your analysis
  \end{itemize}
\end{frame}

\section{Procedure}
\begin{frame}
  \frametitle{The Lab!}
  \begin{itemize}
  \item This lab is meant to hone your measurement skills, so the rest of the semester can go smoothly, specifically:
    \begin{itemize}
    \item Getting a central value and error
    \item Propagating error
    \item Finding averages \& standard deviations
    \item Finding error on averages
    \end{itemize}
  \item This will all be done via testing a supposed relationship between the kinetic energy $(E)$ of a meteor to the diameter ($d$) of the crater it creates:
    \begin{align*}
      d=CE^{1/4}
    \end{align*}
  \end{itemize}
\end{frame}

\begin{frame}
  \frametitle{The Idea}
  \begin{itemize}
  \item Our method of ``generating'' kinetic energy is through the height of the drop $h$, given by:
    \begin{align*}
      E=mgh
    \end{align*}
    Where $m$ is the mass of the ball and $g=9.80\pm0.01\unit{ms^{-2}}$
  \item So we need a number of \textbf{diameters} and \textbf{heights} to test this relationship
  \end{itemize}
\end{frame}

\begin{frame}
  \frametitle{One Diameter Measurement}
  \begin{itemize}
  \item Setting the height, you will take between 8-10 measurements of the diameter
  \item The actual data point you will be using will be the {\color{red}average} of those values, and its error will be the {\color{blue}error on the average}
    \begin{gather*}
      {\color{red}\bar{d}=\frac1N\sum_{i=1}^Nd_i}\qquad
      {\color{blue}\sigma_{\bar{d}}=\frac{\sigma}{\sqrt{N}}}
    \end{gather*}
    Note $N$ is the number of measurements, $d_i$ are the diameter for each trial, and $\sigma$ is the standard deviation.
  \item Repeat the measurement for at least 6 different heights
  \end{itemize}
\end{frame}

\begin{frame}
  \frametitle{Efficient Data Taking}
  \begin{itemize}
  \item You should be in groups of 2, one should be dropping the ball and recording data, while the other
  \item Take time to set up your excel spreadsheet before starting to take data, it can significantly speed up your data taking
  \item Please note however, you should NOT use the built in functions for the average and standard deviation from excel, as they are formulae you should know by hand\footnote{You can however still use functions like \texttt{COUNT} or \texttt{SUM}, if unsure, ask}
  \end{itemize}
\end{frame}

\section{Analysis}
\begin{frame}
  \begin{columns}
    \begin{column}{0.5\textwidth}
      \frametitle{Data Analysis}
      \begin{itemize}
      \item Once you finish collecting data, you will need to create a ``log-log'' plot, with $\ln(E)$ on the $x$ axis, and $\ln(d)$ on the $y$ axis
      \item They should form a relatively straight line, so create a trendline
      \item \textbf{Question}: What should the slope be? How could you find this?
      \end{itemize}
    \end{column}
    \pause
    \begin{column}{0.5\textwidth}
      \begin{figure}[H]
        \centering
        \includegraphics[width=7.0cm]{chart}
        \caption{Example Log-Log Plot}
      \end{figure}
    \end{column}
  \end{columns}
\end{frame}

\begin{frame}
  \frametitle{Error on the Fit Parameter(s)}
  \begin{columns}
    \begin{column}{0.6\textwidth}
      \begin{itemize}
      \item The trendline will show you only the central value of the fit parameters, we use the excel function \texttt{LINEST} to get the error
      \item In a cell, type \texttt{=LINEST(y\_data,x\_data,TRUE,TRUE)}
      \item Upon pressing enter, you will see the something like in the figure
      \item Top row is values of fit parameters
      \item Second row is the error associated with the parameter above it
      \end{itemize}
    \end{column}
    \begin{column}{0.4\textwidth}
      \begin{figure}[H]
        \centering
        \includegraphics[width=5.0cm]{linest}
        \caption{Example \texttt{LINEST}}
      \end{figure}
    \end{column}
  \end{columns}
\end{frame}

\begin{frame}
  \frametitle{Error Propagation}
  \begin{itemize}
  \item You will be asked to propagate one error in this lab to get used to it
  \item This will be the energy, $E=mgh$
    \begin{align*}
      \sigma_E^2=\qty(\pdv{E}{m})^2\sigma_m^2+
      \qty(\pdv{E}{g})^2\sigma_g^2+
      \qty(\pdv{E}{h})^2\sigma_h^2
    \end{align*}
  \item Error on $m$ and $g$ is trivial, the complicated part is the height
  \item There are a number of offsets you will need to find (lab manual)
  \item Each offset as well as the raw height measurement has an error, but the errors will add together quite simply
  \end{itemize}
\end{frame}

\section{Conclusion}
\begin{frame}
  \frametitle{Bookkeeping}
  \begin{itemize}
  \item I have office hours, Tuesday 1:00-3:00pm in the Resource Room
  \item Other TAs have office hours, see LATTE page for details
  \item I will give you a short, graded attendance quiz in every lab
  \item Do not be afraid to ask for help
  \item Questions?
  \end{itemize}
\end{frame}

\end{document}