\documentclass[handout,aspectratio=169]{beamer}

\title{Lab 5}
\author{Michael Cardiff}
\subtitle{Work and Position Dependent Forces}

\usepackage[italicdiff]{physics}
\usepackage{hyperref}
\usepackage{url}
\usepackage{wasysym}

% Changes style of actual slides
\usetheme{Dresden}
% Changes color of slides
\usecolortheme{spruce}
% removes controls at bottom right side
\usenavigationsymbolstemplate{}

% for figures
\graphicspath{ {./figs/} }

\begin{document}

\begin{frame}
  \titlepage
\end{frame}

\section{Theory}
\begin{frame}
  \frametitle{Springs}
  \begin{columns}
    \begin{column}{0.5\textwidth}
      \begin{itemize}
      \item A simple spring system has a spring constant $k$ and a rest length $x_0$
      \item The spring ``wants'' to be at its rest length, so the spring force should accelerate the spring to its rest length:
        \begin{align*}
          F_{spring}(x)=k(x-x_0)
        \end{align*}
      \item Sadly, we cannot assume this form for the force:
        \begin{align*}
          F_{spring}=\,???
        \end{align*}
      \end{itemize}
    \end{column}
    \begin{column}{0.5\textwidth}
      \begin{figure}[H]
        \centering
        \includegraphics[width=3.5cm]{simplespring}
        \includegraphics[width=5.0cm]{complexspring}
        \caption{Simplified vs. Complex Spring}
      \end{figure}
    \end{column}
  \end{columns}
\end{frame}

\begin{frame}
  \frametitle{Dynamic vs. Static Force}
  \begin{itemize}
  \item We will measure the force in 2 different ways:
    \begin{enumerate}
    \item With the glider static, balancing the spring force with another force (gravity), so the numerical value of the force is $F_s=mg$
    \item With the glider moving (dynamic) measuring the acceleration, and the force would be $F_s=ma$
    \end{enumerate}
  \item Ideally, these would give the same acceleration, but this is not necessarily trivial
  \end{itemize}
\end{frame}

\begin{frame}
  \frametitle{The Work Energy Theorem}
  \begin{itemize}
  \item In a system with low dissipation (remember $\mu$ from last lab), mechanical energy is conserved:
    \begin{align*}
      E^{mech}_i=E^{mech}_f
    \end{align*}
  \item Mechanical energy here comes in the form of Work and Kinetic Energy, from work we get:
    \begin{align*}
      W=\int_i^fF\dd{x}=\int_i^fm\dv{v}{t}\dd{x}=\int_i^fmv\dv{v}{t}\dd{t}
      =\frac12mv^2_f-\frac12mv^2_i=\Delta K
    \end{align*}
  \item So we have the Work-Energy Theorem:
    \begin{align*}
      W=\Delta K
    \end{align*}
  \end{itemize}
\end{frame}

\section{Procedure}
\begin{frame}
  \frametitle{Static Force Measurement}
  \begin{itemize}
  \item First we will use a string and pulley system to put a force on the system, so the force from the spring will balance out the force from the pulley
  \item We will measure the position of the back of the glider after it reaches equilibrium, this is what the manual calls $x_g$:
    \begin{align*}
      F_{spring}(\underbrace{x_0-x_g}_{x_m})=mg
    \end{align*}
  \item Where $x_0$ is the rest length the position of the back of the glider with no mass attached, do not to forget to measure it.
  \end{itemize}
\end{frame}

\begin{frame}
  \frametitle{Dynamic Force Measurement}
  \begin{itemize}
  \item Now we will let the spring launch the glider, and measure the acceleration at $x_m$
  \item To make this measurement accurately, we need to launch the glider a bit further back from the $x_m$ measured in the previous part, the lab manual suggests launching from $x_m-8cm$, so you will have several points before and after $x_m$
  \item You should align the photogate so its beam will be broken as soon as the air is turned on and the spring begins to accelerate, be accurate with this.
  \item Get 3 measurements of the acceleration for each mass.
  \item There is no need for a measurement from $x_0$ as it should not get launched at all.
  \end{itemize}
\end{frame}

\begin{frame}
  \frametitle{Measuring the Max Velocity}
  \begin{itemize}
  \item This time, place the photogate so that the glider breaks the beam as soon as it loses contact with the spring (should be a bit ahead of $x_0$)
  \item Launch the glider from the $x_m$ for the 50g mass, do not move it back by 8cm
  \item Notice the velocity should slowly decrease after this point (Why? Lab 4), so the ``max'' should be the first velocity value
  \item Measure this value 3 times to get the mean $\pm$ error on the mean
  \end{itemize}
\end{frame}

\section{Analysis}
\begin{frame}
  \frametitle{Checking Work Energy Theorem}
  \begin{itemize}
  \item Notice that work is an integral, we can't do that...
  \item Instead do numerical \emph{integration}:
    \begin{align*}
      \int F\dd{x}\to\sum F\Delta{x}
    \end{align*}
  \item Our $F_n$: The average force over the interval $\Delta{x}$
  \item We label the force for the 50g mass as $F_1$, and $x_1$ would be the corresponding spring stretch, $x_m$
  \item This gives the following formula:
    \begin{align*}
      W=\frac12\sum_{n=1}^{n=10}(F_n+F_{n+1})(x_n-x_{n+1})
    \end{align*}
  \end{itemize}
\end{frame}

\begin{frame}
  \frametitle{Calculating $W$}
  \begin{columns}
    \begin{column}{0.6\textwidth}
      {\footnotesize \begin{align*}
        W=\frac12\sum_{n=1}^{n=10}(F_n+F_{n+1})(x_n-x_{n+1})
      \end{align*}}
      \begin{itemize}
      \item This formula looks a bit complicated, but it really does down to calculating two things:
        \begin{enumerate}
        \item Average force in the interval
        \item Distance traveled in that interval
        \end{enumerate}
      \item I recommend having every individual part of the sum calculated in the table 
      \item Note that $F_{11}$ and $x_{11}$ should both be 0, here I just did not include them in the formula
      \end{itemize}
    \end{column}
    \begin{column}{0.4\textwidth}
      \begin{figure}[H]
        \centering
        \includegraphics[width=5cm]{table}
        \caption{My Table for Calculating $W$}
      \end{figure}
    \end{column}
  \end{columns}
\end{frame}
\begin{frame}
  \begin{table}[H]
    \begin{tabular}{c|c|c|c|c|c|c}
      $n$ & $m_n$ (g) & $F_n$ (N) & $x_n$ (m) & $F_n+F_{n+1}$ (N) & $x_n-x_{n+1}$ (m) & product (J) \\ \hline
      10 & 5.4  & 0.029 & 0.019 & 0.029 & 0.019 & 0.0006 \\
      9  & 10.6 & 0.027 & 0.040 & 0.056 & 0.021 & 0.0012 \\
      8  & 15.6 & 0.080 & 0.049 & 0.107 & 0.009 & 0.0010 \\
      7  & 20.5 & 0.121 & 0.073 & 0.201 & 0.024 & 0.0048 \\
      6  & 25.5 & 0.162 & 0.092 & 0.283 & 0.019 & 0.0054 \\
      5  & 30.5 & 0.182 & 0.107 & 0.344 & 0.015 & 0.0052 \\
      4  & 35.7 & 0.224 & 0.118 & 0.405 & 0.011 & 0.0045 \\
      3  & 40.9 & 0.242 & 0.130 & 0.466 & 0.012 & 0.0056 \\
      2  & 46.1 & 0.282 & 0.139 & 0.524 & 0.009 & 0.0047 \\
      1  & 51.2 & 0.299 & 0.149 & 0.581 & 0.010 & 0.0058
    \end{tabular}
    \caption{Bigger Version of My Table with Masses}
  \end{table}
\end{frame}

\begin{frame}
  \frametitle{Error Analysis}
  \begin{itemize}
  \item[\Large\frownie] Error on $W$ is hard, you can calculate it if you want, but it is not necessary:
    \begin{align*}
      \sigma_W^2=\sum_{i=1}^{10}\sigma_{W_i}^2,\quad
      \sigma_{W_i}^2=\sum_{j=1}^{10}\qty(\qty(\pdv{W_i}{F_j})^2\sigma_{F_j}^2
      +\qty(\pdv{W_i}{x_j})^2\sigma_{x_j}^2)
    \end{align*}
  \item[\Large\smiley] The error on Kinetic energy is much simpler:
    \begin{align*}
      \sigma_K^2=\qty(\pdv{K}{m})^2\sigma^2_{m}+\qty(\pdv{K}{v})^2\sigma^2_{v}
    \end{align*}
  \item Only do the error on $K$, \pause trust me
  \end{itemize}
\end{frame}

\section{Conclusion}
\begin{frame}
  \frametitle{Bookkeeping}
  \begin{itemize}
  \item Lab 5 Due Next Friday, November 10th
  \item Labs 4 on the way
  \end{itemize}
\end{frame}

\end{document}