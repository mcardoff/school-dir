\documentclass[aspectratio=169]{beamer}

\title{PHYS 19a Lab 2}
\author{Michael Cardiff}
\subtitle{Measuring $g$}

\usepackage[italicdiff]{physics}
\usepackage{hyperref}
\usepackage{url}

% Changes style of actual slides
\usetheme{Dresden}
% Changes color of slides
\usecolortheme{spruce}
% removes controls at bottom right side
\usenavigationsymbolstemplate{}

% for figures
\graphicspath{ {./figs/} }

\def\n{5 }

\begin{document}

\begin{frame}
  \titlepage
\end{frame}

\section{Intro}
\begin{frame}
  \frametitle{Who am I?}
  \begin{itemize}
  \item I am a Second year PhD Student in Physics here at Brandeis
  \item My research is in High Energy Experimental Physics
  \item I have a dog named Linguine
  \item I can solve a Rubik's Cube (no I did not bring one today)
  \end{itemize}
\end{frame}

\section{Theory}
\begin{frame}
  \frametitle{Describing a falling object}
  \begin{itemize}
  \item An object in free fall will only accelerate from gravity
  \item Newton then tells us:
    \begin{align*}
      ma=F_{\text{net}}=-mg
    \end{align*}
  \item The equation of motion is:
    \begin{equation}
      \label{eq:1}
      \boxed{y(t)=y_0+v_0t-\frac12gt^2}
    \end{equation}
    Where $y_0$ is the initial position, and $v_0$ is the initial velocity
  \item This is what we will be testing in this lab
  \end{itemize}
\end{frame}

\section{Collecting Data}
\begin{frame}
  \frametitle{One Trial}
  \begin{itemize}
  \item This lab will consist of you taking \n videos of a falling ball
  \item You will record the video using VirtualDub, read the lab manual for the specific setup instructions
  \item I recommend having 1 person in control of the computer, and another dropping the ball and setting it back up.
  \item Once you have all \n videos, you can analyze them
  \end{itemize}
\end{frame}

\begin{frame}
  \frametitle{Analysis in Tracker}
  \begin{itemize}
  \item The Tracker software is on all of the lab computers, you may have to search for it in the search bar. 
  \item Once you open your video in Tracker, you need to set up your \textbf{axes} and your \textbf{calibration stick}
    \begin{itemize}
    \item The axes give you a reference frame to analyze where the ball is coming from
    \item The calibration stick will turn pixels into meters.
    \end{itemize}
  \item The ball will not be very well defined the faster it falls, so try to track the top of the ball as it falls
  \item Once you have tracked the ball as much as you could, copy the data to Excel, this will be your first trial, repeat the analysis for each video
  \end{itemize}
\end{frame}

\begin{frame}
  \frametitle{Make Sure}
  \begin{itemize}
  \item The camera sees the entire meterstick
  \item The camera's FPS is set to 30
  \item The camera is at a right angle to the setup (visible on video preview)
  \item Track the ball only when it is moving
  \end{itemize}
\end{frame}

\section{How to use your data}
\begin{frame}
  \frametitle{Parabolic Fit}
  \begin{columns}
    \begin{column}{0.5\textwidth}
      \begin{itemize}
      \item This lab will have you taking position data from a video of a falling ball.
      \item Plotting position vs. time ($y$ vs $t$) should give you a parabola, where the coefficient of $t^2$ is $g/2$, from eq \eqref{eq:1}
      \item To do this in excel, change the fit type (right hand side) from \texttt{Linear} to \texttt{Polynomial}, and choose Order 2
      \end{itemize}
    \end{column}
    \begin{column}{0.5\textwidth}
      \begin{figure}[H]
        \centering
        \includegraphics[width=2.2cm]{quadfit}
        \caption{Example of a parabolic fit in Excel}
      \end{figure}
    \end{column}
  \end{columns}
\end{frame}

\begin{frame}
  \frametitle{Linear Fit}
  \begin{itemize}
  \item Taking a derivative of \eqref{eq:1} with respect to time gives a velocity vs time equation of:
    \begin{equation}
      \label{eq:2}
      \dv{y(t)}{t}=v(t)=v_0-gt
    \end{equation}
  \item Hence, if we find a way to take the derivative of our position data to get velocity data, and plot that against time, we should have a straight line with slope $-g$
  \item This may change to $v=v_0+gt$ if your axes definition changes, eqs \eqref{eq:1} and \eqref{eq:2} depend on the object decreasing its position from $0$, this may change in practice as you do the lab
  \end{itemize}
\end{frame}

\begin{frame}
  \frametitle{One More Derivative}
  \begin{itemize}
  \item Taking another derivative of eq \eqref{eq:2} will give an equation for the acceleration:
    \begin{equation}
      \label{eq:4}
      \dv[2]{y(t)}{t}=a(t)=-g
    \end{equation}
  \item So if we do this to our velocity data, we should have a lot of independent measurements of $g$.
  \item A best estimate would be taking the average of these values.
  \end{itemize}
\end{frame}

\begin{frame}
  \frametitle{End Result: What did you find for $g$?}
  \begin{itemize}
  \item In the end, you should have 3 estimates of $g$ with varying accuracy:
    \begin{enumerate}
    \item From fitting a parabola to your $y$ data.
    \item From differentiating your $y$ data and fitting a line to that $v$ data.
    \item From differentiating your $y$ data twice, and taking the average of those values.
    \end{enumerate}
  \item To get the errors on 1 and 2, use \texttt{LINEST}, and the error for the last one would be the error on the mean. 
  \end{itemize}
\end{frame}

\section{Analysis Techniques}
\begin{frame}
  \frametitle{How to take a derivative}
  \begin{itemize}
  \item The problem now is that we cannot take a proper derivative of discrete data. 
  \item The solution: discretize the derivative:
    \begin{equation}
      \label{eq:3}
      f'(t)=\lim_{\Delta t\to0}\frac{f(t+\Delta t)-f(t)}{\Delta t}
      \approx \frac{f(t+\Delta t)-f(t-\Delta t)}{2\Delta t}
    \end{equation}
  \item More details in the lab manual
  \item Doing this in excel, notice how for the endpoints of your data will not have a next ($y(t+\Delta t)$) or previous ($y(t+\Delta t)$) datapoint, keep this in mind when doing your analysis
  \end{itemize}
\end{frame}

\begin{frame}
  \frametitle{Derivative Example in Excel}
  \begin{columns}
    \begin{column}{0.5\textwidth}
      \begin{itemize}
      \item To properly implment the right hand side of eq \eqref{eq:3}, be mindful of which points you can actually take the derivative of
      \item Notably the end points, if you are trying to take a derivative at $t=0$, there is no $t=-\Delta t$ in your dataset, so you will get junk out
      \end{itemize}
    \end{column}
    \begin{column}{0.5\textwidth}
      \begin{figure}[H]
        \centering
        \includegraphics[width=6.0cm]{derivex}
        \caption{Example of a derivative in excel}
        \label{fig:2}
      \end{figure}
    \end{column}
  \end{columns}
\end{frame}

\begin{frame}
  \frametitle{Error on Quadratic Fit Parameters}
  \begin{itemize}
  \item To get the error on fit parameters, you need to change a few things about your call to \texttt{LINEST}:
    \begin{center}
      Before: \texttt{LINEST(y\_data,x\_data,TRUE,TRUE)}
      \footnote{Still use this one for the $v$ vs $t$ fit}\\
      Now: \texttt{LINEST(y\_data,{\color{red} x\_data\^{}\{1,2\}},TRUE,TRUE)}
    \end{center}
  \item Note because now we have 3 fit parameters ($ax^2+bx+c$) as opposed to 2 ($ax+b$), you will need to select a $3\times2$ area of cells as opposed to a $2\times2$
  \end{itemize}
\end{frame}

\section{Conclusion}
\begin{frame}
  \frametitle{Notes About Labs}
  \begin{itemize}
  \item Rubric will tell you everything you need to put in the lab report
  \item Take note of table number as well as partner's name
  \end{itemize}
\end{frame}
\begin{frame}
  \frametitle{Bookkeeping}
  \begin{itemize}
  \item This lab is due Next Friday, September 29
  \item Labs 1 will be returned before then so you can use the feedback
  \end{itemize}
\end{frame}

\end{document}