\documentclass[aspectratio=169]{beamer}

\title{PHYS 19a Lab 4}
\author{Michael Cardiff}
\subtitle{The Incline Plane}

\usepackage[italicdiff]{physics}
\usepackage{hyperref}
\usepackage{url}

% Changes style of actual slides
\usetheme{Dresden}
% Changes color of slides
\usecolortheme{spruce}
% removes controls at bottom right side
\usenavigationsymbolstemplate{}

% for figures
\graphicspath{ {./figs/} }

\begin{document}

\begin{frame}
  \titlepage
\end{frame}

\section{Theory}
\begin{frame}
  \frametitle{The Physics}
  \begin{columns}
    \begin{column}{0.5\textwidth}
      \begin{itemize}
      \item On an inclined plane, it is best to use a reference frame where the axes are aligned with the angle of the plane
      \item Result, $(x,y)$ are now $(\parallel,\perp)$, but we can still call them $(x,y)$
      \item This allows us to resolve only 1 force into these different components (gravity) rather than the other two that are present in these problems (normal force and friction)
      \end{itemize}
    \end{column}
    \begin{column}{0.5\textwidth}
      \begin{figure}[H]
        \centering
        \includegraphics[width=7.0cm]{plane}
        \caption{Inclined Plane Axes}
      \end{figure}
    \end{column}
  \end{columns}
\end{frame}

\begin{frame}
  \frametitle{Friction}
  \begin{itemize}
  \item The friction force depends on ``how much'' an object presses down on the surface
    \begin{align*}
      \vb{F}_{fric}=\mu\vb{N}
    \end{align*}
  \item We characterize this by the normal force $\vb{N}$ (Newton 3), and the friction force is proportional to it by the coefficient of (kinetic or static) friction.
  \item This lab focuses on kinetic friction, but both are given by:
    \begin{align*}
      \vb{F}_{kinetic}=\mu_k\vb{N}\qquad \vb{F}_{static}\leq\mu_s\vb{N}
    \end{align*}
  \item The friction force's direction is always in opposition to the motion
  \end{itemize}
\end{frame}

\begin{frame}
  \frametitle{Up the Ramp}
  \begin{columns}
    \begin{column}{0.5\textwidth}
      \begin{itemize}
      \item Going up the ramp, the friction force goes \emph{down} the ramp
      \item Procedure: resolve $\vb{F}_g$ into parallel and perpendicular axes, solve for $\vb{F}_{N}$ and $\vb{F}_{fric}$
      \item Use Newton 2 to solve for $\vb{a}$
      \end{itemize}
    \end{column}
    \begin{column}{0.5\textwidth}
      \begin{figure}[H]
        \centering
        \includegraphics[width=7.0cm]{upramp}
        \caption{FBD for up the ramp}
      \end{figure}
    \end{column}
  \end{columns}
\end{frame}

\begin{frame}
  \frametitle{Solving Up the Ramp}
  \begin{itemize}
  \item Gravity resolves into 2 components:
    \begin{align*}
      F_{y}&=-mg\cos\theta\\
      F_{x}&= mg\sin\theta
    \end{align*}
  \item No acceleration in $y$ direction (it doesn't float off the ramp), so we have:
    \begin{align*}
      \sum F_y=0\implies F_N-mg\cos\theta=0\implies F_N=mg\cos\theta
    \end{align*}
  \item Acceleration in $x$ gives $a_{up}$:
    \begin{align*}
      ma_{up}&=\mu mg\cos\theta+mg\sin\theta\\
      \implies a_{up}&=\mu g\cos\theta+g\sin\theta
    \end{align*}
  \end{itemize}
\end{frame}

\begin{frame}
  \frametitle{Down the Ramp}
  \begin{columns}
    \begin{column}{0.5\textwidth}
      \begin{itemize}
      \item Going up the ramp, the friction force goes \emph{up} the ramp
      \item Same procedure as before
      \end{itemize}
    \end{column}
    \begin{column}{0.5\textwidth}
      \begin{figure}[H]
        \centering
        \includegraphics[width=7.0cm]{downramp}
        \caption{FBD for down the ramp}
      \end{figure}
    \end{column}
  \end{columns}
\end{frame}

\begin{frame}
  \frametitle{Solving Down the Ramp}
  \begin{itemize}
  \item Gravity resolves into 2 components:
    \begin{align*}
      F_{y}&=-mg\cos\theta\\
      F_{x}&= mg\sin\theta
    \end{align*}
  \item No acceleration in $y$ direction (it doesn't float off the ramp), so we have:
    \begin{align*}
      \sum F_y=0\implies F_N-mg\cos\theta=0\implies F_N=mg\cos\theta
    \end{align*}
  \item Acceleration in $x$ gives $a_{up}$:
    \begin{align*}
      ma_{down}&=-\mu mg\cos\theta+mg\sin\theta\\
      \implies a_{down}&=-\mu g\cos\theta+g\sin\theta
    \end{align*}
  \end{itemize}
\end{frame}

\begin{frame}
  \frametitle{Summary}
  \begin{itemize}
  \item Two different accelerations, due to the different directions of the friction force:
    \begin{equation*}
      \boxed{\begin{aligned}
        a_{down}&=g\sin\theta-\mu g\cos\theta\\
        a_{up}&=g\sin\theta+\mu g\cos\theta
      \end{aligned}}
    \end{equation*}
  \item Goal of this lab, by measuring the accelerations, extract $\mu$ and $\theta$ through algebra. 
  \item Test $\theta$ value by measuring with an electronic level, since both methods will have uncertainty, so acceptance measurement is different:
    \begin{align*}
      \boxed{n_{\sigma}=\frac{\theta_{level}-\theta_{data}}
      {\sqrt{\sigma_{level}^2+\sigma_{data}^2}}}
    \end{align*}
  \item Only need to do error propagation for $\theta$
  \end{itemize}
\end{frame}

\section{Procedure}
\begin{frame}
  \frametitle{Data taking}
  \begin{itemize}
  \item Using a ramp with air being pumped through it.
  \item This reduces the friction but does not completely eliminate it (hence why we have $\mu$)
  \item The ramp is slightly tilted, so releasing it from the top will make it fall, video from here will give data for $a_{down}$
  \item At the bottom it will bounce back up the ramp, video from here will give the data for $a_{up}$
  \item Only one video is necessary, but 2 different runs in tracker
    \begin{itemize}
    \item One will be down the ramp
    \item The other will be up the ramp
    \end{itemize}
  \item Use tape to mark out a distance of 50cm on the ramp's meterstick, ensure it does not interfere with the glider
  \end{itemize}
\end{frame}

\begin{frame}
  \frametitle{Important Notes}
  \begin{itemize}
  \item Align your axes in Tracker with the ramp
  \item DO NOT FORGET TO MEASURE THE ANGLE OF THE RAMP WHEN YOU ARE DONE TAKING DATA
  \end{itemize}
\end{frame}

\section{Analysis}
\begin{frame}
  \frametitle{Analysis}
  \begin{itemize}
  \item Solving for $\sin\theta$:
    \pause
    \begin{align*}
      a_d+a_u=2g\sin\theta\implies
      \boxed{\sin\theta=\frac{a_d+a_u}{2g}}
    \end{align*}
  \item Use $\arcsin$ (in excel) to go from $\sin\theta$ to $\theta$, and $g=9.8$
    \pause
  \item Solving for $\mu$
    \pause
    \begin{align*}
      a_d-a_u=-2\mu g\cos\theta\implies
      \boxed{\mu=\frac{a_u-a_d}{2g\cos\theta}}
    \end{align*}
  \item Only other thing you need to calculate is $\sigma_\theta$ (in radians)
  \end{itemize}
\end{frame}

\begin{frame}
  \frametitle{Analysis Notes}
  \begin{itemize}
  \item You will find $a_u/a_d$ using same 3 methods from lab 2
    \begin{itemize}
    \item Num. Diff. on $x$ vs $t$ $\to$ $v$ vs $t$ Num. Diff. on $v$ vs $t$ $\to$ $a$ vs $t$
    \item Final value: mean of all $a$ values $\pm$ error on mean
    \item Num. Diff. on $x$ vs $t$ $\to$ $v$ vs $t$, Fit a line to $v$ vs. $t$
    \item Final value: slope of best fit line $\pm$ error from linest
    \item Fit a quadratic/second order polynomial on $x$ vs $t$
    \item Final value: 2*(coeff of $t^2$) $\pm$ 2*(error from linest)
    \end{itemize}
  \item For your final calculation of $\theta$, use the most \textbf{precise} \pause (=lowest error) \pause
  \item Excel's $\arcsin$ will give you a value in radians, while the level gives a value in degrees, I recommend converting the level's value to radians:
    \begin{align*}
      x^{\circ}=x^\circ*\frac{\pi\,\mathrm{rad}}{180^\circ}
    \end{align*}
  \end{itemize}
\end{frame}

\section{Bookkeeping}
\begin{frame}
  \frametitle{Bookkeeping}
  \begin{itemize}
  \item Lab is due next Friday @ 9:00 am
  \item Labs 3 on their way (if you dont have it already)
  \item Exam questions?
  \end{itemize}
\end{frame}

\end{document}